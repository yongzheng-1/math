\documentclass[utf8]{book}
\usepackage{titletoc}
\usepackage{titlesec}
\usepackage{ctexcap}
\usepackage[a4paper,text={125mm,195mm},centering,left=1in,right=1in,top=1in,bottom=1in]{geometry}
\usepackage[]{geometry}
\usepackage{imakeidx}
\usepackage{hyperref}
\usepackage{amsthm}
\usepackage{amsmath,amssymb}
\usepackage{extarrows}% http://ctan.org/pkg/extarrows

%%%% 定理类环境的定义 %%%%
\newtheorem{example}{}[section]             % 整体编号
\newtheorem{algorithm}{算法}
\newtheorem{theorem}{定理}[section]  % 按 section 编号
\newtheorem{definition}{定义}
\newtheorem{axiom}{公理}
\newtheorem{property}{性质}
\newtheorem{proposition}{命题}
\newtheorem{lemma}{引理}
\newtheorem{corollary}{推论}
\newtheorem{remark}{注}
\newtheorem{condition}{条件}
\newtheorem{conclusion}{结论}
\newtheorem{assumption}{假设}
\newtheorem{solution}{}
\renewcommand*{\proofname}{\normalfont\bfseries 证明}
\renewcommand*{\thesolution}{\normalfont\bfseries 解}
\renewcommand{\theexample}{\arabic{example}}

\DeclareMathOperator*\lowlim{\underline{lim}}
\DeclareMathOperator*\uplim{\overline{lim}}

\makeindex
\bibliographystyle{plain}
\begin{document}
\title{\heiti 徐森林,薛春华~编 \\ 《数学分析》题解}
\author{\fangsong 西海岸民工}
\date{2024年11月}

\frontmatter
\maketitle


\renewcommand\contentsname{目~录}
\tableofcontents

\mainmatter

%\part{总论}

\chapter{数列极限}

\section{数列极限的概念}
\subsection{练习题}
\begin{example}用数列极限定义证明:
\begingroup
\renewcommand\labelenumi{\normalfont(\theenumi)}
\begin{enumerate}
        \item $\displaystyle \lim_{n\to +\infty} 0.\underbrace{99\cdots9}_{n} = 1;$
        \begin{proof}
        对于 $\forall \varepsilon > 0$, 取 $N = \left [ \frac{-\ln\varepsilon}{\ln 10}\right ] + 1$, 当 $n > N$时,
        $$\left | 0.\underbrace{99\cdots9}_{n} - 1\right | = \frac{1}{10^n} < \varepsilon.$$
        所以, $$\lim_{n\to +\infty} 0.\underbrace{99\cdots9}_{n} = 1.$$
        \end{proof}
       
        \item $\displaystyle \lim_{n\to +\infty}\frac{3n+4}{7n-3} = \frac{3}{7};$
        \begin{proof}
        对于 $\forall \varepsilon > 0$, 取 $N = \left [ \frac{6}{\varepsilon}\right ] + 1$, 当 $n > N$时,
        $$\left | \frac{3n+4}{7n-3} - \frac{3}{7}\right | = \frac{37}{7(7n-3)} < \frac{37}{7n} < \frac{6}{n} < \varepsilon$$
        所以, $$\lim_{n\to +\infty}\frac{3n+4}{7n-3} = \frac{3}{7}.$$
        \end{proof}
        
        \item $\displaystyle \lim_{n\to +\infty}\frac{5n+6}{n^2-n-1000} = 0;$
        \begin{proof}
        对于 $\forall \varepsilon > 0$, 取 $N = \max\{50, \left [ \frac{12}{\varepsilon}\right ] + 1\}$, 当 $n > N$时,
        $$\frac{1}{2}n^2 - n - 1000 > 1$$ 且
        $$\left | \frac{5n+6}{n^2-n-1000} - 0\right | < \frac{5n + 6}{\frac{1}{2}n^2 + (\frac{1}{2}n^2 - n - 1000)} < \frac{6n}{\frac{1}{2}n^2} < \frac{12}{n} < \varepsilon$$
        所以, $$\lim_{n\to +\infty}\frac{5n+6}{n^2-n-1000} = 0.$$  
        \end{proof}
        
        \item $\displaystyle \lim_{n\to +\infty}\frac{8}{2^n+5} = 0;$
        \begin{proof}
         对于 $\forall \varepsilon > 0$, 取 $N = \left [ \frac{-\ln\varepsilon}{\ln2}\right ] + 4$, 当 $n > N$时,
        $$\left | \frac{8}{2^n+5} - 0\right | < \frac{8}{2^n} = \frac{1}{2^{n-3}} < \varepsilon$$
        所以, $$\lim_{n\to +\infty}\frac{8}{2^n+5} = 0.$$  
        \end{proof}
        
        \item $\displaystyle \lim_{n\to +\infty}\frac{\sin n!}{n^{1/2}} = 0;$
        \begin{proof}
         对于 $\forall \varepsilon > 0$, 取 $N = \left [ \left(\frac{1}{\varepsilon}\right)^2\right ] + 1$, 当 $n > N$时,
        $$\left | \frac{\sin n!}{n^{1/2}} - 0\right | < \frac{1}{n^{1/2}} < \varepsilon$$
        所以, $$\lim_{n\to +\infty}\frac{\sin n!}{n^{1/2}} = 0.$$  
        \end{proof}
        
        \item $\displaystyle \lim_{n\to +\infty}(\sqrt{n+2}-\sqrt{n-2}) = 0;$
        \begin{proof}
         对于 $\forall \varepsilon > 0$, 取 $N = \max\{2, \left [ \left(\frac{4}{\varepsilon}\right)^2\right ] + 1\}$, 当 $n > N$时,
        $$\left | \sqrt{n+2}-\sqrt{n-2}\right | = \frac{4}{\sqrt{n+2}+\sqrt{n-2}} < \frac{4}{\sqrt{n}} < \varepsilon$$
        所以, $$\lim_{n\to +\infty}(\sqrt{n+2}-\sqrt{n-2}) = 0.$$  
        \end{proof}
        
        \item $\displaystyle \lim_{n\to +\infty}(\sqrt[3]{n+2}-\sqrt[3]{n-2}) = 0;$
        \begin{proof}
         对于 $\forall \varepsilon > 0$, 取 $N = \max\{2, \left [ \sqrt{\left(\frac{4}{\varepsilon}\right)^3}\right ]\}$, 当 $n > N$时,
         \begin{equation*}
         \begin{split}
		\left | \sqrt[3]{n+2}-\sqrt[3]{n-2}\right | &= \frac{4}{\sqrt[3]{(n+2)^2}+\sqrt[3]{(n+2)(n-2)} + \sqrt[3]{(n-2)^2}} \\
 													 &< \frac{4}{\sqrt[3]{(n+2)^2}} \\
													 &< \varepsilon
		\end{split}
		\end{equation*}
        所以, $$\lim_{n\to +\infty}(\sqrt[3]{n+2}-\sqrt[3]{n-2}) = 0.$$  
        \end{proof}
        
        \item $\displaystyle \lim_{n\to +\infty}\frac{n^{3/2}\arctan{n}}{1+n^2} = 0;$
        \begin{proof}
         对于 $\forall \varepsilon > 0$, 取 $N = \left [ \left(\frac{\pi}{2\varepsilon}\right)^2\right ] + 1$, 当 $n > N$时,
         $$\left | \frac{n^{3/2}\arctan{n}}{1+n^2} \right | < \frac{\frac{\pi}{2}n^{3/2}}{n^2} < \frac{\pi / 2}{\sqrt{n}} < \varepsilon$$
        所以, $$\lim_{n\to +\infty}\frac{n^{3/2}\arctan{n}}{1+n^2} = 0.$$  
        \end{proof} 
        
        \item $\displaystyle \lim_{n\to +\infty}a_n = 1,$ 其中 
        $a_n = 
         \begin{cases}
         \displaystyle\frac{n-1}{n}, \text{n 为偶数,}\\
         \displaystyle\frac{\sqrt{n^2+ n}}{n}, \text{n 为奇数;}
		\end{cases}$      
		
		\begin{proof}
         对于 $\forall \varepsilon > 0$, 取 $N = \left [ \frac{1}{\varepsilon}\right ] + 1$, 当 $n > N$时,
         \begin{equation*}
         \begin{aligned}
         \left | a_n - 1\right | &<
         \begin{cases}
         \displaystyle\frac{1}{n}, \text{n 为偶数,}\\
         \displaystyle\frac{1}{\sqrt{n^2 + n} + n} , \text{n 为奇数}
		\end{cases}\\
		&< \frac{1}{n} \\
		&< \varepsilon
		\end{aligned}
		\end{equation*}
        所以, $$\lim_{n\to +\infty}a_n = 1.$$  
        \end{proof} 
        
        \item $\displaystyle \lim_{n\to +\infty}(n^3 - 4n - 5) = +\infty.$		
		\begin{proof}
         对于 $\forall A > 0$, 取 $N = \max\{5, \left [ \sqrt[3]{2A}\right ] + 1\}$, 当 $n > N$时,
         $$1 - \frac{4}{n^2} - \frac{5}{n^3} > 1 - \frac{9}{n^2} > \frac{1}{2}$$且
         $$n^3 - 4n - 5 = n^3(1 - \frac{4}{n^2} - \frac{5}{n^3}) > \frac{1}{2}n^3 > A$$
        所以, $$\lim_{n\to +\infty}(n^3 - 4n - 5) = +\infty.$$  
        \end{proof}  
\end{enumerate}
\endgroup
\end{example}
%%%%%%%%%%%%%%%%%%%%%%%%%%%%%%
\begin{example}
设$\displaystyle \lim_{n\to +\infty}a_n = a.$,证明: $\forall k \in \mathbb{N}$,有 $\displaystyle \lim_{n\to +\infty}a_{n+k} = a.$
\begin{proof}
我们分以下几种情况证明此命题:
\renewcommand\labelenumi{\normalfont(\theenumi)}
\begin{enumerate}
\item 当 $a\in \mathbb{R}$时。由于$\displaystyle \lim_{n\to +\infty}a_n = a$,对$\forall \varepsilon > 0, \exists N \in \mathbb{N}$,当$n > N$时,有$\left | a_n - a\right | < \varepsilon$。显然 $n+k > n > N$, 从而 $\left | a_{n + k} - a\right | < \varepsilon$. 即 $\displaystyle \lim_{n\to +\infty}a_{n+k} = a$.
\item 当 $a$ 是 $+\infty$时。由于$\displaystyle \lim_{n\to +\infty}a_n = +\infty$, 对$\forall A > 0, \exists N \in \mathbb{N}$, 当$n > N$时, $a_n > A$. 显然 $n+k > n > N$, 从而 $a_{n + k} > A$. 即 $\displaystyle \lim_{n\to +\infty}a_{n+k} = +\infty$.
\item 当 $a$ 是 $-\infty$时。由于$\displaystyle \lim_{n\to +\infty}a_n = -\infty$, 对$\forall A < 0, \exists N \in \mathbb{N}$, 当$n > N$时, $a_n < A$. 显然 $n+k > n > N$, 从而 $a_{n + k} < A$. 即 $\displaystyle \lim_{n\to +\infty}a_{n+k} = -\infty$.
\end{enumerate}
\end{proof}
\end{example}
%%%%%%%%%%%%%%%%%%%%%%%%%%%%%%

\begin{example}
设$\displaystyle \lim_{n\to +\infty}a_n = a$,证明$\displaystyle \lim_{n\to +\infty}\left |a_n\right | = \left |a\right |$:举例说明,这个命题的逆命题不真。
\begin{proof}
我们只证明$a$是有限实数的情况。当$a$是$+\infty$和$-\infty$时也成立。
由极限的定义有:对$\forall \varepsilon > 0, \exists N \in \mathbb{N}$,当$n > N$时,$$\left|a_n - a\right| < \varepsilon.$$
从而
$$\left| \left|a_n \right| - \left|a\right| \right | < \left|a_n - a\right| < \varepsilon.$$
所以 $$\lim_{n\to +\infty}\left |a_n\right | = \left |a\right |.$$

如果我们取$a_n = (-1)^n$, 则$\left | a_n\right | = 1$, 从而 $\displaystyle \lim_{n\to +\infty}\left |a_n\right | = \left |a\right |$。 但是很显然$a_n$是发散的。

\end{proof}

\end{example}
%%%%%%%%%%%%%%%%%%%%%%%%%%%%%%

\begin{example}

设$x_n \leq a \leq y_n, n \in \mathbb{N}$,且$\displaystyle \lim_{n\to +\infty}(y_n-x_n) = 0$。证明:
$$\displaystyle \lim_{n\to +\infty}x_n = \lim_{n\to +\infty}y_n = a$$

\begin{proof}
$\forall \varepsilon > 0, \exists N \in \mathbb{N}$, s.t. $y_n-x_n = \left | y_n - x_n\right | < \varepsilon$. 从而
$$\left| y_n - a\right| = y_n - a = y_n - x_n + x_n - a < y_n - x_n < \varepsilon.$$
即 $\displaystyle \lim_{n\to +\infty}y_n = a.$

同理可证
$$-\varepsilon < x_n - y_n < x_n - a < 0 < \varepsilon.$$ 即$\displaystyle \lim_{n\to +\infty}x_n = a.$

\end{proof}
\end{example}
%%%%%%%%%%%%%%%%%%%%%%%%%%%%%%
\begin{example}
设$\{a_n\}$为一个收敛数列。证明:数列$\{a_n\}$中或者有最大的数,或者有最小的数。 举出两者都有的例子; 再举出只有一个的例子。
\end{example}
\begin{proof}
假设$\displaystyle \lim_{n\to +\infty}a_n = a$,我们分以下几种情况讨论:
\renewcommand\labelenumi{\normalfont(\theenumi)}
\begin{enumerate}
\item 如果$a_n = a,\forall n\in \mathbb{N}$. 此时,数列$\{a_n\}$既有最小值也有最大值,且相等
\item 如果$\exists n_{0} \in \mathbb{N}$,使得$a_{n_{0}} \neq a$. 不妨假设$a_{n_{0}} < a$. 对于$\varepsilon = \frac{a - a_{n_{0}}}{2}$, $\exists N\in \mathbb {N}, 
N > n_{0}$使得
$a_n - a > a - \varepsilon = \frac{a + a_{n_{0}}}{2} > a_{n_{0}}, \forall n > N$. 取
$$m = \min\{a_{1}, a_{2}, \cdots, a_{N}\}.$$
我们有:
\renewcommand\labelenumi{\normalfont(\theenumi)}
\begin{enumerate}
\item $m \in \{a_n\}_{n = 1}^{+\infty},$
\item $a_n >= m, \forall n.$
\end{enumerate}
即,$m$ 是数列$\{a_n\}$的最小值. \\
如果$a_{n_{0}} > a$. 我们可以证明$\{a_n\}$有最大值。
\end{enumerate}

考虑下列收敛数列:
\renewcommand\labelenumi{\normalfont(\theenumi)}
\begin{enumerate}
\item 如果$a_n = \frac{1}{n}$, 则该数列有最大值$a_n \leq a_1 = 1$, 没有最小值。
\item 如果$a_n = -\frac{1}{n}$, 该数列有最小值$-1 = a_1 <= a_n$, 没有最大值。
\item 如果$a_n =(-1)^n \frac{1}{n}$, 则$-1 = a_1 <= a_n <= a_2 = \frac{1}{2}$ 
\end{enumerate}

\end{proof}
%%%%%%%%%%%%%%%%%%%%%%%%%%%%%%
\begin{example}
证明下列数列发散:
\renewcommand\labelenumi{\normalfont(\theenumi)}
\begin{enumerate}
\item $\{n^{(-1)^n}\}$
\begin{proof}
该数列发散,因为:
$$0 = \displaystyle \lim_{n\to +\infty}(2n-1)^{(-1)^{2n-1}} \neq \displaystyle \lim_{n\to +\infty}(2n)^{(-1)^{2n}} = +\infty$$
\end{proof}
\item $\{\cos n\}$
\begin{proof}
取两个整数子列$\{k_n\}, \{l_n\}$使得
\renewcommand\labelenumi{\normalfont(\theenumi)}
\begin{enumerate}
\item $k_n \in (2m\pi - \frac{\pi}{6},2m\pi + \frac{\pi}{6})$,
\item $l_n \in (2m\pi + \frac{5\pi}{6}, (2m+1)\pi + \frac{\pi}{6})$.
\end{enumerate}
显然,我们有
\renewcommand\labelenumi{\normalfont(\theenumi)}
\begin{enumerate}
\item $\cos k_n \in (\frac{\sqrt 3}{2}, 1], \forall n$,
\item $\cos l_n \in [-1, -\frac{\sqrt 3}{2}), \forall n$.
\end{enumerate}
因此,$\{\cos n\}$是发散的。
\end{proof}
\end{enumerate}
\end{example}
%%%%%%%%%%%%%%%%%%%%%%%%%%%%%%
\begin{example}
证明:数列$\{a_n\}$收敛$\Leftrightarrow$三个数列$\{a_{3k-2}\}, \{a_{3k-1}\}, \{a_{3k}\}$都收敛且有相同的极限。
\begin{proof}
$(\Rightarrow)$由定理1.1.2,收敛数列的子列也收敛,且极限相同。\\
$(\Leftarrow)$假设三个子列的极限都是$a$。由极限的定义, 对于$\forall \varepsilon > 0$, 
\begin{enumerate}
\renewcommand\labelenumi{\normalfont(\theenumi)}
\item $\exists N_{1} \in \mathbb{N}$, 使得$\left |a_{3k-2} - a\right | < \varepsilon, \forall k > N_{1}$,
\item $\exists N_{2} \in \mathbb{N}$, 使得$\left |a_{3k-1} - a\right | < \varepsilon, \forall k > N_{2}$,
\item $\exists N_{3} \in \mathbb{N}$, 使得$\left |a_{3k} - a\right | < \varepsilon, \forall k > N_{3}$。
\end{enumerate}
取$N = 3\max\{N_1,N_2,N_3\}$, 我们有$$\left | a_n - a \right | < \varepsilon, \forall n > N,$$
即 $\displaystyle \lim_{n\to +\infty}a_n = a.$
\end{proof}
\begin{remark}
这个命题对于$a = +\infty. -\infty, \infty$也成立。
\end{remark}
\begin{remark}
对于$\forall p \in \mathbb{N}$, $$\displaystyle \lim_{n\to +\infty}a_n = a \Leftrightarrow \lim_{k\to +\infty}a_{pk-p+1} = \lim_{k\to +\infty}a_{pk-p+2} 
=\cdots = \lim_{k\to +\infty}a_{pk} = a.$$

\end{remark}

\end{example}
%%%%%%%%%%%%%%%%%%%%%%%%%%%%%%
\begin{example}
设$\displaystyle \lim_{n\to +\infty}(a_n - a_{n-1}) = d$。证明:$\displaystyle \lim_{n\to +\infty}\frac{a_n}{n} = d$。
\begin{proof}
$$\frac{a_n - a_1}{n} = \frac{(a_n - a_{n-1}) + (a_{n-1} - a_{n-2}) + \cdots + (a_2 - a_1)}{n}.$$
由例1.1.15知:
$$\displaystyle \lim_{n\to +\infty}\frac{a_n - a_1}{n} = d.$$
由于$\displaystyle \lim_{n\to +\infty}\frac{a_1}{n} = 0$, 易知
$$\displaystyle \lim_{n\to +\infty}\frac{a_n}{n} = d.$$
\end{proof}
\end{example}

%%%%%%%%%%%%%%%%%%%%%%%%%%%%%%
\begin{example}
设$\displaystyle \lim_{n\to +\infty}a_n = a$。用$\varepsilon-N$法,$A-N$法证明:
$$\displaystyle \lim_{n\to +\infty}\frac{a_1+2a_2+\cdots+na_n}{n^2} = \frac{a}{2}, (a \text{为实数}, +\infty, -\infty).$$

\end{example}
\begin{proof}
我们只证$a$为实数的情形。其他的情况证明类似。
由极限的定义,对于$\varepsilon > 0$, $\exists N_0\in \mathbb{N}$, 使得$$\left|a_n - a\right| < \frac{\varepsilon}{3}, \forall n > N_0.$$
\begin{equation*}
\begin{split}
&\left | \frac{a_1+2a_2+\cdots+na_n}{n^2}  - \frac{a}{2} \right | \\&= \left | \frac{(a_1-a)+2(a_2-a)+\cdots+n(a_n-a)}{n^2} +\frac{n(n+1)}{2n^2}a - \frac{a}{2} \right |\\
 													 &< \left | \frac{(a_1-a)+2(a_2-a)+\cdots+n(a_n-a)}{n^2}\right | + \frac{a}{2n} \\
													 &<\left | \frac{(a_1-a)+2(a_2-a)+\cdots+N_0(a_{N_0}-a)}{n^2}\right | + \frac{(N_0+1+n)(n-N_0)}{2n^2}\frac{\varepsilon}{3} + \frac{a}{2n}.\\
	\end{split}
\end{equation*}
取$N_1 \in \mathbb{N}$,使得
$$\left | \frac{(a_1-a)+2(a_2-a)+\cdots+N_0(a_{N_0}-a)}{n^2}\right | <\frac{\varepsilon}{3}, \forall n > N_1.$$
取$N_2 \in \mathbb{N}$,使得
$$\frac{a}{2n} < \frac{\varepsilon}{3}, \forall n > N_2.$$
取$N_3\in \mathbb{N}$, 使得
$$\frac{(N_0+1+n)(n-N_0)}{2n^2}\frac{\varepsilon}{3} < \frac{\varepsilon}{3}, \forall n> N_3.$$

最后, 取$N = \max\{N_0, N_1, N_2, N_3\}$, $\forall n > N$, 我们有
$$\left | \frac{a_1+2a_2+\cdots+na_n}{n^2}  - \frac{a}{2} \right | < \varepsilon.$$
即$$\displaystyle \lim_{n\to +\infty}\frac{a_1+2a_2+\cdots+na_n}{n^2} = \frac{a}{2}.$$
\end{proof}
%%%%%%%%%%%%%%%%%%%%%%%%%%%%%%
\subsection{思考题}
%%%%%%%%%%%%%%%%%%%%%%%%%%%%%%
\begin{example}
设$\displaystyle \lim_{n\to +\infty}a_n = a$,$\left |q \right | < 1$。用$\varepsilon-N$法证明:
$$\displaystyle \lim_{n\to +\infty}(a_n + a_{n-1}q + \cdots + a_1q^{n-1}) = \frac{a}{1-q}.$$
\end{example}
\begin{proof}

对于$\forall \varepsilon > 0$, 由$\displaystyle \lim_{n\to +\infty}a_n = a$,则存在整数$M > 0$和$N_0 \in \mathbb{N}$ 使得
$$\left | a_n - a \right | < M, \forall n\in\mathbb{N},$$
$$\left | a_n - a \right | < \frac{(1-|q|)\varepsilon}{3}, \forall n > N_0.$$
我们知道,当$\left| q\right| < 1$时,$\displaystyle \lim_{n\to +\infty}q^n = 0$。于是$\exists N_1 \in \mathbb {N}$使得
$$\left|q^{n-k}\right| < \max\left \{\frac{1}{3MN_0}, \frac{1-|q|}{3(|a|+1)}\right\}\varepsilon, \forall n > N_1, k=0, 1, 2, \cdots, N_0.$$
我们现在取$N = \max\{N_0, N_1\}$。对任意的$n > N$时,有
\begin{equation*}
\begin{split}
&\left | (a_n + a_{n-1}q + \cdots + a_1q^{n-1}) - \frac{a}{1-q} \right | \\
&= \left | (a_n + a_{n-1}q + \cdots + a_1q^{n-1}) - a\frac{1-q^n}{1-q} + \frac{aq^n}{1-q}\right|\\
&< \left | (a_n - a) + (a_{n-1} - a)q + \cdots + (a_1- a)q^{n-1} \right | + \frac{\left|a\right|\left|q\right|^n}{\left|1-q\right|} \\
&<\frac{(1-|q|)\varepsilon}{3}(1 + |q| + \cdots + |q|^{n-N_0}) + \left(MN_0+\frac{a}{1-|q|}\right)\max\left \{\frac{1}{3MN_0}, \frac{1-|q|}{3(|a|+1)}\right\}
\varepsilon\\&< \varepsilon.
	\end{split}
\end{equation*}
即,$\displaystyle \lim_{n\to +\infty}(a_n + a_{n-1}q + \cdots + a_1q^{n-1}) = \frac{a}{1-q}.$ 

\end{proof}
%%%%%%%%%%%%%%%%%%%%%%%%%%%%%%
\begin{example}
设$\displaystyle \lim_{n\to +\infty}a_n = a$,$\displaystyle \lim_{n\to +\infty}b_n = b$。用$\varepsilon-N$法证明:
$$\displaystyle \lim_{n\to +\infty}\frac{a_0b_n+a_1b_{n-1}+\cdots+a_{n-1}b_1+a_nb_0}{n}= ab.$$
\end{example}
\begin{proof}
首先我们证明命题在$b=0$时成立。
\begin{enumerate}
\renewcommand\labelenumi{\normalfont(\theenumi)}
\item 由于$\{a_n\}$收敛,则$\exists M > 0$使得$|a_n| < M, \forall n \in \mathbb {N}$.
\item 对于$\forall \varepsilon > 0$, 由于$\{b_n\}$收敛到$0$, 则$\exists N_0 \in \mathbb {N}$使得$|b_n| < \frac{\varepsilon}{2M}, \forall n > N_0$.
\item 由于$|a_n| < M$, 对上述的$\varepsilon > 0$, $\exists N_1 \in \mathbb{N}$使得
$$\left|\frac{a_{n - N_0}b_{N_0} + a_{n - N_0 + 1}b_{N_0 - 1}+\cdots + a_nb_0}{n}\right| < \frac{\varepsilon}{2}.$$
\end{enumerate}
取$N = \max\{N_0, N_1\}$, 对于上述的$\varepsilon > 0$, 当$n > N$, 有
\begin{equation*}
\begin{split}
&\left | \frac{a_0b_n+a_1b_{n-1}+\cdots+a_{n-1}b_1+a_nb_0}{n} \right | \\
&< \left | \frac{a_0b_n+a_1b_{n-1}+\cdots+a_{n-N_0-1}b_{N_0+1}}{n}\right| 
+ \left | \frac{a_{n-N_0}b_{N_0}+ a_{n-N_0 + 1}b_{N_0-1}+\cdots+a_{n}b_0}{n}\right|\\
&< \frac{\varepsilon}{2M}\frac{(n - N_0)M}{n} + \frac{\varepsilon}{2}\\
&< \varepsilon.
	\end{split}
\end{equation*}
即$$\displaystyle \lim_{n\to +\infty}\frac{a_0b_n+a_1b_{n-1}+\cdots+a_{n-1}b_1+a_nb_0}{n}= 0.$$

下面证明命题在$b\neq 0$时也成立。
\begin{enumerate}
\renewcommand\labelenumi{\normalfont(\theenumi)}
\item 由于$\displaystyle \lim_{n\to +\infty}a_n = a$收敛,则$\displaystyle \lim_{n\to +\infty}(a_n - a)b = 0$. 由此可知
$$\displaystyle \lim_{n\to +\infty}\frac{(a_0-a)b+(a_1-a)b+\cdots+(a_{n-1}-a)b+(a_n-a)b}{n}= 0.$$
\item 由于$\displaystyle \lim_{n\to +\infty}a_n = a$和$\displaystyle \lim_{n\to +\infty}(b_n - b) = 0$, 则
$$\displaystyle \lim_{n\to +\infty}\frac{a_0(b_n-b)+a_1(b_{n-1} - b)+\cdots+a_{n-1}(b_1 - b)+a_n(b_n-b)}{n}= 0.$$
\item 
\begin{equation*}
\begin{split}
&\left | \frac{a_0b_n+a_1b_{n-1}+\cdots+a_{n-1}b_1+a_nb_0}{n} - ab \right | \\
&= \left | \frac{a_0(b_n-b)+a_1(b_{n-1}-b)+\cdots+a_n(b_0-b)}{n} + \frac{(a_0-a)b+(a_1 - a)b+\cdots+(a_n-a)b)}{n}\right|
\end{split}
\end{equation*}
\item 对于$\forall \varepsilon > 0$, $\exists N \in \mathbb {N}$,使得当$n > N$时, 
$$\left | \frac{a_0(b_n-b)+a_1(b_{n-1}-b)+\cdots+a_n(b_0-b)}{n}\right| < \frac{\varepsilon}{2},$$
$$\left | \frac{(a_0-a)b+(a_1 - a)b+\cdots+(a_n-a)b)}{n}\right| < \frac{\varepsilon}{2}.$$
从而 $$\left | \frac{a_0b_n+a_1b_{n-1}+\cdots+a_{n-1}b_1+a_nb_0}{n} - ab \right | < \varepsilon.$$
即$$\displaystyle \lim_{n\to +\infty}\frac{a_0b_n+a_1b_{n-1}+\cdots+a_{n-1}b_1+a_nb_0}{n}= ab.$$
\end{enumerate}
\end{proof}
%%%%%%%%%%%%%%%%%%%%%%%%%%%%%%
\begin{example}
设$\displaystyle \lim_{n\to +\infty}a_n = a$,$b_n \geq 0 (n\in\mathbb{N})$,$\displaystyle \lim_{n\to +\infty}(b_1+b_2+\cdots+b_n) = S$。证明:$\displaystyle \lim_{n\to +\infty}(a_nb_1+a_{n-1}b_2+\cdots+a_1b_n) = aS$.
\end{example}
\begin{proof}
我们分以下步骤证明该命题。
\begin{enumerate}
\renewcommand\labelenumi{\normalfont(\theenumi)}
\item 首先我们证明$\displaystyle \lim_{n\to +\infty}b_n = 0$.
\item 
\begin{equation*}
\begin{split}
&\left | (a_nb_1+a_{n-1}b_2+\cdots+a_1b_n) - aS \right | \\
&= \left | (a_nb_1+a_{n-1}b_2+\cdots+a_1b_n) - a(b_1+b_2+\cdots+b_n)+a(b_1+b_2+\cdots+b_n -S)\right| \\
&< \left | (a_n-a)b_1+(a_{n-1}-a)b_2+\cdots+(a_1-a)b_n\right| + \left|a\right|\left|(b_1 + b_2+\cdots + b_n) -S\right|
\end{split}
\end{equation*}
由$\displaystyle \lim_{n\to +\infty}a_n = a$ 和 $\displaystyle \lim_{n\to +\infty}b_n = 0$得知
$$\left | (a_n-a)b_1+(a_{n-1}-a)b_2+\cdots+(a_1-a)b_n\right| < \frac{\varepsilon}{2}.$$
由$\displaystyle \lim_{n\to +\infty}(b_1+b_2+\cdots+b_n) = S$可得知
$$\left|a\right|\left|(b_1 + b_2+\cdots + b_n) -S\right| < \frac{\varepsilon}{2}.$$
\end{enumerate}
综上,$$\displaystyle \lim_{n\to +\infty}(a_nb_1+a_{n-1}b_2+\cdots+a_1b_n) = aS.$$
\end{proof}
\begin{remark}
这题里的条件$b_m \geq 0 (n\in\mathbb{N})$不是必须的。只要$\displaystyle \lim_{n\to +\infty}(|b_1|+|b_2|+\cdots+|b_n|) = S$就够了。
\end{remark}
\begin{remark}
这题是第10题的推广。如果$b_n = q^{n-1}, 0 < q < 1$,则
$$\displaystyle \lim_{n\to +\infty}(b_1 + b_2 + \cdots + b_n) = \lim_{n\to +\infty}\frac{1-q^n}{1-q}=\frac{1}{1-q}.$$ 
由这题的结论,第10题得证。
\end{remark}
%%%%%%%%%%%%%%%%%%%%%%%%%%%%%%
\begin{example}
(Toeplitz定理) 设$n,k\in\mathbb{N}$,$t_{nk} \geq 0$且$\displaystyle\sum_{k=1}^nt_{nk}=1$,$\displaystyle \lim_{n\to +\infty}t_{nk} = 0$。
如果$\displaystyle \lim_{n\to +\infty}a_n = a$,证明:$\displaystyle \lim_{n\to +\infty}\sum_{k=1}^nt_{nk}a_k = a$。说明例1.1.15为Toeplitze定理的特殊情形。
\end{example}
\begin{proof}
对于$\forall \varepsilon > 0$, 我们有:
\begin{enumerate}
\renewcommand\labelenumi{\normalfont(\theenumi)}
\item $\exists N_0\in \mathbb{N}$,当$n > N_0$时, $\left| a_n - a\right| < \frac{\varepsilon}{2}$.
\item 我们取$M = \max\{|a_1 - a|, |a_2 - a|, \cdots, |a_{N_0} - a|\}$.
\item 对于$l\in\mathbb{N}, 1 \leq l \leq N_{0}$, 存在$N_l \in \mathbb{N}$使得$t_{nl} < \frac{\varepsilon}{2N_0M}, \forall n > N_l$.
\item 取$N = \max \{N_0, N_1, \cdots, N_{N_0}\}$, 当$n > N$时,我们有:
\begin{equation*}
\begin{split}
&\left | \displaystyle \sum_{k=1}^nt_{nk}a_k - a \right | \\
&= \displaystyle \sum_{k=1}^{N_0}t_{nk}\left |(a_k-a)\right| + \displaystyle \sum_{k=N_{0}}^{n}t_{nk}\left |(a_k-a)\right|\\
&< \displaystyle \sum_{k=1}^{N_0}\frac{\varepsilon}{2N_0M}M + \displaystyle \sum_{k=N_{0}}^{n}t_{nk}\frac{\varepsilon}{2}\\
&=\varepsilon
\end{split}
\end{equation*}
\end{enumerate}
所以$$\displaystyle \lim_{n\to +\infty}\sum_{k=1}^nt_{nk}a_k = a.$$
如果我们取$b_{nk} = \frac{1}{n}$, 则例1.1.15就可以由这题得证。
\end{proof}
%%%%%%%%%%%%%%%%%%%%%%%%%%%%%%
\begin{example}
设$a,b,c$为三个给定的实数,令$a_0=a,b_0=b,c_0=c$,并归纳定义
\begin{equation*}
\begin{cases}
a_n = \frac{b_{n-1}+c_{n-1}}{2},\\
b_n = \frac{a_{n-1}+c_{n-1}}{2}, \quad n=1,2,\cdots.\\
c_n = \frac{a_{n-1}+b_{n-1}}{2},
\end{cases}
\end{equation*}
证明:$\displaystyle \lim_{n\to +\infty}a_n = \lim_{n\to +\infty}b_n=\lim_{n\to +\infty}c_n = \frac{a+b+c}{3}$.
\end{example}

\begin{proof}
我们通过以下结论去证明该命题:
\begin{enumerate}
\renewcommand\labelenumi{\normalfont(\theenumi)}
\item $\displaystyle \lim_{n\to +\infty}(a_n + b_n + c_n) = a+b+c$. 这是因为$a_n + b_n + c_n = a_{n-1}+b_{n-1}+c_{n-1} = \cdots = a+b+c$.
\item $\displaystyle \lim_{n\to +\infty}(a_n -b_n) = 0$, $\displaystyle \lim_{n\to +\infty}(a_n -c_n) = 0$, $\displaystyle \lim_{n\to +\infty}(c_n -b_n) = 0$.
这是因为$$a_n - b_n = \left(-\frac{1}{2}\right)(a_{n-1}-b_{n-1}) = \cdots = \left(-\frac{1}{2}\right)^n(a-b),$$
$$a_n - c_n = \left(-\frac{1}{2}\right)(a_{n-1}-c_{n-1}) = \cdots = \left(-\frac{1}{2}\right)^n(a-c),$$
$$c_n - b_n = \left(-\frac{1}{2}\right)(c_{n-1}-b_{n-1}) = \cdots = \left(-\frac{1}{2}\right)^n(c-b).$$
\item $$\displaystyle \lim_{n\to +\infty}3a_n = \lim_{n\to +\infty}(a_n+b_n+c_n +(a_n-b_n) + (a_n-c_n))=a+b+c,$$
从而, $$\displaystyle \lim_{n\to +\infty}a_n  = \frac{a+b+c}{3}.$$
\item 同理可证,$\displaystyle \lim_{n\to +\infty}b_n=\lim_{n\to +\infty}c_n = \frac{a+b+c}{3}$.
\end{enumerate}

\end{proof}
%%%%%%%%%%%%%%%%%%%%%%%%%%%%%%
\begin{example}
设$a_1,a_2$为实数,令$$a_n = pa_{n-1} + qa_{n-2}, n = 3,4,5,\cdots,$$
其中$p>0$,$q>0$, $p+q = 1$。证明:数列$\{a_n\}$收敛,且$\displaystyle \lim_{n\to +\infty}a_n=\frac{a_2+a_1q}{1+q}.$
\end{example}
\begin{proof}
由递推公式,我们可以证明$$a_n-a_{n-1} =\left(-q\right)^{n-2}(a_2 - a_1), \forall n \geq 3.$$
由此我们可以得出$a_n$的通项公式
$$a_n = a_2 + \displaystyle \sum_{k=1}^{n-2}\left(-q\right)^k(a_2 - a_1) = a_2 - \frac{q+(-q)^{n-1}}{1+q}(a_2-a_1).$$
从而,$$\displaystyle \lim_{n\to +\infty}a_n = a_2 - \frac{q}{1+q}(a_2-a_1) = \frac{a_2+qa_1}{1+q}.$$
\end{proof}
%%%%%%%%%%%%%%%%%%%%%%%%%%%%%%
\begin{example}
设数列$\{a_n\}$,$\{b_n\}$,$\{c_n\}$满足$a_1 > 0$,$4 \leq b_n \leq 5$,$4 \leq c_n \leq 5$,$$\displaystyle a_n=\frac{\sqrt{b_n^2+c_n^2}}{b_n+c_n}a_{n-1}$$
证明:$\displaystyle \lim_{n\to +\infty}a_n=0$.
\end{example}
\begin{proof}
由通项公式定义有$$0\leq a_n \leq \frac{5\sqrt 2}{8}a_{n-1} \leq \cdots \leq \left(\frac{5\sqrt 2}{8}\right)^{n-1}a_1.$$
由$\displaystyle\frac{5\sqrt 2}{8} < 1$知$\displaystyle \lim_{n\to +\infty}a_n=0$。
\end{proof}

%%%%%%%%%%%%%%%%%%%%%%%%%%%%%%
\section{数列极限的基本性质}
%%%%%%%%%%%%%%%%%%%%%%%%%%%%%%
\subsection{练习题}
\begin{example}应用数列极限的基本性质求下列极限:
\begingroup
\renewcommand\labelenumi{\normalfont(\theenumi)}
\begin{enumerate}
\item $\displaystyle \lim_{n\to +\infty}\frac{4n^2-n +5}{3n^2 -2n -7}$
\begin{solution}
$\displaystyle \lim_{n\to +\infty}\frac{4n^2-n +5}{3n^2 -2n -7}=\lim_{n\to +\infty}\frac{4-1/n +5/n^2}{3-2/n -7/n^2} = 4/3$
\end{solution}
\item $\displaystyle \lim_{n\to +\infty}\frac{3^n+(-2)^n}{3^{n+1} +(-2)^{n+1}}$
\begin{solution}
$\displaystyle \lim_{n\to +\infty}\frac{3^n+(-2)^n}{3^{n+1} +(-2)^{n+1}}=\lim_{n\to +\infty}\frac{1+(-2/3)^n}{3+(-2)(-2/3)^n} = 1/3$
\end{solution}
\item $\displaystyle \lim_{n\to +\infty}\left(1-\frac{1}{n}\right)^{\frac{1}{n}}$
\begin{solution}
$1 = \displaystyle \lim_{n\to +\infty}\frac{1}{\sqrt[n]2} \leq \lim_{n\to +\infty}\left(1-\frac{1}{n}\right)^{\frac{1}{n}} \leq 1$. 于是 $\displaystyle \lim_{n\to +\infty}\left(1-\frac{1}{n}\right)^{\frac{1}{n}} = 1$.
\end{solution}
\item $\displaystyle \lim_{n\to +\infty}(2\sin^2n+\cos^2n)^{\frac{1}{n}}$
\begin{solution}
$1 \leq \displaystyle \lim_{n\to +\infty}(2\sin^2n+\cos^2n)^{\frac{1}{n}}\leq \lim_{n\to +\infty}\sqrt[n]2 \leq 1$. 
于是 $\displaystyle \lim_{n\to +\infty}(2\sin^2n+\cos^2n)^{\frac{1}{n}} = 1$.
\end{solution}
\item $\displaystyle \lim_{n\to +\infty}(\arctan n)^{\frac{1}{n}}$
\begin{solution}
$1 \leq \displaystyle \lim_{n\to +\infty}(\arctan n)^{\frac{1}{n}}\leq \lim_{n\to +\infty}\sqrt[n]{\frac{\pi}{2}} \leq 1$. 
于是 $\displaystyle \lim_{n\to +\infty}(\arctan n)^{\frac{1}{n}} = 1$.
\end{solution}
\item $\displaystyle \lim_{n\to +\infty}\frac{1+a+\cdots+a^{n-1}}{1+b+\cdots +b^{n-1}}, |a| < 1, |b| < 1$
\begin{solution}
$\displaystyle \lim_{n\to +\infty}\frac{1+a+\cdots+a^{n-1}}{1+b+\cdots +b^{n-1}} = 
\lim_{n\to +\infty}\left(\frac{1-a^n}{1-a}\right)\left(\frac{1-b}{1-b^n}\right) = \frac{1-b}{1-a}$. 
\end{solution}

\item $\displaystyle \lim_{n\to +\infty}\left(\frac{1}{1\cdot 2}+\frac{1}{2\cdot 3} +\cdots+\frac{1}{n(n+1)}\right)$
\begin{solution}
$\displaystyle \lim_{n\to +\infty}\left(\frac{1}{1\cdot 2}+\frac{1}{2\cdot 3} +\cdots+\frac{1}{n(n+1)}\right) = 
\lim_{n\to +\infty}\left(1 - \frac{1}{n+1}\right) = 1$
\end{solution}

\item $\displaystyle \lim_{n\to +\infty}\left(1-\frac{1}{2^2}\right)\left(1-\frac{1}{3^2}\right)\cdots\left(1-\frac{1}{n^2}\right)$
\begin{solution}
$\displaystyle \lim_{n\to +\infty}\left(1-\frac{1}{2^2}\right)\left(1-\frac{1}{3^2}\right)\cdots\left(1-\frac{1}{n^2}\right) = 
\lim_{n\to +\infty}\left(1 - \frac{1}{2}\right)\left(1 + \frac{1}{n}\right) = \frac{1}{2}$
\end{solution}

\item $\displaystyle \lim_{n\to +\infty}\left(\frac{1}{2}+\frac{3}{2^2}+\cdots+\frac{2n-1}{2^n}\right)$
\begin{solution}
记$$S_n = \frac{1}{2}+\frac{3}{2^2}+\cdots+\frac{2n-1}{2^n},$$ 则
$$\frac{1}{2}S_n = \frac{1}{2^2}+\frac{3}{2^3}+\cdots+\frac{2(n-1)-1}{2^{n}}+\frac{2n-1}{2^{n+1}}.$$
于是
\begin{equation*}
\begin{split}
\frac{1}{2}S_n &= \frac{1}{2}+\left(\frac{1}{2} + \frac{1}{2^2}+\cdots+\frac{1}{2^{n-1}}\right)-\frac{2n-1}{2^{n+1}}\\
&=\frac{3}{2}-\frac{1}{2^{n-1}}-\frac{2n-1}{2^{n+1}}
\end{split}
\end{equation*}
从而$$\displaystyle \lim_{n\to +\infty}\left(\frac{1}{2}+\frac{3}{2^2}+\cdots+\frac{2n-1}{2^n}\right) = 3.$$
\end{solution}

\item $\displaystyle \lim_{n\to +\infty}\left(1-\frac{1}{1+2}\right)\left(1-\frac{1}{1+2+3}\right)+\cdots+\left(1-\frac{1}{1+2+\cdots+n}\right)$
\begin{solution}
$\displaystyle 1-\frac{1}{1+2+\cdots+k} = \frac{(k-1)(k+2)}{k(k+1)}$.从而
\begin{equation*}
\begin{split}
&\left(1-\frac{1}{1+2}\right)\left(1-\frac{1}{1+2+3}\right)+\cdots+\left(1-\frac{1}{1+2+\cdots+n}\right) \\
&=\frac{1\cdot 4}{2\cdot 3}\frac{2 \cdot 5}{3\cdot 4}\cdots \frac{(n-1)\cdot (n+2)}{n\cdot (n+1)}
\end{split}
\end{equation*}
分子的$2n$项的积:奇数项的积是$(n-1)!$, 偶数项的积是$\frac{1}{2\cdot 3}(n+2)!$.\\
分母的$2n$项的积:奇数项的积是$n!$, 偶数项的积是$\frac{1}{2}(n+1)!$.\\
于是$\displaystyle \lim_{n\to +\infty}\left(1-\frac{1}{1+2}\right)\left(1-\frac{1}{1+2+3}\right)+\cdots+\left(1-\frac{1}{1+2+\cdots+n}\right) = 
\lim_{n\to +\infty}\frac{n+2}{3n} = \frac{1}{3}.$
\end{solution}

\item $\displaystyle \lim_{n\to +\infty}\left[\frac{1^2}{n^3}+\frac{3^2}{n^3}+\cdots+\frac{(2n-1)^2}{n^3}\right]$
\begin{solution}
$$\sum_{k=1}^{n}(2k-1)^2 =\sum_{k=1}^{2n}k^2 - 4\sum_{k=1}^{n}k^2= \frac{8n^3 - 2n}{6}.$$
于是$$\displaystyle \lim_{n\to +\infty}\left[\frac{1^2}{n^3}+\frac{3^2}{n^3}+\cdots+\frac{(2n-1)^2}{n^3}\right] = \frac{4}{3}.$$
\end{solution}

\item $\displaystyle \lim_{n\to +\infty}(1+x)(1+x^2)(1+x^4)\cdots(1+x^{2^{n - 1}})$
\begin{solution}
$\displaystyle \lim_{n\to +\infty}(1+x)(1+x^2)(1+x^4)\cdots(1+x^{2^{n - 1}}) = \displaystyle \lim_{n\to +\infty}\frac{1-x^{2^n}}{1-x}=\frac{1}{1-x}.$
\end{solution}

\item $\displaystyle \lim_{n\to +\infty}(\sqrt{n+2}-2\sqrt{n+1}+\sqrt{n})$
\begin{solution}

\end{solution}
$\displaystyle \lim_{n\to +\infty}(\sqrt{n+2}-2\sqrt{n+1}+\sqrt{n})=
\lim_{n\to +\infty}\left(\frac{1}{\sqrt{n+2}+\sqrt{n+1}}-\frac{1}{\sqrt{n+1}+\sqrt{n}}\right)=0$.
\end{enumerate}
\endgroup
\end{example}
%%%%%%%%%%%%%%%%%%%%%%%%
\begin{example}
设$a_n > 0$,$n\in\mathbb{N}$,$\displaystyle \lim_{n\to +\infty}\frac{a_{n+1}}{a_n} = a$。应用例1.2.6证明:$\displaystyle \lim_{n\to +\infty}\sqrt[n]{a_n} = a$.
\end{example}
\begin{proof}
$$\sqrt[n]{a_n} = \sqrt[n]{\frac{a_n}{a_{n-1}}\cdot \frac{a_{n-1}}{a_{n-2}}\cdots\frac{a_2}{a_1}}\cdot \sqrt[n]{a_1}$$
于是$\displaystyle \lim_{n\to +\infty}\sqrt[n]{a_n} = a$.
\end{proof}
%%%%%%%%%%%%%%%%%%%%%%%

\begin{example}
设$\displaystyle \lim_{n\to +\infty}a_n = a$。应用夹逼定理证明:$\displaystyle \lim_{n\to +\infty}\frac{[na_n]}{n} = a$,其中$[x]$表示不超过的最大整数。
\end{example}
\begin{proof}
$$a=\displaystyle \lim_{n\to +\infty}\frac{na_n-1}{n} \leq \displaystyle \lim_{n\to +\infty}\frac{[na_n]}{n} \leq 
\displaystyle \lim_{n\to +\infty}\frac{na_n}{n} = a.$$
\end{proof}
%%%%%%%%%%%%%%%%%%%%%%

\begin{example}
设$a_n \neq 0$且$\displaystyle \lim_{n\to +\infty}\left|\frac{a_{n+1}}{a_n}\right| = r > 1$。证明:$\displaystyle \lim_{n\to +\infty}a_n = \infty$.
\end{example}
\begin{proof}
取$\varepsilon = \frac{r - 1}{2}$.由极限的定义,存在$N\in\mathbb{N}$使得$\left|\frac{a_{n+1}}{a_n}\right| > r - \varepsilon = \frac{r + 1}{2} > 1$. 
于是 $$\left|a_n\right| > \left(\frac{r+1}{2}\right)^{n_N}\left|a_N\right|.$$
即$\displaystyle \lim_{n\to +\infty}a_n = \infty$。
\end{proof}
%%%%%%%%%%%%%%%%%%%%%%
\begin{example}

\renewcommand\labelenumi{\normalfont(\theenumi)}
\begin{enumerate}
\item 应用数学归纳法或$\displaystyle\frac{2k-1}{2k}<\frac{2k}{2k+1}$证明不等式:
$$\frac{1}{2}\cdot\frac{3}{4}\cdot\cdots\cdot\frac{2n-1}{2n} < \frac{1}{\sqrt{2n+1}}.$$
\begin{proof}
记$S_n = \frac{1}{2}\cdot\frac{3}{4}\cdot\cdots\cdot\frac{2n-1}{2n}$. 利用不等式$\displaystyle\frac{2k-1}{2k}<\frac{2k}{2k+1}$, 我们有
$$S_n < \frac{2}{3}\cdot\frac{4}{5}\cdot\cdots\cdot\frac{2n}{2n+1} = \frac{1}{S_n(2n+1)}.$$
于是$S_n < \frac{1}{\sqrt{2n+1}}$

\end{proof}

\item 证明:$\displaystyle \lim_{n\to +\infty}\left(\frac{1}{2}\cdot\frac{3}{4}\cdot\cdots\cdot\frac{2n-1}{2n}\right) = 0$
\begin{proof}
$$0<\displaystyle \lim_{n\to +\infty}\left(\frac{1}{2}\cdot\frac{3}{4}\cdot\cdots\cdot\frac{2n-1}{2n}\right) \leq \lim_{n\to +\infty}\frac{1}{\sqrt{2n+1}}=0.$$
\end{proof}
\end{enumerate}

\end{example}
%%%%%%%%%%%%%%%%%%%%

\begin{example}
设$a_n > 0 (n\in\mathbb{N})$且$\displaystyle \lim_{n\to +\infty}a_n = a > 0$。应用夹逼定理证明:$\displaystyle \lim_{n\to +\infty}\sqrt[n]{a_n} = 1$
\end{example}
\begin{proof}
由于$\displaystyle \lim_{n\to +\infty}a_n = a > 0$,我们有一下结论:
$$\displaystyle \lim_{n\to +\infty}\frac{a_n}{n} = 0, \displaystyle \lim_{n\to +\infty}\frac{\frac{1}{a_n}}{n} = 0.$$
同时,我们有
\begin{equation*}
\begin{split}
1&=\lim_{n\to +\infty}\left(\frac{1}{\left(1 + 1 +\cdots+1+\frac{1}{a_n}\right)/n}\right)\\
&=\left(\frac{1}{\displaystyle\lim_{n\to +\infty}\left(n- 1 +\frac{1}{a_n}\right)/n}\right)\\
&\leq \lim_{n\to +\infty}\sqrt[n]{a_n} =\lim_{n\to +\infty}\sqrt[n]{\left(1\cdot 1\cdot \cdots \cdot 1\cdot a_n\right)}\\
&\leq \lim_{n\to +\infty}\frac{\left(1 + 1 +\cdots+1+a_n\right)}{n} \\
&=\lim_{n\to +\infty}\frac{\left(n-1+a_n\right)}{n}\\
&=1.
\end{split}
\end{equation*}
\end{proof}
%%%%%%%%%%%%%%%%%%%%%%%
\begin{example}
证明$\displaystyle \lim_{n\to +\infty}\frac{\displaystyle\sum_{k=1}^nk!}{n!} = 1$:$\left(\text{提示}: 1+\frac{1}{n}\leq\frac{\displaystyle\sum_{k=1}^nk!}{n!}\leq 1+ \frac{2}{n}\right)$
\end{example}
\begin{proof}
\begin{equation*}
\begin{split}
1+\frac{1}{n} &= \frac{(n-1)!+n!}{n!} \\&< \frac{\displaystyle\sum_{k=1}^nk!}{n!} \\&< \frac{(n-1)(n-2)!+(n-1)!+n!}{n!} \\
&= 1 + \frac{1}{n} + \frac{n-1}{n\cdot(n-1)} \\&= 1 +\frac{2}{n}
\end{split}
\end{equation*}
于是$\displaystyle \lim_{n\to +\infty}\frac{\displaystyle\sum_{k=1}^nk!}{n!} = 1$。
\end{proof}
%%%%%%%%%%%%%%%%%%%%%%%
\begin{example}
设$\displaystyle \lim_{n\to +\infty}a_n = a$,$\displaystyle \lim_{n\to +\infty}b_n = b$。记
$$S_n=\max\{a_n, b_n\}, \quad T_n = \min\{a_n, b_n\}, \quad n = 1,2,\cdots.$$
应用$\varepsilon-N$法$(\text{分} a < b, a>b, a=b)$或$\max\{a_n,b_n\} = \frac{1}{2}(a_n+b_n+|a_n-b_n|$与$\min\{a_n,b_n\} = \frac{1}{2}(a_n+b_n-|a_n-b_n|)$,
证明:
$$(1)\quad \displaystyle \lim_{n\to +\infty}S_n = \max\{a, b\}; \quad (2)\quad \displaystyle \lim_{n\to +\infty}T_n = \min\{a, b\}.$$
\end{example}
\begin{proof}
显然我们有$$\displaystyle \lim_{n\to +\infty}|a_n-b_n| = |a-b|.$$
由此可知:
$$\displaystyle \lim_{n\to +\infty}S_n = \frac{1}{2}(a+b+|a-b|) = \max\{a,b\},$$
$$\displaystyle \lim_{n\to +\infty}T_n = \frac{1}{2}(a+b-|a-b|) = \min\{a,b\}.$$
\end{proof}
%%%%%%%%%%%%%%%%%%%%%%%%

\begin{example}
应用例1.1.7与例1.1.15证明:
$$\displaystyle \lim_{n\to +\infty}\frac{1+\sqrt{2}+\sqrt[3]{3}+\cdots+\sqrt[n]{n}}{n} = 1.$$
\end{example}
\begin{proof}
取$a_n = \sqrt[n]{n}$,显然$$\displaystyle \lim_{n\to +\infty}a_n=1.$$于是
$$\displaystyle \lim_{n\to +\infty}\frac{1+\sqrt{2}+\sqrt[3]{3}+\cdots+\sqrt[n]{n}}{n} = \displaystyle \lim_{n\to +\infty}\frac{a_1+a_2+\cdots+a_n}{n} = 1.$$
\end{proof}
%%%%%%%%%%%%%%%%%%%%%%%%
\begin{example}
证明:$\displaystyle \lim_{n\to +\infty}\left(\sin\frac{\ln 2}{2}+\sin\frac{\ln 3}{3} +\cdots+\sin\frac{\ln n}{n}\right) = 1$
\end{example}
\begin{proof}
考虑数列$\sqrt[n]{n}$.这个数列在$n=3$是取得最大值且$\displaystyle \sqrt[n+1]{n+1} < \sqrt[n]{n}, \forall n \geq 3$。
这就是说数列$\{\sin\frac{\ln n}{n}\}$在$n=3$时取得最大值且$$\displaystyle \sin\frac{\ln{(n+1)}}{n+1} < \sin\frac{\ln{n}}{n}, \forall n \geq 3.$$
从而,
$$\left(\sin\frac{\ln 3}{3}\right)^{\frac{1}{n}} < \left(\sin\frac{\ln 2}{2}+\sin\frac{\ln 3}{3} +\cdots+\sin\frac{\ln n}{n}\right)^{\frac{1}{n}} < \left((n-1)\sin\frac{\ln 3}{3}\right)^{\frac{1}{n}}.$$
由于$$\displaystyle \lim_{n\to +\infty}\left(\sin\frac{\ln 3}{3}\right)^{\frac{1}{n}} = 1, \quad\displaystyle \lim_{n\to +\infty}\left((n-1)\sin\frac{\ln 3}{3}\right)^{\frac{1}{n}} = 1,$$我们有
$$\displaystyle \lim_{n\to +\infty}\left(\sin\frac{\ln 2}{2}+\sin\frac{\ln 3}{3} +\cdots+\sin\frac{\ln n}{n}\right) = 1.$$ 
\end{proof}
%%%%%%%%%%%%%%%%%%%%%%%%%
\begin{example}
证明:$\displaystyle \lim_{n\to +\infty}\displaystyle\sum_{k=n^2}^{(n+1)^2}\frac{1}{\sqrt k}= 2$.
\end{example}
\begin{proof}
$$\frac{2n+2}{n+1} = \displaystyle\sum_{k=n^2}^{(n+1)^2}\frac{1}{\sqrt{(n+1)^2}} \leq \displaystyle\sum_{k=n^2}^{(n+1)^2}\frac{1}{\sqrt k} \leq \displaystyle\sum_{k=n^2}^{(n+1)^2}\frac{1}{\sqrt{n^2}} = \frac{2n+2}{n}.$$
由夹逼定理,$\displaystyle \lim_{n\to +\infty}\displaystyle\sum_{k=n^2}^{(n+1)^2}\frac{1}{\sqrt k}= 2$。
\end{proof}
%%%%%%%%%%%%%%%%%%%%%%%%
\subsection{思考题}
\begin{example}
用$p(n)$表示能整除$n$的素数的个数。证明:$\displaystyle \lim_{n\to +\infty}\frac{p(n)}{n} = 0.$
\end{example}
\begin{proof}
假设$n=p_1^{m_1}p_2^{m_2}\cdots p_l^{m_l}$, 其中$p_1 < p_2 <\cdots < p_l$是互异的素数,$m_k \geq 1, k=1,2,\cdots, l$。这里$l=p(n)$.
$$\ln{n} = \displaystyle\sum_{k=1}^{l}m_k\ln{p_k} \geq \displaystyle\sum_{k=1}^{p(n)}\ln{2}=p(n)\ln{2}.$$
因此
$$0 \leq \frac{p(n)}{n} \leq \frac{\ln{n}}{n\ln{2}}.$$
由夹逼定理可知,$\displaystyle \lim_{n\to +\infty}\frac{p(n)}{n} = 0.$

\end{proof}
%%%%%%%%%%%%%%%%%%%%%%%
\begin{example}
设$x_n = \displaystyle \sum_{k=1}^n\left(\sqrt{1+\frac{k}{n^2}} -1\right)$。证明:$\displaystyle \lim_{n\to +\infty}x_n = \frac{1}{4}$.
\end{example}
\begin{proof}
\begin{equation*}
\begin{split}
\frac{n(n+1)}{2n^2\left(\sqrt{1 +\frac{1}{n}} + 1\right)}&= \frac{1}{\sqrt{1 +\frac{1}{n}} + 1}\displaystyle \sum_{k=1}^n\frac{k}{n^2} \\&< \displaystyle \sum_{k=1}^n\frac{\frac{k}{n^2}}{\sqrt{1+\frac{k}{n^2}}+1} \\&= \displaystyle\sum_{k=1}^n\left(\sqrt{1+\frac{k}{n^2}}-1\right) \\
&< \frac{1}{2}\displaystyle \sum_{k=1}^n\frac{k}{n^2}\\&= \frac{n(n+1)}{4n^2}.
\end{split}
\end{equation*}
由夹逼定理可知,$\displaystyle \lim_{n\to +\infty}x_n = \frac{1}{4}$。
\end{proof}
%%%%%%%%%%%%%%%%%%%%%%%%%%
\section{实数理论,实数连续性命题}
\subsection{练习题}
\subsection{思考题}

%%%%%%%%%%%%%%%%%%%%%%%%%%%%%%
\section{Cauchy收敛准则(原理),单调数列的极限,数$e=\displaystyle \lim_{n\to +\infty}\left(1+\frac{1}{n}\right)^n$}
\subsection{练习题}
\begin{example}
证明下列数列收敛:
\renewcommand\labelenumi{\normalfont(\theenumi)}
\begin{enumerate}
\item $\displaystyle\left(1-\frac{1}{2}\right)\left(1-\frac{1}{2^2}\right)\cdots\left(1-\frac{1}{2^n}\right), n\in\mathbb{N}$;

\begin{proof}
记$S_n = \left(1-\frac{1}{2}\right)\left(1-\frac{1}{2^2}\right)\cdots\left(1-\frac{1}{2^n}\right)$, 我们有$S_{n+1} = S_n\left(1-\frac{1}{2^{n+1}}\right) < 
S_n$.很显然$S_n > 0,\forall n\in\mathbb{N}$. 由实数连续性命题(二)可知,$S_n$收敛。
\end{proof}

\item $\displaystyle\frac{10}{1}\cdot\frac{11}{3}\cdots\frac{n+9}{2n-1}, n\in\mathbb{N}$.

\begin{proof}
记$S_n = \frac{10}{1}\cdot\frac{11}{3}\cdots\frac{n+9}{2n-1}$. 当$n>10$时,$\frac{n+9}{2n-1} < 1$.即$S_{n+1} < S_{n},\forall n \in \mathbb{N}, n > 10$.另一方面
$S_n > 0, \forall n\in\mathbb{N}$. 由实数连续性命题(二)可知,$S_n$收敛。
\end{proof}
\end{enumerate}
\end{example}
%%%%%%%%%%%%%%%%%%%%%%
\begin{example}
设$0<a_n<1$且$a_{n+1}(1-a_n)\geq \frac{1}{4}$,$n\in\mathbb{N}$。证明:$\{a_n\}$收敛,且$\displaystyle \lim_{n\to +\infty}a_n = \frac{1}{2}$。
\end{example}
\begin{proof}
考虑函数$f(x) = (1-x)x, x\in (0,1)$, 我们有$$f(x) > 0, f(x) \leq \frac{1}{4}, x\in (0,1).$$
所以$\displaystyle\frac{a_{n+1}}{a_n} \geq \frac{1}{4(1-a_n)a_n} \geq 1$,即$a_n$是单调递增的函数。由实数连续性命题(二)可知,$a_n$收敛。由递推公式可知,$\frac{1}{4}\geq a(1-a)\geq \frac{1}{4}$.所以$a=\frac{1}{2}$, 即$\displaystyle \lim_{n\to +\infty}a_n = \frac{1}{2}$。
\end{proof}
%%%%%%%%%%%%%%%%%%%%%%

\begin{example}
给定两正数$x_0=a$与$y_0=b$,归纳定义$$x_n=\sqrt{x_{n-1}y_{n-1}}, \quad y_n=\frac{x_{n-1}+y_{n-1}}{2},$$ $n=1,2,\cdots$。证明:数列$\{x_n\}$与$\{y_n\}$收敛,
且$\displaystyle \lim_{n\to +\infty}x_n=\displaystyle \lim_{n\to +\infty}y_n$,并称此极限为与的算术-几何平均数。
\end{example}
\begin{proof}
由算术-几何平均不等式知:$x_n \leq y_n, \forall n\in\mathbb{N}$。
于是:
$$x_{n+1} = \sqrt{x_{n}y_{n}} \geq \sqrt{x_{n}x_{n}} = x_n,\quad\forall n = 1,2,\cdots,$$
$$y_{n+1} = \frac{x_{n}+y_{n}}{2} \leq \frac{y_{n}+y_{n}}{2} = y_n, \quad\forall n = 1,2,\cdots.$$
于是$$a = x_0 \leq x_1 \leq \cdots \leq x_n \leq\cdots \leq y_n \leq y_1 \leq y_0 = b.$$
令$\displaystyle \lim_{n\to +\infty}x_n=A, \displaystyle \lim_{n\to +\infty}y_n=B$。由递推公式可知:$A=\sqrt{AB}$,从而$A = B$.
\end{proof}
%%%%%%%%%%%%%%%%%%%%%%
\begin{example}
$\forall n\in\mathbb{N}$,用$x_n$表示方程$x+x^2+\cdots+x^n=1$在闭区间$[0,1]$上的根,求极限$\displaystyle \lim_{n\to +\infty}x_n$.
\end{example}
\begin{solution}
设$f_n(x) = x+x^2+\cdots+x^n -1$
对于给定的$n\in \mathbb{N}$, $f_n(x)$在$[0,1]$是单调增函数,所以$f_n(x)$只会有唯一的根$x_n$。
由于$$f_{n+1}(x_{n+1}) = 0 < f_{n}(x_n) + x_n^{n+1} = f_{n+1}(x_n),$$
所以$$x_{n+1} \leq x_n,\forall n\in\mathbb{N}.$$
由实数连续性命题(二)可知, $\displaystyle \lim_{n\to +\infty}x_n$存在。由于$\displaystyle\frac{x_n-x_n^{n+1}}{1-x_n} = 1$知,
$\displaystyle \lim_{n\to +\infty}x_n=\frac{1}{2}$.
\end{solution}
%%%%%%%%%%%%%%%%%%%%%
\begin{example}
设$c> 0$,$x_1=\sqrt{c}$,$\displaystyle x_2=\sqrt{c+\sqrt{c}}$,$x_{n+1}=\sqrt{c+x_n}$。证明:数列$\{x_n\}$收敛,且$\displaystyle \lim_{n\to +\infty}x_n=\frac{1+\sqrt{1+4c}}{2}$.
\end{example}
\begin{proof}
我们用归纳法证明$x_n \leq \displaystyle\frac{1+\sqrt{1+4c}}{2}, \forall n\in\mathbb{N}$.
\renewcommand\labelenumi{\normalfont(\theenumi)}
\begin{enumerate}
\item $x_1 = \sqrt{c} < \displaystyle\frac{1+\sqrt{1+4c}}{2}$.
\item 假设$x_k < \displaystyle\frac{1+\sqrt{1+4c}}{2}$。我们证明$x_{k+1} < \displaystyle\frac{1+\sqrt{1+4c}}{2}$.
\begin{equation*}
\begin{split}
x_{k+1} &=\sqrt{c+x_{k}} \\&< \sqrt{c+\frac{1+\sqrt{1+4c}}{2}}\\&=\sqrt{\frac{4c+2+2\sqrt{1+4c}}{4}}\\
&=\frac{1+\sqrt{1+4c}}{2}.
\end{split}
\end{equation*}
\end{enumerate}

现在考虑函数$f(x) =c+x-x^2$。很显然$$f(x) > 0, \quad x\in \left(0, \displaystyle\frac{1+\sqrt{1+4c}}{2}\right).$$
于是$$x_{n+1}^2 -x_{n}^2 = c+ x_{n}-x_n^{2} > 0.$$即$\{x_n\}$是单调递增的数列。由实数连续性命题(二)可知,数列$\{x_n\}$收敛. 设$\displaystyle \lim_{n\to +\infty}x_n=a$. 对递归公式取极限得$a=\sqrt{c+a}$。从而
$$\displaystyle \lim_{n\to +\infty}x_n=a =\frac{1+\sqrt{1+4c}}{2}.$$
\end{proof}
%%%%%%%%%%%%%%%%%%%%
\begin{example}
设$x_1=c>0$,令$\displaystyle x_{n+1} = c + \frac{1}{x_n}$,$n\in\mathbb{N}$。求极限$\displaystyle \lim_{n\to +\infty}x_n$.
\end{example}
\begin{solution}
由递推公式,我们有
$$x_{n+1}-x_{n} = \frac{x_{n-1}- x_n}{x_n\cdot x_{n-1}}.$$
下面我们证明:
\renewcommand\labelenumi{\normalfont(\theenumi)}
\begin{enumerate}
\item $x_{2k-1} < x_{2k+1},\quad\forall k\in\mathbb{N}$;
\item $x_{2k} > x_{2(k+1)},\quad\forall k\in\mathbb{N}$.
\end{enumerate}
$$x_{n}-x_{n-2} = \frac{x_{n-3} - x_{n-1}}{x_{n-1}\cdot x_{n-3}} = \frac{x_{n-2} - x_{n-4}}{x_{n-1}\cdot x_{n-2}\cdot x_{n-3}\cdot x_{n-4}}.$$
由于$x_3-x_1 = \displaystyle\frac{1}{x_2} > 0$, 从而知$(1)$成立。由于$x_4-x_2 = \displaystyle\frac{x_1-x_3}{x_1\cdot x_3} < 0$,从而知$(2)$成立。

另一方面:
$$x_{2k}-x_1 = c+\frac{1}{x_{2k-1}}-c = \frac{1}{x_{2k-1}} > 0,\quad \forall k\in\mathbb{N};$$
$$x_{2k+1} - x_2 = c+\frac{1}{x_{2k}} - c -\frac{1}{x_1} = \frac{x_1 - x_{2k}}{x_1\cdot x_{2k}} < 0,\quad\forall k\in\mathbb{N}.$$

由实数连续性命题(二)可知,奇数列和偶数列都是收敛子列。假设
$$\displaystyle \lim_{k\to +\infty}x_{2k-1} = a, \lim_{k\to +\infty}x_{2k} = b.$$
由递推公式,我们有
$$x_{2k+1} =c+\frac{1}{c+\frac{1}{x_{2k-1}}}\Rightarrow a^2-ac-1 = 0\Rightarrow a = \frac{c+\sqrt{c^2 + 4}}{2},$$
$$x_{2k+2} =c+\frac{1}{c+\frac{1}{x_{2k}}}\Rightarrow b^2-bc-1 = 0\Rightarrow b = \frac{c+\sqrt{c^2 + 4}}{2}.$$
即,$\{x_n\}$收敛,且$\displaystyle \lim_{n\to +\infty}x_n=\frac{c+\sqrt{c^2+4}}{2}$.
\end{solution}
%%%%%%%%%%%%%%%%%%%

\begin{example}
证明:$\displaystyle \sqrt{1+\sqrt{1+\sqrt{1+\cdots}}}=\frac{1+\sqrt{5}}{2} =  1 + \displaystyle \frac{1}{1+\displaystyle \frac{1}{1+\cdots}}$
\end{example}
\begin{proof}
第一个等式是题5的特例:$c=1$.第二个等式是题6的特例:$c=1$.
\end{proof}
%%%%%%%%%%%%%%%%%%
\begin{example}
设$c>0$,$a_1=\frac{c}{2}$,$a_{n+1}=\frac{c}{2}+\frac{a_n^2}{2}$,$n=1,2,\cdots$。证明:
\begin{equation*}
\displaystyle \lim_{n\to +\infty}a_n=
\begin{cases}
1-\sqrt{1-c},\quad&0<c\leq 1,\\
+\infty, \quad&c > 1.
\end{cases}
\end{equation*}
\end{example}
\begin{proof}
当$c>1$时,由递推公式可知,$$a_{n+1}\geq 2\displaystyle\sqrt{\frac{c}{2}{\frac{a_n^2}{2}}}=\sqrt{c}a_n\geq \cdots\geq c^{\frac{n}{2}}a_1.$$
所以$$+\infty \geq \displaystyle \lim_{n\to +\infty}a_n \geq \displaystyle \lim_{n\to +\infty}(c^{\frac{n}{2}}a_1) = +\infty.$$

当$0<c\leq 1$时,我们可以证明数列$\{a_n\}$是单调递增有界。
\renewcommand\labelenumi{\normalfont(\theenumi)}
\begin{enumerate}
\item $a_1 = \frac{c}{2} < 1-\sqrt{1-c}$.
\item 设$a_k < 1-\sqrt{1-c}$.下面我们证明$a_{k+1} \geq a_{k}$且$a_{k+1} < 1-\sqrt{1-c}$.
$$a_{k+1} = \frac{c}{2} + \frac{a_{k}^2}{2} <\frac{c}{2} + \frac{1}{2}\left(1-\sqrt{1-c}\right)^2 = 1-\sqrt{1-c}.$$
考察函数$f(x) = x^2-2x+c$. $$f(x) > 0, \quad x\in(-\infty, 1-\sqrt{1-c}).$$
因此 $$a_{k+1} - a_{k} = \frac{1}{2}(a_k^2-2a_k+c) > 0.$$
即$\{a_n\}$是单调增的数列. 由实数连续性命题(二)可知, $\displaystyle \lim_{n\to +\infty}a_n$存在。设极限为$a$, 则$a=\frac{c}{2}+\frac{a^2}{2}\Rightarrow a = 1-\sqrt{1-c}$.
\end{enumerate}\end{proof}
%%%%%%%%%%%%%%%%%%%%%%
\begin{example}
设数列$\{a_n\}$单调增,$\{b_n\}$单调减,且$\displaystyle \lim_{n\to +\infty}(a_n-b_n)=0$。证明:$\{a_n\}$与$\{b_n\}$都收敛,且$\displaystyle \lim_{n\to +\infty}a_n=\displaystyle \lim_{n\to +\infty}b_n$.
\end{example}
\begin{proof}
很显然$\{a_n-b_n\}$是单调增。又由于$\displaystyle \lim_{n\to +\infty}(a_n-b_n)=0$, 可知 $a_n \leq b_n,\forall n\in\mathbb{N}$, 从而
$$a_1 \leq a_2 \leq\cdots\leq a_n \leq \cdots \leq b_n \leq \cdots \leq b_2\leq b_1.$$
由实数连续性命题(二)可知, $\{a_n\}$与$\{b_n\}$都收敛. 再由$\displaystyle \lim_{n\to +\infty}(a_n-b_n)=0$知,$\displaystyle \lim_{n\to +\infty}a_n=\displaystyle \lim_{n\to +\infty}b_n$。
\end{proof}
%%%%%%%%%%%%%%%%%%%%%
\begin{example}
设数列$\{a_n\}$满足:存在正数$M$,$\forall n\in\mathbb{N}$,有$$A_n = \left|a_2-a_1 \right| +\left|a_3-a_2 \right| + \left|a_n-a_{n-1} \right| \leq M.$$
证明:数列$\{a_n\}$与$\{A_n\}$都收敛。
\end{example}
\begin{proof}
很显然数列$\{A_n\}$是单调增有界数列,由实数连续性命题(二)可知,$\{A_n\}$是收敛的。
$$\{A_n\}\text{收敛}\Rightarrow \{A_n\}\text{是Cauchy列}\Rightarrow \{a_n\}\text{是Cauchy列} \Rightarrow \{a_n\}\text{收敛}.$$
\end{proof}
%%%%%%%%%%%%%%%%%%%%%%
\begin{example}
应用Cauchy收敛准则证明下列数列收敛:
\renewcommand\labelenumi{\normalfont(\theenumi)}
\begin{enumerate}
\item $\displaystyle x_n=\frac{\cos{1!}}{1\cdot 2}+\frac{\cos{2!}}{2\cdot 3}+\cdots+\frac{\cos{n!}}{n\cdot (n+1)}$;
\begin{proof}
$$\left|a_{n+p} - a_{n}\right| = \left|\frac{\cos{(n+1)!}}{(n+1)\cdot (n+2)}+\cdots+\frac{\cos{(n+p)!}}{(n+p)\cdot (n+p+1)}\right|\leq 
\frac{1}{n+1} - \frac{1}{n+p+1} < \frac{1}{n+1}.$$
即$\{x_n\}$是Cauchy列,从而收敛。
\end{proof}
\item $\displaystyle x_n=1+\frac{1}{2^2}+\frac{1}{3^2}+\cdots+\frac{1}{n^2}$;
\begin{proof}
$$\left|x_{n+p}-x_n\right| = \frac{1}{(n+1)^2}+\frac{1}{(n+2)^2}+\cdots+\frac{1}{(n+p)^2}<\frac{1}{n}-\frac{1}{n+p}<
\frac{1}{n}.$$
即$\{x_n\}$是Cauchy列,从而收敛。
\end{proof}
\item $\displaystyle x_n = \frac{\arctan{1}}{1(1+\cos{1!})}+\frac{\arctan{2}}{2(2+\cos{2!})}+\cdots+\frac{\arctan{n}}{n(n+\cos{n!})}$.
\begin{proof}
\begin{equation*}
\begin{split}
\left|x_{n+p}-x_{n}\right| &= \left|\frac{\arctan{(n+1)}}{(n+1)((n+1)+\cos{(n+1)!})}+\cdots+\frac{\arctan{(n+p)}}{(n+p)((n+p)+\cos{(n+p)!})}\right| \\
&<\frac{\pi}{2}\left(\frac{1}{n(n+1)}+\cdots+\frac{1}{(n+p-1)(n+p)}\right)\\
&=\frac{\pi}{2}\left(\frac{1}{n}-\frac{1}{n+p}\right)\\
&<\frac{\pi}{2n}.
\end{split}
\end{equation*}
即$\{x_n\}$是Cauchy列,从而收敛。
\end{proof}
\end{enumerate}
\end{example}
%%%%%%%%%%%%%%%%%%%%%%
\begin{example}
应用$\displaystyle \lim_{n\to +\infty}\left(1+\frac{1}{n}\right)^n = e$与$\displaystyle \lim_{n\to +\infty}\left(1-\frac{1}{n}\right)^n = e^{-1}$,求下列极限:
\renewcommand\labelenumi{\normalfont(\theenumi)}
\begin{enumerate}
\item $\displaystyle \lim_{n\to +\infty}\left(1+\frac{1}{n-3}\right)^n;$
\begin{solution}
$\displaystyle \lim_{n\to +\infty}\left(1+\frac{1}{n-3}\right)^n=\displaystyle \lim_{n\to +\infty}\left(1+\frac{1}{n-3}\right)^{(n-3)\frac{n}{n-3}}=e.$
\end{solution}
\item $\displaystyle \lim_{n\to +\infty}\left(1-\frac{1}{n-2}\right)^n;$
\begin{solution}
$\displaystyle \lim_{n\to +\infty}\left(1-\frac{1}{n-2}\right)^n=\displaystyle \lim_{n\to +\infty}\left(1-\frac{1}{n-2}\right)^{(-n+2)\frac{n}{-n+2}}=e^{-1}.$
\end{solution}
\item $\displaystyle \lim_{n\to +\infty}\left(\frac{1+n}{2+n}\right)^n;$
\begin{proof}
$\displaystyle \lim_{n\to +\infty}\left(\frac{1+n}{2+n}\right)^n = \lim_{n\to +\infty}\left(1-\frac{1}{2+n}\right)^{(-2-n)\frac{n}{-2-n}}=e^{-1}$.
\end{proof}
\item $\displaystyle \lim_{n\to +\infty}\left(1+\frac{1}{2n^2}\right)^{4n^2};$
\begin{proof}
$\displaystyle \lim_{n\to +\infty}\left(1+\frac{1}{2n^2}\right)^{4n^2} = \lim_{n\to +\infty}\left(\left(1+\frac{1}{2n^2}\right)^{2n^2}\right)^2=e^2$.
\end{proof}
\item $\displaystyle \lim_{n\to +\infty}\left(1+\frac{3}{n}\right)^{n}$.
\begin{proof}
$\displaystyle \lim_{n\to +\infty}\left(1+\frac{3}{n}\right)^{n}=\lim_{n\to +\infty}\left(\left(1+\frac{3}{n}\right)^{\frac{n}{3}}\right)^3=e^3$.
\end{proof}
\end{enumerate}
\end{example}
%%%%%%%%%%%%%%%%%%%%
\begin{example}
$\forall n\in\mathbb{N}$,证明:
\renewcommand\labelenumi{\normalfont(\theenumi)}
\begin{enumerate}
\item $0<e-\left(1+\frac{1}{n}\right)^n <\frac{3}{n}$
\begin{proof}
由不等式$\left(1+\frac{1}{n}\right)^n<e<\left(1+\frac{1}{n}\right)^{n+1}$可知:
$$0<e-\left(1+\frac{1}{n}\right)^n < \left(1+\frac{1}{n}\right)^{n+1} - \left(1+\frac{1}{n}\right)^{n}=\left(1+\frac{1}{n}\right)^{n}\frac{1}{n}<\frac{3}{n}.$$
\end{proof}
\item $\displaystyle \lim_{n\to +\infty}\left[e-\left(1+\frac{1}{n}\right)^n\right]=0$.
\begin{proof}
由(1)和夹逼原理,可知$\displaystyle \lim_{n\to +\infty}\left[e-\left(1+\frac{1}{n}\right)^n\right] = 0$。
\end{proof}
\end{enumerate}
\end{example}
%%%%%%%%%%%%%%%%%%%%%
\begin{example}
设$\alpha<1$,证明:
\renewcommand\labelenumi{\normalfont(\theenumi)}
\begin{enumerate}
\item $\displaystyle  0< n^{\alpha}\left[e-\left(1+\frac{1}{n}\right)^n\right] < \frac{e}{n^{1-\alpha}}.$
\begin{proof}
由不等式$\left(1+\frac{1}{n}\right)^n<e<\left(1+\frac{1}{n}\right)^{n+1}$可知:
\begin{equation*}
\begin{split}
0 &< n^{\alpha}\left[e-\left(1+\frac{1}{n}\right)^n\right] \\
&< n^{\alpha}\left[\left(1+\frac{1}{n}\right)^{n+1}-\left(1+\frac{1}{n}\right)^n\right] \\
&=n^{\alpha}\left[\left(1+\frac{1}{n}\right)^n\frac{1}{n}\right] \\&<\frac{e}{n^{1-\alpha}}.
\end{split}
\end{equation*}
\end{proof}
\item $\displaystyle  \lim_{n\to +\infty}n^{\alpha}\left[e-\left(1+\frac{1}{n}\right)^n\right] = 0$.
\begin{proof}
由(1)和夹逼原理可知,$\displaystyle  \lim_{n\to +\infty}n^{\alpha}\left[e-\left(1+\frac{1}{n}\right)^n\right] = 0$。
\end{proof}
\end{enumerate}
\end{example}
%%%%%%%%%%%%%%%%%%%%%%
\begin{example}
\renewcommand\labelenumi{\normalfont(\theenumi)}
\begin{enumerate}
\item 设$0<a<b$,$\forall n\in\mathbb{N}$。证明:
$$b^{n+1}-a^{n+1}< (n+1)b^n(b-a),$$
$$a^{n+1}>b^n\left[(n+1)a-nb\right];$$
\begin{proof}
$$b^{n+1}-a^{n+1}=(b-a)(b^{n}+b^{n-1}a+\cdots+a^n) < (n+1)b^n(b-a).$$
由此式可知:
$$a^{n+1} > b^{n+1}-(n+1)b^n(b-a) = b^n\left[b-(n+1)b +(n+1)a\right] = b^n\left[(n+1)a -nb\right].$$
\end{proof}
\item 在(1)中,令$a=1+\frac{1}{n+1}$,$b=1+\frac{1}{n}$推出$\left(1+\frac{1}{n}\right)^n$为严格增的数列;
\begin{proof}
将$a=1+\frac{1}{n+1}, b = 1+\frac{1}{n}$代入(1)中的第二式,可知 
$$\left(1+\frac{1}{n+1}\right)^{n+1} > \left(1+\frac{1}{n}\right)^n\left[(n+1)\left(1+\frac{1}{n+1}\right)
-n\left(1+\frac{1}{n}\right)\right]=\left(1+\frac{1}{n}\right)^n.$$
\end{proof}
\item 在(1)中,令$a=1$,$b=1+\frac{1}{2n}$推出当$n$为偶数时,有$\left(1+\frac{1}{n}\right)^n<4$;由此得到$\forall n\in\mathbb{N}$,有$\left(1+\frac{1}{n}\right)^n<4$,即$4$为该数列的上界,从而$\left(1+\frac{1}{n}\right)^n$收敛。
\begin{proof}
将$a=1, b=1+\frac{1}{2n}$代入(1)的第二个不等式,我们有:
$$1\geq \left(1+\frac{1}{2n}\right)^n\left[(n+1) - n\left(1+\frac{1}{2n}\right)\right],$$
即
$$\left(1+\frac{1}{2n}\right)^n \leq 2\Rightarrow \left(1+\frac{1}{2n}\right)^{2n} \leq 4.$$
由于$\left(1+\frac{1}{n}\right)^n$是单调递增的,我们可知,$\left(1+\frac{1}{n}\right)^{n} \leq 4, \forall n\in\mathbb{N}$.
\end{proof}
\end{enumerate}
\end{example}
%%%%%%%%%%%%%%%%%%%%%%%%
\begin{example}
应用不等式$b^{n+1}-a^{n+1}>(n+1)a^n(b-a)$,$0<a<b$,证明:数列$\left(1+\frac{1}{n}\right)^{n+1}$是严格单减的,并由此推出$\left(1+\frac{1}{n}\right)^{n+1}$为有界数列。
\end{example}
\begin{proof}
$$b^{n+1}-a^{n+1}=(b-a)(b^{n}+b^{n-1}a+\cdots+a^n) > (n+1)a^n(b-a).$$
即$$b^{n+1}> a^n\left[(n+1)b-na\right].$$
取$a=1+\frac{1}{n+1}, b = 1+\frac{1}{n}$,我们有
\begin{equation*}
\begin{split}
\left(1+\frac{1}{n}\right)^{n+1} &> \left(1+\frac{1}{n+1}\right)^n\left[(n+1)\frac{n+1}{n}-n\frac{n+2}{n+1}\right]\\
&=\left(1+\frac{1}{n+1}\right)^n\left(\frac{n^2+3n+1}{n(n+1)}\right)\\
&=\left(1+\frac{1}{n+1}\right)^{n+2}\left[\frac{n^2+3n+1}{n(n+1)}\cdot\left(\frac{n+1}{n+2}\right)^2\right]\\
&=\left(1+\frac{1}{n+1}\right)^{n+2}\left(\frac{n^3+4n^2+4n+1}{n^3+4n^2+4n}\right)\\
&>\left(1+\frac{1}{n+1}\right)^{n+2}
\end{split}
\end{equation*}
由此可见$\{\left(1+\frac{1}{n}\right)^{n+1}\}$是严格单调减,且有界。
\end{proof}
%%%%%%%%%%%%%%%%%%%%%%%%
\begin{example}
证明:$\left(1+\frac{1}{n}\right)^{n+1} < \frac{3}{n}+\left(1+\frac{1}{n}\right)^n, \forall n\in\mathbb{N}$.
\end{example}
\begin{proof}
$$\left(1+\frac{1}{n}\right)^{n+1} - \left(1+\frac{1}{n}\right)^n = \left(1+\frac{1}{n}\right)^n\left(1+\frac{1}{n} - 1\right)<\frac{3}{n}, \quad\forall n\in\mathbb{N}.$$
\end{proof}
%%%%%%%%%%%%%%%%%%%%%%%
\begin{example}
设$\{a_n\}$为有界数列。记
$$\overline{a}_n = \sup\{a_n, a_{n+1}, \cdots\}, \quad\underline{a}_n= \inf\{a_n, a_{n+1},\cdots\}.$$
证明:
\renewcommand\labelenumi{\normalfont(\theenumi)}
\begin{enumerate}
\item $\forall n\in\mathbb{N}$,有$\overline{a}_n \geq\underline{a}_n;$
\begin{proof}
这是显然的。$\overline{a}_n \geq a_n \geq \underline{a}_n$, $\forall n\in \mathbb{N}$.
\end{proof}
\item $\{\overline{a}_n\}$为单调减有界数列;$\{\underline{a}_n\}$为单调增有界数列,且$\forall n,m\in\mathbb{N}$,有$\overline{a}_n \geq\underline{a}_m;$
\begin{proof}
由$\overline{a}_n$和$\underline{a}_n$的定义可知,
$$\overline{a}_n = \max\{a_n, \overline{a}_{n+1}\}, \quad \underline{a}_n = \min\{a_n, \underline{a}_{n+1}\}.$$
由此可见$\{\overline{a}_n\}$是单调减,$\{\underline{a}_n\}$是单调增, 且
$$\underline{a}_1 \leq \underline{a}_2\leq \cdots\leq \underline{a}_n \cdots \leq \cdots \leq \overline{a}_n \leq \cdots \leq \overline{a}_2 \leq \overline{a}_1.$$
由于数列$\{a_n\}$是有界数列,故$\underline{a}_1$和$\overline{a}_1$都是有界数。命题得证。
\end{proof}
\item 设$\overline{a}=\displaystyle  \lim_{n\to +\infty}\overline{a}_n$,$\underline{a}=\displaystyle  \lim_{n\to +\infty}\underline{a}_n$,则$\overline{a}\geq \underline{a};$
\begin{proof}
应用定理1.2.5,这个命题就可以得证。
\end{proof}
\item $\{a_n\}$收敛$\Leftrightarrow \overline{a}=\underline{a}$.
\begin{proof}
(反证法)假设$\overline{a} > \underline{a}$.对于$\forall \varepsilon > 0, \varepsilon < \frac{\overline{a} - \underline{a}}{3}$, 存在$N\in\mathbb{N}$, 当$n>N$有
$$\underline{a}_n < \underline{a} +\varepsilon < \underline{a} + \frac{\overline{a}-\underline{a}}{3} < \overline{a} - \frac{\overline{a}-\underline{a}}{3} < \overline{a}-\varepsilon < \overline{a}_n.$$
于是在$\{a_n\}$存在两个子列$\{a_{k_n}\}$和
$\{a_{l_n}\}$使得
$$a_{k_n} < \underline{a} + \frac{\overline{a}-\underline{a}}{3} < \overline{a} - \frac{\overline{a}-\underline{a}}{3} < a_{l_n}, \quad \forall n\in\mathbb{N}.$$
从而$\{a_n\}$发散。矛盾。
\end{proof}
\end{enumerate}
\end{example}
%%%%%%%%%%%%%%%%%%%%
\subsection{思考题}
\begin{example}
设$a_1\geq 0$,$a_{n+1} = \displaystyle\frac{3(1+a_n)}{3+a_n}$,$n=1,2,\cdots$.证明:$\{a_n\}$收敛,且$\displaystyle  \lim_{n\to +\infty}a_n=\sqrt{3}$.
\begin{proof}
首先我们证明,如果$a_n < \sqrt{3}$,则 $a_{n+1} > a_{n}$.如果$a_n > \sqrt{3}$,则$a_{n+1} < a_{n}$. 
$$\displaystyle \frac{a_{n+1}}{a_n} = \frac{3 + 3a_n}{3a_n + a_n^2}.$$
于是我们有
\renewcommand\labelenumi{\normalfont(\theenumi)}
\begin{enumerate}
\item $a_n < \sqrt{3}\Rightarrow 3 + 3a_n > 3a_n + a_n^2 \Rightarrow  a_{n+1}>a_{n}$;
\item $a_n > \sqrt{3}\Rightarrow 3 + 3a_n < 3a_n + a_n^2 \Rightarrow  a_{n+1}< a_{n}$;
\item $a_n = \sqrt{3}\Rightarrow 3 + 3a_n = 3a_n + a_n^2 \Rightarrow  a_{n+1}= a_{n}$.
\end{enumerate}
基于以上的计算,不管$a_1$取何值,我们都有
$$\left|a_{n+1}-\sqrt{3}\right| = \left|\frac{3+3a_n - 3\sqrt{3} -\sqrt{3}a_n}{a_n+3}\right|=\left|\frac{(3-\sqrt{3})(a_n-\sqrt{3})}{a_n+3}\right|<\left(\frac{3-\sqrt{3}}{3}\right)\left| a_n-\sqrt{3}\right|.$$
因为$\displaystyle \frac{3-\sqrt{3}}{3} < 1$,所以$(a_n-\sqrt{3})\rightarrow 0 \Rightarrow a_n\rightarrow \sqrt{3}.$
\end{proof}
\end{example}
%%%%%%%%%%%%%%%%%%%%%%%%
\begin{example}
设$a>0$,$x_1>0$,$x_{n+1} = \displaystyle\frac{x_n(x_n^2+3a)}{3x_n^2+a}$,$n=1,2,\cdots$。证明:$\{x_n\}$收敛,且$\displaystyle  \lim_{n\to +\infty}x_n=\sqrt{a}$
\end{example}
\begin{proof}这题和上一题类似。我们计算
$$x_{n+1}-x_{n} = \frac{2x_n(\sqrt{a}-x_n)(\sqrt{a}+x_n)}{3x_n^2+a},$$
所以:
\renewcommand\labelenumi{\normalfont(\theenumi)}
\begin{enumerate}
\item 如果$x_1\leq \sqrt{a}$,则$\{x_n\}$是单调增且有上界$\sqrt{a}$的数列;
\item 如果$x_1 > \sqrt{a}$, 则$\{x_n\}$是单调减且有下界$\sqrt{a}$的数列;
\end{enumerate}
无论那种情况发生,由实数连续性命题(二)可知,$\{x_n\}$收敛,且$\displaystyle  \lim_{n\to +\infty}x_n=\sqrt{a}$。

\noindent 对于以上两种情况我们可以分别用数学归纳法讨论:\\
\noindent(1): 当$x_1 \leq \sqrt{a}$时:\\
$x_2 - x_1 \geq 0\Rightarrow x_2 \geq x_1$且
$x_2 - \sqrt{a} = \displaystyle\frac{(x_1-\sqrt{a})^3}{3x_1^2+a} \leq 0\Rightarrow x_2 \leq \sqrt{a}.$\\
假设$n=k$时也有$x_{k}\geq x_{k-1}$且$x_k \leq \sqrt{a}$.\\
现证$n=k+1$时也有这些。$x_{k+1} \geq x_{k}$是显然的。$x_{k+1} - \sqrt{a} = \displaystyle\frac{(x_{k}-\sqrt{a})^3}{3x_k^2+a} \leq 0\Rightarrow x_{k+1} \leq \sqrt{a}.$\\
\noindent(2): 当$x_1 > \sqrt{a}$时:\\
$x_2 - x_1 < 0\Rightarrow x_2 < x_1$且
$x_2 - \sqrt{a} = \displaystyle\frac{(x_1-\sqrt{a})^3}{3x_1^2+a} > 0\Rightarrow x_2 > \sqrt{a}.$\\
假设$n=k$时也有$x_{k}< x_{k-1}$且$x_k > \sqrt{a}$.\\
现证$n=k+1$时也有这些。$x_{k+1} < x_{k}$是显然的。$x_{k+1} - \sqrt{a} = \displaystyle\frac{(x_{k}-\sqrt{a})^3}{3x_k^2+a} > 0\Rightarrow x_{k+1} > \sqrt{a}.$
\end{proof}

%%%%%%%%%%%%%%%%%%%
\begin{example}
设$a> 0$,$x_1 =  \sqrt[3]{a}$,$x_n=\sqrt[3]{ax_{n-1}}(n> 1)$。证明: $\{x_n\}$收敛,且$\displaystyle  \lim_{n\to +\infty}x_n=\sqrt{a}$
\end{example}
\begin{proof}
我们只需要证明:
\renewcommand\labelenumi{\normalfont(\theenumi)}
\begin{enumerate}
\item 如果$a\geq 1$,则数列$\{x_n\}$是单调递增,且有上界$\sqrt{a}$.
\item 如果$a<1$,则数列$\{x_n\}$是单调递减,且有下界$\sqrt{a}$.
\end{enumerate}
无论上述那种情况发生,由实数连续性命题(二)可知, $\{x_n\}$收敛,且$\displaystyle  \lim_{n\to +\infty}x_n=\sqrt{a}$。

$$x_n - x_{n-1} = \displaystyle\frac{x_{n-1}(\sqrt{a}-x_{n-1})(\sqrt{a}+x_{n-1})}{\left(\sqrt[3]{ax_{n-1}}\right)^2 + \sqrt[3]{ax_{n-1}}x_{n-1}+x_{n-1}^2}.$$

\noindent(1):当$a\geq1$时:\\
$\sqrt[3]{a} \leq \sqrt{a}$.\\
当$n=2$时,$\sqrt{a}-x_1 = \sqrt{a}-\sqrt[3]{a} \geq 0 \Rightarrow x_2-x_1 \geq  0$且$x_2 = \sqrt[3]{ax_1} \leq \sqrt[3]{a \sqrt{a}} = \sqrt{a}$.\\
假设$n=k$时,$x_k \geq x_{k-1}$且$x_k \leq \sqrt{a}$.\\
由归纳法可得知,$n=k+1$时有,$x_{k+1} \geq x_k$且$x_{k+1}\leq \sqrt{a}$.\\

\noindent(2):当$a<1$时:\\
$\sqrt[3]{a} > \sqrt{a}$.\\
当$n=2$时,$\sqrt{a}-x_1 = \sqrt{a}-\sqrt[3]{a} < 0 \Rightarrow x_2-x_1 < 0$且$x_2 = \sqrt[3]{ax_1} > \sqrt[3]{a \sqrt{a}} = \sqrt{a}$.\\
假设$n=k$时,$x_k < x_{k-1}$且$x_k > \sqrt{a}$.\\
由归纳法可得知,$n=k+1$时有,$x_{k+1} < x_k$且$x_{k+1} > \sqrt{a}$.
\end{proof}
%%%%%%%%%%%%%%%%%%%%
\begin{example}
设$0<a_1<b_1<c_1$。令
$$a_{n+1}=\frac{3}{\frac{1}{a_n}+\frac{1}{b_n}+\frac{1}{c_n}},\quad
b_{n+1} =  \sqrt[3]{a_nb_nc_n},\quad
c_{n+1}= \frac{a_n+b_n+c_n}{3}
$$
证明:$\{a_n\}$,$\{b_n\}$,$\{c_n\}$收敛于同一实数。
\end{example}
\begin{proof}
由定义可知$$0< a_1< a_n \leq b_n \leq c_n < c_1, \quad\forall n >1.$$
于是 $\{a_n\}$单调增,$\{c_n\}$单调减。
由实数连续性命题(二)可知, 数列$\{a_n\}, \{c_n\}$收敛。设$\displaystyle  \lim_{n\to +\infty}a_n=a, \displaystyle  \lim_{n\to +\infty}c_n=c$。易知,$$0<a_1 \leq a\leq c\leq c_1.$$
因为$b_n = 3c_{n+1} - a_n - c_n$可知,数列$\{b_n\}$收敛。
设$\displaystyle \lim_{n\to +\infty}b_n=b$。显然$a\leq b\leq c$, 且
\begin{equation*}
a = \frac{3}{\displaystyle\frac{1}{a}+\frac{1}{b}+\frac{1}{c}},\quad
b = \displaystyle\sqrt[3]{abc},\quad
c = \displaystyle\frac{a+b+c}{3}
\end{equation*}
解方程组可得知$a=b=c$.
\end{proof}
%%%%%%%%%%%%%%%%%%%%%%%%
\begin{example}
设$a_n > 0$,$S_n=a_1+\cdots+a_n$,$T_n=\displaystyle\frac{a_1}{S_1}+\cdots+\frac{a_n}{S_n}$,且$\displaystyle  \lim_{n\to +\infty}S_n=+\infty$。证明:$\displaystyle  \lim_{n\to +\infty}T_n=+\infty.$
\end{example}
\begin{proof}
由于$S_n \rightarrow  +\infty$, 我们可以找到一个子列$\{n_k\}$使得
$$\frac{S_{n_{k-1}}}{S_{n_k}} < \frac{1}{2}, \quad\forall k\in\mathbb{N}.$$
现在我们计算$T_{n_k}$:
\begin{equation*}
\begin{split}
T_{n_k} &= \left(\frac{a_1}{S_1}+\cdots \frac{a_{n_1}}{S_{n_1}}\right) + \left(\frac{a_{n_1+1}}{S_{n_1+1}}+\cdots \frac{a_{n_2}}{S_{n_2}}\right) +\cdots+\left(\frac{a_{n_{k-1}+1}}{S_{n_{k-1}+1}}+\cdots \frac{a_{n_k}}{S_{n_k}}\right) \\
&> \left(\frac{a_1}{S_{n_1}}+\cdots \frac{a_{n_1}}{S_{n_1}}\right) + \left(\frac{a_{n_1+1}}{S_{n_2}}+\cdots \frac{a_{n_2}}{S_{n_2}}\right) +\cdots+\left(\frac{a_{n_{k-1}+1}}{S_{n_k}}+\cdots \frac{a_{n_k}}{S_{n_k}}\right)\\
&=\frac{S_{n_1}- 0}{S_{n_1}} + \frac{S_{n_2}-S_1}{S_{n_2}} + \cdots +  \frac{S_{n_k}-S_{n_{k-1}}}{S_{n_{k}}}\\
&>\frac{k}{2}.
\end{split}
\end{equation*}
即:$\displaystyle \lim_{k\to +\infty}T_{n_k}=+\infty.$
因为$T_n$是单调增的数列,从而$\displaystyle \lim_{n\to +\infty}T_{n}=+\infty.$命题 得证。
\end{proof}
%%%%%%%%%%%%%%%%%%%
\begin{example}
设$a_1=1$,$a_{n+1}=\displaystyle\frac{1}{1+a_n}$,$n=1,2,\cdots$.证明:$\displaystyle  \lim_{n\to +\infty}a_n=\frac{\sqrt{5}-1}{2}.$
\end{example}
\begin{proof}我们只需证明$\left|a_n -\frac{\sqrt{5}-1}{2}\right|$收敛于$0$.
\begin{equation*}
\begin{split}
\left|a_{n+1} - \frac{\sqrt{5}-1}{2}\right| &= \left|\frac{3-\sqrt{5} - (\sqrt{5}-1)a_n}{2(1+a_n)}\right|\\
&=\left|\frac{-(\sqrt{5}-1)\left(a_n-\displaystyle\frac{\sqrt{5}-1}{2}\right)}{2(1+a_n)}\right|\\
&<\left|\frac{\sqrt{5}-1}{2}\right|\cdot\left|a_n-\frac{\sqrt{5}-1}{2}\right|.
\end{split}
\end{equation*}
由此可见,$\left\{\left|a_n-\frac{\sqrt{5}-1}{2}\right|\right\}$收敛于$0$.
\end{proof}
%%%%%%%%%%%%%%%%%%
\begin{example}
设$a_n\geq 0$,$S_n=\displaystyle\sum_{k=1}^na_k$收敛于$S$。证明:$b_n=(1+a_1)(1+a_2)\cdots(1+a_n)$收敛。
\end{example}
\begin{proof}很显然,$\{b_n\}$是单调增数列。下面我们证明$\{b_n\}$是有界数列。
由于$a_n\geq 0$且$S_n\rightarrow S$, 则$S_n$是单调增的收敛于$S$.从而$$\displaystyle\sum_{k=1}^na_k < S,\quad\forall n\in \mathbb{N}.$$

另一方面,
$$b_n\leq \left(\frac{1}{n}\displaystyle\sum_{k=1}^n(1+a_k)\right)^n\leq \left(1+\frac{S}{n}\right)^n.$$
数列$\left\{\left(1+\displaystyle\frac{S}{n}\right)^n\right\}$是单调增的,且$$\displaystyle  \lim_{n\to +\infty}\left(1+\frac{S}{n}\right)^n=e^{S}.$$
于是
$$b_n \leq e^{S}, \forall n\in\mathbb{N}.$$
由实数连续性命题(二)可知, 数列$\{b_n\}$是收敛数列。
\end{proof}
%%%%%%%%%%%%%%%%%%%%%%%%%%%%%
\section{上极限与下极限}
\subsection{练习题}
\begin{example}求$\displaystyle\lowlim_{n\to +\infty}a_n$与$\displaystyle\uplim_{n\to +\infty}a_n$:
\renewcommand\labelenumi{\normalfont(\theenumi)}
\begin{enumerate}
\item $a_n = \displaystyle\frac{(-1)^n}{n}+\displaystyle\frac{1+(-1)^n}{2}$;
\begin{solution}
$\displaystyle\lowlim_{n\to +\infty}a_n = 0$, $\displaystyle\uplim_{n\to +\infty}a_n = 1$.
\end{solution}
\item $a_n = n^{(-1)^n}$;
\begin{solution}
$\displaystyle\lowlim_{n\to +\infty}a_n = 0$, $\displaystyle\uplim_{n\to +\infty}a_n = +\infty$.
\end{solution}
\item $a_n = [1+2^{(-1)^nn}]^{\frac{1}{n}}$;
\begin{solution}
$\displaystyle\lowlim_{n\to +\infty}a_n = 1$, $\displaystyle\uplim_{n\to +\infty}a_n = 2$.
\end{solution}
\item $a_n = \frac{n^2}{1+n^2}\cos{\frac{2n\pi}{3}}$;
\begin{solution}
$\displaystyle\lowlim_{n\to +\infty}a_n = -\frac{1}{2}$, $\displaystyle\uplim_{n\to +\infty}a_n = 1$.
\end{solution}
\item $a_n = \frac{n^2+1}{n^2}\sin\frac{\pi}{n}$;
\begin{solution}
$\displaystyle\lowlim_{n\to +\infty}a_n = 0$, $\displaystyle\uplim_{n\to +\infty}a_n = 0$.
\end{solution}
\item $a_n=\displaystyle\sqrt[n]{\left|\cos\frac{n\pi}{3}\right|}$;
\begin{solution}
$\displaystyle\lowlim_{n\to +\infty}a_n = 1$, $\displaystyle\uplim_{n\to +\infty}a_n = 1$.
\end{solution}
\item $a_n = 
\begin{cases}
0, \quad &\text{n为奇数},\\
\frac{n}{\sqrt[n]{n!}},\quad &\text{n为偶数}.
\end{cases}$
\begin{solution}
$\displaystyle\lowlim_{n\to +\infty}a_n = 0$, $\displaystyle\uplim_{n\to +\infty}a_n = e$.
\end{solution}
\end{enumerate}
\end{example}
%%%%%%%%%%%%%%%%%%%
\begin{example}
证明下面各式当两端有意义时成立:
\renewcommand\labelenumi{\normalfont(\theenumi)}
\begin{enumerate}
\item $\displaystyle\lowlim_{n\to +\infty}a_n + \displaystyle\lowlim_{n\to +\infty}b_n \leq \displaystyle\lowlim_{n\to +\infty}(a_n + b_n) \leq \displaystyle\lowlim_{n\to +\infty}a_n + \displaystyle\uplim_{n\to +\infty}b_n$,\\
$\displaystyle\lowlim_{n\to +\infty}a_n + \displaystyle\uplim_{n\to +\infty}b_n \leq \displaystyle\uplim_{n\to +\infty}(a_n + b_n) \leq \displaystyle\uplim_{n\to +\infty}a_n + \displaystyle\uplim_{n\to +\infty}b_n$
\begin{proof}
有上,下极限的定义,我们可以得到两个简单的无需证明的事实:对于任何数列$\{a_n\}$, 
$\{a_{n_k}\}$是一个任意一个子列, 则
\renewcommand\labelenumi{\normalfont(\theenumi)}
\begin{enumerate}
\item $\displaystyle\lowlim_{n\to +\infty}a_n \leq \displaystyle\lowlim_{k\to +\infty}a_{n_k}$
\item $\displaystyle\uplim_{k\to +\infty}a_{n_k} \leq \displaystyle\uplim_{n\to +\infty}a_n$
\end{enumerate}
我们取子列$\{a_{n_k} + b_{n_k}\}$使得
$$\displaystyle\lim_{k\to +\infty}(a_{n_k} + b_{n_k}) = \displaystyle\lowlim_{n\to +\infty}(a_n + b_n).$$
再在子列$\{a_{n_k}\}$中取子列$\{a_{n_{k_l}}\}$使得
$$\displaystyle\lim_{l\to +\infty}a_{n_{k_l}} = \displaystyle\lowlim_{k\to +\infty}a_{n_k}\geq \displaystyle\lowlim_{n\to +\infty}a_n.$$
对于子列$\{b_{n_{k_l}}\}$,我们会有
$$\displaystyle\lim_{l\to +\infty}b_{n_{k_l}} \geq \displaystyle\lowlim_{k\to +\infty}b_{n_k}\geq \displaystyle\lowlim_{n\to +\infty}b_n.$$

从而
\begin{equation*}
\begin{split}
\displaystyle\lowlim_{n\to +\infty}a_n+\displaystyle\lowlim_{n\to +\infty}b_n &\leq \displaystyle\lim_{l\to +\infty}a_{n_{k_l}} + \displaystyle\lim_{l\to +\infty}b_{n_{k_l}} \\&= \displaystyle\lim_{l\to +\infty}(a_{n_{k_l}}+b_{n_{k_l}}) \\&= \displaystyle\lim_{k\to +\infty}(a_{n_k} + b_{n_k})  \\&= \displaystyle\lowlim_{n\to +\infty}(a_n + b_n).
\end{split}
\end{equation*}

取子列$\{a_{n_k}\}$,使得
$$\displaystyle\lim_{k\to +\infty}a_{n_k}=\displaystyle\lowlim_{n\to +\infty}a_n.$$
考虑子列$\{a_{n_k}+b_{n_k}\}$,在其中取子列$\{a_{n_{k_l}} + b_{n_{k_l}}\}$,使得
$$\displaystyle\lim_{l\to +\infty}(a_{n_{k_l}}+b_{n_{k_l}}) = \displaystyle\lowlim_{k\to +\infty}(a_{n_k} + b_{n_k}) \geq \displaystyle\lowlim_{n\to +\infty}(a_n + b_n).$$
对于同样下标的子列$\{b_{n_{k_l}}\}$, 我们有
$$\displaystyle\lim_{l\to +\infty}b_{n_{k_l}} \leq \displaystyle\uplim_{k\to +\infty}b_{n_k}\leq \displaystyle\uplim_{n\to +\infty}b_n.$$

由此可见
\begin{equation*}
\begin{split}
\displaystyle\lowlim_{n\to +\infty}(a_n + b_n) &\leq \displaystyle\lim_{l\to +\infty}(a_{n_{k_l}}+b_{n_{k_l}}) \\&= \displaystyle\lim_{l\to +\infty}a_{n_{k_l}}+\displaystyle\lim_{l\to +\infty}b_{n_{k_l}}\\&\leq \displaystyle\lim_{k\to +\infty}a_{n_k} + \displaystyle\uplim_{n\to +\infty}b_n\\&=\displaystyle\lowlim_{n\to +\infty}a_n+ \displaystyle\uplim_{n\to +\infty}b_n
\end{split}
\end{equation*}

下面我们用类似的办法证明第二式亦成立。
取子列$\{b_{n_k}\}$使得
$$\displaystyle\lim_{k\to +\infty}b_{n_k} = \displaystyle\uplim_{n\to +\infty}b_n.$$
对于相应的$\{a_{n_k}+b_{n_k}\}$可以取子列$\{a_{n_{k_l}}+b_{n_{k_l}}\}$使得
$$\displaystyle\lim_{l\to +\infty}(a_{n_{k_l}}+b_{n_{k_l}}) = \displaystyle\uplim_{k\to +\infty}(a_{n_k}+b_{n_k}) \leq   \displaystyle\uplim_{n\to +\infty}(a_n+b_n).$$
显然
$$\displaystyle\lim_{l\to +\infty}a_{n_{k_l}}\geq \displaystyle\lowlim_{n\to +\infty}a_n.$$
于是
\begin{equation*}
\begin{split}
\displaystyle\lowlim_{n\to +\infty}a_n+\displaystyle\uplim_{n\to +\infty}b_n &\leq \displaystyle\lim_{l\to +\infty}a_{n_{k_l}} + \displaystyle\lim_{l\to +\infty}b_{n_{k_l}}\\&=\displaystyle\lim_{l\to +\infty}(a_{n_{k_l}}+b_{n_{k_l}}) \\&=\displaystyle\lim_{l\to +\infty}(a_{n_{k_l}}+b_{n_{k_l}}) \\&\leq \displaystyle\uplim_{n\to +\infty}(a_n+b_n)
\end{split}
\end{equation*}

取子列$\{a_{n_k} + b_{n_k}\}$,使得
$$\displaystyle\lim_{k\to +\infty}(a_{n_k} + b_{n_k}) =  \displaystyle\uplim_{n\to +\infty}(a_n + b_n).$$
取子列$\{b_{n_{k_l}}\}$使得
$$\displaystyle\lim_{l\to +\infty}b_{n_{k_l}} = \displaystyle\uplim_{k\to +\infty}b_{n_k} \leq \displaystyle\uplim_{n\to +\infty}b_n.$$
另一方面
$$\displaystyle\lim_{l\to +\infty}a_{n_{k_l}}\leq \displaystyle\uplim_{k\to +\infty}a_{n_k} \leq \displaystyle\uplim_{n\to +\infty}a_n.$$
所以
$$\displaystyle\uplim_{n\to +\infty}(a_n + b_n) \leq \displaystyle\uplim_{n\to +\infty}a_n + \displaystyle\uplim_{n\to +\infty}b_n.$$
\end{proof}
\item 设$\displaystyle\lim_{n\to +\infty}b_n = b$,则\\
$\displaystyle\lowlim_{n\to +\infty}(a_n + b_n) = \displaystyle\lowlim_{n\to +\infty}a_n + b$,\\
$\displaystyle\uplim_{n\to +\infty}(a_n + b_n) = \displaystyle\uplim_{n\to +\infty}a_n + b$
\begin{proof}
这题是上面一题的直接应用。
\end{proof}
\item $\displaystyle\lowlim_{n\to +\infty}(-a_n)=-\displaystyle\uplim_{n\to +\infty}a_n, \displaystyle\uplim_{n\to +\infty}(-a_n)=-\displaystyle\lowlim_{n\to +\infty}a_n$;
\begin{proof}$\displaystyle\inf_{k\geq n}\{-a_k\} = -\displaystyle \sup_{k \geq n}\{a_k\} \Rightarrow \displaystyle\lowlim_{n\to +\infty}(-a_n)=-\displaystyle\uplim_{n\to +\infty}a_n$.\\
$\displaystyle\sup_{k\geq n}\{-a_k\} = -\displaystyle \inf_{k \geq n}\{a_k\} \Rightarrow \displaystyle\uplim_{n\to +\infty}(-a_n)=-\displaystyle\lowlim_{n\to +\infty}a_n$
\end{proof}
\item 设$\{a_n\}$与$\{b_n\}$均为非负数列,则\\
$\displaystyle\lowlim_{n\to +\infty}a_n\cdot \displaystyle\lowlim_{n\to +\infty}b_n \leq \displaystyle\lowlim_{n\to +\infty}a_nb_n\leq \displaystyle\lowlim_{n\to +\infty}a_n\cdot\displaystyle\uplim_{n\to +\infty}b_n$,\\
$\displaystyle\lowlim_{n\to +\infty}a_n\cdot \displaystyle\uplim_{n\to +\infty}b_n \leq \displaystyle\uplim_{n\to +\infty}a_nb_n\leq \displaystyle\uplim_{n\to +\infty}a_n\cdot\displaystyle\uplim_{n\to +\infty}b_n$;
\begin{proof}
我们现证第一式。
取子列$\{a_{n_k}b_{n_k}\}$使得
$$\displaystyle\lim_{k\to +\infty}a_{n_k}b_{n_k}=\displaystyle\lowlim_{n\to +\infty}a_nb_n.$$
对于上述的$\{a_{n_k}\}$,我们再取子列$\{a_{n_{k_l}}\}$,使得
$$\displaystyle\lim_{l\to +\infty}a_{n_{k_l}}=\displaystyle\lowlim_{k\to +\infty}a_{n_k}\geq \displaystyle\lowlim_{n\to +\infty}a_n \geq 0.$$
对于$\{b_{n_{k_l}}\}$, 我们会有
$$\displaystyle\lowlim_{l\to +\infty}b_{n_{k_l}}\geq \displaystyle\lowlim_{k\to +\infty}b_{n_k}\geq \displaystyle\lowlim_{n\to +\infty}b_n \geq 0.$$
所以
\begin{equation*}
\begin{split}
\displaystyle\lowlim_{n\to +\infty}a_n\cdot \displaystyle\lowlim_{n\to +\infty}b_n &\leq \displaystyle\lim_{l\to +\infty}a_{n_{k_l}}\cdot \displaystyle\lowlim_{l\to +\infty}b_{n_{k_l}} \\&= \displaystyle\lim_{l\to +\infty}a_{n_{k_l}}b_{n_{k_l}} \\&= \displaystyle\lim_{k\to +\infty}a_{n_k}b_{n_k} \\&=
\displaystyle\lowlim_{n\to +\infty}a_nb_n 
\end{split}
\end{equation*}
取$\{a_{n_k}\}$使得
$$\displaystyle\lim_{k\to +\infty}a_{n_k}=\displaystyle\lowlim_{n\to +\infty}a_n.$$
对于对应的子列$\{b_{n_k}\}$,我们再取子列$\{b_{n_{k_l}}\}$使得
$$\displaystyle\lim_{l\to +\infty}b_{n_{k_l}} = \displaystyle\uplim_{k\to +\infty}b_{n_k} \leq \displaystyle\uplim_{n\to +\infty}b_n.$$
所以
\begin{equation*}
\begin{split}
\displaystyle\lowlim_{n\to +\infty}a_nb_n &\leq \displaystyle\lowlim_{k\to +\infty}a_{n_k}b_{n_k}\\
&\leq \displaystyle\lowlim_{l\to +\infty}a_{n_{k_l}}b_{n_{k_l}} \\&= \displaystyle\lim_{l\to +\infty}a_{n_{k_l}}\cdot \displaystyle\lim_{l\to +\infty}b_{n_{k_l}} \\&\leq \displaystyle\lowlim_{n\to +\infty}a_n\cdot\displaystyle\uplim_{n\to +\infty}b_n
\end{split}
\end{equation*}

现在我们来证明第二式。取子列$\{b_{n_k}\}$使得
$$\displaystyle\lim_{k\to +\infty}b_{n_k}=\displaystyle\uplim_{n\to +\infty}b_n.$$
对于相对应的$\{a_{n_k}\}$, 我们再取子列$\{a_{n_{k_l}}\}$使得
$$\displaystyle\lim_{l\to +\infty}a_{n_{k_l}} = \displaystyle\lowlim_{k\to +\infty}a_{n_k}\geq \displaystyle\lowlim_{n\to +\infty}a_n.$$
所以
\begin{equation*}
\begin{split}
\displaystyle\lowlim_{n\to +\infty}a_n \cdot \displaystyle\uplim_{n\to +\infty}b_n &\leq \displaystyle\lim_{l\to +\infty}a_{n_{k_l}}\cdot \displaystyle\lim_{l\to +\infty}b_{n_{k_l}} \\&=\displaystyle\lim_{l\to +\infty}a_{n_{k_l}}b_{n_{k_l}} \\&\leq \displaystyle\uplim_{k\to +\infty}a_{n_k}b_{n_k}\\&\leq \displaystyle\uplim_{n\to +\infty}a_nb_n
\end{split}
\end{equation*}
取子列$\{a_{n_k}b_{n_k}\}$使得
$$\displaystyle\lim_{k\to +\infty}a_{n_k}b_{n_k}=\displaystyle\uplim_{n\to +\infty}a_nb_n.$$
在$\{a_{n_k}\}$取收敛子列$\{a_{n_{k_l}}\}$使得
$$\displaystyle\lim_{l\to +\infty}a_{n_{k_l}} =\displaystyle\uplim_{k\to +\infty}a_{n_k}\leq \displaystyle\uplim_{n\to +\infty}a_n.$$
另一方面
$$\displaystyle\uplim_{l\to +\infty}b_{n_{k_l}}\leq \displaystyle\uplim_{k\to +\infty}b_{n_k}\leq \displaystyle\uplim_{n\to +\infty}b_n.$$
所以
\begin{equation*}
\begin{split}
\displaystyle\uplim_{n\to +\infty}a_nb_n &= \displaystyle\uplim_{l\to +\infty}a_{n_{k_l}}b_{n_{k_l}} \\&\leq \displaystyle\lim_{l\to +\infty}a_{n_{k_l}} \cdot\displaystyle\uplim_{l\to +\infty}b_{n_{k_l}}\\&\leq \displaystyle\uplim_{n\to +\infty}a_n\cdot \displaystyle\uplim_{n\to +\infty}b_n
\end{split}
\end{equation*}
\end{proof}
\item 设$\{b_n\}$非负,且$\displaystyle\lim_{n\to +\infty}b_n=b$,则\\
$\displaystyle\lowlim_{n\to +\infty}a_nb_n=b\displaystyle\lowlim_{n\to +\infty}a_n, \displaystyle\uplim_{n\to +\infty}a_nb_n=b\displaystyle\uplim_{n\to +\infty}a_n$;
\begin{proof}
直接用上一题的结论就可以得证。
\end{proof}
\item 设$a_n > 0(n\in\mathbb{N})$, $\displaystyle\lowlim_{n\to +\infty}a_n > 0$,则
$$\displaystyle\uplim_{n\to +\infty}\frac{1}{a_n} = \frac{1}{\displaystyle\lowlim_{n\to +\infty}a_n}.$$
\begin{proof}
因为$a_n > 0, \forall n$, 从而
$$\displaystyle\sup_{k \geq n}\left\{\frac{1}{a_k}\right\} = \frac{1}{\displaystyle\inf_{k \geq n}\{a_k\}}.$$
由上,下极限的定义,命题得证。
\end{proof}
\end{enumerate}
\end{example}
%%%%%%%%%%%%%%%%%%%
\begin{example}
设$a_n>0(n\in\mathbb{N})$,且$\displaystyle\uplim_{n\to +\infty}a_n\cdot\displaystyle\uplim_{n\to +\infty}\frac{1}{a_n}=1$,证明:数列$\{a_n\}$收敛。
\end{example}
\begin{proof}
由题可知$\displaystyle\uplim_{n\to +\infty}a_n > 0$。我们可以证明
$$\displaystyle\lowlim_{n\to +\infty}\frac{1}{a_n} = \frac{1}{\displaystyle\uplim_{n\to +\infty}a_n}.$$
由此可知$$\displaystyle\lowlim_{n\to +\infty}\frac{1}{a_n} = \displaystyle\uplim_{n\to +\infty}\frac{1}{a_n}.$$
故$\left\{\frac{1}{a_n}\right\}$是收敛数列。当然$\{a_n\}$收敛。
\end{proof}
%%%%%%%%%%%%%%%%%%%%%
\begin{example}
设数列$\{a_n\}, a_n\leq 1,n=1,2,\cdots$,且满足:
$$a_m+a_n-1<a_{m+n}<a_m+a_n+1.$$
证明: (1) $\displaystyle\lim_{n\to +\infty}\frac{a_n}{n}=\omega$,其中$\omega$为有限数;(2) $n\omega -1 \leq a_n \leq n\omega+1$.
\end{example}
\begin{proof}
先证明(1).
$$a_1 -\frac{1}{n}< \frac{a_n}{n} < a_1 +\frac{1}{n}.$$
由此得证(1),且$\omega=a_1$。
由此(2)得证。
\end{proof}
%%%%%%%%%%%%%%%%%%%%%%%%%%%%%%%%
\begin{example}
设$a_n\geq 0,n\in\mathbb{N}$。证明:
$$\displaystyle\lim_{n\to +\infty}\sqrt[n]{a_n}\leq 1 \iff \text{对任何}l > 1,\text{有} \displaystyle\uplim_{n\to +\infty}\frac{a_n}{l^n} = 0.$$
如果删去“任何”两字,结论如何?
\end{example}
\begin{remark}
这个题目的结论是不对的。比如
\begin{equation*}
a_n  = \begin{cases} \frac{1}{2^n}, &\quad n\text{是奇数}\\
1, &\quad n\text{是偶数}
\end{cases}
\end{equation*}
于是
$$\displaystyle\uplim_{n\to +\infty}\frac{a_n}{l^n} \leq \displaystyle\uplim_{n\to +\infty}\frac{1}{l^n}  = 0,$$
但是,$\displaystyle\lim_{n\to +\infty}\sqrt[n]{a_n}$ 不存在。所以下面我们证明:
$$\displaystyle\uplim_{n\to +\infty}\sqrt[n]{a_n}\leq 1 \iff \text{对任何}l > 1,\text{有} \displaystyle\uplim_{n\to +\infty}\frac{a_n}{l^n} = 0.$$
\end{remark}
\begin{proof}
($\Rightarrow$): 对于任何的$l > 1$, 取$\varepsilon=\frac{l-1}{2} > 0$,我们可以找到
$N>0$,使得$$\sqrt[n]{a_n} \leq 1 + \frac{l -1}{2} = \frac{l + 1}{2} < l,\quad\forall n > N.$$ 
于是 $$\frac{a_n}{l^n} < \left(\frac{l + 1}{2l}\right)^n, \quad\forall n > N.$$
由$\frac{l + 1}{2l} < 1$可知,$$\displaystyle\uplim_{n\to +\infty}\frac{a_n}{l^n} = 0.$$

($\Leftarrow$): 由定义可知,对任意的$l > 1$, 存在$N > 0$, 当$n>N$时有,
$$\frac{a_n}{l^n} < 1 \Rightarrow a_n < l^n\Rightarrow \sqrt[n]{a_n} < l.$$
从而$$\displaystyle\uplim_{n\to +\infty}\sqrt[n]{a_n} \leq l\Rightarrow \displaystyle\uplim_{n\to +\infty}\sqrt[n]{a_n} \leq 1.$$

如果删去“任何”两字, 结论不成立。
\end{proof}
%%%%%%%%%%%%%%%%%%%%%%%%%%%%%%
\subsection{思考题}
%%%%%%%%%%%%%%%%%%%
\begin{example}
设数列$\{x_n\}$有界,且$\displaystyle\lim_{n\to +\infty}(x_{n+1}-x_n) = 0$, 令
$$l=\displaystyle\lowlim_{n\to +\infty}x_n, \quad L=\displaystyle\uplim_{n\to +\infty}x_n.$$
证明:$\{a\in\mathbb{R} | \text{有子列}x_{n_k}\rightarrow a(k\rightarrow \infty)\}=[l,L]$.如果删去条件$\displaystyle\lim_{n\to +\infty}(x_{n+1}-x_n) = 0$,结论如何?
\end{example}
\begin{proof}
对于$a\in (l, L)$和任意的$\varepsilon < \frac{1}{2}\min\{a-l, L-a\}$, 我们有存在$N$
使得:
\renewcommand\labelenumi{\normalfont(\theenumi)}
\begin{enumerate}
\item 当$n>N$时,$-\varepsilon \leq a_{n+1} - a_n \leq \varepsilon$.
\item 存在子列$n_{k}$, 使得$a_{n_k} \leq l +\varepsilon, \quad\forall k$
\item 存在子列$n_{l}$, 使得$a_{n_l}\geq L -\varepsilon,\quad\forall l$
\end{enumerate}
选择子列$n_{t}$,并且重新标记下标,使得
\begin{equation*}
a_{n_{t}}
\begin{cases} 
\leq l + \varepsilon, &\quad \text{当t是奇数}\\
\geq L -\varepsilon, &\quad \text{当t是偶数}
\end{cases}
\end{equation*}
对于任意上述子列的任意两个相邻的数$a_{n_{2t}}, a_{n_{2t+1}}$, 在院数列中一定有至少一个$a_{n_{t'}}$落在区间$(a-\varepsilon, a+\varepsilon)$. 从而由极限的定义知这个子列收敛到$a$.命题得证。
\end{proof}
%%%%%%%%%%%%%%%%%%%%%%%%%%
\begin{example}
设$0\leq a_{n+m}\leq a_n\cdot a_m(n,m=1,2,\cdots)$.证明$\displaystyle\uplim_{n\to +\infty}\sqrt[n]{a_n} = \displaystyle\lowlim_{n\to +\infty}\sqrt[n]{a_n}$且$\sqrt[n]{a_n}$收敛。
\end{example}
\begin{proof}
设$\displaystyle\uplim_{n\to +\infty}\sqrt[n]{a_n} 
=a > b=\displaystyle\lowlim_{n\to +\infty}\sqrt[n]{a_n}$
取$N > 0$使得 $$a_N < b +\frac{a-b}{3}.$$
又取子列$a_{n_k}$使得
$$\displaystyle\uplim_{k\to +\infty}\sqrt[n_{k}]{a_{n_k}} = a.$$
当$k$充分大后,$n_k > N$,且$n_k = m_kN + l$,其中$l = 0, 1, \cdots, N - 1$.
于是$$a_{n_k} \leq a_{m_kN}\cdot a_l = a_N^{m_k}\cdot a_l = a_N^{n_k}\cdot \frac{a_l}{a_N^{l}}.$$
由此可知
$$0 \leq \sqrt[n_k]{a_{n_k}}\leq \sqrt[n_k]{a_N^{n_k}}\sqrt[n_k]{\frac{a_l}{a_N^{l}}}= a_N\cdot\sqrt[n_k]{\frac{a_l}{a_N^{l}}}.$$
于是
$$\displaystyle\uplim_{n\to +\infty}\sqrt[n_k]{a_{n_k}} \leq a_N \leq b+\frac{a-b}{3} < a.$$
这与$\{a_{n_k}\}$的选择矛盾。从而知$a=b$.命题得证。
\end{proof}
%%%%%%%%%%%%%%%%%%%%%%%
\section{Stolz公式}
\subsection{练习题}
\begin{example}
设$C_n^k=\displaystyle\frac{n!}{k!(n-k)!}$为组合数。应用Stolz公式证明:
$$\displaystyle\lim_{n\to +\infty}\frac{\displaystyle\sum_{k=0}^n\ln{C_n^k}}{n^2}=\frac{1}{2}.$$
\end{example}
\begin{proof}我们先计算
$$\ln{C_n^k}+\ln{C_n^{n-k}} = 2\ln{n!} - 2\ln{k!} - 2\ln{(n-k)!}, \quad\forall k\leq n.$$
所以
$$\displaystyle\sum_{k=0}^n\ln{C_n^k} = n\ln{n!} - 2\sum_{k=0}^n\ln{k!}.$$
设$x_n=n^2, y_n=\displaystyle\sum_{k=0}^n\ln{C_n^k}$,于是
\begin{equation*}
\begin{split}
\lim_{n\to +\infty}\frac{\displaystyle\sum_{k=0}^n\ln{C_n^k}}{n^2}&\xlongequal{\text{Stolz公式}}\lim_{n\to +\infty}\frac{(n-1)\ln{n}-\ln{n!}}{2n-1}\\&=\lim_{n\to +\infty}\left(\ln{\frac{n}{\sqrt[n]{n!}}}\cdot\frac{n}{2n-1}\right) +\lim_{n\to +\infty}\frac{\ln{n}}{2n-1}\\
&=\frac{1}{2}.
\end{split}
\end{equation*}
命题得证
\end{proof}
%%%%%%%%%%%%%%%%%
\begin{example}
应用Stolz公式证明:
\renewcommand\labelenumi{\normalfont(\theenumi)}
\begin{enumerate}
\item $\displaystyle\lim_{n\to +\infty}\frac{\displaystyle\sum_{k=1}^n\sqrt{k}}{n^{\frac{3}{2}}}=\frac{2}{3}$;
\begin{proof}
设$y_n=\sum_{k=1}^n\sqrt{k}, x_n = n^{\frac{3}{2}}$,则
\begin{equation*}
\begin{split}
\lim_{n\to +\infty}\frac{\displaystyle\sum_{k=1}^n\sqrt{k}}{n^{\frac{3}{2}}}&\xlongequal{\text{Stolz公式}}\lim_{n\to +\infty}\frac{\sqrt{n}}{n^{\frac{3}{2}}-(n-1)^{\frac{3}{2}}}\\&=
\lim_{n\to +\infty}\frac{\sqrt{n}(\sqrt{n^3}+\sqrt{(n-1)^3})}{n^3-(n-1)^3}\\
&=\lim_{n\to +\infty}\frac{n^2\left(1+\sqrt{(1-\frac{1}{n})^3}\right)}{3n^2-3n+1}\\
&=\frac{2}{3}.
\end{split}
\end{equation*}
\end{proof}
\item $\displaystyle\lim_{n\to +\infty}n\left[\frac{\displaystyle\sum_{k=1}^n\sqrt{k}}{n^{\frac{3}{2}}}-\frac{2}{3}\right]=\frac{1}{2}$.
\begin{proof}我们设$y_n = 3\displaystyle\sum_{k=1}^n\sqrt{k} -2n^{\frac{3}{2}}$, $x_n = 3\sqrt{n}$.于是
\begin{equation*}
\begin{split}
\frac{y_n-y_{n-1}}{x_n-x_{n-1}}&=\frac{3\sqrt{n}-2n^{\frac{3}{2}}+2(n-1)^{\frac{3}{2}}}{3(\sqrt{n}-\sqrt{n-1})}\\
&=\frac{3\sqrt{n}(\sqrt{n^3}+\sqrt{(n-1)^3})-6n^2+6n-2}{3(\sqrt{n}-\sqrt{n-1})(\sqrt{n^3}+\sqrt{(n-1)^3})}\\
&=\frac{3n^2(\sqrt{(1-1/n)^3}-1)+6n-2}{3(\sqrt{n}-\sqrt{n-1})(\sqrt{n^3}+\sqrt{(n-1)^3})}\\
&=\frac{-9n+9-3/n+(6n-2)(\sqrt{(1-1/n)^3}+1)}{3(\sqrt{n}-\sqrt{n-1})(\sqrt{n^3}+\sqrt{(n-1)^3})(\sqrt{(1-1/n)^3}+1)}\\
&=\frac{(-9n+9-3/n+(6n-2)(\sqrt{(1-1/n)^3}+1))(\sqrt{n}+\sqrt{n-1})}{3(\sqrt{n^3}+\sqrt{(n-1)^3})(\sqrt{(1-1/n)^3}+1)}\\
&=\frac{n^{\frac{3}{2}}(-9+9/n-3/n^2+(6-2/n)(\sqrt{(1-1/n)^3}+1))(1+\sqrt{1-1/n})}{3n^{\frac{3}{2}}(1+\sqrt{(1-1/n)^3})(\sqrt{(1-1/n)^3}+1)}
\end{split}
\end{equation*}
于是:
$$\displaystyle\lim_{n\to +\infty}n\left[\frac{\displaystyle\sum_{k=1}^n\sqrt{k}}{n^{\frac{3}{2}}}-\frac{2}{3}\right] \xlongequal{\text{Stolz公式}} \displaystyle\lim_{n\to +\infty}\frac{y_n-y_{n-1}}{x_n-x_{n-1}}=\frac{1}{2}.$$
命题得证
\end{proof}
\end{enumerate}
\end{example}
%%%%%%%%%%%%%%%%%%
\subsection{思考题}
\begin{example}
设$0<x_1<1$,$x_{n+1} = x_n(1-x_n), n =1,2,\cdots.$ 证明:$\displaystyle\lim_{n\to +\infty}nx_n=1$.进而设$0<x_1\leq\frac{1}{q}$,其中$0<q\leq 1$,并且$x_{n+1}=x_n(1-qx_n), n\in\mathbb{N}$.证明:$\displaystyle\lim_{n\to +\infty}nx_n=\frac{1}{q}.$
\end{example}
\begin{proof}
如果我们能证明$\{x_n\}$单调减且$\displaystyle\lim_{n\to +\infty}x_n = 0$,则
\begin{equation*}
\begin{split}
\displaystyle\lim_{n\to +\infty}nx_n&\xlongequal{\text{Stolz公式}}\displaystyle\lim_{n\to +\infty}\frac{n-(n-1)}{\frac{1}{x_n} - \frac{1}{x_{n-1}}}\\&=\displaystyle\lim_{n\to +\infty}\frac{x_{n-1}(1-qx_{n-1})}{qx_{n-1}}\\&=\frac{1}{q}.
\end{split}
\end{equation*}
下面我们证明$\{x_n\}$单调减且$\displaystyle\lim_{n\to +\infty}x_n = 0$.
由于$x_n-x_{n-1} = -qx^2_{n-1}, \forall n\in\mathbb{N}$可知,$\{x_n\}$是单调减的。
很明显$x_n > 0, \forall n\in\mathbb{N}$. 由实数连续性命题(二)可知, $\{x_n\}$是收敛的。
设$\displaystyle\lim_{n\to +\infty}x_n = a$, 则$a=a(1-qa)\Rightarrow qa^2 = 0\Rightarrow a = 0$.命题得证。
\end{proof}
%%%%%%%%%%%%%%%%%%%%%%%%%%%%%
\begin{example}
由Toeplitz定理导出$\frac{\infty}{\infty}$型的Stolz公式。
\end{example}
\begin{proof}
取$$t_{nk} = \frac{x_k - x_{k-1}}{x_n-x_0}, \quad\forall k = 1,2,\cdots, n.$$
\renewcommand\labelenumi{\normalfont(\theenumi)}
\begin{enumerate}
\item 由于$\{x_n\}$是单调增数列,$t_{nk} > 0$. 
\item 因为$\displaystyle\lim_{n\to +\infty}x_n=+\infty$, $\displaystyle\lim_{n\to +\infty}t_{nk} = 0$. 
\item $\displaystyle\sum_{k=1}^{n}t_{nk} = \frac{x_n - x_0}{x_n-x_0}= 1$.
\end{enumerate}
由Toeplitz定理可知
\begin{equation*}
\begin{split}
\lim_{n\to +\infty}\frac{y_n}{x_n} &= \lim_{n\to +\infty}\frac{y_0}{x_n} +\lim_{n\to +\infty}\frac{x_n - x_0}{x_n}\lim_{n\to +\infty}\frac{y_ n - y_0}{x_n-x_0}\\&=\lim_{n\to +\infty}\sum_{k=1}^{n-1}\frac{x_{k+1}-x_{k}}{x_n-x_0}\cdot\frac{y_{k+1}-y_k}{x_{k+1}-x_k}\\&\lim_{n\to +\infty}\sum_{k=1}^{n-1}t_{nk}\frac{y_{k+1}-y_k}{x_{k+1}-x_k}\\
&\xlongequal{\text{Toeplitz公式}}\lim_{n\to +\infty}\frac{y_n-y_{n-1}}{x_{n}-x_{n-1}} = a.
\end{split}
\end{equation*}
命题得证。
\end{proof}
%%%%%%%%%%%%%%%%%%%%%%%%%%%%%
\begin{example}
设数列$\{a_n\}$满足$\displaystyle\lim_{n\to +\infty}a_n \displaystyle\sum_{i=1}^na_i^2 = 1$。证明:$\displaystyle\lim_{n\to +\infty}\sqrt[3]{3n}a_n=1$.
\end{example}
\begin{proof}
记$S_n = \displaystyle\sum_{i=1}^na_i^2.$我们很容易证明以下结论:
\renewcommand\labelenumi{\normalfont(\theenumi)}
\begin{enumerate}
\item $\{S_n\}$是单增的, 且$\displaystyle\lim_{n\to +\infty}S_n=+\infty$
\item $\displaystyle\lim_{n\to +\infty}a_n = 0$
\end{enumerate}
现在计算
\begin{equation*}
\begin{split}
S_n^3 - S_{n-1}^3 &= (S_n-S_{n-1})(S_n^2+S_nS_{n-1}+S_{n-1}^2) \\
&=a_n^2(S_n^2+S_n(S_n - a_n^2)) +(S_n-a_n^2)^2)\\
&=3(a_nS_n)^2-3a_n^3(a_nS_n) +a_n^6
\end{split}
\end{equation*}
从而
$$\displaystyle\lim_{n\to +\infty}\frac{S_n^3}{3n}\xlongequal{\text{Stolz公式}}\displaystyle\lim_{n\to +\infty}\frac{S_n^3-S_{n-1}^3}{3} = 1.$$
于是$$\displaystyle\lim_{n\to +\infty}\frac{1}{3na^3_n} = \displaystyle\lim_{n\to +\infty}\frac{1}{(a_nS_n)^3}\displaystyle\lim_{n\to +\infty}\frac{S^3_n}{3n} = 1.$$
由此可知$\displaystyle\lim_{n\to +\infty}\sqrt[3]{3n}a_n=1$。命题得证。
\end{proof}
%%%%%%%%%%%%%%%%%%%%%%%%%%%%%
\section{复习题1}
%%%%%%%%%%%%%%%%%%%%%%%%%%%%%
\begin{example}
设$a_0=1$,$a_{n+1}=a_n+\frac{1}{a_n}$,$n=0,1,2,\cdots$.证明:$\displaystyle\lim_{n\to +\infty}\frac{a_n}{\sqrt{2n}}= 1$.
\end{example}
\begin{proof}
由递归定义,
$$a_{n-1}^2+2 < a_{n-1}^2+\frac{1}{a^2_{n-1}} + 2< a_{n}^2,\quad\forall n > 1\Rightarrow a^2_{n}\geq 2*(n-1) + a_1^2=2n-1.$$
于是
$$0\leq \frac{1}{a_n^2} < \frac{1}{2n-1}, \forall n > 1\Rightarrow \displaystyle\lim_{n\to +\infty}\frac{1}{a_n^2}= 0.$$
算术平均
$$\displaystyle\lim_{n\to +\infty}\frac{\displaystyle\frac{1}{a_1^2}+\frac{1}{a_2^2} +\cdots +\frac{1}{a_n^2}}{n} = 0.$$
现在计算
\begin{equation*}
\begin{split}
\frac{a^2_n}{2n} &= \frac{2n-1}{2n} +\frac{\frac{1}{a_1^2}+\frac{1}{a_2^2} +\cdots +\frac{1}{a_{n-1}^2}}{2n}\\
&=\frac{2n-1}{2n} +\frac{\frac{1}{a_1^2}+\frac{1}{a_2^2} +\cdots +\frac{1}{a_{n-1}^2}}{n-1}\cdot\frac{n-1}{2n}
\end{split}
\end{equation*}
由此可知
$$\displaystyle\lim_{n\to +\infty}\frac{a^2_n}{2n}= 1\Rightarrow \displaystyle\lim_{n\to +\infty}\frac{a_n}{\sqrt{2n}}= 1.$$
命题得证。
\end{proof}
%%%%%%%%%%%%%%%%%%
\begin{example}
设$\displaystyle\lim_{n\to +\infty}x_n=\lim_{n\to +\infty}y_n=0$,并且存在常数$K$使得$\forall n\in\mathbb{N}$,有$$|y_1|+|y_2|+\cdots+|y_n|\leq K.$$
令$$z_n=x_1y_n+x_2y_{n-1}+\cdots+x_ny_1,\quad n\in\mathbb{N}.$$证明:$\displaystyle\lim_{n\to +\infty}z_n=0$.
\end{example}
\begin{proof}
由于$\displaystyle\lim_{n\to +\infty}x_n=0$,存在$M > 0$使得$|a_n| < M, \forall n\in\mathbb{N}$.

对$\forall \varepsilon > 0$, 存在$N_1\in\mathbb{N}$,当$n>N_1$时有$$\left|x_n\right| < \frac{\varepsilon}{2K}, \quad\forall n > N_1.$$

设$s_n=\displaystyle\sum_{k=1}^{n}\left|y_k\right|$.很显然$\{s_n\}$是一个收敛数列。于是,存在$N_2\in\mathbb{N}$,当$n>N_2$时有
$$s_{n} - s_{n-N_1} <\frac{\varepsilon}{2MN_1}.$$

综上,
\begin{equation*}
\begin{split}
\left|z_n\right|&=\left|x_1y_n+x_2y_{n-1}+\cdots+x_ny_1\right|\\
&<\left|x_1y_n+x_2y_{n-1}+\cdots+x_{N_1}y_{n-N_1+1}\right| + \left|x_{N_1+1}y_{n-N_1}+x_{N_1+2}y_{n-N_1-1}+\cdots+x_{n}y_1\right|\\
&<MN_1(s_n-s_{n-N_1}) + \frac{\varepsilon}{2K}(|y_1|+|y_2|+\cdots +|y_{n-N_1}|)\\
&<MN_1\frac{\varepsilon}{2MN_1}+K\frac{\varepsilon}{2K} \\&= \varepsilon
\end{split}
\end{equation*}
命题得证。
\end{proof}
%%%%%%%%%%%%%%%
\begin{example}
设数列$\{a_n\}$与$\{b_n\}$满足:
\renewcommand\labelenumi{\normalfont(\theenumi)}
\begin{enumerate}
\item $b_n>0, b_0+b_1+\cdots+b_n\rightarrow +\infty(n\rightarrow+\infty)$;
\item $\displaystyle\lim_{n\to +\infty}\frac{a_n}{b_n}=s$.
\end{enumerate}
应用Toeplitz定理证明:
$$\displaystyle\lim_{n\to +\infty}\frac{a_0+a_1+\cdots+a_n}{b_0+b_1+\cdots+b_n}=s.$$
\end{example}
\begin{proof}
在这题里我们可以取$$t_{nk} = \displaystyle\frac{b_k}{b_0+b_1+b_2+\cdots+b_n}.$$ 
\renewcommand\labelenumi{\normalfont(\theenumi)}
\begin{enumerate}
\item $t_{nk} > 0, \forall n > 0, k=0,2,\cdots,n$;
\item 给定$n$, $\displaystyle\sum_{k=0}^{n}t_{nk} = 1$;
\item 由于$b_0+b_1+\cdots+b_n\rightarrow +\infty(n\rightarrow+\infty)$,给定$k$, $\displaystyle\lim_{n\to +\infty}t_{nk} = 0$;
\end{enumerate}

于是
$$\frac{a_0+a_1+\cdots+a_n}{b_0+b_1+\cdots+b_n}=\displaystyle\sum_{k=0}^{n}t_{nk}\frac{a_k}{b_k}.$$
由Toeplitz定理,命题可以得证。
\end{proof}
%%%%%%%%%%%%%%%%%
\begin{example}
设$p_k>0, k=1,2,\cdots$,且$\displaystyle\lim_{n\to +\infty}\frac{p_n}{p_1+p_2+\cdots+p_n}=0$,$\displaystyle\lim_{n\to +\infty}a_n=a$.证明:
$$\displaystyle\lim_{n\to +\infty}\frac{p_1a_n+p_2a_{n-1}+\cdots+p_na_1}{p_1+p_2+\cdots+p_n}=a.$$
\end{example}
\begin{proof}
对于给定的正整数$k>0$,我们可以证明$\displaystyle\lim_{n\to +\infty}\frac{p_{n-k}}{p_1+p_2+\cdots+p_n}=0$.
这是因为
$$0\leq \frac{p_{n-k}}{p_1+p_2+\cdots+p_n} < \frac{p_{n-k}}{p_1+p_2+\cdots+p_{n-k}}.$$
夹逼定理说明$\displaystyle\lim_{n\to +\infty}\frac{p_{n-k}}{p_1+p_2+\cdots+p_n}=0$。

下面我们就可以用极限的定义来证明命题了。对于$\forall \varepsilon > 0$,存在$N_0$,当$n>N_0$时有$$\left|a_n-a\right| < \frac{\varepsilon}{2}.$$

设$M = \max\{\left|a_0 -a\right|, \left|a_1 -a\right|, \cdots, \left|a_{N_0} -a\right|\}$.

对于每一个$k, 1\leq k \leq N_0$, 存在$N_k$, 当$n>N_k$时有,
$$\frac{p_{n-k-1}}{p_1+p_2+\cdots+p_n} \leq \frac{\varepsilon}{2MN_0}.$$

取$N=\max\{N_0, N_1, \cdots, N_{N_0}\}$, 当$n>N$时
\begin{equation*}
\begin{split}
\left|\frac{p_1a_n+p_2a_{n-1}+\cdots+p_na_1}{p_1+p_2+\cdots+p_n} - a\right| &= \left|\frac{p_1(a_n-a)+p_2(a_{n-1}-a)+\cdots+p_n(a_1-a)}{p_1+p_2+\cdots+p_n}\right|\\
&<\sum_{k=1}^{N_0} \frac{p_{n-k-1}|a_k - a|}{p_1+p_2+\cdots+p_n} + \frac{p_1 + p_2 + \cdots + p_{n-N_0}}{p_1+p_2+\cdots+p_n}\cdot\frac{\varepsilon}{2} \\
&<\sum_{k=1}^{N_0}\frac{M\varepsilon}{2MN_0} + \frac{\varepsilon}{2} = \varepsilon.
\end{split}
\end{equation*}
命题得证。
\end{proof}
%%%%%%%%%%%%%%%
\begin{example}
设$\{a_n\}$为单调增的数列,令$\sigma_n=\displaystyle\frac{a_1+a_2+\cdots+a_n}{n}$,如果$\displaystyle\lim_{n\to +\infty}\sigma_n = a$,证明:$\displaystyle\lim_{n\to +\infty}a_n = a$.若“单调增”的条件删去,结论是否成立。
\end{example}
\begin{proof}
存在$N > 0$,当$n>N$时,$\sigma_n < a+1$.于是
$$\frac{a_1}{2}+\frac{a_n}{2} < \sigma_{2n} < a + 1.$$从而$$a_n < 2\left(a+1-\frac{a_1}{2}\right).$$
单调增有上界的数列是收敛数列。即$\displaystyle\lim_{n\to +\infty}a_n$存在。
设$\displaystyle\lim_{n\to +\infty}a_n=b$, 由例1.1.15可知,
$$b=\displaystyle\lim_{n\to +\infty}\sigma_n = a.$$

如果$\{a_n\}$不是单调增的,结论不成立。例如$a_n=(-1)^n$,$\sigma_n = 0 \text{或} -\frac{1}{n}$.所以$\displaystyle\lim_{n\to +\infty}\sigma_n = 0$.但是$\{a_n\}$不收敛。
\end{proof}
%%%%%%%%%%%%%%%%%%
\begin{example}
设$\{S_n\}$为数列,$a_n=S_n-S_{n-1}$,$\sigma_n = \displaystyle\frac{S_0+S_1+\cdots+S_n}{n+1}$.如果$\displaystyle\lim_{n\to +\infty}na_n=0$且$\{\sigma_n\}$收敛,证明$\{S_n\}$也收敛,且$\displaystyle\lim_{n\to +\infty}S_n = \displaystyle\lim_{n\to +\infty}\sigma_n$.
\end{example}
\begin{proof}
我们直接计算
\begin{equation*}
\begin{split}
\displaystyle\lim_{n\to +\infty}(S_n - \sigma_n) &= \displaystyle\lim_{n\to +\infty}\frac{nS_n - S_0 -S_1-\cdots-S_{n-1}}{n+1}\\
&\xlongequal{\text{Stolz公式}}\displaystyle\lim_{n\to +\infty}(nS_n-nS_{n-1})\\
&=\displaystyle\lim_{n\to +\infty}a_n = 0.
\end{split}
\end{equation*}
于是$\displaystyle\lim_{n\to +\infty}S_n=\displaystyle\lim_{n\to +\infty}(S_n - \sigma_n) +\displaystyle\lim_{n\to +\infty}\sigma_n = \displaystyle\lim_{n\to +\infty}\sigma_n.$
\end{proof}
%%%%%%%%%%%%%%%%%%%
\begin{example}
设数列$\{x_n\}$满足:$\displaystyle\lim_{n\to +\infty}(x_n-x_{n-2})=0$.证明$\displaystyle\lim_{n\to +\infty}\frac{x_n-x_{n-1}}{n} = 0$.
\end{example}
\begin{proof}如果我们能证明$\displaystyle\lim_{n\to +\infty}\frac{\left(-1\right)^n\left(x_n-x_{n-1}\right)}{n} = 0$,则命题得证。

通过简单的计算,我们有
$$\left(-1\right)^n\left(x_n-x_{n-1}\right) = \sum_{k=1}^{n-2}(-1)^k(x_{k+2}-x_k)+x_2-x_1.$$
由于$\displaystyle\lim_{n\to +\infty}(-1)^n(x_n-x_{n-2})=0$, 可知
$$\lim_{n\to +\infty}\displaystyle\frac{\displaystyle\sum_{k=1}^{n-2}(-1)^k(x_{k+2}-x_k)}{n-2} = 0.$$
于是
$$\displaystyle\lim_{n\to +\infty}\frac{\left(-1\right)^n\left(x_n-x_{n-1}\right)}{n}=\lim_{n\to +\infty}\displaystyle\frac{\displaystyle\sum_{k=1}^{n-2}(-1)^k(x_{k+2}-x_k)}{n-2}\cdot \lim_{n\to +\infty}\frac{n-2}{n}+\lim_{n\to +\infty}\frac{x_2-x_1}{n} = 0.$$
\end{proof}
%%%%%%%%%%%%%%%%
\begin{example}
设$u_0,u_1,\cdots$为满足$u_n=\displaystyle\sum_{k=1}^{\infty}u^2_{n+k}(n=0,1,2,\cdots)$的实数列,且$\displaystyle\sum_{k=1}^{\infty}u_{n}$收敛。证明$\forall k\in\mathbb{N}$,有$u_k=0$.
\end{example}
%%%%%%%%%%%%%%%%%
\begin{example}
设$\displaystyle\lim_{n\to +\infty}a_n =a$证明$\displaystyle\lim_{n\to +\infty}\frac{1}{2^n}\displaystyle\sum_{k=0}^nC_n^ka_k =a$
\end{example}
\begin{proof}
取$t_{nk} = \frac{1}{2^n}C_n^k$, 我们有
\renewcommand\labelenumi{\normalfont(\theenumi)}
\begin{enumerate}
\item $t_{nk} > 0$且 $\displaystyle\sum_{k=0}^nt_{nk} = \frac{1}{2^n}(1+1)^n = 1$;
\item 很容易验证$\displaystyle\lim_{n\to +\infty}t_{nk} = 0$. 因为$2^n=(1+1)^n > C_n^{k+1}$, 从而
$$0\leq t_{nk} \leq \frac{C_n^k}{C_n^{k+1}} = \frac{k+1}{n-k}.$$
由夹逼定理,$\displaystyle\lim_{n\to +\infty}t_{nk} = 0$。
\end{enumerate}
由Toeplitz定理知,$\displaystyle\lim_{n\to +\infty}\frac{1}{2^n}\displaystyle\sum_{k=0}^nC_n^ka_k =a$。
\end{proof}
%%%%%%%%%%%%%%%%%%%%%
\begin{example}
给定实数$a_0,a_1$,并令$$a_n=\frac{a_{n-1}+a_{n-2}}{2}, \quad n = 2,3,\cdots.$$
证明:数列$\{a_n\}$收敛,且$\displaystyle\lim_{n\to +\infty}a_n =\frac{a_0+2a_1}{3}$
\end{example}
\begin{proof}
由递归公式有:
$$a_n -a_{n-1} = \left(-\frac{1}{2}\right)\left(a_{n-1}-a_{n-2}\right)=\cdots= \left(-\frac{1}{2}\right)^{n-1}(a_1-a_0).$$
于是:
\begin{equation*}
\begin{split}
a_n-a_0 &= (a_n-a_{n-1}) + (a_{n-1}-a_{n-2}) +\cdots+(a_1-a_0)\\
&=\left(-\frac{1}{2}\right)^{n-1}(a_1-a_0) + \left(-\frac{1}{2}\right)^{n-2}(a_1-a_0) + \cdots + (a_1-a_0)\\
&=\frac{1-\left(-\frac{1}{2}\right)^{n}}{1+\frac{1}{2}}(a_1-a_0)
\end{split}
\end{equation*}
从而$\displaystyle\lim_{n\to +\infty}a_n =a_0+\displaystyle\lim_{n\to +\infty}\left(\frac{1-\left(-\frac{1}{2}\right)^{n}}{1+\frac{1}{2}}(a_1-a_0)\right)=a_0+\frac{2}{3}(a_1-a_0) = \frac{a_0+2a_1}{3}$。
\end{proof}
%%%%%%%%%%%%%%%%%%
\begin{example}
设$x_1,x_2,\cdots, x_n$为任意给定的实数。令
$$x_i^{(1)}=\frac{x_i+x_{i+1}}{2}, \quad i=1,2,\cdots, n,$$
其中$x_{n+1}$应理解为$x_1$.归纳定义
$$x_i^{(k)}=\frac{x_i^{(k-1)}+x_{i+1}^{(k-1)}}{2}, \quad i=1,2,\cdots, n,$$
$x^{(k-1)}_{n+1}$应理解为$x_1^{(k-1)},k=2,3,\cdots$.证明:
$$\displaystyle\lim_{k\to +\infty}x_i^{(k)}=\frac{x_1+x_2+\cdots+x_n}{n}, \quad \forall i = 1,2,\cdots,n.$$
\end{example}
%%%%%%%%%%%%%%%%%%
\begin{example}
设$\{a_n\}$为一个数列,且$\displaystyle\lim_{n\to +\infty}(a_{n+1}-a_n) = l$。证明:$$\displaystyle\lim_{n\to +\infty}\frac{a_n}{n} = l; \quad \displaystyle\lim_{n\to +\infty}\frac{\displaystyle\sum_{k=1}^na_k}{n^2} =\frac{l}{2}.$$
\end{example}
\begin{proof}
$$a_n = (a_n - a_{n-1}) + (a_{n-1}-a_{n-2})+\cdots + (a_2 - a_1) + a_1.$$
由于$\displaystyle\lim_{n\to +\infty}(a_{n+1}-a_n) = l$,易知
$$\displaystyle\lim_{n\to +\infty}\frac{a_n}{n} = l.$$

现在计算
\begin{equation*}
\begin{split}
\displaystyle\sum_{k=1}^na_k - na_1 &= \displaystyle\sum_{k=2}^{n}(a_k - a_1)\\
&=\displaystyle\sum_{k=2}^{n}\displaystyle\sum_{l=1}^{k-1}(a_{l+1}-a_l)\\
&=\displaystyle\sum_{l=1}^{n-1}(n-l)(a_{l+1}-a_l)
\end{split}
\end{equation*}
如果取$t_{nk}=\displaystyle\frac{2(n-k)}{(n-1)n}$,则
\renewcommand\labelenumi{\normalfont(\theenumi)}
\begin{enumerate}
\item $t_{nk} \geq 0, \forall n\in\mathbb{N}, k=1,2,\cdots, n$且$\displaystyle\sum_{k=1}^{n}t_{nk} = \displaystyle\sum_{k=1}^{n}\frac{2(n-k)}{(n-1)n}=1$;
\item $\displaystyle\lim_{n\to +\infty}t_{nk} = 0, \quad\forall k$.
\end{enumerate}
由Toeplitz定理可知
$$\displaystyle\lim_{n\to +\infty}\frac{\displaystyle\sum_{k=1}^na_k - na_1}{(n-1)n}=\displaystyle\lim_{n\to +\infty}\displaystyle\sum_{k=1}^{n-1}\frac{2(n-k)}{(n-1)n}\frac{a_{k+1}-a_k}{2}=\displaystyle\lim_{n\to +\infty}\displaystyle\sum_{k=1}^{n-1}t_{nk}\frac{a_{k+1}-a_k}{2}=\frac{l}{2}.$$
从而
$$\displaystyle\lim_{n\to +\infty}\frac{\displaystyle\sum_{k=1}^na_k}{n^2}=\displaystyle\lim_{n\to +\infty}\frac{\displaystyle\sum_{k=1}^na_k - na_1}{(n-1)n}\cdot\displaystyle\lim_{n\to +\infty}\frac{(n-1)n}{n^2}+\displaystyle\lim_{n\to +\infty}\frac{a_1}{n}=\frac{l}{2}.$$
命题得证。
\end{proof}
%%%%%%%%%%%%%%%%%%%
\begin{example}
设$x_1\in[0,1], \forall n\geq 2$,令
\begin{equation*}
x_n =
\begin{cases}
\frac{1}{2}x_{n-1}, \quad \text{n为偶数,}\\
\frac{1+x_{n-1}}{2},\quad\text{n为奇数.}
\end{cases}
\end{equation*}
证明:$\displaystyle\lim_{n\to +\infty}x_{2k}=\frac{1}{3};\displaystyle\lim_{n\to +\infty}x_{2k+1}=\frac{2}{3}$.
\end{example}
\begin{proof}
由递推公式可知
\begin{equation*}
\begin{split}
x_{2k} &= \frac{1}{2}\left(\frac{1+x_{2(k-1)}}{2}\right)\\&=\frac{1}{4}+\frac{1}{4}x_{2(k-1)}\\&= \displaystyle\sum_{l=1}^{k-1}\left(\frac{1}{4}\right)^l+\left(\frac{1}{4}\right)^{k-1}x_2 \\&= \frac{1}{3}-\frac{4}{3}\left(\frac{1}{4}\right)^k+\left(\frac{1}{4}\right)^{k-1}x_2.
\end{split}
\end{equation*}
因此$\displaystyle\lim_{n\to +\infty}x_{2k}=\frac{1}{3}$.

$$x_{2k+1}=\frac{1}{2}+\frac{1}{4}x_{2(k-1)+1}
=\frac{1}{2}\displaystyle\sum_{l=0}^{k-1}\left(\frac{1}{4}\right)^{l} + \left(\frac{1}{4}\right)^{k}x_1
=\frac{2}{3}-\frac{2}{3}\left(\frac{1}{4}\right)^{k}+\left(\frac{1}{4}\right)^{k}x_1.$$
因此$\displaystyle\lim_{n\to +\infty}x_{2k+1}=\frac{2}{3}$.
\end{proof}
%%%%%%%%%%%%%%%%%
\begin{example}
定初始值$a_0$,并递推定义$$a_n=2^{n-1}-3a_{n-1},\quad n=1,2,\cdots.$$
求$a_0$的所有可能的值,使得数列$\{a_n\}$是严格增的。
\end{example}
\begin{solution}
考虑数列$\left\{\displaystyle\frac{a_{n}}{3^{n}}\right\}$.显然
$$\frac{a_n}{3^n} = \frac{2^{n-1}}{3^n}-\frac{a_{n-1}}{3^{n-1}}$$
\end{solution}
%%%%%%%%%%%%%%
\begin{example}
设$c>0$,$a_1=\frac{c}{2}$,$a_{n+1} = \frac{c}{2}+\frac{a_n^2}{2}$,$n=1,2,\cdots$.证明:
\begin{equation*}
\displaystyle\lim_{n\to +\infty}a_n=
\begin{cases}
1-\sqrt{1-c}, \quad &0<c\leq 1,\\
+\infty,\quad &c >1.
\end{cases}
\end{equation*}
试问:当时$-3\leq c < 0$,数列$\{a_n\}$的收敛性如何?
\end{example}
\begin{proof}
现证$c>1$的情形:
$$a_{n} \geq \sqrt{ca_{n-1}^2} = \sqrt{c}a_{n-1} \geq \cdots \geq \left(\sqrt{c}\right)^{n-1}a_1.$$
由于$\displaystyle\lim_{n\to +\infty}\left(\sqrt{c}\right)^{n-1} = +\infty$, $\displaystyle\lim_{n\to +\infty}a_{n} = +\infty$.命题得证。

现在$c\leq 1$的情形:
\renewcommand\labelenumi{\normalfont(\theenumi)}
\begin{enumerate}
\item 考虑函数$f(x) = x^2-2x+c$。 当$x\in (1-\sqrt{1-c}, 1+\sqrt{1-c})$, $f(x) < 0$且$\displaystyle\min_{x\in \mathbb{R}}f(x) = -1+c$.
\item 很显然 $a_1=\frac{c}{2} \in (1-\sqrt{1-c}, 1+\sqrt{1-c})$.于是$a_2 - a_1 = \frac{1}{2}f(a_1) < 0 \Rightarrow a_2 < a_1$.
$$a^2_1\in(2-c-2\sqrt{1-c}, 2-c+2\sqrt{1-c})\Rightarrow a_2 = \frac{c}{2} + \frac{a_1^2}{2}\in (1-\sqrt{1-c}, 1+\sqrt{1-c}).$$
\item 假设$n=k$时, $a_k \leq a_{k-1}$ 且$a_k\in (1-\sqrt{1-c}, 1+\sqrt{1-c})$. 显然我们有$a_{k+1} \leq a_k, a_{k+1}\in (1-\sqrt{1-c}, 1+\sqrt{1-c})$.
\end{enumerate}
综上,
$$a_{n+1}\leq a_{n}, \quad a_n\geq 1-\sqrt{1-c}, \quad\forall n.$$
于是$\displaystyle\lim_{n\to +\infty}a_{n}=1-\sqrt{1-c}$.

现考虑$-3\leq c < 0$的情形。

\end{proof}
%%%%%%%%%%%%%%%%%%
\begin{example}
数列$\{u_n\}$定义如下:$u_1=b,u_{n+1}=u_n^2+(1-2a)u_n+a^2$,$n\in\mathbb{N}$.问: $a,b$为何值时$\{u_n\}$收敛,并求出其极限值。
\end{example}
%%%%%%%%%%%%%%%%
\begin{example}
设$A > 0, 0< y_0 < A^{-1}, y_{n+1} = y_n(2-Ay_n), n\in\mathbb{N}$. 证明:$\displaystyle\lim_{n\to +\infty}y_{n}=A^{-1}$.
\end{example}
\begin{proof}如果我们能证明$0< y_n < A^{-1}, \forall n\in\mathbb{N}$,则
$$\frac{y_{n+1}}{y_n} = 2-Ay_n\geq 2-AA^{-1}=1\Rightarrow y_{n+1} \geq y_n.$$
从而$\{y_n\}$是单调递增有上界的数列。故收敛,且$\displaystyle\lim_{n\to +\infty}y_{n}=A^{-1}$.

下面我们就证明$y_n < A^{-1},\forall n\in\mathbb{N}$. 考虑函数$f(x)=x(2-Ax)$.显然该函函数在$x=A^{-1}$是取得最大值。即$f(x) < A^{-1},\forall x\in \mathbb{R}$.由于$0< y_1 < A^{-1}$,由归纳法, $0< y_n < A^{-1}, \forall n\in\mathbb{N}$.
\end{proof}
%%%%%%%%%%%%%%%%%%
\begin{example}
设数列$\{a_n\}$满足$(2-a_n)a_{n+1} = 1$。证明:$\displaystyle\lim_{n\to +\infty}a_{n}=1$。
\end{example}
\begin{proof}
由于$(2-a_n)a_{n+1} = 1$,则$a_n \neq 2$,从而$a_1\neq \frac{3}{2}$.
我们现证无论$a_1$取何值,都存在$N$使得$a_N \leq 1$.
\renewcommand\labelenumi{\normalfont(\theenumi)}
\begin{enumerate}
\item 如果$a_1 \leq 1$, 取$N=1$, $a_N = a_1 \leq 1$
\item 如果$a_1 > \frac{3}{2}$, 则$a_2 > 2$, $a_3 < 0$. 于是取$N=3$, $a_N = a_3 < 0 \leq 1$.
\item 如果$1< a_1 < \frac{3}{2}$.记$a_1 = 1 + h$, 则$$a_k = 1 + \frac{h}{1-(k-1)h}, \quad k = 2, 3\cdots.$$取$k=\left[\frac{1}{h}\right]$, 我们有
$$1-(k-1)h \geq h > 0, \quad \frac{h}{1-(k-1)h}\geq\frac{1}{2}.$$
从而取$N = k+2 = \left[\frac{1}{h}\right]$,我们有$a_{N-2}=a_k\geq \frac{3}{2}, a_{N}< 0\leq 1$.
\end{enumerate}
综上,我们不妨假设$a_1 \leq 1$. 下面我们可以用数学归纳法证明$\{a_n\}$是单调增的,且$a_n \leq 1,\forall n$.
由于$$a_{n+1}- a_n = \frac{(a_n - 1)^2}{2-a_n},$$
我们可知当$a_n < 1$时,$a_{n+1} > a_n$. 又因为$2-a_{n} > 1\rightarrow a_{n+1} \leq 1$. 到此我们证明了$\displaystyle\lim_{n\to +\infty}a_{n}$存在。设$\displaystyle\lim_{n\to +\infty}a_{n}=a$,则$(2-a)a=1$.解方程知$a=1$.
\end{proof}
%%%%%%%%%%%%%%%%%%%
\begin{example}
设数列$\{a_n\}$满足不等式$0\leq a_k\leq 100a_n(n\leq k\leq 2n, n=1,2,\cdots)$,且无穷级数$\displaystyle\sum_{n=1}^{\infty}a_n$收敛。证明$\displaystyle\lim_{n\to +\infty}na_n=0$
\end{example}
%%%%%%%%%%%%%
\begin{example}
证明:$\displaystyle\lim_{n\to +\infty}\left(1+\frac{1}{n^2}\right)\left(1+\frac{2}{n^2}\right)\cdots\left(1+\frac{n}{n^2}\right)=e^{\frac{1}{2}}$
\end{example}
\begin{proof}
通过简单计算,我们可以得知
$$\left(1+\frac{n}{2n^2}\right)^2 \leq \left(1+\frac{k}{n^2}\right)\left(1+\frac{n-k+1}{n^2}\right), \quad\forall k \leq \frac{n}{2}, k\in\mathbb{N}.$$
由此可见
$$\left(1+\frac{n}{2n^2}\right)^n < \left(1+\frac{1}{n^2}\right)\left(1+\frac{2}{n^2}\right)\cdots\left(1+\frac{n}{n^2}\right).$$

另一方面
$$\left(1+\frac{1}{n^2}\right)\left(1+\frac{2}{n^2}\right)\cdots\left(1+\frac{n}{n^2}\right) \leq \left(\frac{\displaystyle\sum_{k=1}^{n}\left(1+\frac{k}{n^2}\right)}{n}\right)^n=\left(1+\frac{(n+1)}{2n^2}\right)^n.$$

很显然
$$\displaystyle\lim_{n\to +\infty}\left(1+\frac{n}{2n^2}\right)^n=e^{\frac{1}{2}},$$
$$\displaystyle\lim_{n\to +\infty}\left(1+\frac{(n+1)}{2n^2}\right)^n=e^{\frac{1}{2}}.$$
由夹逼定理,命题得证。
\end{proof}
%%%%%%%%%%%%%%%
\begin{example}
设$a_1>b_1>0$,令$$a_n=\displaystyle\frac{a_{n-1}+b_{n-1}}{2},\quad b_n=\frac{2a_{n-1}b_{n-1}}{a_{n-1}+b_{n-1}}, \quad n = 2,3,\cdots.$$ 证明:数列$\{a_n\}$与$\{b_n\}$都收敛,且$\displaystyle\lim_{n\to +\infty}a_n=\displaystyle\lim_{n\to +\infty}b_n=\sqrt{a_1b_1}$.
\end{example}
\begin{proof}很显然
$$b_n \leq \sqrt{a_{n-1}b_{n-1}}\leq a_n,\quad\forall n = 2,3\cdots.$$
于是
$$a_n =\displaystyle\frac{a_{n-1}+b_{n-1}}{2} \leq \displaystyle\frac{a_{n-1}+a_{n-1}}{2} = a_{n-1},$$
$$b_n -b_{n-1} = \frac{(a_{n-1}-b_{n-1})\cdot b_{n-1}}{a_{n-1}+b_{n-1}} \geq 0\Rightarrow b_n \geq b_{n-1}.$$
所以
$$b_1 \leq b_2\leq \cdots \leq b_n \leq \cdots \leq a_n \leq \cdots \leq a_2\leq a_1.$$
由实数连续性命题(二)可知,$\displaystyle\lim_{n\to +\infty}a_n$和$\displaystyle\lim_{n\to +\infty}b_n$都存在。 

设$\displaystyle\lim_{n\to +\infty}a_n=a$,$\displaystyle\lim_{n\to +\infty}b_n=b$,

$$a_nb_n=a_{n-1}b_{n-1}=\cdots = a_1b_1\Rightarrow ab = a_1b_1.$$
$$a_n=\displaystyle\frac{a_{n-1}+b_{n-1}}{2}\Rightarrow a=\frac{a+b}{2}\Rightarrow a=b.$$
于是$a=b=\sqrt{a_1b_1}$.
\end{proof}
%%%%%%%%%%%%%%%%%
\begin{example}
当$n\geq 3$时,证明:
$$\displaystyle\sum_{k=0}^n\frac{1}{k!}-\frac{3}{2n}<\left(1+\frac{1}{n}\right)^n < \displaystyle\sum_{k=0}^n\frac{1}{k!}.$$
\end{example}
%%%%%%%%%%%%%%%%
\begin{example}
设$a_1=1, a_n=n(a_{n-1}+1), n=2,3,\cdots$,且
\[
x_n = \prod_{k=1}^n\left(1+\frac{1}{a_k}\right).
\]求$\displaystyle\lim_{n\to +\infty}x_n$(其中$\displaystyle\prod_{k=1}^n$表示从$k=1$到$k=n$的连乘积).
\end{example}
\begin{solution}

\end{solution}
%%%%%%%%%%%%%%%%%%
\begin{example}
设$H_n=1+\frac{1}{2}+\cdots+\frac{1}{n}$,$n\in\mathbb{N}$,用$K_n$表示使得$H_k\geq n$的最小下标,求$\displaystyle\lim_{n\to +\infty}\frac{K_{n+1}}{K_n}$
\end{example}
\begin{solution}
\end{solution}
%%%%%%%%%%%%%%%%%%
\begin{example}
设$y_0\geq 2, y_n = y_{n-1}^2 - 2(n\in\mathbb{N})$,
\[
S_n=\frac{1}{y_0}+\frac{1}{y_0y_1} +\cdots + \frac{1}{y_0y_1\cdots y_n}.
\]证明:$\displaystyle\lim_{n\to +\infty}S_n=\frac{y_0-\sqrt{y_0^2-4}}{2}$.
\end{example}
\begin{proof}
首先$$y_n - y_{n-1} = y_{n-1}^2 - y_{n-1} - 2 = (y_{n-1}-2)(y_{n-1} + 1).$$
由于$y_0\geq 2$, 由数学归纳法, 我们可知
$$y_n \geq y_{n-1} \geq 2, \forall n \in\mathbb{N}.$$
由此可见
$$S_n \geq S_{n-1}, \forall n\in\mathbb{N}.$$
另一方面,由于$$\frac{1}{y_0y_1\cdots y_n} < \frac{1}{2^{n+1}},$$
可知$$S_n \leq 1 -\frac{1}{2^{n+1}}, \forall n = 0, 1, \cdots.$$
由此可见,$\displaystyle\lim_{n\to +\infty}S_n$ 存在。

假设$\displaystyle\lim_{n\to +\infty}S_n = a$.
\end{proof}
%%%%%%%%%%%%%%%%%%%%%%%
\begin{example}
令数列$\{b_n\}$满足
$$b_n=\displaystyle\sum_{k=0}^n\frac{1}{C^k_n}, \quad n =1,2,\cdots.$$
证明:(1)当$n\geq 2$时,$b_n=\displaystyle\frac{n+1}{2n}b_{n-1} + 1;$\\
(2)$\displaystyle\lim_{n\to +\infty}b_n = 2.$
\end{example}
%%%%%%%%%%%%%%%%%%%%%%%%%%
%\subsubsection{三级节标题}

%\begin{multicols}{2}

%\end{multicols}


%\part{植物多样性分区概述}



\include{angiosperms}

\appendix

%\chapter{}

\renewcommand\indexname{索~~引}
\printindex
\addcontentsline{toc}{chapter}{索~引}

\backmatter

\addcontentsline{toc}{chapter}{参考文献}

\begin{thebibliography}{参考文献}
\bibitem[徐薛]{XX1} 徐森林,薛春华编著 《数学分析》, 清华大学出版社, 2005.
\end{thebibliography}

\chapter{后~~记}

\begin{flushright}

\end{flushright}

\end{document}
