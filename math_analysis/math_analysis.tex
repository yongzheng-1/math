\documentclass[utf8]{book}
\usepackage{titletoc}
\usepackage{titlesec}
\usepackage{ctexcap}
\usepackage[a4paper,text={125mm,195mm},centering,left=1in,right=1in,top=1in,bottom=1in]{geometry}
\usepackage[]{geometry}
\usepackage{imakeidx}
\usepackage{hyperref}
\usepackage{amsthm}
\usepackage{amsmath,amssymb}

%%%% 定理类环境的定义 %%%%
\newtheorem{example}{}[section]             % 整体编号
\newtheorem{algorithm}{算法}
\newtheorem{theorem}{定理}[section]  % 按 section 编号
\newtheorem{definition}{定义}
\newtheorem{axiom}{公理}
\newtheorem{property}{性质}
\newtheorem{proposition}{命题}
\newtheorem{lemma}{引理}
\newtheorem{corollary}{推论}
\newtheorem{remark}{注解}
\newtheorem{condition}{条件}
\newtheorem{conclusion}{结论}
\newtheorem{assumption}{假设}
\newtheorem{solution}{}
\renewcommand*{\proofname}{\normalfont\bfseries 证明}
\renewcommand*{\thesolution}{\normalfont\bfseries 解}
\renewcommand{\theexample}{\arabic{example}}

\makeindex
\bibliographystyle{plain}
\begin{document}
\title{\heiti 徐森林,薛春华~编 \\ 《数学分析》题解}
\author{\fangsong 西海岸民工}
\date{2024年11月}

\frontmatter
\maketitle


\renewcommand\contentsname{目~录}
\tableofcontents

\mainmatter

%\part{总论}

\chapter{数列极限}

\section{数列极限的概念}
\subsection{练习题}
\begin{example}用数列极限定义证明:
\begingroup
\renewcommand\labelenumi{\normalfont(\theenumi)}
\begin{enumerate}
        \item $\displaystyle \lim_{n\to +\infty} 0.\underbrace{99\cdots9}_{n} = 1;$
        \begin{proof}
        对于 $\forall \varepsilon > 0$, 取 $N = \left [ \frac{-\ln\varepsilon}{\ln 10}\right ] + 1$, 当 $n > N$时,
        $$\left | 0.\underbrace{99\cdots9}_{n} - 1\right | = \frac{1}{10^n} < \varepsilon.$$
        所以, $$\lim_{n\to +\infty} 0.\underbrace{99\cdots9}_{n} = 1.$$
        \end{proof}
       
        \item $\displaystyle \lim_{n\to +\infty}\frac{3n+4}{7n-3} = \frac{3}{7};$
        \begin{proof}
        对于 $\forall \varepsilon > 0$, 取 $N = \left [ \frac{6}{\varepsilon}\right ] + 1$, 当 $n > N$时,
        $$\left | \frac{3n+4}{7n-3} - \frac{3}{7}\right | = \frac{37}{7(7n-3)} < \frac{37}{7n} < \frac{6}{n} < \varepsilon$$
        所以, $$\lim_{n\to +\infty}\frac{3n+4}{7n-3} = \frac{3}{7}.$$
        \end{proof}
        
        \item $\displaystyle \lim_{n\to +\infty}\frac{5n+6}{n^2-n-1000} = 0;$
        \begin{proof}
        对于 $\forall \varepsilon > 0$, 取 $N = \max\{50, \left [ \frac{12}{\varepsilon}\right ] + 1\}$, 当 $n > N$时,
        $$\frac{1}{2}n^2 - n - 1000 > 1$$ 且
        $$\left | \frac{5n+6}{n^2-n-1000} - 0\right | < \frac{5n + 6}{\frac{1}{2}n^2 + (\frac{1}{2}n^2 - n - 1000)} < \frac{6n}{\frac{1}{2}n^2} < \frac{12}{n} < \varepsilon$$
        所以, $$\lim_{n\to +\infty}\frac{5n+6}{n^2-n-1000} = 0.$$  
        \end{proof}
        
        \item $\displaystyle \lim_{n\to +\infty}\frac{8}{2^n+5} = 0;$
        \begin{proof}
         对于 $\forall \varepsilon > 0$, 取 $N = \left [ \frac{-\ln\varepsilon}{\ln2}\right ] + 4$, 当 $n > N$时,
        $$\left | \frac{8}{2^n+5} - 0\right | < \frac{8}{2^n} = \frac{1}{2^{n-3}} < \varepsilon$$
        所以, $$\lim_{n\to +\infty}\frac{8}{2^n+5} = 0.$$  
        \end{proof}
        
        \item $\displaystyle \lim_{n\to +\infty}\frac{\sin n!}{n^{1/2}} = 0;$
        \begin{proof}
         对于 $\forall \varepsilon > 0$, 取 $N = \left [ \left(\frac{1}{\varepsilon}\right)^2\right ] + 1$, 当 $n > N$时,
        $$\left | \frac{\sin n!}{n^{1/2}} - 0\right | < \frac{1}{n^{1/2}} < \varepsilon$$
        所以, $$\lim_{n\to +\infty}\frac{\sin n!}{n^{1/2}} = 0.$$  
        \end{proof}
        
        \item $\displaystyle \lim_{n\to +\infty}(\sqrt{n+2}-\sqrt{n-2}) = 0;$
        \begin{proof}
         对于 $\forall \varepsilon > 0$, 取 $N = \max\{2, \left [ \left(\frac{4}{\varepsilon}\right)^2\right ] + 1\}$, 当 $n > N$时,
        $$\left | \sqrt{n+2}-\sqrt{n-2}\right | = \frac{4}{\sqrt{n+2}+\sqrt{n-2}} < \frac{4}{\sqrt{n}} < \varepsilon$$
        所以, $$\lim_{n\to +\infty}(\sqrt{n+2}-\sqrt{n-2}) = 0.$$  
        \end{proof}
        
        \item $\displaystyle \lim_{n\to +\infty}(\sqrt[3]{n+2}-\sqrt[3]{n-2}) = 0;$
        \begin{proof}
         对于 $\forall \varepsilon > 0$, 取 $N = \max\{2, \left [ \sqrt{\left(\frac{4}{\varepsilon}\right)^3}\right ]\}$, 当 $n > N$时,
         \begin{equation*}
         \begin{split}
		\left | \sqrt[3]{n+2}-\sqrt[3]{n-2}\right | &= \frac{4}{\sqrt[3]{(n+2)^2}+\sqrt[3]{(n+2)(n-2)} + \sqrt[3]{(n-2)^2}} \\
 													 &< \frac{4}{\sqrt[3]{(n+2)^2}} \\
													 &< \varepsilon
		\end{split}
		\end{equation*}
        所以, $$\lim_{n\to +\infty}(\sqrt[3]{n+2}-\sqrt[3]{n-2}) = 0.$$  
        \end{proof}
        
        \item $\displaystyle \lim_{n\to +\infty}\frac{n^{3/2}\arctan{n}}{1+n^2} = 0;$
        \begin{proof}
         对于 $\forall \varepsilon > 0$, 取 $N = \left [ \left(\frac{\pi}{2\varepsilon}\right)^2\right ] + 1$, 当 $n > N$时,
         $$\left | \frac{n^{3/2}\arctan{n}}{1+n^2} \right | < \frac{\frac{\pi}{2}n^{3/2}}{n^2} < \frac{\pi / 2}{\sqrt{n}} < \varepsilon$$
        所以, $$\lim_{n\to +\infty}\frac{n^{3/2}\arctan{n}}{1+n^2} = 0.$$  
        \end{proof} 
        
        \item $\displaystyle \lim_{n\to +\infty}a_n = 1,$ 其中 
        $a_n = 
         \begin{cases}
         \displaystyle\frac{n-1}{n}, \text{n 为偶数,}\\
         \displaystyle\frac{\sqrt{n^2+ n}}{n}, \text{n 为奇数;}
		\end{cases}$      
		
		\begin{proof}
         对于 $\forall \varepsilon > 0$, 取 $N = \left [ \frac{1}{\varepsilon}\right ] + 1$, 当 $n > N$时,
         \begin{equation*}
         \begin{aligned}
         \left | a_n - 1\right | &<
         \begin{cases}
         \displaystyle\frac{1}{n}, \text{n 为偶数,}\\
         \displaystyle\frac{1}{\sqrt{n^2 + n} + n} , \text{n 为奇数}
		\end{cases}\\
		&< \frac{1}{n} \\
		&< \varepsilon
		\end{aligned}
		\end{equation*}
        所以, $$\lim_{n\to +\infty}a_n = 1.$$  
        \end{proof} 
        
        \item $\displaystyle \lim_{n\to +\infty}(n^3 - 4n - 5) = +\infty.$		
		\begin{proof}
         对于 $\forall A > 0$, 取 $N = \max\{5, \left [ \sqrt[3]{2A}\right ] + 1\}$, 当 $n > N$时,
         $$1 - \frac{4}{n^2} - \frac{5}{n^3} > 1 - \frac{9}{n^2} > \frac{1}{2}$$且
         $$n^3 - 4n - 5 = n^3(1 - \frac{4}{n^2} - \frac{5}{n^3}) > \frac{1}{2}n^3 > A$$
        所以, $$\lim_{n\to +\infty}(n^3 - 4n - 5) = +\infty.$$  
        \end{proof}  
\end{enumerate}
\endgroup
\end{example}
%%%%%%%%%%%%%%%%%%%%%%%%%%%%%%
\begin{example}
设$\displaystyle \lim_{n\to +\infty}a_n = a.$,证明: $\forall k \in \mathbb{N}$,有 $\displaystyle \lim_{n\to +\infty}a_{n+k} = a.$
\begin{proof}
我们分以下几种情况证明此命题:
\renewcommand\labelenumi{\normalfont(\theenumi)}
\begin{enumerate}
\item 当 $a\in \mathbb{R}$时。由于$\displaystyle \lim_{n\to +\infty}a_n = a$,对$\forall \varepsilon > 0, \exists N \in \mathbb{N}$,当$n > N$时,有$\left | a_n - a\right | < \varepsilon$。显然 $n+k > n > N$, 从而 $\left | a_{n + k} - a\right | < \varepsilon$. 即 $\displaystyle \lim_{n\to +\infty}a_{n+k} = a$.
\item 当 $a$ 是 $+\infty$时。由于$\displaystyle \lim_{n\to +\infty}a_n = +\infty$, 对$\forall A > 0, \exists N \in \mathbb{N}$, 当$n > N$时, $a_n > A$. 显然 $n+k > n > N$, 从而 $a_{n + k} > A$. 即 $\displaystyle \lim_{n\to +\infty}a_{n+k} = +\infty$.
\item 当 $a$ 是 $-\infty$时。由于$\displaystyle \lim_{n\to +\infty}a_n = -\infty$, 对$\forall A < 0, \exists N \in \mathbb{N}$, 当$n > N$时, $a_n < A$. 显然 $n+k > n > N$, 从而 $a_{n + k} < A$. 即 $\displaystyle \lim_{n\to +\infty}a_{n+k} = -\infty$.
\end{enumerate}
\end{proof}
\end{example}
%%%%%%%%%%%%%%%%%%%%%%%%%%%%%%

\begin{example}
设$\displaystyle \lim_{n\to +\infty}a_n = a$,证明$\displaystyle \lim_{n\to +\infty}\left |a_n\right | = \left |a\right |$:举例说明,这个命题的逆命题不真。
\begin{proof}
我们只证明$a$是有限实数的情况。当$a$是$+\infty$和$-\infty$时也成立。
由极限的定义有:对$\forall \varepsilon > 0, \exists N \in \mathbb{N}$,当$n > N$时,$$\left|a_n - a\right| < \varepsilon.$$
从而
$$\left| \left|a_n \right| - \left|a\right| \right | < \left|a_n - a\right| < \varepsilon.$$
所以 $$\lim_{n\to +\infty}\left |a_n\right | = \left |a\right |.$$

如果我们取$a_n = (-1)^n$, 则$\left | a_n\right | = 1$, 从而 $\displaystyle \lim_{n\to +\infty}\left |a_n\right | = \left |a\right |$。 但是很显然$a_n$是发散的。

\end{proof}

\end{example}
%%%%%%%%%%%%%%%%%%%%%%%%%%%%%%

\begin{example}

设$x_n \leq a \leq y_n, n \in \mathbb{N}$,且$\displaystyle \lim_{n\to +\infty}(y_n-x_n) = 0$。证明:
$$\displaystyle \lim_{n\to +\infty}x_n = \lim_{n\to +\infty}y_n = a$$

\begin{proof}
$\forall \varepsilon > 0, \exists N \in \mathbb{N}$, s.t. $y_n-x_n = \left | y_n - x_n\right | < \varepsilon$. 从而
$$\left| y_n - a\right| = y_n - a = y_n - x_n + x_n - a < y_n - x_n < \varepsilon.$$
即 $\displaystyle \lim_{n\to +\infty}y_n = a.$

同理可证
$$-\varepsilon < x_n - y_n < x_n - a < 0 < \varepsilon.$$ 即$\displaystyle \lim_{n\to +\infty}x_n = a.$

\end{proof}
\end{example}
%%%%%%%%%%%%%%%%%%%%%%%%%%%%%%
\begin{example}
设$\{a_n\}$为一个收敛数列。证明:数列$\{a_n\}$中或者有最大的数,或者有最小的数。 举出两者都有的例子; 再举出只有一个的例子。
\end{example}
\begin{proof}
假设$\displaystyle \lim_{n\to +\infty}a_n = a$,我们分以下几种情况讨论:
\renewcommand\labelenumi{\normalfont(\theenumi)}
\begin{enumerate}
\item 如果$a_n = a,\forall n\in \mathbb{N}$. 此时,数列$\{a_n\}$既有最小值也有最大值,且相等
\item 如果$\exists n_{0} \in \mathbb{N}$,使得$a_{n_{0}} \neq a$. 不妨假设$a_{n_{0}} < a$. 对于$\varepsilon = \frac{a - a_{n_{0}}}{2}$, $\exists N\in \mathbb {N}, 
N > n_{0}$使得
$a_n - a > a - \varepsilon = \frac{a + a_{n_{0}}}{2} > a_{n_{0}}, \forall n > N$. 取
$$m = \min\{a_{1}, a_{2}, \cdots, a_{N}\}.$$
我们有:
\renewcommand\labelenumi{\normalfont(\theenumi)}
\begin{enumerate}
\item $m \in \{a_n\}_{n = 1}^{+\infty},$
\item $a_n >= m, \forall n.$
\end{enumerate}
即,$m$ 是数列$\{a_n\}$的最小值. \\
如果$a_{n_{0}} > a$. 我们可以证明$\{a_n\}$有最大值。
\end{enumerate}

考虑下列收敛数列:
\renewcommand\labelenumi{\normalfont(\theenumi)}
\begin{enumerate}
\item 如果$a_n = \frac{1}{n}$, 则该数列有最大值$a_n \leq a_1 = 1$, 没有最小值。
\item 如果$a_n = -\frac{1}{n}$, 该数列有最小值$-1 = a_1 <= a_n$, 没有最大值。
\item 如果$a_n =(-1)^n \frac{1}{n}$, 则$-1 = a_1 <= a_n <= a_2 = \frac{1}{2}$ 
\end{enumerate}

\end{proof}
%%%%%%%%%%%%%%%%%%%%%%%%%%%%%%
\begin{example}
证明下列数列发散:
\renewcommand\labelenumi{\normalfont(\theenumi)}
\begin{enumerate}
\item $\{n^{(-1)^n}\}$
\begin{proof}
该数列发散,因为:
$$0 = \displaystyle \lim_{n\to +\infty}(2n-1)^{(-1)^{2n-1}} \neq \displaystyle \lim_{n\to +\infty}(2n)^{(-1)^{2n}} = +\infty$$
\end{proof}
\item $\{\cos n\}$
\begin{proof}
取两个整数子列$\{k_n\}, \{l_n\}$使得
\renewcommand\labelenumi{\normalfont(\theenumi)}
\begin{enumerate}
\item $k_n \in (2m\pi - \frac{\pi}{6},2m\pi + \frac{\pi}{6})$,
\item $l_n \in (2m\pi + \frac{5\pi}{6}, (2m+1)\pi + \frac{\pi}{6})$.
\end{enumerate}
显然,我们有
\renewcommand\labelenumi{\normalfont(\theenumi)}
\begin{enumerate}
\item $\cos k_n \in (\frac{\sqrt 3}{2}, 1], \forall n$,
\item $\cos l_n \in [-1, -\frac{\sqrt 3}{2}), \forall n$.
\end{enumerate}
因此,$\{\cos n\}$是发散的。
\end{proof}
\end{enumerate}
\end{example}
%%%%%%%%%%%%%%%%%%%%%%%%%%%%%%
\begin{example}
证明:数列$\{a_n\}$收敛$\Leftrightarrow$三个数列$\{a_{3k-2}\}, \{a_{3k-1}\}, \{a_{3k}\}$都收敛且有相同的极限。
\begin{proof}
$(\Rightarrow)$由定理1.1.2,收敛数列的子列也收敛,且极限相同。\\
$(\Leftarrow)$假设三个子列的极限都是$a$。由极限的定义, 对于$\forall \varepsilon > 0$, 
\begin{enumerate}
\renewcommand\labelenumi{\normalfont(\theenumi)}
\item $\exists N_{1} \in \mathbb{N}$, 使得$\left |a_{3k-2} - a\right | < \varepsilon, \forall k > N_{1}$,
\item $\exists N_{2} \in \mathbb{N}$, 使得$\left |a_{3k-1} - a\right | < \varepsilon, \forall k > N_{2}$,
\item $\exists N_{3} \in \mathbb{N}$, 使得$\left |a_{3k} - a\right | < \varepsilon, \forall k > N_{3}$。
\end{enumerate}
取$N = 3\max\{N_1,N_2,N_3\}$, 我们有$$\left | a_n - a \right | < \varepsilon, \forall n > N,$$
即 $\displaystyle \lim_{n\to +\infty}a_n = a.$
\end{proof}
\begin{remark}
这个命题对于$a = +\infty. -\infty, \infty$也成立。
\end{remark}
\begin{remark}
对于$\forall p \in \mathbb{N}$, $$\displaystyle \lim_{n\to +\infty}a_n = a \Leftrightarrow \lim_{k\to +\infty}a_{pk-p+1} = \lim_{k\to +\infty}a_{pk-p+2} 
=\cdots = \lim_{k\to +\infty}a_{pk} = a.$$

\end{remark}

\end{example}
%%%%%%%%%%%%%%%%%%%%%%%%%%%%%%
\begin{example}
设$\displaystyle \lim_{n\to +\infty}(a_n - a_{n-1}) = d$。证明:$\displaystyle \lim_{n\to +\infty}\frac{a_n}{n} = d$。
\begin{proof}
$$\frac{a_n - a_1}{n} = \frac{(a_n - a_{n-1}) + (a_{n-1} - a_{n-2}) + \cdots + (a_2 - a_1)}{n}.$$
由例1.1.15知:
$$\displaystyle \lim_{n\to +\infty}\frac{a_n - a_1}{n} = d.$$
由于$\displaystyle \lim_{n\to +\infty}\frac{a_1}{n} = 0$, 易知
$$\displaystyle \lim_{n\to +\infty}\frac{a_n}{n} = d.$$
\end{proof}
\end{example}

%%%%%%%%%%%%%%%%%%%%%%%%%%%%%%
\begin{example}
设$\displaystyle \lim_{n\to +\infty}a_n = a$。用$\varepsilon-N$法,$A-N$法证明:
$$\displaystyle \lim_{n\to +\infty}\frac{a_1+2a_2+\cdots+na_n}{n^2} = \frac{a}{2}, (a \text{为实数}, +\infty, -\infty).$$

\end{example}
\begin{proof}
我们只证$a$为实数的情形。其他的情况证明类似。
由极限的定义,对于$\varepsilon > 0$, $\exists N_0\in \mathbb{N}$, 使得$$\left|a_n - a\right| < \frac{\varepsilon}{3}, \forall n > N_0.$$
\begin{equation*}
\begin{split}
&\left | \frac{a_1+2a_2+\cdots+na_n}{n^2}  - \frac{a}{2} \right | \\&= \left | \frac{(a_1-a)+2(a_2-a)+\cdots+n(a_n-a)}{n^2} +\frac{n(n+1)}{2n^2}a - \frac{a}{2} \right |\\
 													 &< \left | \frac{(a_1-a)+2(a_2-a)+\cdots+n(a_n-a)}{n^2}\right | + \frac{a}{2n} \\
													 &<\left | \frac{(a_1-a)+2(a_2-a)+\cdots+N_0(a_{N_0}-a)}{n^2}\right | + \frac{(N_0+1+n)(n-N_0)}{2n^2}\frac{\varepsilon}{3} + \frac{a}{2n}.\\
	\end{split}
\end{equation*}
取$N_1 \in \mathbb{N}$,使得
$$\left | \frac{(a_1-a)+2(a_2-a)+\cdots+N_0(a_{N_0}-a)}{n^2}\right | <\frac{\varepsilon}{3}, \forall n > N_1.$$
取$N_2 \in \mathbb{N}$,使得
$$\frac{a}{2n} < \frac{\varepsilon}{3}, \forall n > N_2.$$
取$N_3\in \mathbb{N}$, 使得
$$\frac{(N_0+1+n)(n-N_0)}{2n^2}\frac{\varepsilon}{3} < \frac{\varepsilon}{3}, \forall n> N_3.$$

最后, 取$N = \max\{N_0, N_1, N_2, N_3\}$, $\forall n > N$, 我们有
$$\left | \frac{a_1+2a_2+\cdots+na_n}{n^2}  - \frac{a}{2} \right | < \varepsilon.$$
即$$\displaystyle \lim_{n\to +\infty}\frac{a_1+2a_2+\cdots+na_n}{n^2} = \frac{a}{2}.$$
\end{proof}
%%%%%%%%%%%%%%%%%%%%%%%%%%%%%%
\subsection{思考题}
%%%%%%%%%%%%%%%%%%%%%%%%%%%%%%
\begin{example}
设$\displaystyle \lim_{n\to +\infty}a_n = a$,$\left |q \right | < 1$。用$\varepsilon-N$法证明:
$$\displaystyle \lim_{n\to +\infty}(a_n + a_{n-1}q + \cdots + a_1q^{n-1}) = \frac{a}{1-q}.$$
\end{example}
\begin{proof}

对于$\forall \varepsilon > 0$, 由$\displaystyle \lim_{n\to +\infty}a_n = a$,则存在整数$M > 0$和$N_0 \in \mathbb{N}$ 使得
$$\left | a_n - a \right | < M, \forall n\in\mathbb{N},$$
$$\left | a_n - a \right | < \frac{(1-|q|)\varepsilon}{3}, \forall n > N_0.$$
我们知道,当$\left| q\right| < 1$时,$\displaystyle \lim_{n\to +\infty}q^n = 0$。于是$\exists N_1 \in \mathbb {N}$使得
$$\left|q^{n-k}\right| < \max\left \{\frac{1}{3MN_0}, \frac{1-|q|}{3(|a|+1)}\right\}\varepsilon, \forall n > N-1, k=0, 1, 2, \cdots, N_0.$$
我们现在取$N = \max\{N_0, N_1\}$。对任意的$n > N$时,有
\begin{equation*}
\begin{split}
&\left | (a_n + a_{n-1}q + \cdots + a_1q^{n-1}) - \frac{a}{1-q} \right | \\
&= \left | (a_n + a_{n-1}q + \cdots + a_1q^{n-1}) - a\frac{1-q^n}{1-q} + \frac{aq^n}{1-q}\right|\\
&< \left | (a_n - a) + (a_{n-1} - a)q + \cdots + (a_1- a)q^{n-1} \right | + \frac{\left|a\right|\left|q\right|^n}{\left|1-q\right|} \\
&<\frac{(1-|q|)\varepsilon}{3}(1 + |q| + \cdots + |q|^{n-N_0}) + \left(MN_0+\frac{a}{1-|q|}\right)\max\left \{\frac{1}{3MN_0}, \frac{1-|q|}{3(|a|+1)}\right\}
\varepsilon\\&< \varepsilon.
	\end{split}
\end{equation*}
即,$\displaystyle \lim_{n\to +\infty}(a_n + a_{n-1}q + \cdots + a_1q^{n-1}) = \frac{a}{1-q}.$ 

\end{proof}
%%%%%%%%%%%%%%%%%%%%%%%%%%%%%%
\begin{example}
设$\displaystyle \lim_{n\to +\infty}a_n = a$,$\displaystyle \lim_{n\to +\infty}b_n = b$。用$\varepsilon-N$法证明:
$$\displaystyle \lim_{n\to +\infty}\frac{a_0b_n+a_1b_{n-1}+\cdots+a_{n-1}b_1+a_nb_0}{n}= ab.$$
\end{example}
\begin{proof}
首先我们证明命题在$b=0$时成立。
\begin{enumerate}
\renewcommand\labelenumi{\normalfont(\theenumi)}
\item 由于$\{a_n\}$收敛,则$\exists M > 0$使得$|a_n| < M, \forall n \in \mathbb {N}$.
\item 对于$\forall \varepsilon > 0$, 由于$\{b_n\}$收敛到$0$, 则$\exists N_0 \in \mathbb {N}$使得$|b_n| < \frac{\varepsilon}{2M}, \forall n > N_0$.
\item 由于$|a_n| < M$, 对上述的$\varepsilon > 0$, $\exists N_1 \in \mathbb{N}$使得
$$\left|\frac{a_{n - N_0}b_{N_0} + a_{n - N_0 + 1}b_{N_0 - 1}+\cdots + a_nb_0}{n}\right| < \frac{\varepsilon}{2}.$$
\end{enumerate}
取$N = \max\{N_0, N_1\}$, 对于上述的$\varepsilon > 0$, 当$n > N$, 有
\begin{equation*}
\begin{split}
&\left | \frac{a_0b_n+a_1b_{n-1}+\cdots+a_{n-1}b_1+a_nb_0}{n} \right | \\
&< \left | \frac{a_0b_n+a_1b_{n-1}+\cdots+a_{n-N_0-1}b_{N_0+1}}{n}\right| 
+ \left | \frac{a_{n-N_0}b_{N_0}+ a_{n-N_0 + 1}b_{N_0-1}+\cdots+a_{n}b_0}{n}\right|\\
&< \frac{\varepsilon}{2M}\frac{(n - N_0)M}{n} + \frac{\varepsilon}{2}\\
&< \varepsilon.
	\end{split}
\end{equation*}
即$$\displaystyle \lim_{n\to +\infty}\frac{a_0b_n+a_1b_{n-1}+\cdots+a_{n-1}b_1+a_nb_0}{n}= 0.$$

下面证明命题在$b\neq 0$时也成立。
\begin{enumerate}
\renewcommand\labelenumi{\normalfont(\theenumi)}
\item 由于$\displaystyle \lim_{n\to +\infty}a_n = a$收敛,则$\displaystyle \lim_{n\to +\infty}(a_n - a)b = 0$. 由此可知
$$\displaystyle \lim_{n\to +\infty}\frac{(a_0-a)b+(a_1-a)b+\cdots+(a_{n-1}-a)b+(a_n-a)b}{n}= 0.$$
\item 由于$\displaystyle \lim_{n\to +\infty}a_n = a$和$\displaystyle \lim_{n\to +\infty}(b_n - b) = 0$, 则
$$\displaystyle \lim_{n\to +\infty}\frac{a_0(b_n-b)+a_1(b_{n-1} - b)+\cdots+a_{n-1}(b_1 - b)+a_n(b_n-b)}{n}= 0.$$
\item 
\begin{equation*}
\begin{split}
&\left | \frac{a_0b_n+a_1b_{n-1}+\cdots+a_{n-1}b_1+a_nb_0}{n} - ab \right | \\
&= \left | \frac{a_0(b_n-b)+a_1(b_{n-1}-b)+\cdots+a_n(b_0-b)}{n} + \frac{(a_0-a)b+(a_1 - a)b+\cdots+(a_n-a)b)}{n}\right|
\end{split}
\end{equation*}
\item 对于$\forall \varepsilon > 0$, $\exists N \in \mathbb {N}$,使得当$n > N$时, 
$$\left | \frac{a_0(b_n-b)+a_1(b_{n-1}-b)+\cdots+a_n(b_0-b)}{n}\right| < \frac{\varepsilon}{2},$$
$$\left | \frac{(a_0-a)b+(a_1 - a)b+\cdots+(a_n-a)b)}{n}\right| < \frac{\varepsilon}{2}.$$
从而 $$\left | \frac{a_0b_n+a_1b_{n-1}+\cdots+a_{n-1}b_1+a_nb_0}{n} - ab \right | < \varepsilon.$$
即$$\displaystyle \lim_{n\to +\infty}\frac{a_0b_n+a_1b_{n-1}+\cdots+a_{n-1}b_1+a_nb_0}{n}= ab.$$
\end{enumerate}
\end{proof}
%%%%%%%%%%%%%%%%%%%%%%%%%%%%%%
\begin{example}
设$\displaystyle \lim_{n\to +\infty}a_n = a$,$b_n \geq 0 (n\in\mathbb{N})$,$\displaystyle \lim_{n\to +\infty}(b_1+b_2+\cdots+b_n) = S$。证明:$\displaystyle \lim_{n\to +\infty}(a_nb_1+a_{n-1}b_2+\cdots+a_1b_n) = aS$.
\end{example}
\begin{proof}
我们分以下步骤证明该命题。
\begin{enumerate}
\renewcommand\labelenumi{\normalfont(\theenumi)}
\item 首先我们证明$\displaystyle \lim_{n\to +\infty}b_n = 0$.
\item 
\begin{equation*}
\begin{split}
&\left | (a_nb_1+a_{n-1}b_2+\cdots+a_1b_n) - aS \right | \\
&= \left | (a_nb_1+a_{n-1}b_2+\cdots+a_1b_n) - a(b_1+b_2+\cdots+b_n)+a(b_1+b_2+\cdots+b_n -S)\right| \\
&< \left | (a_n-a)b_1+(a_{n-1}-a)b_2+\cdots+(a_1-a)b_n)\right| + \left|a\right|\left|(b_1 + b_2+\cdots + b_n) -S\right|
\end{split}
\end{equation*}
由$\displaystyle \lim_{n\to +\infty}a_n = a$ 和 $\displaystyle \lim_{n\to +\infty}b_n = 0$得知
$$\left | (a_n-a)b_1+(a_{n-1}-a)b_2+\cdots+(a_1-a)b_n)\right| < \frac{\varepsilon}{2}.$$
由$\displaystyle \lim_{n\to +\infty}(b_1+b_2+\cdots+b_n) = S$可得知
$$\left|a\right|\left|(b_1 + b_2+\cdots + b_n) -S\right| < \frac{\varepsilon}{2}.$$
\end{enumerate}
综上,$$\displaystyle \lim_{n\to +\infty}(a_nb_1+a_{n-1}b_2+\cdots+a_1b_n) = aS.$$
\end{proof}
\begin{remark}
这题里的条件$b_m \geq 0 (n\in\mathbb{N})$不是必须的。只要$\displaystyle \lim_{n\to +\infty}(|b_1|+|b_2|+\cdots+|b_n|) = S$就够了。
\end{remark}
\begin{remark}
这题是第10题的推广。如果$b_n = q^{n-1}, 0 < q < 1$,则
$$\displaystyle \lim_{n\to +\infty}(b_1 + b_2 + \cdots + b_n) = \lim_{n\to +\infty}\frac{1-q^n}{1-q}=\frac{1}{1-q}.$$ 
由这题的结论,第10题得证。
\end{remark}
%%%%%%%%%%%%%%%%%%%%%%%%%%%%%%
\begin{example}
(Toeplitz定理) 设$n,k\in\mathbb{N}$,$t_{nk} \geq 0$且$\displaystyle\sum_{k=1}^nt_{nk}=1$,$\displaystyle \lim_{n\to +\infty}t_{nk} = 0$。
如果$\displaystyle \lim_{n\to +\infty}a_n = a$,证明:$\displaystyle \lim_{n\to +\infty}\sum_{k=1}^nt_{nk}a_k = a$。说明例1.1.15为Toeplitze定理的特殊情形。
\end{example}
\begin{proof}
对于$\forall \varepsilon > 0$, 我们有:
\begin{enumerate}
\renewcommand\labelenumi{\normalfont(\theenumi)}
\item $\exists N_0\in \mathbb{N}$,当$n > N_0$时, $\left| a_n - a\right| < \frac{\varepsilon}{2}$.
\item 我们取$M = \max\{|a_1 - a|, |a_2 - a|, \cdots, |a_{N_0} - a|\}$.
\item 对于$l\in\mathbb{N}, 1 \leq l \leq N_{0}$, 存在$N_l \in \mathbb{N}$使得$t_{nl} < \frac{\varepsilon}{2N_0M}, \forall n > N_l$.
\item 取$N = \max \{N_0, N_1, \cdots, N_{N_0}\}$, 当$n > N$时,我们有:
\begin{equation*}
\begin{split}
&\left | \displaystyle \sum_{k=1}^nt_{nk}a_k - a \right | \\
&= \displaystyle \sum_{k=1}^{N_0}t_{nk}\left |(a_k-a)\right| + \displaystyle \sum_{k=N_{0}}^{n}t_{nk}\left |(a_k-a)\right|\\
&< \displaystyle \sum_{k=1}^{N_0}\frac{\varepsilon}{2N_0M}M + \displaystyle \sum_{k=N_{0}}^{n}t_{nk}\frac{\varepsilon}{2}\\
&=\varepsilon
\end{split}
\end{equation*}
\end{enumerate}
所以$$\displaystyle \lim_{n\to +\infty}\sum_{k=1}^nt_{nk}a_k = a.$$
如果我们取$b_{nk} = \frac{1}{n}$, 则例1.1.15就可以由这题得证。
\end{proof}
%%%%%%%%%%%%%%%%%%%%%%%%%%%%%%
\begin{example}
设$a,b,c$为三个给定的实数,令$a_0=a,b_0=b,c_0=c$,并归纳定义
\begin{equation*}
\begin{cases}
a_n = \frac{b_{n-1}+c_{n-1}}{2},\\
b_n = \frac{a_{n-1}+c_{n-1}}{2}, \quad n=1,2,\cdots.\\
c_n = \frac{a_{n-1}+b_{n-1}}{2},
\end{cases}
\end{equation*}
证明:$\displaystyle \lim_{n\to +\infty}a_n = \lim_{n\to +\infty}b_n=\lim_{n\to +\infty}c_n = \frac{a+b+c}{3}$.
\end{example}

\begin{proof}
我们通过以下结论去证明该命题:
\begin{enumerate}
\renewcommand\labelenumi{\normalfont(\theenumi)}
\item $\displaystyle \lim_{n\to +\infty}(a_n + b_n + c_n) = a+b+c$. 这是因为$a_n + b_n + c_n = a_{n-1}+b_{n-1}+c_{n-1} = \cdots = a+b+c$.
\item $\displaystyle \lim_{n\to +\infty}(a_n -b_n) = 0$, $\displaystyle \lim_{n\to +\infty}(a_n -c_n) = 0$, $\displaystyle \lim_{n\to +\infty}(c_n -b_n) = 0$.
这是因为$$a_n - b_n = \left(-\frac{1}{2}\right)(a_{n-1}-b_{n-1}) = \cdots = \left(-\frac{1}{2}\right)^n(a-b),$$
$$a_n - c_n = \left(-\frac{1}{2}\right)(a_{n-1}-c_{n-1}) = \cdots = \left(-\frac{1}{2}\right)^n(a-c),$$
$$c_n - b_n = \left(-\frac{1}{2}\right)(c_{n-1}-b_{n-1}) = \cdots = \left(-\frac{1}{2}\right)^n(c-b).$$
\item $$\displaystyle \lim_{n\to +\infty}3a_n = \lim_{n\to +\infty}(a_n+b_n+c_n +(a_n-b_n) + (a_n-c_n))=a+b+c,$$
从而, $$\displaystyle \lim_{n\to +\infty}a_n  = \frac{a+b+c}{3}.$$
\item 同理可证,$\displaystyle \lim_{n\to +\infty}b_n=\lim_{n\to +\infty}c_n = \frac{a+b+c}{3}$.
\end{enumerate}

\end{proof}
%%%%%%%%%%%%%%%%%%%%%%%%%%%%%%
\begin{example}
设$a_1,a_2$为实数,令$$a_n = pa_{n-1} + qa_{n-2}, n = 3,4,5,\cdots,$$
其中$p>0$,$q>0$, $p+q = 1$。证明:数列$\{a_n\}$收敛,且$\displaystyle \lim_{n\to +\infty}a_n=\frac{a_2+a_1q}{1+q}.$
\end{example}
\begin{proof}
由递推公式,我们可以证明$$a_n-a_{n-1} =\left(-q\right)^{n-2}(a_2 - a_1), \forall n \geq 3.$$
由此我们可以得出$a_n$的通项公式
$$a_n = a_2 + \displaystyle \sum_{k=1}^{n-2}\left(-q\right)^k(a_2 - a_1) = a_2 - \frac{q+(-q)^{n-1}}{1+q}(a_2-a_1).$$
从而,$$\displaystyle \lim_{n\to +\infty}a_n = a_2 - \frac{q}{1+q}(a_2-a_1) = \frac{a_2+qa_1}{1+q}.$$
\end{proof}
%%%%%%%%%%%%%%%%%%%%%%%%%%%%%%
\begin{example}
设数列$\{a_n\}$,$\{b_n\}$,$\{c_n\}$满足$a_1 > 0$,$4 \leq b_n \leq 5$,$4 \leq c_n \leq 5$,$$\displaystyle a_n=\frac{\sqrt{b_n^2+c_n^2}}{b_n+c_n}a_{n-1}$$
证明:$\displaystyle \lim_{n\to +\infty}a_n=0$.
\end{example}
\begin{proof}
由通项公式定义有$$0\leq a_n \leq \frac{5\sqrt 2}{8}a_{n-1} \leq \cdots \leq \left(\frac{5\sqrt 2}{8}\right)^{n-1}a_1.$$
由$\displaystyle\frac{5\sqrt 2}{8} < 1$知$\displaystyle \lim_{n\to +\infty}a_n=0$。
\end{proof}

%%%%%%%%%%%%%%%%%%%%%%%%%%%%%%
\section{数列极限的基本性质}
%%%%%%%%%%%%%%%%%%%%%%%%%%%%%%
\subsection{练习题}
\begin{example}应用数列极限的基本性质求下列极限:
\begingroup
\renewcommand\labelenumi{\normalfont(\theenumi)}
\begin{enumerate}
\item $\displaystyle \lim_{n\to +\infty}\frac{4n^2-n +5}{3n^2 -2n -7}$
\begin{solution}
$\displaystyle \lim_{n\to +\infty}\frac{4n^2-n +5}{3n^2 -2n -7}=\lim_{n\to +\infty}\frac{4-1/n +5/n^2}{3-2/n -7/n^2} = 4/3$
\end{solution}
\item $\displaystyle \lim_{n\to +\infty}\frac{3^n+(-2)^n}{3^{n+1} +(-2)^{n+1}}$
\begin{solution}
$\displaystyle \lim_{n\to +\infty}\frac{3^n+(-2)^n}{3^{n+1} +(-2)^{n+1}}=\lim_{n\to +\infty}\frac{1+(-2/3)^n}{3+(-2)(-2/3)^n} = 1/3$
\end{solution}
\item $\displaystyle \lim_{n\to +\infty}\left(1-\frac{1}{n}\right)^{\frac{1}{n}}$
\begin{solution}
$1 = \displaystyle \lim_{n\to +\infty}\frac{1}{\sqrt[n]2} \leq \lim_{n\to +\infty}\left(1-\frac{1}{n}\right)^{\frac{1}{n}} \leq 1$. 于是 $\displaystyle \lim_{n\to +\infty}\left(1-\frac{1}{n}\right)^{\frac{1}{n}} = 1$.
\end{solution}
\item $\displaystyle \lim_{n\to +\infty}(2\sin^2n+\cos^2n)^{\frac{1}{n}}$
\begin{solution}
$1 \leq \displaystyle \lim_{n\to +\infty}(2\sin^2n+\cos^2n)^{\frac{1}{n}}\leq \lim_{n\to +\infty}\sqrt[n]2 \leq 1$. 
于是 $\displaystyle \lim_{n\to +\infty}(2\sin^2n+\cos^2n)^{\frac{1}{n}} = 1$.
\end{solution}
\item $\displaystyle \lim_{n\to +\infty}(\arctan n)^{\frac{1}{n}}$
\begin{solution}
$1 \leq \displaystyle \lim_{n\to +\infty}(\arctan n)^{\frac{1}{n}}\leq \lim_{n\to +\infty}\sqrt[n]{\frac{\pi}{2}} \leq 1$. 
于是 $\displaystyle \lim_{n\to +\infty}(\arctan n)^{\frac{1}{n}} = 1$.
\end{solution}
\item $\displaystyle \lim_{n\to +\infty}\frac{1+a+\cdots+a^{n-1}}{1+b+\cdots +b^{n-1}}, |a| < 1, |b| < 1$
\begin{solution}
$\displaystyle \lim_{n\to +\infty}\frac{1+a+\cdots+a^{n-1}}{1+b+\cdots +b^{n-1}} = 
\lim_{n\to +\infty}\left(\frac{1-a^n}{1-a}\right)\left(\frac{1-b}{1-b^n}\right) = \frac{1-b}{1-a}$. 
\end{solution}

\item $\displaystyle \lim_{n\to +\infty}\left(\frac{1}{1\cdot 2}+\frac{1}{2\cdot 3} +\cdots+\frac{1}{n(n+1)}\right)$
\begin{solution}
$\displaystyle \lim_{n\to +\infty}\left(\frac{1}{1\cdot 2}+\frac{1}{2\cdot 3} +\cdots+\frac{1}{n(n+1)}\right) = 
\lim_{n\to +\infty}\left(1 - \frac{1}{n+1}\right) = 1$
\end{solution}

\item $\displaystyle \lim_{n\to +\infty}\left(1-\frac{1}{2^2}\right)\left(1-\frac{1}{3^2}\right)\cdots\left(1-\frac{1}{n^2}\right)$
\begin{solution}
$\displaystyle \lim_{n\to +\infty}\left(1-\frac{1}{2^2}\right)\left(1-\frac{1}{3^2}\right)\cdots\left(1-\frac{1}{n^2}\right) = 
\lim_{n\to +\infty}\left(1 - \frac{1}{2}\right)\left(1 + \frac{1}{n}\right) = \frac{1}{2}$
\end{solution}

\item $\displaystyle \lim_{n\to +\infty}\left(\frac{1}{2}+\frac{3}{2^2}+\cdots+\frac{2n-1}{2^n}\right)$
\begin{solution}
记$$S_n = \frac{1}{2}+\frac{3}{2^2}+\cdots+\frac{2n-1}{2^n},$$ 则
$$\frac{1}{2}S_n = \frac{1}{2^2}+\frac{3}{2^3}+\cdots+\frac{2(n-1)-1}{2^{n}}+\frac{2n-1}{2^{n+1}}.$$
于是
\begin{equation*}
\begin{split}
\frac{1}{2}S_n &= \frac{1}{2}+\left(\frac{1}{2} + \frac{1}{2^2}+\cdots+\frac{1}{2^{n-1}}\right)-\frac{2n-1}{2^{n+1}}\\
&=\frac{3}{2}-\frac{1}{2^{n-1}}-\frac{2n-1}{2^{n+1}}
\end{split}
\end{equation*}
从而$$\displaystyle \lim_{n\to +\infty}\left(\frac{1}{2}+\frac{3}{2^2}+\cdots+\frac{2n-1}{2^n}\right) = 3.$$
\end{solution}

\item $\displaystyle \lim_{n\to +\infty}\left(1-\frac{1}{1+2}\right)\left(1-\frac{1}{1+2+3}\right)+\cdots+\left(1-\frac{1}{1+2+\cdots+n}\right)$
\begin{solution}
$\displaystyle 1-\frac{1}{1+2+\cdots+k} = \frac{(k-1)(k+2)}{k(k+1)}$.从而
\begin{equation*}
\begin{split}
&\left(1-\frac{1}{1+2}\right)\left(1-\frac{1}{1+2+3}\right)+\cdots+\left(1-\frac{1}{1+2+\cdots+n}\right) \\
&=\frac{1\cdot 4}{2\cdot 3}\frac{2 \cdot 5}{3\cdot 4}\cdots \frac{(n-1)\cdot (n+2)}{n\cdot (n+1)}
\end{split}
\end{equation*}
分子的$2n$项的积:奇数项的积是$(n-1)!$, 偶数项的积是$\frac{1}{2\cdot 3}(n+2)!$.\\
分母的$2n$项的积:奇数项的积是$n!$, 偶数项的积是$\frac{1}{2}(n+1)!$.\\
于是$\displaystyle \lim_{n\to +\infty}\left(1-\frac{1}{1+2}\right)\left(1-\frac{1}{1+2+3}\right)+\cdots+\left(1-\frac{1}{1+2+\cdots+n}\right) = 
\lim_{n\to +\infty}\frac{n+2}{3n} = \frac{1}{3}.$
\end{solution}

\item $\displaystyle \lim_{n\to +\infty}\left[\frac{1^2}{n^3}+\frac{3^2}{n^3}+\cdots+\frac{(2n-1)^2}{n^3}\right]$
\begin{solution}
$$\sum_{k=1}^{n}(2k-1)^2 =\sum_{k=1}^{2n}k^2 - 4\sum_{k=1}^{n}k^2= \frac{8n^3 - 2n}{6}.$$
于是$$\displaystyle \lim_{n\to +\infty}\left[\frac{1^2}{n^3}+\frac{3^2}{n^3}+\cdots+\frac{(2n-1)^2}{n^3}\right] = \frac{4}{3}.$$
\end{solution}

\item $\displaystyle \lim_{n\to +\infty}(1+x)(1+x^2)(1+x^4)\cdots(1+x^{2^{n - 1}})$
\begin{solution}
$\displaystyle \lim_{n\to +\infty}(1+x)(1+x^2)(1+x^4)\cdots(1+x^{2^{n - 1}}) = \displaystyle \lim_{n\to +\infty}\frac{1-x^{2^n}}{1-x}=\frac{1}{1-x}.$
\end{solution}

\item $\displaystyle \lim_{n\to +\infty}(\sqrt{n+2}-2\sqrt{n+1}+\sqrt{n})$
\begin{solution}

\end{solution}
$\displaystyle \lim_{n\to +\infty}(\sqrt{n+2}-2\sqrt{n+1}+\sqrt{n})=
\lim_{n\to +\infty}\left(\frac{1}{\sqrt{n+2}+\sqrt{n+1}}-\frac{1}{\sqrt{n+1}+\sqrt{n}}\right)=0$.
\end{enumerate}
\endgroup
\end{example}
%%%%%%%%%%%%%%%%%%%%%%%%
\begin{example}
设$a_n > 0$,$n\in\mathbb{N}$,$\displaystyle \lim_{n\to +\infty}\frac{a_{n+1}}{a_n} = a$。应用例1.2.6证明:$\displaystyle \lim_{n\to +\infty}\sqrt[n]{a_n} = a$.
\end{example}
\begin{proof}
$$\sqrt[n]{a_n} = \sqrt[n]{\frac{a_n}{a_{n-1}}\cdot \frac{a_{n-1}}{a_{n-2}}\cdots\frac{a_2}{a_1}}\cdot \sqrt[n]{a_1}$$
于是$\displaystyle \lim_{n\to +\infty}\sqrt[n]{a_n} = a$.
\end{proof}
%%%%%%%%%%%%%%%%%%%%%%%

\begin{example}
设$\displaystyle \lim_{n\to +\infty}a_n = a$。应用夹逼定理证明:$\displaystyle \lim_{n\to +\infty}\frac{[na_n]}{n} = a$,其中$[x]$表示不超过的最大整数。
\end{example}
\begin{proof}
$$a=\displaystyle \lim_{n\to +\infty}\frac{na_n-1}{n} \leq \displaystyle \lim_{n\to +\infty}\frac{[na_n]}{n} \leq 
\displaystyle \lim_{n\to +\infty}\frac{na_n}{n} = a.$$
\end{proof}
%%%%%%%%%%%%%%%%%%%%%%

\begin{example}
设$a_n \neq 0$且$\displaystyle \lim_{n\to +\infty}\left|\frac{a_{n+1}}{a_n}\right| = r > 1$。证明:$\displaystyle \lim_{n\to +\infty}a_n = \infty$.
\end{example}
\begin{proof}
取$\varepsilon = \frac{r - 1}{2}$.由极限的定义,存在$N\in\mathbb{N}$使得$\left|\frac{a_{n+1}}{a_n}\right| > r - \varepsilon = \frac{r + 1}{2} > 1$. 
于是 $$\left|a_n\right| > \left(\frac{r+1}{2}\right)^{n_N}\left|a_N\right|.$$
即$\displaystyle \lim_{n\to +\infty}a_n = \infty$。
\end{proof}
%%%%%%%%%%%%%%%%%%%%%%
\begin{example}

\renewcommand\labelenumi{\normalfont(\theenumi)}
\begin{enumerate}
\item 应用数学归纳法或$\displaystyle\frac{2k-1}{2k}<\frac{2k}{2k+1}$证明不等式:
$$\frac{1}{2}\cdot\frac{3}{4}\cdot\cdots\cdot\frac{2n-1}{2n} < \frac{1}{\sqrt{2n+1}}.$$
\begin{proof}
记$S_n = \frac{1}{2}\cdot\frac{3}{4}\cdot\cdots\cdot\frac{2n-1}{2n}$. 利用不等式$\displaystyle\frac{2k-1}{2k}<\frac{2k}{2k+1}$, 我们有
$$S_n < \frac{2}{3}\cdot\frac{4}{5}\cdot\cdots\cdot\frac{2n}{2n+1} = \frac{1}{S_n(2n+1)}.$$
于是$S_n < \frac{1}{\sqrt{2n+1}}$

\end{proof}

\item 证明:$\displaystyle \lim_{n\to +\infty}\left(\frac{1}{2}\cdot\frac{3}{4}\cdot\cdots\cdot\frac{2n-1}{2n}\right) = 0$
\begin{proof}
$$0<\displaystyle \lim_{n\to +\infty}\left(\frac{1}{2}\cdot\frac{3}{4}\cdot\cdots\cdot\frac{2n-1}{2n}\right) \leq \lim_{n\to +\infty}\frac{1}{\sqrt{2n+1}}=0.$$
\end{proof}
\end{enumerate}

\end{example}
%%%%%%%%%%%%%%%%%%%%

\begin{example}
设$a_n > 0 (n\in\mathbb{N})$且$\displaystyle \lim_{n\to +\infty}a_n = a > 0$。应用夹逼定理证明:$\displaystyle \lim_{n\to +\infty}\sqrt[n]{a_n} = 1$
\end{example}
\begin{proof}
由于$\displaystyle \lim_{n\to +\infty}a_n = a > 0$,我们有一下结论:
$$\displaystyle \lim_{n\to +\infty}\frac{a_n}{n} = 0, \displaystyle \lim_{n\to +\infty}\frac{\frac{1}{a_n}}{n} = 0.$$
同时,我们有
\begin{equation*}
\begin{split}
1&=\lim_{n\to +\infty}\left(\frac{1}{\left(1 + 1 +\cdots+1+\frac{1}{a_n}\right)/n}\right)\\
&=\left(\frac{1}{\displaystyle\lim_{n\to +\infty}\left(n- 1 +\frac{1}{a_n}\right)/n}\right)\\
&\leq \lim_{n\to +\infty}\sqrt[n]{a_n} =\lim_{n\to +\infty}\sqrt[n]{\left(1\cdot 1\cdot \cdots \cdot 1\cdot a_n\right)}\\
&\leq \lim_{n\to +\infty}\frac{\left(1 + 1 +\cdots+1+a_n\right)}{n} \\
&=\lim_{n\to +\infty}\frac{\left(n-1+a_n\right)}{n}\\
&=1.
\end{split}
\end{equation*}
\end{proof}
%%%%%%%%%%%%%%%%%%%%%%%
\begin{example}
证明$\displaystyle \lim_{n\to +\infty}\frac{\displaystyle\sum_{k=1}^nk!}{n!} = 1$:$\left(\text{提示}: 1+\frac{1}{n}\leq\frac{\displaystyle\sum_{k=1}^nk!}{n!}\leq 1+ \frac{2}{n}\right)$
\end{example}
\begin{proof}
\begin{equation*}
\begin{split}
1+\frac{1}{n} &= \frac{(n-1)!+n!}{n!} \\&< \frac{\displaystyle\sum_{k=1}^nk!}{n!} \\&< \frac{(n-1)(n-2)!+(n-1)!+n!}{n!} \\
&= 1 + \frac{1}{n} + \frac{n-1}{n\cdot(n-1)} \\&= 1 +\frac{2}{n}
\end{split}
\end{equation*}
于是$\displaystyle \lim_{n\to +\infty}\frac{\displaystyle\sum_{k=1}^nk!}{n!} = 1$。
\end{proof}
%%%%%%%%%%%%%%%%%%%%%%%
\begin{example}
设$\displaystyle \lim_{n\to +\infty}a_n = a$,$\displaystyle \lim_{n\to +\infty}b_n = b$。记
$$S_n=\max\{a_n, b_n\}, \quad T_n = \min\{a_n, b_n\}, \quad n = 1,2,\cdots.$$
应用$\varepsilon-N$法$(\text{分} a < b, a>b, a=b)$或$\max\{a_n,b_n\} = \frac{1}{2}(a_n+b_n+|a_n-b_n|$与$\min\{a_n,b_n\} = \frac{1}{2}(a_n+b_n-|a_n-b_n|)$,
证明:
$$(1)\quad \displaystyle \lim_{n\to +\infty}S_n = \max\{a, b\}; \quad (2)\quad \displaystyle \lim_{n\to +\infty}T_n = \min\{a, b\}.$$
\end{example}
\begin{proof}
显然我们有$$\displaystyle \lim_{n\to +\infty}|a_n-b_n| = |a-b|.$$
由此可知:
$$\displaystyle \lim_{n\to +\infty}S_n = \frac{1}{2}(a+b+|a-b|) = \max\{a,b\},$$
$$\displaystyle \lim_{n\to +\infty}T_n = \frac{1}{2}(a+b-|a-b|) = \min\{a,b\}.$$
\end{proof}
%%%%%%%%%%%%%%%%%%%%%%%%

\begin{example}
应用例1.1.7与例1.1.15证明:
$$\displaystyle \lim_{n\to +\infty}\frac{1+\sqrt{2}+\sqrt[3]{3}+\cdots+\sqrt[n]{n}}{n} = 1.$$
\end{example}
\begin{proof}
取$a_n = \sqrt[n]{n}$,显然$$\displaystyle \lim_{n\to +\infty}a_n=1.$$于是
$$\displaystyle \lim_{n\to +\infty}\frac{1+\sqrt{2}+\sqrt[3]{3}+\cdots+\sqrt[n]{n}}{n} = \displaystyle \lim_{n\to +\infty}\frac{a_1+a_2+\cdots+a_n}{n} = 1.$$
\end{proof}
%%%%%%%%%%%%%%%%%%%%%%%%
\begin{example}
证明:$\displaystyle \lim_{n\to +\infty}\left(\sin\frac{\ln 2}{2}+\sin\frac{\ln 3}{3} +\cdots+\sin\frac{\ln n}{n}\right) = 1$
\end{example}
\begin{proof}
考虑数列$\sqrt[n]{n}$.这个数列在$n=3$是取得最大值且$\displaystyle \sqrt[n+1]{n+1} < \sqrt[n]{n}, \forall n \geq 3$。
这就是说数列$\{\sin\frac{\ln n}{n}\}$在$n=3$时取得最大值且$$\displaystyle \sin\frac{\ln{(n+1)}}{n+1} < \sin\frac{\ln{n}}{n}, \forall n \geq 3.$$
从而,
$$\left(\sin\frac{\ln 3}{3}\right)^{\frac{1}{n}} < \left(\sin\frac{\ln 2}{2}+\sin\frac{\ln 3}{3} +\cdots+\sin\frac{\ln n}{n}\right)^{\frac{1}{n}} < \left((n-1)\sin\frac{\ln 3}{3}\right)^{\frac{1}{n}}.$$
由于$$\displaystyle \lim_{n\to +\infty}\left(\sin\frac{\ln 3}{3}\right)^{\frac{1}{n}} = 1, \quad\displaystyle \lim_{n\to +\infty}\left((n-1)\sin\frac{\ln 3}{3}\right)^{\frac{1}{n}} = 1,$$我们有
$$\displaystyle \lim_{n\to +\infty}\left(\sin\frac{\ln 2}{2}+\sin\frac{\ln 3}{3} +\cdots+\sin\frac{\ln n}{n}\right) = 1.$$ 
\end{proof}
%%%%%%%%%%%%%%%%%%%%%%%%%
\begin{example}
证明:$\displaystyle \lim_{n\to +\infty}\displaystyle\sum_{k=n^2}^{(n+1)^2}\frac{1}{\sqrt k}= 2$.
\end{example}
\begin{proof}
$$\frac{2n+2}{n+1} = \displaystyle\sum_{k=n^2}^{(n+1)^2}\frac{1}{\sqrt{(n+1)^2}} \leq \displaystyle\sum_{k=n^2}^{(n+1)^2}\frac{1}{\sqrt k} \leq \displaystyle\sum_{k=n^2}^{(n+1)^2}\frac{1}{\sqrt{n^2}} = \frac{2n+2}{n}.$$
由夹逼定理,$\displaystyle \lim_{n\to +\infty}\displaystyle\sum_{k=n^2}^{(n+1)^2}\frac{1}{\sqrt k}= 2$。
\end{proof}
%%%%%%%%%%%%%%%%%%%%%%%%
\subsection{思考题}
\begin{example}
用$p(n)$表示能整除$n$的素数的个数。证明:$\displaystyle \lim_{n\to +\infty}\frac{p(n)}{n} = 0.$
\end{example}
\begin{proof}
假设$n=p_1^{m_1}p_2^{m_2}\cdots p_l^{m_l}$, 其中$p_1 < p_2 <\cdots < p_l$是互异的素数,$m_k \geq 1, k=1,2,\cdots, l$。这里$l=p(n)$.
$$\ln{n} = \displaystyle\sum_{k=1}^{l}m_k\ln{p_l} \geq \displaystyle\sum_{k=1}^{p(n)}\ln{2}=p(n)\ln{2}.$$
因此
$$0 \leq \frac{p(n)}{n} \leq \frac{\ln{n}}{n\ln{2}}.$$
由夹逼定理可知,$\displaystyle \lim_{n\to +\infty}\frac{p(n)}{n} = 0.$

\end{proof}
%%%%%%%%%%%%%%%%%%%%%%%
\begin{example}
设$x_n = \displaystyle \sum_{k=1}^n\left(\sqrt{1+\frac{k}{n^2}} -1\right)$。证明:$\displaystyle \lim_{n\to +\infty}x_n = \frac{1}{4}$.
\end{example}
\begin{proof}
\begin{equation*}
\begin{split}
\frac{n(n+1)}{2n^2\left(\sqrt{1 +\frac{1}{n}} + 1\right)}&= \frac{1}{\sqrt{1 +\frac{1}{n}} + 1}\displaystyle \sum_{k=1}^n\frac{k}{n^2} \\&< \displaystyle \sum_{k=1}^n\frac{\frac{k}{n^2}}{\sqrt{1+\frac{k}{n^2}}+1} \\&= \displaystyle\sum_{k=1}^n\left(\sqrt{1+\frac{k}{n^2}}-1\right) \\
&< \frac{1}{2}\displaystyle \sum_{k=1}^n\frac{k}{n^2}\\&= \frac{n(n+1)}{4n^2}.
\end{split}
\end{equation*}
由夹逼定理可知,$\displaystyle \lim_{n\to +\infty}x_n = \frac{1}{4}$。
\end{proof}
%%%%%%%%%%%%%%%%%%%%%%%%%%
\section{实数理论,实数连续性命题}
\subsection{练习题}
\subsection{思考题}

%\subsubsection{三级节标题}

%\begin{multicols}{2}

%\end{multicols}


%\part{植物多样性分区概述}



\include{angiosperms}

\appendix

%\chapter{}

\renewcommand\indexname{索~~引}
\printindex
\addcontentsline{toc}{chapter}{索~引}

\backmatter

\addcontentsline{toc}{chapter}{参考文献}

\begin{thebibliography}{参考文献}
\bibitem[徐森林,薛春华]{XX1} 徐森林,薛春华编著 《数学分析》, 清华大学出版社, 2005.
\end{thebibliography}

\chapter{后~~记}

\begin{flushright}

\end{flushright}

\end{document}
