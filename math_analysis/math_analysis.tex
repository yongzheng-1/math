\documentclass[utf8]{book}
\usepackage{titletoc}
\usepackage{titlesec}
\usepackage{ctexcap}
\usepackage[a4paper,text={125mm,195mm},centering,left=1in,right=1in,top=1in,bottom=1in]{geometry}
\usepackage[]{geometry}
\usepackage{imakeidx}
\usepackage{hyperref}
\usepackage{amsthm}
\usepackage{amsmath,amssymb}
\usepackage{extarrows}% http://ctan.org/pkg/extarrows

%%%% 定理类环境的定义 %%%%
\newtheorem{example}{}[section]             % 整体编号
\newtheorem{algorithm}{算法}
\newtheorem{theorem}{定理}[section]  % 按 section 编号
\newtheorem{definition}{定义}
\newtheorem{axiom}{公理}
\newtheorem{property}{性质}
\newtheorem{proposition}{命题}
\newtheorem{lemma}{引理}
\newtheorem{corollary}{推论}
\newtheorem{remark}{注}
\newtheorem{condition}{条件}
\newtheorem{conclusion}{结论}
\newtheorem{assumption}{假设}
\newtheorem{solution}{}
\renewcommand*{\proofname}{\normalfont\bfseries 证明}
\renewcommand*{\thesolution}{\normalfont\bfseries 解}
\renewcommand{\theexample}{\arabic{example}}

\DeclareMathOperator*\lowlim{\underline{lim}}
\DeclareMathOperator*\uplim{\overline{lim}}
\DeclareMathOperator{\sign}{sign}
\let\oldemptyset\emptyset
\let\emptyset\varnothing

\makeindex
\bibliographystyle{plain}
\begin{document}
\title{\heiti 徐森林,薛春华~编 \\ 《数学分析》题解}
\author{\fangsong 西海岸民工}
\date{2024年11月}

\frontmatter
\maketitle


\renewcommand\contentsname{目~录}
\tableofcontents

\mainmatter

%\part{总论}

\chapter{数列极限}

\section{数列极限的概念}
\subsection{练习题}
\begin{example}用数列极限定义证明:
\renewcommand\labelenumi{\normalfont(\theenumi)}
\begin{enumerate}
        \item $\displaystyle \lim_{n\to +\infty} 0.\underbrace{99\cdots9}_{n} = 1;$
        \begin{proof}
        对$\forall \varepsilon > 0$, 取$N = \left [ \frac{-\ln\varepsilon}{\ln 10}\right ] + 1$, 当 $n > N$时,
        $$\left | 0.\underbrace{99\cdots9}_{n} - 1\right | = \frac{1}{10^n} < \varepsilon.$$
        由极限的定义知,$\displaystyle\lim_{n\to +\infty} 0.\underbrace{99\cdots9}_{n} = 1$。
        \end{proof}
        
        \item $\displaystyle \lim_{n\to +\infty}\frac{3n+4}{7n-3} = \frac{3}{7};$
        \begin{proof}
        对$\forall \varepsilon > 0$, 取 $N = \left [ \frac{6}{\varepsilon}\right ] + 1$, 当 $n > N$时,
        $$\left | \frac{3n+4}{7n-3} - \frac{3}{7}\right | = \frac{37}{7(7n-3)} < \frac{37}{7n} < \frac{6}{n} < \varepsilon$$
        由极限定义知,$\displaystyle\lim_{n\to +\infty}\frac{3n+4}{7n-3} = \frac{3}{7}$。
        \end{proof}
        
        \item $\displaystyle \lim_{n\to +\infty}\frac{5n+6}{n^2-n-1000} = 0;$
        \begin{proof}
        对$\forall \varepsilon > 0$, 取 $N = \max\left\{50, \left [ \frac{12}{\varepsilon}\right ] + 1\right\}$, 当 $n > N$时,$\frac{1}{2}n^2 - n - 1000 > 1$ 且
        $$\left|\frac{5n+6}{n^2-n-1000} - 0\right| < \frac{5n + 6}{\frac{1}{2}n^2 + (\frac{1}{2}n^2 - n - 1000)} < \frac{6n}{\frac{1}{2}n^2} < \frac{12}{n} < \varepsilon$$
        由极限定义知,$\displaystyle\lim_{n\to +\infty}\frac{5n+6}{n^2-n-1000} = 0$。  
        \end{proof}
        
        \item $\displaystyle \lim_{n\to +\infty}\frac{8}{2^n+5} = 0;$
        \begin{proof}
         对$\forall \varepsilon > 0$, 取 $N = \left [ \frac{-\ln\varepsilon}{\ln2}\right ] + 4$, 当 $n > N$时,
        $$\left | \frac{8}{2^n+5} - 0\right | < \frac{8}{2^n} = \frac{1}{2^{n-3}} < \varepsilon$$
        由极限定义知$\displaystyle\lim_{n\to +\infty}\frac{8}{2^n+5} = 0$.  
        \end{proof}
        
        \item $\displaystyle \lim_{n\to +\infty}\frac{\sin n!}{n^{1/2}} = 0;$
        \begin{proof}
         对$\forall \varepsilon > 0$, 取 $N = \left [\frac{1}{\varepsilon^2}\right ] + 1$, 当 $n > N$时,
        $$\left | \frac{\sin n!}{n^{1/2}} - 0\right | < \frac{1}{n^{1/2}} < \varepsilon$$
        由极限定义知,$\displaystyle\lim_{n\to +\infty}\frac{\sin n!}{n^{1/2}} = 0$。  
        \end{proof}
        
        \item $\displaystyle \lim_{n\to +\infty}(\sqrt{n+2}-\sqrt{n-2}) = 0;$
        \begin{proof}
         对$\forall \varepsilon > 0$, 取 $N = \max\left\{2, \left [ \left(\frac{4}{\varepsilon}\right)^2\right ] + 1\right\}$, 当 $n > N$时,
        $$\left | \sqrt{n+2}-\sqrt{n-2}\right | = \frac{4}{\sqrt{n+2}+\sqrt{n-2}} < \frac{4}{\sqrt{n}} < \varepsilon$$
        由极限定义知,$\displaystyle\lim_{n\to +\infty}(\sqrt{n+2}-\sqrt{n-2}) = 0$。  
        \end{proof}
        
        \item $\displaystyle \lim_{n\to +\infty}(\sqrt[3]{n+2}-\sqrt[3]{n-2}) = 0;$
        \begin{proof}
         对 $\forall \varepsilon > 0$, 取 $N = \max\left\{2, \left [ \sqrt{\left(\frac{4}{\varepsilon}\right)^3}\right ]\right\}$, 当 $n > N$时,
$$\left|\sqrt[3]{n+2}-\sqrt[3]{n-2}\right| = \frac{4}{\sqrt[3]{(n+2)^2}+\sqrt[3]{(n+2)(n-2)} + \sqrt[3]{(n-2)^2}} < \frac{4}{\sqrt[3]{(n+2)^2}}< \varepsilon.$$
        由极限定义知,$\displaystyle\lim_{n\to +\infty}(\sqrt[3]{n+2}-\sqrt[3]{n-2}) = 0$.  
        \end{proof}
        
        \item $\displaystyle \lim_{n\to +\infty}\frac{n^{3/2}\arctan{n}}{1+n^2} = 0;$
        \begin{proof}
         对$\forall \varepsilon > 0$, 取 $N = \left [ \left(\frac{\pi}{2\varepsilon}\right)^2\right ] + 1$, 当 $n > N$时,
         $$\left | \frac{n^{3/2}\arctan{n}}{1+n^2} \right | < \frac{\frac{\pi}{2}n^{3/2}}{n^2} < \frac{\pi / 2}{\sqrt{n}} < \varepsilon$$
        由极限定义知,$\displaystyle\lim_{n\to +\infty}\frac{n^{3/2}\arctan{n}}{1+n^2} = 0$.  
        \end{proof} 
        
        \item $\displaystyle \lim_{n\to +\infty}a_n = 1,$ 其中 
        $a_n = 
         \begin{cases}
         \displaystyle\frac{n-1}{n}, &\text{n 为偶数,}\\
         \displaystyle\frac{\sqrt{n^2+ n}}{n}, &\text{n 为奇数;}
		\end{cases}$      
		
		\begin{proof}
         对$\forall \varepsilon > 0$, 取 $N = \left [ \frac{1}{\varepsilon}\right ] + 1$, 当 $n > N$时,
         \begin{equation*}
         \begin{aligned}
         \left | a_n - 1\right | &<
         \begin{cases}
         \displaystyle\frac{1}{n}, &\text{n 为偶数,}\\
         \displaystyle\frac{1}{\sqrt{n^2 + n} + n} , &\text{n 为奇数}
		\end{cases}\\
		&< \frac{1}{n} < \varepsilon
		\end{aligned}
		\end{equation*}
        由极限定义知,$\displaystyle\lim_{n\to +\infty}a_n = 1$.  
        \end{proof} 
        
        \item $\displaystyle \lim_{n\to +\infty}(n^3 - 4n - 5) = +\infty.$		
		\begin{proof}
         对$\forall A > 0$, 取 $N = \max\left\{5, \left [ \sqrt[3]{2A}\right ] + 1\right\}$, 当 $n > N$时,
         $1 - \frac{4}{n^2} - \frac{5}{n^3} > 1 - \frac{9}{n^2} > \frac{1}{2}$且 $$n^3 - 4n - 5 = n^3(1 - \frac{4}{n^2} - \frac{5}{n^3}) > \frac{1}{2}n^3 > A$$
        由极限定义知,$\displaystyle\lim_{n\to +\infty}(n^3 - 4n - 5) = +\infty$。  
        \end{proof}  
\end{enumerate}
\end{example}
%%%%%%%%%%%%%%%%%%%%%%%%%%%%%%
\begin{example}
设$\displaystyle \lim_{n\to +\infty}a_n = a$,证明: $\forall k \in \mathbb{N}$,有 $\displaystyle \lim_{n\to +\infty}a_{n+k} = a.$
\begin{proof}按$a$分类讨论:
\renewcommand\labelenumi{\normalfont(\theenumi)}
\begin{enumerate}
\item $a\in \mathbb{R}$:由极限定义,对$\forall \varepsilon > 0, \exists N \in \mathbb{N}$,当$n > N$时,$\left|a_n - a\right| < \varepsilon$。显然 $n+k > n > N$, 从而$\left| a_{n + k} - a\right| < \varepsilon$. 即 $\displaystyle \lim_{n\to +\infty}a_{n+k} = a$.
\item $a = +\infty$:由极限定义, 对$\forall A > 0, \exists N \in \mathbb{N}$, 当$n > N$时,$a_n > A$. 显然 $n+k > n > N$, 从而 $a_{n + k} > A$. 即 $\displaystyle \lim_{n\to +\infty}a_{n+k} = +\infty$.
\item $a=-\infty$:由极限定义,对$\forall A < 0, \exists N \in \mathbb{N}$, 当$n > N$时, $a_n < A$. 显然 $n+k > n > N$, 从而 $a_{n + k} < A$. 即 $\displaystyle \lim_{n\to +\infty}a_{n+k} = -\infty$.
\end{enumerate}
命题得证。
\end{proof}
\end{example}
%%%%%%%%%%%%%%%%%%%%%%%%%%%%%%
\begin{example}
设$\displaystyle \lim_{n\to +\infty}a_n = a$,证明$\displaystyle \lim_{n\to +\infty}\left |a_n\right | = \left |a\right |$:举例说明,这个命题的逆命题不真。
\begin{proof}当$a\in\mathbb{R}, a=+\infty, a=-\infty$时,此命题都是成立。我们只证$a\in\mathbb{R}$的情况。

由极限的定义,对$\forall \varepsilon > 0, \exists N \in \mathbb{N}$,当$n > N$时,$\left|a_n - a\right| < \varepsilon$. 于是:
$$\left| \left|a_n \right| - \left|a\right| \right | < \left|a_n - a\right| < \varepsilon.$$
再由极限定义,$\displaystyle\lim_{n\to +\infty}\left |a_n\right | = \left |a\right |$。

如果我们取$a_n = (-1)^n$, 则$\left | a_n\right | = 1$, 从而 $\displaystyle \lim_{n\to +\infty}\left |a_n\right | = \left |a\right |$。 但$\{a_n\}$是发散的。
\end{proof}
\end{example}
%%%%%%%%%%%%%%%%%%%%%%%%%%%%%%
\begin{example}
设$x_n \leq a \leq y_n, n \in \mathbb{N}$,且$\displaystyle \lim_{n\to +\infty}(y_n-x_n) = 0$。证明:
$$\displaystyle \lim_{n\to +\infty}x_n = \lim_{n\to +\infty}y_n = a$$
\end{example}
\begin{proof}
对$\forall \varepsilon > 0, \exists N \in \mathbb{N}$, 当$n> N$时,$y_n-x_n = \left|y_n - x_n\right| < \varepsilon$. 从而
$$\left| y_n - a\right| = y_n - a = y_n - x_n + x_n - a < y_n - x_n < \varepsilon.$$
i.e.$\displaystyle \lim_{n\to +\infty}y_n = a.$

同理可证
$$-\varepsilon < x_n - y_n < x_n - a < 0 < \varepsilon.$$ i.e.$\displaystyle \lim_{n\to +\infty}x_n = a.$
\end{proof}
%%%%%%%%%%%%%%%%%%%%%%%%%%%%%%
\begin{example}
设$\{a_n\}$为一个收敛数列。证明:数列$\{a_n\}$中或者有最大的数,或者有最小的数。 举出两者都有的例子; 再举出只有一个的例子。
\end{example}
\begin{proof}
假设$\displaystyle \lim_{n\to +\infty}a_n = a$,我们分以下几种情况讨论:
\renewcommand\labelenumi{\normalfont(\theenumi)}
\begin{enumerate}
\item 如果$a_n = a,\forall n\in \mathbb{N}$. 此时,数列$\{a_n\}$既有最小值也有最大值,且相等
\item 如果$\exists n_{0} \in \mathbb{N}$,使得$a_{n_{0}} \neq a$. 不妨假设$a_{n_{0}} < a$. 对于$\varepsilon = \displaystyle\frac{a - a_{n_{0}}}{2}$, $\exists N\in \mathbb {N}, 
N > n_{0}$使得
$a_n - a > a - \varepsilon = \frac{a + a_{n_{0}}}{2} > a_{n_{0}}, \forall n > N$. 取
$m = \min\left\{a_{1}, a_{2}, \cdots, a_{N}\right\}$,则
$$m \in \{a_n | n\geq 1\}, \quad a_n \geq m, \forall n.$$
即$m$是数列$\{a_n\}$的最小值. 如果$a_{n_{0}} > a$,类似地,我们可以证明$\{a_n\}$有最大值。
\end{enumerate}

考虑下列收敛数列:
\renewcommand\labelenumi{\normalfont(\theenumi)}
\begin{enumerate}
\item 如果$a_n = \frac{1}{n}$, 则该数列有最大值$a_n \leq a_1 = 1$, 没有最小值。
\item 如果$a_n = -\frac{1}{n}$, 该数列有最小值$-1 = a_1 \leq a_n$, 没有最大值。
\item 如果$a_n =(-1)^n \frac{1}{n}$, 则$-1 = a_1 \leq a_n \leq a_2 = \frac{1}{2}$ 
\end{enumerate}
命题得证。
\end{proof}
\begin{remark}
如果$\displaystyle \lim_{n\to +\infty}a_n = +\infty$,则$\{a_n\}$有最小值。如果$\displaystyle \lim_{n\to +\infty}a_n = -\infty$,则$\{a_n\}$有最大值。
\end{remark}
%%%%%%%%%%%%%%%%%%%%%%%%%%%%%%
\begin{example}
证明下列数列发散:
\renewcommand\labelenumi{\normalfont(\theenumi)}
\begin{enumerate}
\item $\{n^{(-1)^n}\}$
\begin{proof}
该数列发散,因为:
$0 = \displaystyle \lim_{n\to +\infty}(2n-1)^{(-1)^{2n-1}} \neq \displaystyle \lim_{n\to +\infty}(2n)^{(-1)^{2n}} = +\infty$.
\end{proof}
\item $\{\cos n\}$
\begin{proof}
取两个整数子列$\{k_n\}, \{l_n\}$使得
\renewcommand\labelenumi{\normalfont(\theenumi)}
\begin{enumerate}
\item $k_n \in (2n\pi - \frac{\pi}{6},2n\pi + \frac{\pi}{6})$,
\item $l_n \in (2n\pi + \frac{5\pi}{6}, (2n+1)\pi + \frac{\pi}{6})$.
\end{enumerate}
显然,我们有
\renewcommand\labelenumi{\normalfont(\theenumi)}
\begin{enumerate}
\item $\cos k_n \in (\frac{\sqrt 3}{2}, 1], \forall n$,
\item $\cos l_n \in [-1, -\frac{\sqrt 3}{2}), \forall n$.
\end{enumerate}
因此,$\{\cos n\}$是发散的。
\end{proof}
\end{enumerate}
\end{example}
%%%%%%%%%%%%%%%%%%%%%%%%%%%%%%
\begin{example}
证明:数列$\{a_n\}$收敛$\Leftrightarrow$三个数列$\{a_{3k-2}\}, \{a_{3k-1}\}, \{a_{3k}\}$都收敛且有相同的极限。
\begin{proof}
$(\Rightarrow)$由定理1.1.2,收敛数列的子列也收敛,且极限相同。\\
$(\Leftarrow)$假设三个子列的极限都是$a$。由极限的定义, 对于$\forall \varepsilon > 0$, 
\begin{enumerate}
\renewcommand\labelenumi{\normalfont(\theenumi)}
\item $\exists N_{1} \in \mathbb{N}$, 使得$\left |a_{3k-2} - a\right | < \varepsilon, \forall k > N_{1}$,
\item $\exists N_{2} \in \mathbb{N}$, 使得$\left |a_{3k-1} - a\right | < \varepsilon, \forall k > N_{2}$,
\item $\exists N_{3} \in \mathbb{N}$, 使得$\left |a_{3k} - a\right | < \varepsilon, \forall k > N_{3}$。
\end{enumerate}
取$N = 3\max\{N_1,N_2,N_3\}$, 我们有$$\left | a_n - a \right | < \varepsilon, \forall n > N,$$
即 $\displaystyle \lim_{n\to +\infty}a_n = a.$
\end{proof}
\begin{remark}
这个命题当$a = +\infty$或者$-\infty$时,该命题也成立。
\end{remark}
\begin{remark}
对于$\forall p \in \mathbb{N}$, $$\displaystyle \lim_{n\to +\infty}a_n = a \Leftrightarrow \lim_{k\to +\infty}a_{pk-p+1} = \lim_{k\to +\infty}a_{pk-p+2} 
=\cdots = \lim_{k\to +\infty}a_{pk} = a.$$
\end{remark}
\end{example}
%%%%%%%%%%%%%%%%%%%%%%%%%%%%%%
\begin{example}
设$\displaystyle \lim_{n\to +\infty}(a_n - a_{n-1}) = d$。证明:$\displaystyle \lim_{n\to +\infty}\frac{a_n}{n} = d$。
\begin{proof}
$$\frac{a_n - a_1}{n} = \frac{(a_n - a_{n-1}) + (a_{n-1} - a_{n-2}) + \cdots + (a_2 - a_1)}{n}.$$
由例1.1.15知:
$\displaystyle \lim_{n\to +\infty}\frac{a_n}{n}=\displaystyle \lim_{n\to +\infty}\frac{a_n - a_1}{n} = d.$
\end{proof}
\end{example}
%%%%%%%%%%%%%%%%%%%%%%%%%%%%%%
\begin{example}
设$\displaystyle \lim_{n\to +\infty}a_n = a$。用$\varepsilon-N$法,$A-N$法证明:
$$\displaystyle \lim_{n\to +\infty}\frac{a_1+2a_2+\cdots+na_n}{n^2} = \frac{a}{2}, (a \text{为实数}, +\infty, -\infty).$$
\end{example}
\begin{proof}
我们只证$a$为实数的情形。其他情况证明类似。
由极限的定义,对于$\varepsilon > 0$, $\exists N_0\in \mathbb{N}$, 使得$$\left|a_n - a\right| < \frac{\varepsilon}{3}, \forall n > N_0.$$
\begin{equation*}
\begin{split}
&\left | \frac{a_1+2a_2+\cdots+na_n}{n^2}  - \frac{a}{2} \right | \\&= \left | \frac{(a_1-a)+2(a_2-a)+\cdots+n(a_n-a)}{n^2} +\frac{n(n+1)}{2n^2}a - \frac{a}{2} \right |\\
 													 &< \left | \frac{(a_1-a)+2(a_2-a)+\cdots+n(a_n-a)}{n^2}\right | + \frac{a}{2n} \\
													 &<\left | \frac{(a_1-a)+2(a_2-a)+\cdots+N_0(a_{N_0}-a)}{n^2}\right | + \frac{(N_0+1+n)(n-N_0)}{2n^2}\frac{\varepsilon}{3} + \frac{a}{2n}.\\
	\end{split}
\end{equation*}
取$N_1 \in \mathbb{N}$,使得
$$\left | \frac{(a_1-a)+2(a_2-a)+\cdots+N_0(a_{N_0}-a)}{n^2}\right | <\frac{\varepsilon}{3}, \forall n > N_1.$$
取$N_2 \in \mathbb{N}$,使得
$$\frac{a}{2n} < \frac{\varepsilon}{3}, \forall n > N_2.$$
取$N_3\in \mathbb{N}$, 使得
$$\frac{(N_0+1+n)(n-N_0)}{2n^2}\frac{\varepsilon}{3} < \frac{\varepsilon}{3}, \forall n> N_3.$$

最后, 取$N = \max\{N_0, N_1, N_2, N_3\}$, $\forall n > N$, 我们有
$$\left | \frac{a_1+2a_2+\cdots+na_n}{n^2}  - \frac{a}{2} \right | < \varepsilon.$$
即$$\displaystyle \lim_{n\to +\infty}\frac{a_1+2a_2+\cdots+na_n}{n^2} = \frac{a}{2}.$$
命题得证。\end{proof}
\begin{remark}
考虑如下数列
\begin{equation*}
b_k=
\begin{cases}
a_k,\quad&k=1;\\
a_n,\quad&k\in \left[\frac{(n-1)n}{2}+ 1,\frac{n(n+1)}{2}\right], \forall n\geq 2.
\end{cases}
\end{equation*}
显然$\displaystyle \lim_{n\to +\infty}b_n = a$.
记$$S_n = \frac{b_1+b_2+\cdots+b_{\frac{n(n+1)}{2}}}{\frac{n(n+1)}{2}},$$ 则
\begin{equation*}
\begin{split}
\frac{a_1+2a_2+\cdots+na_n}{n^2} &= \frac{b_1+b_2+\cdots+b_{\frac{n(n+1)}{2}}}{n^2}\\&=\frac{b_1+b_2+\cdots+b_{\frac{n(n+1)}{2}}}{\frac{n(n+1)}{2}}\cdot\frac{n(n+1)}{2n^2}\\&=\frac{S_n}{2}+\frac{S_n}{2n^2}
\end{split}
\end{equation*}
由例1.1.15知,$\displaystyle \lim_{n\to +\infty}S_n = a$, 从而$$\displaystyle \lim_{n\to +\infty}\frac{a_1+2a_2+\cdots+na_n}{n^2} = \frac{a}{2}.$$
\end{remark}
%%%%%%%%%%%%%%%%%%%%%%%%%%%%%%
\subsection{思考题}
%%%%%%%%%%%%%%%%%%%%%%%%%%%%%%
\begin{example}
设$\displaystyle \lim_{n\to +\infty}a_n = a$,$\left |q \right | < 1$. 用$\varepsilon-N$法证明:
$$\displaystyle \lim_{n\to +\infty}(a_n + a_{n-1}q + \cdots + a_1q^{n-1}) = \frac{a}{1-q}.$$
\end{example}
\begin{proof}

对于$\forall \varepsilon > 0$, 由$\displaystyle \lim_{n\to +\infty}a_n = a$,则存在整数$M > 0$和$N_0 \in \mathbb{N}$ 使得
$$\left | a_n - a \right | < M, \forall n\in\mathbb{N},$$
$$\left | a_n - a \right | < \frac{(1-|q|)\varepsilon}{3}, \forall n > N_0.$$
我们知道,当$\left| q\right| < 1$时,$\displaystyle \lim_{n\to +\infty}q^n = 0$。于是$\exists N_1 \in \mathbb {N}$使得
$$\left|q^{n-k}\right| < \max\left \{\frac{1}{3MN_0}, \frac{1-|q|}{3(|a|+1)}\right\}\varepsilon, \forall n > N_1, k=0, 1, 2, \cdots, N_0.$$
取$N = \max\left\{N_0, N_1\right\}$。当$n > N$时,有
\begin{equation*}
\begin{split}
&\left | (a_n + a_{n-1}q + \cdots + a_1q^{n-1}) - \frac{a}{1-q} \right | \\
&= \left | (a_n + a_{n-1}q + \cdots + a_1q^{n-1}) - a\frac{1-q^n}{1-q} - \frac{aq^n}{1-q}\right|\\
&< \left | (a_n - a) + (a_{n-1} - a)q + \cdots + (a_1- a)q^{n-1} \right | + \frac{\left|a\right|\left|q\right|^n}{\left|1-q\right|} \\
&<\frac{(1-|q|)\varepsilon}{3}(1 + |q| + \cdots + |q|^{n-N_0}) + \left(MN_0+\frac{|a|}{1-|q|}\right)\max\left \{\frac{1}{3MN_0}, \frac{1-|q|}{3(|a|+1)}\right\}
\varepsilon\\&< \varepsilon.
	\end{split}
\end{equation*}
即,$\displaystyle \lim_{n\to +\infty}(a_n + a_{n-1}q + \cdots + a_1q^{n-1}) = \frac{a}{1-q}.$ 

\end{proof}
%%%%%%%%%%%%%%%%%%%%%%%%%%%%%%
\begin{example}
设$\displaystyle \lim_{n\to +\infty}a_n = a$,$\displaystyle \lim_{n\to +\infty}b_n = b$。用$\varepsilon-N$法证明:
$$\displaystyle \lim_{n\to +\infty}\frac{a_0b_n+a_1b_{n-1}+\cdots+a_{n-1}b_1+a_nb_0}{n}= ab.$$
\end{example}
\begin{proof}
首先我们证明命题在$b=0$时成立。
\begin{enumerate}
\renewcommand\labelenumi{\normalfont(\theenumi)}
\item 由于$\{a_n\}$收敛,则$\exists M > 0$使得$|a_n| < M, \forall n \in \mathbb {N}$.
\item 对于$\forall \varepsilon > 0$, 由于$\{b_n\}$收敛到$0$, 则$\exists N_0 \in \mathbb {N}$使得$|b_n| < \frac{\varepsilon}{2M}, \forall n > N_0$.
\item 由于$|a_n| < M$, 对上述的$\varepsilon > 0$, $\exists N_1 \in \mathbb{N}$使得
$$\left|\frac{a_{n - N_0}b_{N_0} + a_{n - N_0 + 1}b_{N_0 - 1}+\cdots + a_nb_0}{n}\right| < \frac{\varepsilon}{2}.$$
\end{enumerate}
取$N = \max\{N_0, N_1\}$。对于上述的$\varepsilon > 0$, 当$n > N$时,
\begin{equation*}
\begin{split}
&\left | \frac{a_0b_n+a_1b_{n-1}+\cdots+a_{n-1}b_1+a_nb_0}{n} \right | \\
&< \left | \frac{a_0b_n+a_1b_{n-1}+\cdots+a_{n-N_0-1}b_{N_0+1}}{n}\right| 
+ \left | \frac{a_{n-N_0}b_{N_0}+ a_{n-N_0 + 1}b_{N_0-1}+\cdots+a_{n}b_0}{n}\right|\\
&< \frac{\varepsilon}{2M}\frac{(n - N_0)M}{n} + \frac{\varepsilon}{2}\\
&< \varepsilon.
	\end{split}
\end{equation*}
由极限的定义知,命题成立在$b=0$时成立。下面证明命题在$b\neq 0$时也成立。
\begin{enumerate}
\renewcommand\labelenumi{\normalfont(\theenumi)}
\item 由于$\displaystyle \lim_{n\to +\infty}a_n = a$收敛,则$\displaystyle \lim_{n\to +\infty}(a_n - a)b = 0$. 由此可知
$$\displaystyle \lim_{n\to +\infty}\frac{(a_0-a)b+(a_1-a)b+\cdots+(a_{n-1}-a)b+(a_n-a)b}{n}= 0.$$
\item 由于$\displaystyle \lim_{n\to +\infty}a_n = a$和$\displaystyle \lim_{n\to +\infty}(b_n - b) = 0$, 则
$$\displaystyle \lim_{n\to +\infty}\frac{a_0(b_n-b)+a_1(b_{n-1} - b)+\cdots+a_{n-1}(b_1 - b)+a_n(b_n-b)}{n}= 0.$$
\item 对于$\forall \varepsilon > 0$, $\exists N \in \mathbb {N}$,使得当$n > N$时, 
$$\left | \frac{a_0(b_n-b)+a_1(b_{n-1}-b)+\cdots+a_n(b_0-b)}{n}\right| < \frac{\varepsilon}{2},$$
$$\left | \frac{(a_0-a)b+(a_1 - a)b+\cdots+(a_n-a)b)}{n}\right| < \frac{\varepsilon}{2}.$$
从而 
\begin{equation*}
\begin{split}
&\left | \frac{a_0b_n+a_1b_{n-1}+\cdots+a_{n-1}b_1+a_nb_0}{n} - ab \right | \\
&\leq \left | \frac{a_0(b_n-b)+a_1(b_{n-1}-b)+\cdots+a_n(b_0-b)}{n}\right| + \left|\frac{(a_0-a)b+(a_1 - a)b+\cdots+(a_n-a)b)}{n}\right|\\&\leq \varepsilon
\end{split}
\end{equation*}
\end{enumerate}
\end{proof}
%%%%%%%%%%%%%%%%%%%%%%%%%%%%%%
\begin{example}
设$\displaystyle \lim_{n\to +\infty}a_n = a$,$b_n \geq 0 (n\in\mathbb{N})$,$\displaystyle \lim_{n\to +\infty}(b_1+b_2+\cdots+b_n) = S$。证明:$\displaystyle \lim_{n\to +\infty}(a_nb_1+a_{n-1}b_2+\cdots+a_1b_n) = aS$.
\end{example}
\begin{proof}
我们分以下步骤证明该命题。
\begin{enumerate}
\renewcommand\labelenumi{\normalfont(\theenumi)}
\item 显然$\displaystyle \lim_{n\to +\infty}b_n = 0$.
\item 
由$\displaystyle \lim_{n\to +\infty}a_n = a$ 和 $\displaystyle \lim_{n\to +\infty}b_n = 0$得知
$$\left | (a_n-a)b_1+(a_{n-1}-a)b_2+\cdots+(a_1-a)b_n\right| < \frac{\varepsilon}{2}.$$
由$\displaystyle \lim_{n\to +\infty}(b_1+b_2+\cdots+b_n) = S$可得知
$$\left|a\right|\left|(b_1 + b_2+\cdots + b_n) -S\right| < \frac{\varepsilon}{2}.$$
于是
\begin{equation*}
\begin{split}
&\left | (a_nb_1+a_{n-1}b_2+\cdots+a_1b_n) - aS \right | \\
&= \left | (a_nb_1+a_{n-1}b_2+\cdots+a_1b_n) - a(b_1+b_2+\cdots+b_n)+a(b_1+b_2+\cdots+b_n -S)\right| \\
&< \left | (a_n-a)b_1+(a_{n-1}-a)b_2+\cdots+(a_1-a)b_n\right| + \left|a\right|\left|(b_1 + b_2+\cdots + b_n) -S\right|\\&<\varepsilon
\end{split}
\end{equation*}
\end{enumerate}
由极限的定义,命题得证。
\end{proof}
\begin{remark}
这题是第10题的推广。如果$b_n = q^{n-1}, 0 < q < 1$,则
$$\displaystyle \lim_{n\to +\infty}(b_1 + b_2 + \cdots + b_n) = \lim_{n\to +\infty}\frac{1-q^n}{1-q}=\frac{1}{1-q}.$$ 
由这题的结论,第10题得证。
\end{remark}
%%%%%%%%%%%%%%%%%%%%%%%%%%%%%%
\begin{example}
(Toeplitz定理) 设$n,k\in\mathbb{N}$,$t_{nk} \geq 0$且$\displaystyle\sum_{k=1}^nt_{nk}=1$,$\displaystyle \lim_{n\to +\infty}t_{nk} = 0$。
如果$\displaystyle \lim_{n\to +\infty}a_n = a$,证明:$\displaystyle \lim_{n\to +\infty}\sum_{k=1}^nt_{nk}a_k = a$。说明例1.1.15为Toeplitze定理的特殊情形。
\end{example}
\begin{proof}
对于$\forall \varepsilon > 0$, 我们有:
\begin{enumerate}
\renewcommand\labelenumi{\normalfont(\theenumi)}
\item $\exists N_0\in \mathbb{N}$,当$n > N_0$时, $\left| a_n - a\right| < \frac{\varepsilon}{2}$.
\item 我们取$M = \max\{|a_1 - a|, |a_2 - a|, \cdots, |a_{N_0} - a|\}$.
\item 对于$l\in\mathbb{N}, 1 \leq l \leq N_{0}$, 存在$N_l \in \mathbb{N}$使得$t_{nl} < \frac{\varepsilon}{2N_0M}, \forall n > N_l$.
\item 取$N = \max \{N_0, N_1, \cdots, N_{N_0}\}$, 当$n > N$时,我们有:
\begin{equation*}
\begin{split}
&\left | \displaystyle \sum_{k=1}^nt_{nk}a_k - a \right | \\
&= \displaystyle \sum_{k=1}^{N_0}t_{nk}\left |(a_k-a)\right| + \displaystyle \sum_{k=N_{0}}^{n}t_{nk}\left |(a_k-a)\right|\\
&< \displaystyle \sum_{k=1}^{N_0}\frac{\varepsilon}{2N_0M}M + \displaystyle \sum_{k=N_{0}}^{n}t_{nk}\frac{\varepsilon}{2}\\
&\leq\varepsilon
\end{split}
\end{equation*}
\end{enumerate}
由极限定义知,$\displaystyle \lim_{n\to +\infty}\sum_{k=1}^nt_{nk}a_k = a$.

如果我们取$b_{nk} = \frac{1}{n}$, 则例1.1.15就可以由这题得证。
\end{proof}
%%%%%%%%%%%%%%%%%%%%%%%%%%%%%%
\begin{example}
设$a,b,c$为三个给定的实数,令$a_0=a,b_0=b,c_0=c$,并归纳定义
\begin{equation*}
\begin{cases}
a_n = \displaystyle\frac{b_{n-1}+c_{n-1}}{2},\\
b_n = \displaystyle\frac{a_{n-1}+c_{n-1}}{2}, \quad n=1,2,\cdots.\\
c_n = \displaystyle\frac{a_{n-1}+b_{n-1}}{2},
\end{cases}
\end{equation*}
证明:$\displaystyle \lim_{n\to +\infty}a_n = \lim_{n\to +\infty}b_n=\lim_{n\to +\infty}c_n = \frac{a+b+c}{3}$.
\end{example}
\begin{proof}
我们通过以下结论证明该命题:
\begin{enumerate}
\renewcommand\labelenumi{\normalfont(\theenumi)}
\item $\displaystyle \lim_{n\to +\infty}(a_n + b_n + c_n) = a+b+c$. 这是因为$a_n + b_n + c_n = a_{n-1}+b_{n-1}+c_{n-1} = \cdots = a+b+c$.
\item $\displaystyle \lim_{n\to +\infty}(a_n -b_n) = 0$, $\displaystyle \lim_{n\to +\infty}(a_n -c_n) = 0$, $\displaystyle \lim_{n\to +\infty}(c_n -b_n) = 0$.
这是因为$$a_n - b_n = \left(-\frac{1}{2}\right)(a_{n-1}-b_{n-1}) = \cdots = \left(-\frac{1}{2}\right)^n(a-b),$$
$$a_n - c_n = \left(-\frac{1}{2}\right)(a_{n-1}-c_{n-1}) = \cdots = \left(-\frac{1}{2}\right)^n(a-c),$$
$$c_n - b_n = \left(-\frac{1}{2}\right)(c_{n-1}-b_{n-1}) = \cdots = \left(-\frac{1}{2}\right)^n(c-b).$$
\item $\displaystyle \lim_{n\to +\infty}a_n =\frac{1}{3}\displaystyle \lim_{n\to +\infty}3a_n = \lim_{n\to +\infty}(a_n+b_n+c_n +(a_n-b_n) + (a_n-c_n))=\frac{a+b+c}{3}.$
\end{enumerate}

类似可证,$\displaystyle \lim_{n\to +\infty}b_n=\lim_{n\to +\infty}c_n = \frac{a+b+c}{3}$.
\end{proof}
%%%%%%%%%%%%%%%%%%%%%%%%%%%%%%
\begin{example}
设$a_1,a_2$为实数,令$$a_n = pa_{n-1} + qa_{n-2}, n = 3,4,5,\cdots,$$
其中$p>0$,$q>0$, $p+q = 1$。证明:数列$\{a_n\}$收敛,且$\displaystyle \lim_{n\to +\infty}a_n=\frac{a_2+a_1q}{1+q}.$
\end{example}
\begin{proof}
由递推公式,我们可以证明$$a_n-a_{n-1} =\left(-q\right)^{n-2}(a_2 - a_1), \forall n \geq 3.$$
由此我们可以得出$a_n$的通项公式
$$a_n = a_2 + \displaystyle \sum_{k=1}^{n-2}\left(-q\right)^k(a_2 - a_1) = a_2 - \frac{q+(-q)^{n-1}}{1+q}(a_2-a_1).$$
从而,$\displaystyle \lim_{n\to +\infty}a_n = a_2 - \frac{q}{1+q}(a_2-a_1) = \frac{a_2+qa_1}{1+q}$.
\end{proof}
%%%%%%%%%%%%%%%%%%%%%%%%%%%%%%
\begin{example}
设数列$\{a_n\}$,$\{b_n\}$,$\{c_n\}$满足$a_1 > 0$,$4 \leq b_n \leq 5$,$4 \leq c_n \leq 5$,$$\displaystyle a_n=\frac{\sqrt{b_n^2+c_n^2}}{b_n+c_n}a_{n-1}$$
证明:$\displaystyle \lim_{n\to +\infty}a_n=0$.
\end{example}
\begin{proof}
由通项公式定义有$$0\leq a_n \leq \frac{5\sqrt 2}{8}a_{n-1} \leq \cdots \leq \left(\frac{5\sqrt 2}{8}\right)^{n-1}a_1.$$
由$\displaystyle\frac{5\sqrt 2}{8} < 1$知$\displaystyle \lim_{n\to +\infty}a_n=0$。
\end{proof}
%%%%%%%%%%%%%%%%%%%%%%%%%%%%%%
\section{数列极限的基本性质}
%%%%%%%%%%%%%%%%%%%%%%%%%%%%%%
\subsection{练习题}
\begin{example}应用数列极限的基本性质求下列极限:
\renewcommand\labelenumi{\normalfont(\theenumi)}
\begin{enumerate}
\item $\displaystyle \lim_{n\to +\infty}\frac{4n^2-n +5}{3n^2 -2n -7}$
\begin{solution}
$\displaystyle \lim_{n\to +\infty}\frac{4n^2-n +5}{3n^2 -2n -7}=\lim_{n\to +\infty}\frac{4-1/n +5/n^2}{3-2/n -7/n^2} = 4/3$
\end{solution}
\item $\displaystyle \lim_{n\to +\infty}\frac{3^n+(-2)^n}{3^{n+1} +(-2)^{n+1}}$
\begin{solution}
$\displaystyle \lim_{n\to +\infty}\frac{3^n+(-2)^n}{3^{n+1} +(-2)^{n+1}}=\lim_{n\to +\infty}\frac{1+(-2/3)^n}{3+(-2)(-2/3)^n} = 1/3$
\end{solution}
\item $\displaystyle \lim_{n\to +\infty}\left(1-\frac{1}{n}\right)^{\frac{1}{n}}$
\begin{solution}
$1 = \displaystyle \lim_{n\to +\infty}\frac{1}{\sqrt[n]2} \leq \lim_{n\to +\infty}\left(1-\frac{1}{n}\right)^{\frac{1}{n}} \leq 1$. 于是 $\displaystyle \lim_{n\to +\infty}\left(1-\frac{1}{n}\right)^{\frac{1}{n}} = 1$.
\end{solution}
\item $\displaystyle \lim_{n\to +\infty}(2\sin^2n+\cos^2n)^{\frac{1}{n}}$
\begin{solution}
$1 \leq \displaystyle \lim_{n\to +\infty}(2\sin^2n+\cos^2n)^{\frac{1}{n}}\leq \lim_{n\to +\infty}\sqrt[n]2 = 1$. 
于是 $\displaystyle \lim_{n\to +\infty}(2\sin^2n+\cos^2n)^{\frac{1}{n}} = 1$.
\end{solution}
\item $\displaystyle \lim_{n\to +\infty}(\arctan n)^{\frac{1}{n}}$
\begin{solution}
$1 \leq \displaystyle \lim_{n\to +\infty}(\arctan n)^{\frac{1}{n}}\leq \lim_{n\to +\infty}\sqrt[n]{\frac{\pi}{2}} = 1$. 
于是 $\displaystyle \lim_{n\to +\infty}(\arctan n)^{\frac{1}{n}} = 1$.
\end{solution}
\item $\displaystyle \lim_{n\to +\infty}\frac{1+a+\cdots+a^{n-1}}{1+b+\cdots +b^{n-1}}, |a| < 1, |b| < 1$
\begin{solution}
$\displaystyle \lim_{n\to +\infty}\frac{1+a+\cdots+a^{n-1}}{1+b+\cdots +b^{n-1}} = 
\lim_{n\to +\infty}\left(\frac{1-a^n}{1-a}\right)\left(\frac{1-b}{1-b^n}\right) = \frac{1-b}{1-a}$. 
\end{solution}

\item $\displaystyle \lim_{n\to +\infty}\left(\frac{1}{1\cdot 2}+\frac{1}{2\cdot 3} +\cdots+\frac{1}{n(n+1)}\right)$
\begin{solution}
$\displaystyle \lim_{n\to +\infty}\left(\frac{1}{1\cdot 2}+\frac{1}{2\cdot 3} +\cdots+\frac{1}{n(n+1)}\right) = 
\lim_{n\to +\infty}\left(1 - \frac{1}{n+1}\right) = 1$
\end{solution}

\item $\displaystyle \lim_{n\to +\infty}\left(1-\frac{1}{2^2}\right)\left(1-\frac{1}{3^2}\right)\cdots\left(1-\frac{1}{n^2}\right)$
\begin{solution}
$\displaystyle \lim_{n\to +\infty}\left(1-\frac{1}{2^2}\right)\left(1-\frac{1}{3^2}\right)\cdots\left(1-\frac{1}{n^2}\right) = 
\lim_{n\to +\infty}\left(1 - \frac{1}{2}\right)\left(1 + \frac{1}{n}\right) = \frac{1}{2}$
\end{solution}

\item $\displaystyle \lim_{n\to +\infty}\left(\frac{1}{2}+\frac{3}{2^2}+\cdots+\frac{2n-1}{2^n}\right)$
\begin{solution}
记$$S_n = \frac{1}{2}+\frac{3}{2^2}+\cdots+\frac{2n-1}{2^n},$$ 则
$$\frac{1}{2}S_n = \frac{1}{2^2}+\frac{3}{2^3}+\cdots+\frac{2(n-1)-1}{2^{n}}+\frac{2n-1}{2^{n+1}}.$$
于是
\begin{equation*}
\begin{split}
\frac{1}{2}S_n &= \frac{1}{2}+\left(\frac{1}{2} + \frac{1}{2^2}+\cdots+\frac{1}{2^{n-1}}\right)-\frac{2n-1}{2^{n+1}}\\
&=\frac{3}{2}-\frac{1}{2^{n-1}}-\frac{2n-1}{2^{n+1}}
\end{split}
\end{equation*}
从而$$\displaystyle \lim_{n\to +\infty}\left(\frac{1}{2}+\frac{3}{2^2}+\cdots+\frac{2n-1}{2^n}\right) = 3.$$
\end{solution}

\item $\displaystyle \lim_{n\to +\infty}\left(1-\frac{1}{1+2}\right)\left(1-\frac{1}{1+2+3}\right)+\cdots+\left(1-\frac{1}{1+2+\cdots+n}\right)$
\begin{solution}
$\displaystyle 1-\frac{1}{1+2+\cdots+k} = \frac{(k-1)(k+2)}{k(k+1)}$.从而
\begin{equation*}
\begin{split}
&\left(1-\frac{1}{1+2}\right)\left(1-\frac{1}{1+2+3}\right)+\cdots+\left(1-\frac{1}{1+2+\cdots+n}\right) \\
&=\frac{1\cdot 4}{2\cdot 3}\frac{2 \cdot 5}{3\cdot 4}\cdots \frac{(n-1)\cdot (n+2)}{n\cdot (n+1)}
\end{split}
\end{equation*}
分子的$2n$项的积:奇数项的积是$(n-1)!$, 偶数项的积是$\frac{1}{2\cdot 3}(n+2)!$.\\
分母的$2n$项的积:奇数项的积是$n!$, 偶数项的积是$\frac{1}{2}(n+1)!$.\\
于是$\displaystyle \lim_{n\to +\infty}\left(1-\frac{1}{1+2}\right)\left(1-\frac{1}{1+2+3}\right)+\cdots+\left(1-\frac{1}{1+2+\cdots+n}\right) = 
\lim_{n\to +\infty}\frac{n+2}{3n} = \frac{1}{3}.$
\end{solution}

\item $\displaystyle \lim_{n\to +\infty}\left[\frac{1^2}{n^3}+\frac{3^2}{n^3}+\cdots+\frac{(2n-1)^2}{n^3}\right]$
\begin{solution}
$$\sum_{k=1}^{n}(2k-1)^2 =\sum_{k=1}^{2n}k^2 - 4\sum_{k=1}^{n}k^2= \frac{8n^3 - 2n}{6}.$$
于是$$\displaystyle \lim_{n\to +\infty}\left[\frac{1^2}{n^3}+\frac{3^2}{n^3}+\cdots+\frac{(2n-1)^2}{n^3}\right] = \frac{4}{3}.$$
\end{solution}

\item $\displaystyle \lim_{n\to +\infty}(1+x)(1+x^2)(1+x^4)\cdots(1+x^{2^{n - 1}})$
\begin{solution}
$\displaystyle \lim_{n\to +\infty}(1+x)(1+x^2)(1+x^4)\cdots(1+x^{2^{n - 1}}) = \displaystyle \lim_{n\to +\infty}\frac{1-x^{2^n}}{1-x}=\frac{1}{1-x}.$
\end{solution}

\item $\displaystyle \lim_{n\to +\infty}(\sqrt{n+2}-2\sqrt{n+1}+\sqrt{n})$
\begin{solution}
$\displaystyle \lim_{n\to +\infty}(\sqrt{n+2}-2\sqrt{n+1}+\sqrt{n})=
\lim_{n\to +\infty}\left(\frac{1}{\sqrt{n+2}+\sqrt{n+1}}-\frac{1}{\sqrt{n+1}+\sqrt{n}}\right)=0$.
\end{solution}
\end{enumerate}
\end{example}
%%%%%%%%%%%%%%%%%%%%%%%%
\begin{example}
设$a_n > 0$,$n\in\mathbb{N}$,$\displaystyle \lim_{n\to +\infty}\frac{a_{n+1}}{a_n} = a$。应用例1.2.6证明:$\displaystyle \lim_{n\to +\infty}\sqrt[n]{a_n} = a$.
\end{example}
\begin{proof}
$$\sqrt[n]{a_n} = \sqrt[n]{\frac{a_n}{a_{n-1}}\cdot \frac{a_{n-1}}{a_{n-2}}\cdots\frac{a_2}{a_1}}\cdot \sqrt[n]{a_1}$$
于是$\displaystyle \lim_{n\to +\infty}\sqrt[n]{a_n} = a$.
\end{proof}
%%%%%%%%%%%%%%%%%%%%%%%

\begin{example}
设$\displaystyle \lim_{n\to +\infty}a_n = a$。应用夹逼定理证明:$\displaystyle \lim_{n\to +\infty}\frac{[na_n]}{n} = a$,其中$[x]$表示不超过的最大整数。
\end{example}
\begin{proof}
$a=\displaystyle \lim_{n\to +\infty}\frac{na_n-1}{n} \leq \displaystyle \lim_{n\to +\infty}\frac{[na_n]}{n} \leq 
\displaystyle \lim_{n\to +\infty}\frac{na_n}{n} = a.$
\end{proof}
%%%%%%%%%%%%%%%%%%%%%%
\begin{example}
设$a_n \neq 0$且$\displaystyle \lim_{n\to +\infty}\left|\frac{a_{n+1}}{a_n}\right| = r > 1$。证明:$\displaystyle \lim_{n\to +\infty}a_n = \infty$.
\end{example}
\begin{proof}
取$\varepsilon = \frac{r - 1}{2}$.由极限的定义,存在$N\in\mathbb{N}$使得$\left|\frac{a_{n+1}}{a_n}\right| > r - \varepsilon = \frac{r + 1}{2} > 1$. 
于是$$\left|a_n\right| > \left(\frac{r+1}{2}\right)^{n_N}\left|a_N\right|.$$
即$\displaystyle \lim_{n\to +\infty}a_n = \infty$。
\end{proof}
%%%%%%%%%%%%%%%%%%%%%%
\begin{example}
\renewcommand\labelenumi{\normalfont(\theenumi)}
\begin{enumerate}
\item 应用数学归纳法或$\displaystyle\frac{2k-1}{2k}<\frac{2k}{2k+1}$证明不等式:
$$\frac{1}{2}\cdot\frac{3}{4}\cdot\cdots\cdot\frac{2n-1}{2n} < \frac{1}{\sqrt{2n+1}}.$$
\begin{proof}
记$S_n = \displaystyle\frac{1}{2}\cdot\frac{3}{4}\cdot\cdots\cdot\frac{2n-1}{2n}$. 利用不等式$\displaystyle\frac{2k-1}{2k}<\frac{2k}{2k+1}$, 我们有
$$S_n < \frac{2}{3}\cdot\frac{4}{5}\cdot\cdots\cdot\frac{2n}{2n+1} =\frac{1}{S_n(2n+1)}.$$
于是$S_n < \displaystyle\frac{1}{\sqrt{2n+1}}$.
\end{proof}

\item 证明:$\displaystyle \lim_{n\to +\infty}\left(\frac{1}{2}\cdot\frac{3}{4}\cdot\cdots\cdot\frac{2n-1}{2n}\right) = 0$
\begin{proof}
$0<\displaystyle \lim_{n\to +\infty}\left(\frac{1}{2}\cdot\frac{3}{4}\cdot\cdots\cdot\frac{2n-1}{2n}\right) \leq \lim_{n\to +\infty}\frac{1}{\sqrt{2n+1}}=0$.
\end{proof}
\end{enumerate}
\end{example}
%%%%%%%%%%%%%%%%%%%%

\begin{example}
设$a_n > 0 (n\in\mathbb{N})$且$\displaystyle \lim_{n\to +\infty}a_n = a > 0$。应用夹逼定理证明:$\displaystyle \lim_{n\to +\infty}\sqrt[n]{a_n} = 1$
\end{example}
\begin{proof}
由极限的定义知,存在$N>0$使得$$\frac{a}{2}\leq a_n \leq \frac{3a}{2},\quad\forall n > N.$$
于是$1=\displaystyle \lim_{n\to +\infty}\sqrt[n]{\frac{a}{2}}\leq \displaystyle \lim_{n\to +\infty}\sqrt[n]{a_n} \leq \displaystyle \lim_{n\to +\infty}\sqrt[n]{\frac{3a}{2}}=1$.
\end{proof}
%%%%%%%%%%%%%%%%%%%%%%%
\begin{example}
证明$\displaystyle \lim_{n\to +\infty}\frac{\displaystyle\sum_{k=1}^nk!}{n!} = 1$:$\left(\text{提示}: 1+\displaystyle\frac{1}{n}\leq\frac{\displaystyle\sum_{k=1}^nk!}{n!}\leq 1+ \displaystyle\frac{2}{n}\right)$
\end{example}
\begin{proof}
\begin{equation*}
\begin{split}
1+\frac{1}{n} &= \frac{(n-1)!+n!}{n!} \\&< \frac{\displaystyle\sum_{k=1}^nk!}{n!} \\&< \frac{(n-1)(n-2)!+(n-1)!+n!}{n!} \\
&= 1 +\frac{2}{n}
\end{split}
\end{equation*}
于是$\displaystyle \lim_{n\to +\infty}\frac{\displaystyle\sum_{k=1}^nk!}{n!} = 1$.
\end{proof}
%%%%%%%%%%%%%%%%%%%%%%%
\begin{example}
设$\displaystyle \lim_{n\to +\infty}a_n = a$,$\displaystyle \lim_{n\to +\infty}b_n = b$。记
$$S_n=\max\{a_n, b_n\}, \quad T_n = \min\{a_n, b_n\}, \quad n = 1,2,\cdots.$$
应用$\varepsilon-N$法$(\text{分} a < b, a>b, a=b)$或$\max\{a_n,b_n\} = \frac{1}{2}(a_n+b_n+|a_n-b_n|$与$\min\{a_n,b_n\} = \frac{1}{2}(a_n+b_n-|a_n-b_n|)$,
证明:
$$(1)\quad \displaystyle \lim_{n\to +\infty}S_n = \max\{a, b\}; \quad (2)\quad \displaystyle \lim_{n\to +\infty}T_n = \min\{a, b\}.$$
\end{example}
\begin{proof}
显然我们有$$\displaystyle \lim_{n\to +\infty}|a_n-b_n| = |a-b|.$$
由此可知:
$$\displaystyle \lim_{n\to +\infty}S_n = \frac{1}{2}(a+b+|a-b|) = \max\{a,b\},$$
$$\displaystyle \lim_{n\to +\infty}T_n = \frac{1}{2}(a+b-|a-b|) = \min\{a,b\}.$$
\end{proof}
%%%%%%%%%%%%%%%%%%%%%%%%

\begin{example}
应用例1.1.7与例1.1.15证明:
$$\displaystyle \lim_{n\to +\infty}\frac{1+\sqrt{2}+\sqrt[3]{3}+\cdots+\sqrt[n]{n}}{n} = 1.$$
\end{example}
\begin{proof}
取$a_n = \sqrt[n]{n}$,显然$$\displaystyle \lim_{n\to +\infty}a_n=1.$$于是
$\displaystyle \lim_{n\to +\infty}\frac{1+\sqrt{2}+\sqrt[3]{3}+\cdots+\sqrt[n]{n}}{n} = \displaystyle \lim_{n\to +\infty}\frac{a_1+a_2+\cdots+a_n}{n} = 1$.
\end{proof}
%%%%%%%%%%%%%%%%%%%%%%%%
\begin{example}
证明:$\displaystyle \lim_{n\to +\infty}\left(\sin\frac{\ln 2}{2}+\sin\frac{\ln 3}{3} +\cdots+\sin\frac{\ln n}{n}\right)^{\frac{1}{n}} = 1$
\end{example}
\begin{proof}
$1=\displaystyle \lim_{n\to +\infty}\left(\sin\frac{\ln 2}{2}\right)^{\frac{1}{n}}\leq\displaystyle \lim_{n\to +\infty}\left(\sin\frac{\ln 2}{2}+\sin\frac{\ln 3}{3} +\cdots+\sin\frac{\ln n}{n}\right)^{\frac{1}{n}}\leq \displaystyle \lim_{n\to +\infty}\sqrt[n]{n} = 1.
$\end{proof}
%%%%%%%%%%%%%%%%%%%%%%%%%
\begin{example}
证明:$\displaystyle \lim_{n\to +\infty}\displaystyle\sum_{k=n^2}^{(n+1)^2}\frac{1}{\sqrt k}= 2$.
\end{example}
\begin{proof}
$$\frac{2n+2}{n+1} = \displaystyle\sum_{k=n^2}^{(n+1)^2}\frac{1}{\sqrt{(n+1)^2}} \leq \displaystyle\sum_{k=n^2}^{(n+1)^2}\frac{1}{\sqrt k} \leq \displaystyle\sum_{k=n^2}^{(n+1)^2}\frac{1}{\sqrt{n^2}} = \frac{2n+2}{n}.$$
由夹逼定理,$\displaystyle \lim_{n\to +\infty}\displaystyle\sum_{k=n^2}^{(n+1)^2}\frac{1}{\sqrt k}= 2$。
\end{proof}
%%%%%%%%%%%%%%%%%%%%%%%%
\subsection{思考题}
\begin{example}
用$p(n)$表示能整除$n$的素数的个数。证明:$\displaystyle \lim_{n\to +\infty}\frac{p(n)}{n} = 0.$
\end{example}
\begin{proof}
假设$n=p_1^{m_1}p_2^{m_2}\cdots p_l^{m_l}$, 其中$p_1 < p_2 <\cdots < p_l$是互异的素数,$m_k \geq 1, k=1,2,\cdots, l$。于是$p(n)=\displaystyle\sum_{k=1}^{l}m_k$.
$$\ln{n} = \displaystyle\sum_{k=1}^{l}m_k\ln{p_k} \geq \displaystyle\sum_{k=1}^{p(n)}\ln{2}=p(n)\ln{2}.$$
因此
$$0 \leq \frac{p(n)}{n} \leq \frac{\ln{n}}{n\ln{2}}.$$
由夹逼定理可知,$\displaystyle \lim_{n\to +\infty}\frac{p(n)}{n} = 0.$
\end{proof}
%%%%%%%%%%%%%%%%%%%%%%%
\begin{example}
设$x_n = \displaystyle \sum_{k=1}^n\left(\sqrt{1+\frac{k}{n^2}} -1\right)$。证明:$\displaystyle \lim_{n\to +\infty}x_n = \frac{1}{4}$.
\end{example}
\begin{proof}
\begin{equation*}
\begin{split}
\frac{n(n+1)}{2n^2\left(\sqrt{1 +\frac{1}{n}} + 1\right)}&= \frac{1}{\sqrt{1 +\frac{1}{n}} + 1}\displaystyle \sum_{k=1}^n\frac{k}{n^2} \\&< \displaystyle \sum_{k=1}^n\frac{\frac{k}{n^2}}{\sqrt{1+\frac{k}{n^2}}+1} \\&= \displaystyle\sum_{k=1}^n\left(\sqrt{1+\frac{k}{n^2}}-1\right) \\
&< \frac{1}{2}\displaystyle \sum_{k=1}^n\frac{k}{n^2}\\&= \frac{n(n+1)}{4n^2}.
\end{split}
\end{equation*}
由夹逼定理可知,$\displaystyle \lim_{n\to +\infty}x_n = \frac{1}{4}$。
\end{proof}
%%%%%%%%%%%%%%%%%%%%%%%%%%
\section{实数理论,实数连续性命题}
\subsection{练习题}
\subsection{思考题}

%%%%%%%%%%%%%%%%%%%%%%%%%%%%%%
\section{Cauchy收敛准则(原理),单调数列的极限,数$e=\displaystyle \lim_{n\to +\infty}\left(1+\frac{1}{n}\right)^n$}
\subsection{练习题}
\begin{example}
证明下列数列收敛:
\renewcommand\labelenumi{\normalfont(\theenumi)}
\begin{enumerate}
\item $\displaystyle\left(1-\frac{1}{2}\right)\left(1-\frac{1}{2^2}\right)\cdots\left(1-\frac{1}{2^n}\right), n\in\mathbb{N}$;
\begin{proof}
记$S_n = \left(1-\frac{1}{2}\right)\left(1-\frac{1}{2^2}\right)\cdots\left(1-\frac{1}{2^n}\right)$, 我们有$S_{n+1} = S_n\left(1-\frac{1}{2^{n+1}}\right) < 
S_n$.很显然$S_n > 0,\forall n\in\mathbb{N}$. 由实数连续性命题(二)可知,$S_n$收敛。
\end{proof}

\item $\displaystyle\frac{10}{1}\cdot\frac{11}{3}\cdots\frac{n+9}{2n-1}, n\in\mathbb{N}$.
\begin{proof}
记$S_n = \displaystyle\frac{10}{1}\cdot\frac{11}{3}\cdots\frac{n+9}{2n-1}$. 当$n>10$时,$\frac{n+9}{2n-1} < 1$.即$S_{n+1} < S_{n},\forall n \in \mathbb{N}, n > 10$.另一方面
$S_n > 0, \forall n\in\mathbb{N}$. 由实数连续性命题(二)可知,$S_n$收敛。
\end{proof}
\end{enumerate}
\end{example}
%%%%%%%%%%%%%%%%%%%%%%
\begin{example}
设$0<a_n<1$且$a_{n+1}(1-a_n)\geq \frac{1}{4}$,$n\in\mathbb{N}$。证明:$\{a_n\}$收敛,且$\displaystyle \lim_{n\to +\infty}a_n = \frac{1}{2}$。
\end{example}
\begin{proof}
考虑函数$f(x) = (1-x)x, x\in (0,1)$, 我们有$$f(x) > 0, f(x) \leq \frac{1}{4}, x\in (0,1).$$
所以$\displaystyle\frac{a_{n+1}}{a_n} \geq \frac{1}{4(1-a_n)a_n} \geq 1$,即$\{a_n\}$是单调递增。由实数连续性命题(二)可知,$a_n$收敛。由递推公式可知,$\frac{1}{4}\geq a(1-a)\geq \frac{1}{4}$.所以$a=\frac{1}{2}$, 即$\displaystyle \lim_{n\to +\infty}a_n = \frac{1}{2}$。
\end{proof}
%%%%%%%%%%%%%%%%%%%%%%
\begin{example}
给定两正数$x_0=a$与$y_0=b$,归纳定义$$x_n=\sqrt{x_{n-1}y_{n-1}}, \quad y_n=\frac{x_{n-1}+y_{n-1}}{2},$$ $n=1,2,\cdots$。证明:数列$\{x_n\}$与$\{y_n\}$收敛,
且$\displaystyle \lim_{n\to +\infty}x_n=\displaystyle \lim_{n\to +\infty}y_n$,并称此极限为与的算术-几何平均数。
\end{example}
\begin{proof}
由算术-几何平均不等式知:$x_n \leq y_n, \forall n\in\mathbb{N}$。
于是:
$$x_{n+1} = \sqrt{x_{n}y_{n}} \geq \sqrt{x_{n}x_{n}} = x_n,\quad\forall n = 1,2,\cdots,$$
$$y_{n+1} = \frac{x_{n}+y_{n}}{2} \leq \frac{y_{n}+y_{n}}{2} = y_n, \quad\forall n = 1,2,\cdots.$$
于是$$a = x_0 \leq x_1 \leq \cdots \leq x_n \leq\cdots \leq y_n \leq y_1 \leq y_0 = b.$$
令$\displaystyle \lim_{n\to +\infty}x_n=A, \displaystyle \lim_{n\to +\infty}y_n=B$。由递推公式可知:$A=\sqrt{AB}$,从而$A = B$.
\end{proof}
%%%%%%%%%%%%%%%%%%%%%%
\begin{example}
$\forall n\in\mathbb{N}$,用$x_n$表示方程$x+x^2+\cdots+x^n=1$在闭区间$[0,1]$上的根,求极限$\displaystyle \lim_{n\to +\infty}x_n$.
\end{example}
\begin{solution}
设$f_n(x) = x+x^2+\cdots+x^n -1$
对于给定的$n\in \mathbb{N}$, $f_n(x)$在$[0,1]$是单调增函数,所以$f_n(x)$只会有唯一的根$x_n$。
由于$$f_{n+1}(x_{n+1}) = 0 < f_{n}(x_n) + x_n^{n+1} = f_{n+1}(x_n),$$
所以$$x_{n+1} \leq x_n,\forall n\in\mathbb{N}.$$
由实数连续性命题(二)可知, $\displaystyle \lim_{n\to +\infty}x_n$存在。由于$\displaystyle\frac{x_n-x_n^{n+1}}{1-x_n} = 1$知,
$\displaystyle \lim_{n\to +\infty}x_n=\frac{1}{2}$.
\end{solution}
%%%%%%%%%%%%%%%%%%%%%
\begin{example}
设$c> 0$,$x_1=\sqrt{c}$,$\displaystyle x_2=\sqrt{c+\sqrt{c}}$,$x_{n+1}=\sqrt{c+x_n}$。证明:数列$\{x_n\}$收敛,且$\displaystyle \lim_{n\to +\infty}x_n=\frac{1+\sqrt{1+4c}}{2}$.
\end{example}
\begin{proof}
我们用归纳法证明$x_n \leq \displaystyle\frac{1+\sqrt{1+4c}}{2}, \forall n\in\mathbb{N}$.
\renewcommand\labelenumi{\normalfont(\theenumi)}
\begin{enumerate}
\item $x_1 = \sqrt{c} < \displaystyle\frac{1+\sqrt{1+4c}}{2}$.
\item 假设$x_k < \displaystyle\frac{1+\sqrt{1+4c}}{2}$。我们证明$x_{k+1} < \displaystyle\frac{1+\sqrt{1+4c}}{2}$.
\begin{equation*}
\begin{split}
x_{k+1} &=\sqrt{c+x_{k}} \\&< \sqrt{c+\frac{1+\sqrt{1+4c}}{2}}\\&=\sqrt{\frac{4c+2+2\sqrt{1+4c}}{4}}\\
&=\frac{1+\sqrt{1+4c}}{2}.
\end{split}
\end{equation*}
\end{enumerate}

现在考虑函数$f(x) =c+x-x^2$。很显然$$f(x) > 0, \quad x\in \left(0, \displaystyle\frac{1+\sqrt{1+4c}}{2}\right).$$
于是$x_{n+1}^2 -x_{n}^2 = c+ x_{n}-x_n^{2} > 0$.即$\{x_n\}$是单调递增的数列。由实数连续性命题(二)知,数列$\{x_n\}$收敛. 设$\displaystyle \lim_{n\to +\infty}x_n=a$,则
$$a=\sqrt{c+a}.$$解方程得$\displaystyle \lim_{n\to +\infty}x_n=a =\frac{1+\sqrt{1+4c}}{2}.$
\end{proof}
%%%%%%%%%%%%%%%%%%%%
\begin{example}
设$x_1=c>0$,令$\displaystyle x_{n+1} = c + \frac{1}{x_n}$,$n\in\mathbb{N}$。求极限$\displaystyle \lim_{n\to +\infty}x_n$.
\end{example}
\begin{solution}
我们首先证明:
\renewcommand\labelenumi{\normalfont(\theenumi)}
\begin{enumerate}
\item $x_{2k-1} < x_{2k+1},\quad\forall k\in\mathbb{N}$;
\item $x_{2k} > x_{2(k+1)},\quad\forall k\in\mathbb{N}$.
\end{enumerate}
由递推公式可知
$$x_{n}-x_{n-2} = \frac{x_{n-3} - x_{n-1}}{x_{n-1}\cdot x_{n-3}} = \frac{x_{n-2} - x_{n-4}}{x_{n-1}\cdot x_{n-2}\cdot x_{n-3}\cdot x_{n-4}}.$$
由于$x_3-x_1 = \displaystyle\frac{1}{x_2} > 0$, 从而知$(1)$成立。由于$x_4-x_2 = \displaystyle\frac{x_1-x_3}{x_1\cdot x_3} < 0$,从而知$(2)$成立。

另一方面:
$$x_{2k}-x_1 = c+\frac{1}{x_{2k-1}}-c = \frac{1}{x_{2k-1}} > 0,\quad \forall k\in\mathbb{N};$$
$$x_{2k+1} - x_2 = c+\frac{1}{x_{2k}} - c -\frac{1}{x_1} = \frac{x_1 - x_{2k}}{x_1\cdot x_{2k}} < 0,\quad\forall k\in\mathbb{N}.$$

由实数连续性命题(二)可知,奇数列和偶数列都是收敛子列。假设
$$\displaystyle \lim_{k\to +\infty}x_{2k-1} = a, \lim_{k\to +\infty}x_{2k} = b.$$
由递推公式,我们有
$$x_{2k+1} =c+\frac{1}{c+\frac{1}{x_{2k-1}}}\Rightarrow a^2-ac-1 = 0\Rightarrow a = \frac{c+\sqrt{c^2 + 4}}{2},$$
$$x_{2k+2} =c+\frac{1}{c+\frac{1}{x_{2k}}}\Rightarrow b^2-bc-1 = 0\Rightarrow b = \frac{c+\sqrt{c^2 + 4}}{2}.$$
即,$\{x_n\}$收敛,且$\displaystyle \lim_{n\to +\infty}x_n=\frac{c+\sqrt{c^2+4}}{2}$.
\end{solution}
%%%%%%%%%%%%%%%%%%%

\begin{example}
证明:$\displaystyle \sqrt{1+\sqrt{1+\sqrt{1+\cdots}}}=\frac{1+\sqrt{5}}{2} =  1 + \displaystyle \frac{1}{1+\displaystyle \frac{1}{1+\cdots}}$
\end{example}
\begin{proof}
第一个等式是题5的特例:$c=1$.第二个等式是题6的特例:$c=1$.
\end{proof}
%%%%%%%%%%%%%%%%%%
\begin{example}
设$c>0$,$a_1=\frac{c}{2}$,$a_{n+1}=\displaystyle\frac{c}{2}+\frac{a_n^2}{2}$,$n=1,2,\cdots$。证明:
\begin{equation*}
\displaystyle \lim_{n\to +\infty}a_n=
\begin{cases}
1-\sqrt{1-c},\quad&0<c\leq 1,\\
+\infty, \quad&c > 1.
\end{cases}
\end{equation*}
\end{example}
\begin{proof}
当$c>1$时,由递推公式可知,$$a_{n+1}\geq 2\displaystyle\sqrt{\frac{c}{2}{\frac{a_n^2}{2}}}=\sqrt{c}a_n\geq \cdots\geq c^{\frac{n}{2}}a_1.$$
所以$+\infty \geq \displaystyle \lim_{n\to +\infty}a_n \geq \displaystyle \lim_{n\to +\infty}\left(c^{\frac{n}{2}}a_1\right) = +\infty.$

当$0<c\leq 1$时,我们证明数列$\{a_n\}$是单调递增且有界。
\renewcommand\labelenumi{\normalfont(\theenumi)}
\begin{enumerate}
\item $a_1 = \displaystyle\frac{c}{2} < 1-\sqrt{1-c}$.
\item 设$a_k < 1-\sqrt{1-c}$.下面我们证明$a_{k+1} \geq a_{k}$且$a_{k+1} < 1-\sqrt{1-c}$.
$$a_{k+1} = \frac{c}{2} + \frac{a_{k}^2}{2} <\frac{c}{2} + \frac{1}{2}\left(1-\sqrt{1-c}\right)^2 = 1-\sqrt{1-c}.$$
考察函数$f(x) = x^2-2x+c$. $$f(x) > 0, \quad x\in(-\infty, 1-\sqrt{1-c}).$$
因此 $$a_{k+1} - a_{k} = \frac{1}{2}(a_k^2-2a_k+c) > 0.$$
即$\{a_n\}$是单调增的数列. 由实数连续性命题(二)可知, $\displaystyle \lim_{n\to +\infty}a_n$存在。设极限为$a$, 则$$a=\frac{c}{2}+\frac{a^2}{2}\Rightarrow a = 1-\sqrt{1-c}.$$
\end{enumerate}
命题在$0<c\leq$也得证了。\end{proof}
%%%%%%%%%%%%%%%%%%%%%%
\begin{example}
设数列$\{a_n\}$单调增,$\{b_n\}$单调减,且$\displaystyle \lim_{n\to +\infty}(a_n-b_n)=0$。证明:$\{a_n\}$与$\{b_n\}$都收敛,且$\displaystyle \lim_{n\to +\infty}a_n=\displaystyle \lim_{n\to +\infty}b_n$.
\end{example}
\begin{proof}
很显然$\{a_n-b_n\}$是单调增。又由于$\displaystyle \lim_{n\to +\infty}(a_n-b_n)=0$, 可知 $a_n \leq b_n,\forall n\in\mathbb{N}$, 从而
$$a_1 \leq a_2 \leq\cdots\leq a_n \leq \cdots \leq b_n \leq \cdots \leq b_2\leq b_1.$$
由实数连续性命题(二)可知, $\{a_n\}$与$\{b_n\}$都收敛. 再由$\displaystyle \lim_{n\to +\infty}(a_n-b_n)=0$知,$\displaystyle \lim_{n\to +\infty}a_n=\displaystyle \lim_{n\to +\infty}b_n$。
\end{proof}
%%%%%%%%%%%%%%%%%%%%%
\begin{example}
设数列$\{a_n\}$满足:存在正数$M$,$\forall n\in\mathbb{N}$,有$$A_n = \left|a_2-a_1 \right| +\left|a_3-a_2 \right| + \left|a_n-a_{n-1} \right| \leq M.$$
证明:数列$\{a_n\}$与$\{A_n\}$都收敛。
\end{example}
\begin{proof}
很显然数列$\{A_n\}$是单调增有界数列,由实数连续性命题(二)可知,$\{A_n\}$是收敛的。
$$\{A_n\}\text{收敛}\Rightarrow \{A_n\}\text{是Cauchy列}\Rightarrow \{a_n\}\text{是Cauchy列} \Rightarrow \{a_n\}\text{收敛}.$$
Cauchy数列必收敛,从而知$\{a_n\}$收敛。
\end{proof}
%%%%%%%%%%%%%%%%%%%%%%
\begin{example}
应用Cauchy收敛准则证明下列数列收敛:
\renewcommand\labelenumi{\normalfont(\theenumi)}
\begin{enumerate}
\item $\displaystyle x_n=\frac{\cos{1!}}{1\cdot 2}+\frac{\cos{2!}}{2\cdot 3}+\cdots+\frac{\cos{n!}}{n\cdot (n+1)}$;
\begin{proof}
$$\left|a_{n+p} - a_{n}\right| = \left|\frac{\cos{(n+1)!}}{(n+1)\cdot (n+2)}+\cdots+\frac{\cos{(n+p)!}}{(n+p)\cdot (n+p+1)}\right|\leq 
\frac{1}{n+1} - \frac{1}{n+p+1} < \frac{1}{n+1}.$$
即$\{x_n\}$是Cauchy列,从而收敛。
\end{proof}
\item $\displaystyle x_n=1+\frac{1}{2^2}+\frac{1}{3^2}+\cdots+\frac{1}{n^2}$;
\begin{proof}
$$\left|x_{n+p}-x_n\right| = \frac{1}{(n+1)^2}+\frac{1}{(n+2)^2}+\cdots+\frac{1}{(n+p)^2}<\frac{1}{n}-\frac{1}{n+p}<
\frac{1}{n}.$$
即$\{x_n\}$是Cauchy列,从而收敛。
\end{proof}
\item $\displaystyle x_n = \frac{\arctan{1}}{1(1+\cos{1!})}+\frac{\arctan{2}}{2(2+\cos{2!})}+\cdots+\frac{\arctan{n}}{n(n+\cos{n!})}$.
\begin{proof}
\begin{equation*}
\begin{split}
\left|x_{n+p}-x_{n}\right| &= \left|\frac{\arctan{(n+1)}}{(n+1)((n+1)+\cos{(n+1)!})}+\cdots+\frac{\arctan{(n+p)}}{(n+p)((n+p)+\cos{(n+p)!})}\right| \\
&<\frac{\pi}{2}\left(\frac{1}{n(n+1)}+\cdots+\frac{1}{(n+p-1)(n+p)}\right)\\
&=\frac{\pi}{2}\left(\frac{1}{n}-\frac{1}{n+p}\right)\\
&<\frac{\pi}{2n}.
\end{split}
\end{equation*}
即$\{x_n\}$是Cauchy列,从而收敛。
\end{proof}
\end{enumerate}
\end{example}
%%%%%%%%%%%%%%%%%%%%%%
\begin{example}
应用$\displaystyle \lim_{n\to +\infty}\left(1+\frac{1}{n}\right)^n = e$与$\displaystyle \lim_{n\to +\infty}\left(1-\frac{1}{n}\right)^n = e^{-1}$,求下列极限:
\renewcommand\labelenumi{\normalfont(\theenumi)}
\begin{enumerate}
\item $\displaystyle \lim_{n\to +\infty}\left(1+\frac{1}{n-3}\right)^n;$
\begin{solution}
$\displaystyle \lim_{n\to +\infty}\left(1+\frac{1}{n-3}\right)^n=\displaystyle \lim_{n\to +\infty}\left(1+\frac{1}{n-3}\right)^{(n-3)\frac{n}{n-3}}=e.$
\end{solution}
\item $\displaystyle \lim_{n\to +\infty}\left(1-\frac{1}{n-2}\right)^n;$
\begin{solution}
$\displaystyle \lim_{n\to +\infty}\left(1-\frac{1}{n-2}\right)^n=\displaystyle \lim_{n\to +\infty}\left(1-\frac{1}{n-2}\right)^{(-n+2)\frac{n}{-n+2}}=e^{-1}.$
\end{solution}
\item $\displaystyle \lim_{n\to +\infty}\left(\frac{1+n}{2+n}\right)^n;$
\begin{proof}
$\displaystyle \lim_{n\to +\infty}\left(\frac{1+n}{2+n}\right)^n = \lim_{n\to +\infty}\left(1-\frac{1}{2+n}\right)^{(-2-n)\frac{n}{-2-n}}=e^{-1}$.
\end{proof}
\item $\displaystyle \lim_{n\to +\infty}\left(1+\frac{1}{2n^2}\right)^{4n^2};$
\begin{proof}
$\displaystyle \lim_{n\to +\infty}\left(1+\frac{1}{2n^2}\right)^{4n^2} = \lim_{n\to +\infty}\left(\left(1+\frac{1}{2n^2}\right)^{2n^2}\right)^2=e^2$.
\end{proof}
\item $\displaystyle \lim_{n\to +\infty}\left(1+\frac{3}{n}\right)^{n}$.
\begin{proof}
$\displaystyle \lim_{n\to +\infty}\left(1+\frac{3}{n}\right)^{n}=\lim_{n\to +\infty}\left(\left(1+\frac{3}{n}\right)^{\frac{n}{3}}\right)^3=e^3$.
\end{proof}
\end{enumerate}
\end{example}
%%%%%%%%%%%%%%%%%%%%
\begin{example}
$\forall n\in\mathbb{N}$,证明:
\renewcommand\labelenumi{\normalfont(\theenumi)}
\begin{enumerate}
\item $0<e-\left(1+\frac{1}{n}\right)^n <\frac{3}{n}$
\begin{proof}
由不等式$\left(1+\frac{1}{n}\right)^n<e<\left(1+\frac{1}{n}\right)^{n+1}$知:
$$0<e-\left(1+\frac{1}{n}\right)^n < \left(1+\frac{1}{n}\right)^{n+1} - \left(1+\frac{1}{n}\right)^{n}=\left(1+\frac{1}{n}\right)^{n}\frac{1}{n}<\frac{3}{n}.$$
\end{proof}
\item $\displaystyle \lim_{n\to +\infty}\left[e-\left(1+\frac{1}{n}\right)^n\right]=0$.
\begin{proof}
由(1)和夹逼原理,可知$\displaystyle \lim_{n\to +\infty}\left[e-\left(1+\frac{1}{n}\right)^n\right] = 0$。
\end{proof}
\end{enumerate}
\end{example}
%%%%%%%%%%%%%%%%%%%%%
\begin{example}
设$\alpha<1$,证明:
\renewcommand\labelenumi{\normalfont(\theenumi)}
\begin{enumerate}
\item $\displaystyle  0< n^{\alpha}\left[e-\left(1+\frac{1}{n}\right)^n\right] < \frac{e}{n^{1-\alpha}}.$
\begin{proof}
由不等式$\left(1+\frac{1}{n}\right)^n<e<\left(1+\frac{1}{n}\right)^{n+1}$知:
\begin{equation*}
\begin{split}
0 &< n^{\alpha}\left[e-\left(1+\frac{1}{n}\right)^n\right] \\
&< n^{\alpha}\left[\left(1+\frac{1}{n}\right)^{n+1}-\left(1+\frac{1}{n}\right)^n\right] \\
&=n^{\alpha}\left[\left(1+\frac{1}{n}\right)^n\frac{1}{n}\right] \\&<\frac{e}{n^{1-\alpha}}.
\end{split}
\end{equation*}
\end{proof}
\item $\displaystyle  \lim_{n\to +\infty}n^{\alpha}\left[e-\left(1+\frac{1}{n}\right)^n\right] = 0$.
\begin{proof}
由(1)和夹逼原理可知,$\displaystyle  \lim_{n\to +\infty}n^{\alpha}\left[e-\left(1+\frac{1}{n}\right)^n\right] = 0$。
\end{proof}
\end{enumerate}
\end{example}
\begin{remark}
由上两题可知$\left[e-\left(1+\frac{1}{n}\right)^n\right]$比$\frac{1}{n^{\alpha}},\forall \alpha < 1$.
高阶的无穷小量。
\end{remark}
%%%%%%%%%%%%%%%%%%%%%%
\begin{example}
\renewcommand\labelenumi{\normalfont(\theenumi)}
\begin{enumerate}
\item 设$0<a<b$,$\forall n\in\mathbb{N}$。证明:
$$b^{n+1}-a^{n+1}< (n+1)b^n(b-a),$$
$$a^{n+1}>b^n\left[(n+1)a-nb\right];$$
\begin{proof}
$$b^{n+1}-a^{n+1}=(b-a)(b^{n}+b^{n-1}a+\cdots+a^n) < (n+1)b^n(b-a).$$
由此式可知:
$$a^{n+1} > b^{n+1}-(n+1)b^n(b-a) = b^n\left[b-(n+1)b +(n+1)a\right] = b^n\left[(n+1)a -nb\right].$$
\end{proof}
\item 在(1)中,令$a=1+\displaystyle\frac{1}{n+1}$,$b=1+\displaystyle\frac{1}{n}$推出$\left(1+\frac{1}{n}\right)^n$为严格增的数列;
\begin{proof}
将$a=1+\displaystyle\frac{1}{n+1}, b = 1+\displaystyle\frac{1}{n}$代入(1)中的第二式,可知 
$$\left(1+\frac{1}{n+1}\right)^{n+1} > \left(1+\frac{1}{n}\right)^n\left[(n+1)\left(1+\frac{1}{n+1}\right)
-n\left(1+\frac{1}{n}\right)\right]=\left(1+\frac{1}{n}\right)^n.$$
\end{proof}
\item 在(1)中,令$a=1$,$b=1+\frac{1}{2n}$推出当$n$为偶数时,有$\left(1+\frac{1}{n}\right)^n<4$;由此得到$\forall n\in\mathbb{N}$,有$\left(1+\frac{1}{n}\right)^n<4$,即$4$为该数列的上界,从而$\left(1+\frac{1}{n}\right)^n$收敛。
\begin{proof}
将$a=1, b=1+\displaystyle\frac{1}{2n}$代入(1)的第二个不等式,我们有:
$$1\geq \left(1+\frac{1}{2n}\right)^n\left[(n+1) - n\left(1+\frac{1}{2n}\right)\right],$$
即
$$\left(1+\frac{1}{2n}\right)^n \leq 2\Rightarrow \left(1+\frac{1}{2n}\right)^{2n} \leq 4.$$
由于$\left(1+\frac{1}{n}\right)^n$是单调递增的,我们可知,$\left(1+\frac{1}{n}\right)^{n} \leq 4, \forall n\in\mathbb{N}$.
\end{proof}
\end{enumerate}
\end{example}
%%%%%%%%%%%%%%%%%%%%%%%%
\begin{example}
应用不等式$b^{n+1}-a^{n+1}>(n+1)a^n(b-a)$,$0<a<b$,证明:数列$\left(1+\frac{1}{n}\right)^{n+1}$是严格单减的,并由此推出$\left(1+\frac{1}{n}\right)^{n}$为有界数列。
\end{example}
\begin{proof}
$$b^{n+1}-a^{n+1}=(b-a)(b^{n}+b^{n-1}a+\cdots+a^n) > (n+1)a^n(b-a).$$
即$$b^{n+1}> a^n\left[(n+1)b-na\right].$$
取$a=1+\displaystyle\frac{1}{n+1}, b = 1+\displaystyle\frac{1}{n}$,我们有
\begin{equation*}
\begin{split}
\left(1+\frac{1}{n}\right)^{n+1} &> \left(1+\frac{1}{n+1}\right)^n\left[(n+1)\frac{n+1}{n}-n\frac{n+2}{n+1}\right]\\
&=\left(1+\frac{1}{n+1}\right)^n\left(\frac{n^2+3n+1}{n(n+1)}\right)\\
&=\left(1+\frac{1}{n+1}\right)^{n+2}\left[\frac{n^2+3n+1}{n(n+1)}\cdot\left(\frac{n+1}{n+2}\right)^2\right]\\
&=\left(1+\frac{1}{n+1}\right)^{n+2}\left(\frac{n^3+4n^2+4n+1}{n^3+4n^2+4n}\right)\\
&>\left(1+\frac{1}{n+1}\right)^{n+2}
\end{split}
\end{equation*}
由此可见$\left\{\left(1+\frac{1}{n}\right)^{n+1}\right\}$是严格单调减.
另外
$$\left(1+\frac{1}{n}\right)^{n}< \left(1+\frac{1}{n}\right)^{n+1}<\left(1+\frac{1}{1}\right)^{1+1}=4.$$
即$\left(1+\frac{1}{n}\right)^{n}$为有界数列。
\end{proof}
%%%%%%%%%%%%%%%%%%%%%%%%
\begin{example}
证明:$\left(1+\frac{1}{n}\right)^{n+1} < \frac{3}{n}+\left(1+\frac{1}{n}\right)^n, \forall n\in\mathbb{N}$.
\end{example}
\begin{proof}
$$\left(1+\frac{1}{n}\right)^{n+1} - \left(1+\frac{1}{n}\right)^n = \left(1+\frac{1}{n}\right)^n\left(1+\frac{1}{n} - 1\right)<\frac{3}{n}, \quad\forall n\in\mathbb{N}.$$
\end{proof}
%%%%%%%%%%%%%%%%%%%%%%%
\begin{example}
设$\{a_n\}$为有界数列。记
$$\overline{a}_n = \sup\{a_n, a_{n+1}, \cdots\}, \quad\underline{a}_n= \inf\{a_n, a_{n+1},\cdots\}.$$
证明:
\renewcommand\labelenumi{\normalfont(\theenumi)}
\begin{enumerate}
\item $\forall n\in\mathbb{N}$,有$\overline{a}_n \geq\underline{a}_n;$
\begin{proof}
这是显然的。$\overline{a}_n \geq a_n \geq \underline{a}_n$, $\forall n\in \mathbb{N}$.
\end{proof}
\item $\{\overline{a}_n\}$为单调减有界数列;$\{\underline{a}_n\}$为单调增有界数列,且$\forall n,m\in\mathbb{N}$,有$\overline{a}_n \geq\underline{a}_m;$
\begin{proof}
由$\overline{a}_n$和$\underline{a}_n$的定义可知,
$$\overline{a}_n = \max\{a_n, \overline{a}_{n+1}\}, \quad \underline{a}_n = \min\{a_n, \underline{a}_{n+1}\}.$$
由此可见$\{\overline{a}_n\}$是单调减,$\{\underline{a}_n\}$是单调增, 且
$$\underline{a}_1 \leq \underline{a}_2\leq \cdots\leq \underline{a}_n \cdots \leq \cdots \leq \overline{a}_n \leq \cdots \leq \overline{a}_2 \leq \overline{a}_1.$$
由于数列$\{a_n\}$是有界数列,故$\underline{a}_1$和$\overline{a}_1$都是有界数。命题得证。
\end{proof}
\item 设$\overline{a}=\displaystyle  \lim_{n\to +\infty}\overline{a}_n$,$\underline{a}=\displaystyle  \lim_{n\to +\infty}\underline{a}_n$,则$\overline{a}\geq \underline{a};$
\begin{proof}
应用定理1.2.5,这个命题就可以得证。
\end{proof}
\item $\{a_n\}$收敛$\Leftrightarrow \overline{a}=\underline{a}$.
\begin{proof}
($\Rightarrow$): 由极限定义,对$\forall \varepsilon>0$, 存在$N$使得
$$|a_n-a|<\varepsilon,\forall n>N.$$
于是
$$|\overline{a}_n-a|<\varepsilon, |\underline{a}_n-a|<\varepsilon\forall n>N.$$
由此可知
$$|\overline{a}-a|<\varepsilon, |\underline{a}-a|<\varepsilon.$$
由$\varepsilon$的任意性知,$\overline{a}=\underline{a}=a$.

($\Leftarrow$): 记$\overline{a}=\underline{a}=a$. 再次由极限定义知,对$\forall \varepsilon > 0$,存在$N$使得
$$|\overline{a}- a|<\varepsilon, |\underline{a}-a|<\varepsilon, \forall n> N.$$
从而$$|a_n-a|<\varepsilon,\forall n>N.$$
由极限的定义知,$\displaystyle  \lim_{n\to +\infty}a_n = a$.
\end{proof}
\end{enumerate}
\end{example}
%%%%%%%%%%%%%%%%%%%%
\subsection{思考题}
\begin{example}
设$a_1\geq 0$,$a_{n+1} = \displaystyle\frac{3(1+a_n)}{3+a_n}$,$n=1,2,\cdots$.证明:$\{a_n\}$收敛,且$\displaystyle  \lim_{n\to +\infty}a_n=\sqrt{3}$.
\begin{proof}
很显然$a_n > 0, \forall n$.我们现在计算$\left|a_{n+1}-\sqrt{3}\right|$:
$$\left|a_{n+1}-\sqrt{3}\right| = \left|\frac{3+3a_n - 3\sqrt{3} -\sqrt{3}a_n}{a_n+3}\right|=\left|\frac{(3-\sqrt{3})(a_n-\sqrt{3})}{a_n+3}\right|<\left(\frac{3-\sqrt{3}}{3}\right)\left| a_n-\sqrt{3}\right|.$$
因为$\displaystyle \frac{3-\sqrt{3}}{3} < 1$,所以$\displaystyle  \lim_{n\to +\infty}(a_n-\sqrt{3})=0 \Rightarrow \displaystyle  \lim_{n\to +\infty}a_n=\sqrt{3}.$
\end{proof}
\end{example}
%%%%%%%%%%%%%%%%%%%%%%%%
\begin{example}
设$a>0$,$x_1>0$,$x_{n+1} = \displaystyle\frac{x_n(x_n^2+3a)}{3x_n^2+a}$,$n=1,2,\cdots$。证明:$\{x_n\}$收敛,且$\displaystyle  \lim_{n\to +\infty}x_n=\sqrt{a}$
\end{example}
\begin{proof}我们证明如下结论:
\renewcommand\labelenumi{\normalfont(\theenumi)}
\begin{enumerate}
\item 如果$x_1\leq \sqrt{a}$,则$\{x_n\}$是单调增且$x_n\leq \sqrt{a}, \quad\forall n$;
\item 如果$x_1 > \sqrt{a}$, 则$\{x_n\}$是单调减且$x_n\geq\sqrt{a}, \quad\forall n$;
\end{enumerate}
无论哪种情况发生,由实数连续性命题(二)可知,$\{x_n\}$收敛. 记$\displaystyle  \lim_{n\to +\infty}x_n=b$, 则
$$b=\frac{b(b^2+3a)}{3b^2+a}.$$
解方程得$b=\sqrt{a}$. 即$\displaystyle  \lim_{n\to +\infty}x_n=\sqrt{a}$.

注意到$x_n > 0,\forall n$且
$$x_{n+1}-x_{n} = \frac{2x_n(\sqrt{a}-x_n)(\sqrt{a}+x_n)}{3x_n^2+a}, \quad x_{k+1} - \sqrt{a} = \displaystyle\frac{(x_{k}-\sqrt{a})^3}{3x_k^2+a}.$$
我们很容易看出上述的两个结论都是成立的。
\end{proof}

%%%%%%%%%%%%%%%%%%%
\begin{example}
设$a> 0$,$x_1 = \sqrt[3]{a}$,$x_n=\sqrt[3]{ax_{n-1}}(n> 1)$。证明: $\{x_n\}$收敛,且$\displaystyle  \lim_{n\to +\infty}x_n=\sqrt{a}$
\end{example}
\begin{proof}
我们只需要证明:
\renewcommand\labelenumi{\normalfont(\theenumi)}
\begin{enumerate}
\item 如果$a\geq 1$,则数列$\{x_n\}$是单调递增且$x_n \leq \sqrt{a}, \forall n$
\item 如果$a<1$,则数列$\{x_n\}$是单调递减且$x_n\geq\sqrt{a},\forall n$.
\end{enumerate}
无论上述哪种情况发生,由实数连续性命题(二)可知, $\{x_n\}$收敛. 记$\displaystyle  \lim_{n\to +\infty}x_n=b$, 则
$$b=\sqrt[3]{ab}.$$
解方程得$b=\sqrt{a}$. 即$\displaystyle  \lim_{n\to +\infty}x_n=\sqrt{a}$.

注意到$x_n > 0, \forall n$且
\begin{equation*}
\begin{split}
x_n - x_{n-1} &= \displaystyle\frac{x_{n-1}(\sqrt{a}-x_{n-1})(\sqrt{a}+x_{n-1})}{\left(\sqrt[3]{ax_{n-1}}\right)^2 + \sqrt[3]{ax_{n-1}}x_{n-1}+x_{n-1}^2},\\
x_n - \sqrt{a} &=\sqrt[3]{ax_{n-1}}-\sqrt{a}=\frac{a(x_{n-1}-\sqrt{a})}{\left(\sqrt[3]{ax_{n-1}}\right)^2 + \sqrt[3]{a^{5/2}x_{n-1}}+a}
\end{split}
\end{equation*}
当$a\geq1$, $x_1 = \sqrt[3]{a} \leq \sqrt{a}$.由上两式知,结论(1)成立.\\
当$a < 1$, $x_1 = \sqrt[3]{a} > \sqrt{a}$.由上两式知,结论(2)成立.
\end{proof}
%%%%%%%%%%%%%%%%%%%%
\begin{example}
设$0<a_1<b_1<c_1$。令
$$a_{n+1}=\frac{3}{\displaystyle\frac{1}{a_n}+\frac{1}{b_n}+\frac{1}{c_n}},\quad
b_{n+1} =  \displaystyle\sqrt[3]{a_nb_nc_n},\quad
c_{n+1}= \displaystyle\frac{a_n+b_n+c_n}{3}
$$
证明:$\{a_n\}$,$\{b_n\}$,$\{c_n\}$收敛于同一实数。
\end{example}
\begin{proof}
由定义可知$$0< a_1< a_n \leq b_n \leq c_n < c_1, \quad\forall n >1.$$
于是 $\{a_n\}$单调增,$\{c_n\}$单调减。
由实数连续性命题(二)可知, 数列$\{a_n\}, \{c_n\}$收敛。因为$b_n = 3c_{n+1} - a_n - c_n$可知,数列$\{b_n\}$收敛。

设$\displaystyle  \lim_{n\to +\infty}a_n=a, \displaystyle  \lim_{n\to +\infty}c_n=c, \displaystyle \lim_{n\to +\infty}b_n=b$。易知,$0<a_1 \leq a\leq b\leq c\leq c_1$且
\begin{equation*}
a = \frac{3}{\displaystyle\frac{1}{a}+\frac{1}{b}+\frac{1}{c}},\quad
b = \displaystyle\sqrt[3]{abc},\quad
c = \displaystyle\frac{a+b+c}{3}
\end{equation*}
解方程组可得知$a=b=c$.
\end{proof}
%%%%%%%%%%%%%%%%%%%%%%%%
\begin{example}
设$a_n > 0$,$S_n=a_1+\cdots+a_n$,$T_n=\displaystyle\frac{a_1}{S_1}+\cdots+\frac{a_n}{S_n}$,且$\displaystyle  \lim_{n\to +\infty}S_n=+\infty$。证明:$\displaystyle  \lim_{n\to +\infty}T_n=+\infty.$
\end{example}
\begin{proof}
由于$S_n \rightarrow  +\infty$, 我们可以找到一个子列$\{n_k\}$使得
$$\frac{S_{n_{k-1}}}{S_{n_k}} < \frac{1}{2}, \quad\forall k\in\mathbb{N}.$$于是
\begin{equation*}
\begin{split}
T_{n_k} &= \left(\frac{a_1}{S_1}+\cdots \frac{a_{n_1}}{S_{n_1}}\right) + \left(\frac{a_{n_1+1}}{S_{n_1+1}}+\cdots \frac{a_{n_2}}{S_{n_2}}\right) +\cdots+\left(\frac{a_{n_{k-1}+1}}{S_{n_{k-1}+1}}+\cdots \frac{a_{n_k}}{S_{n_k}}\right) \\
&> \left(\frac{a_1}{S_{n_1}}+\cdots \frac{a_{n_1}}{S_{n_1}}\right) + \left(\frac{a_{n_1+1}}{S_{n_2}}+\cdots \frac{a_{n_2}}{S_{n_2}}\right) +\cdots+\left(\frac{a_{n_{k-1}+1}}{S_{n_k}}+\cdots \frac{a_{n_k}}{S_{n_k}}\right)\\
&=\frac{S_{n_1}- 0}{S_{n_1}} + \frac{S_{n_2}-S_1}{S_{n_2}} + \cdots +  \frac{S_{n_k}-S_{n_{k-1}}}{S_{n_{k}}}\\
&>\frac{k}{2}.
\end{split}
\end{equation*}
即:$\displaystyle \lim_{k\to +\infty}T_{n_k}=+\infty.$
因为$T_n$是单调增的数列,从而$\displaystyle \lim_{n\to +\infty}T_{n}=+\infty.$命题 得证。
\end{proof}
%%%%%%%%%%%%%%%%%%%
\begin{example}
设$a_1=1$,$a_{n+1}=\displaystyle\frac{1}{1+a_n}$,$n=1,2,\cdots$.证明:$\displaystyle  \lim_{n\to +\infty}a_n=\frac{\sqrt{5}-1}{2}.$
\end{example}
\begin{proof}我们只需证明$\left|a_n -\frac{\sqrt{5}-1}{2}\right|$收敛于$0$.
\begin{equation*}
\begin{split}
\left|a_{n+1} - \frac{\sqrt{5}-1}{2}\right| &= \left|\frac{3-\sqrt{5} - (\sqrt{5}-1)a_n}{2(1+a_n)}\right|\\
&=\left|\frac{-(\sqrt{5}-1)\left(a_n-\displaystyle\frac{\sqrt{5}-1}{2}\right)}{2(1+a_n)}\right|\\
&<\left|\frac{\sqrt{5}-1}{2}\right|\cdot\left|a_n-\frac{\sqrt{5}-1}{2}\right|.
\end{split}
\end{equation*}
由此可见,$\left\{\left|a_n-\frac{\sqrt{5}-1}{2}\right|\right\}$收敛于$0$.
\end{proof}
%%%%%%%%%%%%%%%%%%
\begin{example}
设$a_n\geq 0$,$S_n=\displaystyle\sum_{k=1}^na_k$收敛于$S$。证明:$b_n=(1+a_1)(1+a_2)\cdots(1+a_n)$收敛。
\end{example}
\begin{proof}
很显然,$\{b_n\}$是单调增数列。下面证明$\{b_n\}$是有界数列。
由于$a_n\geq 0$且$S_n\rightarrow S$, 则$S_n$是单调增收敛于$S$,所以$$\displaystyle\sum_{k=1}^na_k < S,\quad\forall n\in \mathbb{N}.$$

另一方面,
$$b_n\leq \left(\frac{1}{n}\displaystyle\sum_{k=1}^n(1+a_k)\right)^n\leq \left(1+\frac{S}{n}\right)^n <\left(1+\frac{[S]+1}{n}\right)^n.$$
数列$\left\{\left(1+\displaystyle\frac{[S]+1}{n}\right)^n\right\}$是单调增的,且$$\displaystyle  \lim_{n\to +\infty}\left(1+\frac{[S]+1}{n}\right)^n=e^{[S]+1}.$$
于是
$$b_n \leq e^{[S]+1}, \forall n\in\mathbb{N}.$$
由实数连续性命题(二)可知, 数列$\{b_n\}$是收敛数列。
\end{proof}
%%%%%%%%%%%%%%%%%%%%%%%%%%%%%
\section{上极限与下极限}
\subsection{练习题}
\begin{example}求$\displaystyle\lowlim_{n\to +\infty}a_n$与$\displaystyle\uplim_{n\to +\infty}a_n$:
\renewcommand\labelenumi{\normalfont(\theenumi)}
\begin{enumerate}
\item $a_n = \displaystyle\frac{(-1)^n}{n}+\displaystyle\frac{1+(-1)^n}{2}$;
\begin{solution}
$\displaystyle\lowlim_{n\to +\infty}a_n = 0$, $\displaystyle\uplim_{n\to +\infty}a_n = 1$.
\end{solution}
\item $a_n = n^{(-1)^n}$;
\begin{solution}
$\displaystyle\lowlim_{n\to +\infty}a_n = 0$, $\displaystyle\uplim_{n\to +\infty}a_n = +\infty$.
\end{solution}
\item $a_n = [1+2^{(-1)^nn}]^{\frac{1}{n}}$;
\begin{solution}
$\displaystyle\lowlim_{n\to +\infty}a_n = 1$, $\displaystyle\uplim_{n\to +\infty}a_n = 2$.
\end{solution}
\item $a_n = \frac{n^2}{1+n^2}\cos{\frac{2n\pi}{3}}$;
\begin{solution}
$\displaystyle\lowlim_{n\to +\infty}a_n = -\frac{1}{2}$, $\displaystyle\uplim_{n\to +\infty}a_n = 1$.
\end{solution}
\item $a_n = \frac{n^2+1}{n^2}\sin\frac{\pi}{n}$;
\begin{solution}
$\displaystyle\lowlim_{n\to +\infty}a_n = 0$, $\displaystyle\uplim_{n\to +\infty}a_n = 0$.
\end{solution}
\item $a_n=\displaystyle\sqrt[n]{\left|\cos\frac{n\pi}{3}\right|}$;
\begin{solution}
$\displaystyle\lowlim_{n\to +\infty}a_n = 1$, $\displaystyle\uplim_{n\to +\infty}a_n = 1$.
\end{solution}
\item $a_n = 
\begin{cases}
0, \quad &\text{n为奇数},\\
\frac{n}{\sqrt[n]{n!}},\quad &\text{n为偶数}.
\end{cases}$
\begin{solution}
$\displaystyle\lowlim_{n\to +\infty}a_n = 0$, $\displaystyle\uplim_{n\to +\infty}a_n = e$.
\end{solution}
\end{enumerate}
\end{example}
%%%%%%%%%%%%%%%%%%%
\begin{example}
证明下面各式当两端有意义时成立:
\renewcommand\labelenumi{\normalfont(\theenumi)}
\begin{enumerate}
\item $\displaystyle\lowlim_{n\to +\infty}a_n + \displaystyle\lowlim_{n\to +\infty}b_n \leq \displaystyle\lowlim_{n\to +\infty}(a_n + b_n) \leq \displaystyle\lowlim_{n\to +\infty}a_n + \displaystyle\uplim_{n\to +\infty}b_n$,\\
$\displaystyle\lowlim_{n\to +\infty}a_n + \displaystyle\uplim_{n\to +\infty}b_n \leq \displaystyle\uplim_{n\to +\infty}(a_n + b_n) \leq \displaystyle\uplim_{n\to +\infty}a_n + \displaystyle\uplim_{n\to +\infty}b_n$
\begin{proof}
由上,下极限的定义,我们可以得到两个简单的无需证明的事实:对于任何数列$\{a_n\}$, 
$\{a_{n_k}\}$是一个任意一个子列, 则
\renewcommand\labelenumi{\normalfont(\theenumi)}
\begin{enumerate}
\item $\displaystyle\lowlim_{n\to +\infty}a_n \leq \displaystyle\lowlim_{k\to +\infty}a_{n_k}$
\item $\displaystyle\uplim_{k\to +\infty}a_{n_k} \leq \displaystyle\uplim_{n\to +\infty}a_n$
\end{enumerate}
我们取子列$\{a_{n_k} + b_{n_k}\}$使得
$$\displaystyle\lim_{k\to +\infty}(a_{n_k} + b_{n_k}) = \displaystyle\lowlim_{n\to +\infty}(a_n + b_n).$$
再在子列$\{a_{n_k}\}$中取子列$\{a_{n_{k_l}}\}$使得
$$\displaystyle\lim_{l\to +\infty}a_{n_{k_l}} = \displaystyle\lowlim_{k\to +\infty}a_{n_k}\geq \displaystyle\lowlim_{n\to +\infty}a_n.$$
对于子列$\{b_{n_{k_l}}\}$,我们会有
$$\displaystyle\lim_{l\to +\infty}b_{n_{k_l}} \geq \displaystyle\lowlim_{k\to +\infty}b_{n_k}\geq \displaystyle\lowlim_{n\to +\infty}b_n.$$

从而
\begin{equation*}
\begin{split}
\displaystyle\lowlim_{n\to +\infty}a_n+\displaystyle\lowlim_{n\to +\infty}b_n &\leq \displaystyle\lim_{l\to +\infty}a_{n_{k_l}} + \displaystyle\lim_{l\to +\infty}b_{n_{k_l}} \\&= \displaystyle\lim_{l\to +\infty}(a_{n_{k_l}}+b_{n_{k_l}}) \\&= \displaystyle\lim_{k\to +\infty}(a_{n_k} + b_{n_k})  \\&= \displaystyle\lowlim_{n\to +\infty}(a_n + b_n).
\end{split}
\end{equation*}

取子列$\{a_{n_k}\}$,使得
$$\displaystyle\lim_{k\to +\infty}a_{n_k}=\displaystyle\lowlim_{n\to +\infty}a_n.$$
考虑子列$\{a_{n_k}+b_{n_k}\}$,在其中取子列$\{a_{n_{k_l}} + b_{n_{k_l}}\}$,使得
$$\displaystyle\lim_{l\to +\infty}(a_{n_{k_l}}+b_{n_{k_l}}) = \displaystyle\lowlim_{k\to +\infty}(a_{n_k} + b_{n_k}) \geq \displaystyle\lowlim_{n\to +\infty}(a_n + b_n).$$
对于同样下标的子列$\{b_{n_{k_l}}\}$, 我们有
$$\displaystyle\lim_{l\to +\infty}b_{n_{k_l}} \leq \displaystyle\uplim_{k\to +\infty}b_{n_k}\leq \displaystyle\uplim_{n\to +\infty}b_n.$$

由此可见
\begin{equation*}
\begin{split}
\displaystyle\lowlim_{n\to +\infty}(a_n + b_n) &\leq \displaystyle\lim_{l\to +\infty}(a_{n_{k_l}}+b_{n_{k_l}}) \\&= \displaystyle\lim_{l\to +\infty}a_{n_{k_l}}+\displaystyle\lim_{l\to +\infty}b_{n_{k_l}}\\&\leq \displaystyle\lim_{k\to +\infty}a_{n_k} + \displaystyle\uplim_{n\to +\infty}b_n\\&=\displaystyle\lowlim_{n\to +\infty}a_n+ \displaystyle\uplim_{n\to +\infty}b_n
\end{split}
\end{equation*}

下面我们用类似的办法证明第二式。取子列$\{b_{n_k}\}$使得
$$\displaystyle\lim_{k\to +\infty}b_{n_k} = \displaystyle\uplim_{n\to +\infty}b_n.$$
对于相应的$\{a_{n_k}+b_{n_k}\}$可以取子列$\{a_{n_{k_l}}+b_{n_{k_l}}\}$使得
$$\displaystyle\lim_{l\to +\infty}(a_{n_{k_l}}+b_{n_{k_l}}) = \displaystyle\uplim_{k\to +\infty}(a_{n_k}+b_{n_k}) \leq   \displaystyle\uplim_{n\to +\infty}(a_n+b_n).$$
显然
$$\displaystyle\lim_{l\to +\infty}a_{n_{k_l}}\geq \displaystyle\lowlim_{n\to +\infty}a_n.$$
于是
\begin{equation*}
\begin{split}
\displaystyle\lowlim_{n\to +\infty}a_n+\displaystyle\uplim_{n\to +\infty}b_n &\leq \displaystyle\lim_{l\to +\infty}a_{n_{k_l}} + \displaystyle\lim_{l\to +\infty}b_{n_{k_l}}\\&=\displaystyle\lim_{l\to +\infty}(a_{n_{k_l}}+b_{n_{k_l}}) \\&=\displaystyle\lim_{k\to +\infty}(a_{n_{k}}+b_{n_{k}}) \\&\leq \displaystyle\uplim_{n\to +\infty}(a_n+b_n)
\end{split}
\end{equation*}

取子列$\{a_{n_k} + b_{n_k}\}$,使得
$$\displaystyle\lim_{k\to +\infty}(a_{n_k} + b_{n_k}) =  \displaystyle\uplim_{n\to +\infty}(a_n + b_n).$$
取子列$\{b_{n_{k_l}}\}$使得
$$\displaystyle\lim_{l\to +\infty}b_{n_{k_l}} = \displaystyle\uplim_{k\to +\infty}b_{n_k} \leq \displaystyle\uplim_{n\to +\infty}b_n.$$
另一方面
$$\displaystyle\lim_{l\to +\infty}a_{n_{k_l}}\leq \displaystyle\uplim_{k\to +\infty}a_{n_k} \leq \displaystyle\uplim_{n\to +\infty}a_n.$$
所以
$\displaystyle\uplim_{n\to +\infty}(a_n + b_n) \leq \displaystyle\uplim_{n\to +\infty}a_n + \displaystyle\uplim_{n\to +\infty}b_n.$
\end{proof}
\item 设$\displaystyle\lim_{n\to +\infty}b_n = b$,则\\
$\displaystyle\lowlim_{n\to +\infty}(a_n + b_n) = \displaystyle\lowlim_{n\to +\infty}a_n + b$,\\
$\displaystyle\uplim_{n\to +\infty}(a_n + b_n) = \displaystyle\uplim_{n\to +\infty}a_n + b$
\begin{proof}
这题是上面一题的直接应用。
\end{proof}
\item $\displaystyle\lowlim_{n\to +\infty}(-a_n)=-\displaystyle\uplim_{n\to +\infty}a_n, \displaystyle\uplim_{n\to +\infty}(-a_n)=-\displaystyle\lowlim_{n\to +\infty}a_n$;
\begin{proof}$\displaystyle\inf_{k\geq n}\{-a_k\} = -\displaystyle \sup_{k \geq n}\{a_k\} \Rightarrow \displaystyle\lowlim_{n\to +\infty}(-a_n)=-\displaystyle\uplim_{n\to +\infty}a_n$.\\
$\displaystyle\sup_{k\geq n}\{-a_k\} = -\displaystyle \inf_{k \geq n}\{a_k\} \Rightarrow \displaystyle\uplim_{n\to +\infty}(-a_n)=-\displaystyle\lowlim_{n\to +\infty}a_n$
\end{proof}
\item 设$\{a_n\}$与$\{b_n\}$均为非负数列,则\\
$\displaystyle\lowlim_{n\to +\infty}a_n\cdot \displaystyle\lowlim_{n\to +\infty}b_n \leq \displaystyle\lowlim_{n\to +\infty}a_nb_n\leq \displaystyle\lowlim_{n\to +\infty}a_n\cdot\displaystyle\uplim_{n\to +\infty}b_n$,\\
$\displaystyle\lowlim_{n\to +\infty}a_n\cdot \displaystyle\uplim_{n\to +\infty}b_n \leq \displaystyle\uplim_{n\to +\infty}a_nb_n\leq \displaystyle\uplim_{n\to +\infty}a_n\cdot\displaystyle\uplim_{n\to +\infty}b_n$;
\begin{proof}
我们现证第一式。
取子列$\{a_{n_k}b_{n_k}\}$使得
$$\displaystyle\lim_{k\to +\infty}a_{n_k}b_{n_k}=\displaystyle\lowlim_{n\to +\infty}a_nb_n.$$
对于上述的$\{a_{n_k}\}$,我们再取子列$\{a_{n_{k_l}}\}$,使得
$$\displaystyle\lim_{l\to +\infty}a_{n_{k_l}}=\displaystyle\lowlim_{k\to +\infty}a_{n_k}\geq \displaystyle\lowlim_{n\to +\infty}a_n \geq 0.$$
对于$\{b_{n_{k_l}}\}$, 我们会有
$$\displaystyle\lowlim_{l\to +\infty}b_{n_{k_l}}\geq \displaystyle\lowlim_{k\to +\infty}b_{n_k}\geq \displaystyle\lowlim_{n\to +\infty}b_n \geq 0.$$
所以
\begin{equation*}
\begin{split}
\displaystyle\lowlim_{n\to +\infty}a_n\cdot \displaystyle\lowlim_{n\to +\infty}b_n &\leq \displaystyle\lim_{l\to +\infty}a_{n_{k_l}}\cdot \displaystyle\lowlim_{l\to +\infty}b_{n_{k_l}} \\&= \displaystyle\lim_{l\to +\infty}a_{n_{k_l}}b_{n_{k_l}} \\&= \displaystyle\lim_{k\to +\infty}a_{n_k}b_{n_k} \\&=
\displaystyle\lowlim_{n\to +\infty}a_nb_n 
\end{split}
\end{equation*}
取$\{a_{n_k}\}$使得
$$\displaystyle\lim_{k\to +\infty}a_{n_k}=\displaystyle\lowlim_{n\to +\infty}a_n.$$
对于对应的子列$\{b_{n_k}\}$,我们再取子列$\{b_{n_{k_l}}\}$使得
$$\displaystyle\lim_{l\to +\infty}b_{n_{k_l}} = \displaystyle\uplim_{k\to +\infty}b_{n_k} \leq \displaystyle\uplim_{n\to +\infty}b_n.$$
所以
\begin{equation*}
\begin{split}
\displaystyle\lowlim_{n\to +\infty}a_nb_n &\leq \displaystyle\lowlim_{k\to +\infty}a_{n_k}b_{n_k}\\
&\leq \displaystyle\lowlim_{l\to +\infty}a_{n_{k_l}}b_{n_{k_l}} \\&= \displaystyle\lim_{l\to +\infty}a_{n_{k_l}}\cdot \displaystyle\lim_{l\to +\infty}b_{n_{k_l}} \\&\leq \displaystyle\lowlim_{n\to +\infty}a_n\cdot\displaystyle\uplim_{n\to +\infty}b_n
\end{split}
\end{equation*}

现在我们来证明第二式。取子列$\{b_{n_k}\}$使得
$$\displaystyle\lim_{k\to +\infty}b_{n_k}=\displaystyle\uplim_{n\to +\infty}b_n.$$
对于相对应的$\{a_{n_k}\}$, 我们再取子列$\{a_{n_{k_l}}\}$使得
$$\displaystyle\lim_{l\to +\infty}a_{n_{k_l}} = \displaystyle\lowlim_{k\to +\infty}a_{n_k}\geq \displaystyle\lowlim_{n\to +\infty}a_n.$$
所以
\begin{equation*}
\begin{split}
\displaystyle\lowlim_{n\to +\infty}a_n \cdot \displaystyle\uplim_{n\to +\infty}b_n &\leq \displaystyle\lim_{l\to +\infty}a_{n_{k_l}}\cdot \displaystyle\lim_{l\to +\infty}b_{n_{k_l}} \\&=\displaystyle\lim_{l\to +\infty}a_{n_{k_l}}b_{n_{k_l}} \\&\leq \displaystyle\uplim_{k\to +\infty}a_{n_k}b_{n_k}\\&\leq \displaystyle\uplim_{n\to +\infty}a_nb_n
\end{split}
\end{equation*}
取子列$\{a_{n_k}b_{n_k}\}$使得
$$\displaystyle\lim_{k\to +\infty}a_{n_k}b_{n_k}=\displaystyle\uplim_{n\to +\infty}a_nb_n.$$
在$\{a_{n_k}\}$取收敛子列$\{a_{n_{k_l}}\}$使得
$$\displaystyle\lim_{l\to +\infty}a_{n_{k_l}} =\displaystyle\uplim_{k\to +\infty}a_{n_k}\leq \displaystyle\uplim_{n\to +\infty}a_n.$$
另一方面
$$\displaystyle\uplim_{l\to +\infty}b_{n_{k_l}}\leq \displaystyle\uplim_{k\to +\infty}b_{n_k}\leq \displaystyle\uplim_{n\to +\infty}b_n.$$
所以
\begin{equation*}
\begin{split}
\displaystyle\uplim_{n\to +\infty}a_nb_n &= \displaystyle\uplim_{l\to +\infty}a_{n_{k_l}}b_{n_{k_l}} \\&\leq \displaystyle\lim_{l\to +\infty}a_{n_{k_l}} \cdot\displaystyle\uplim_{l\to +\infty}b_{n_{k_l}}\\&\leq \displaystyle\uplim_{n\to +\infty}a_n\cdot \displaystyle\uplim_{n\to +\infty}b_n
\end{split}
\end{equation*}
\end{proof}
\item 设$\{b_n\}$非负,且$\displaystyle\lim_{n\to +\infty}b_n=b$,则\\
$\displaystyle\lowlim_{n\to +\infty}a_nb_n=b\displaystyle\lowlim_{n\to +\infty}a_n, \displaystyle\uplim_{n\to +\infty}a_nb_n=b\displaystyle\uplim_{n\to +\infty}a_n$;
\begin{proof}
直接用上一题的结论就可以得证。
\end{proof}
\item 设$a_n > 0(n\in\mathbb{N})$, $\displaystyle\lowlim_{n\to +\infty}a_n > 0$,则
$$\displaystyle\uplim_{n\to +\infty}\frac{1}{a_n} = \frac{1}{\displaystyle\lowlim_{n\to +\infty}a_n}.$$
\begin{proof}
因为$a_n > 0, \forall n$, 从而
$$\displaystyle\sup_{k \geq n}\left\{\frac{1}{a_k}\right\} = \frac{1}{\displaystyle\inf_{k \geq n}\{a_k\}}.$$
由上,下极限的定义,命题得证。
\end{proof}
\end{enumerate}
\end{example}
%%%%%%%%%%%%%%%%%%%
\begin{example}
设$a_n>0(n\in\mathbb{N})$,且$\displaystyle\uplim_{n\to +\infty}a_n\cdot\displaystyle\uplim_{n\to +\infty}\frac{1}{a_n}=1$,证明:数列$\{a_n\}$收敛。
\end{example}
\begin{proof}
由题可知$\displaystyle\uplim_{n\to +\infty}a_n > 0$。我们可以证明
$$\displaystyle\lowlim_{n\to +\infty}\frac{1}{a_n} = \frac{1}{\displaystyle\uplim_{n\to +\infty}a_n}.$$
由此可知$$\displaystyle\lowlim_{n\to +\infty}\frac{1}{a_n} = \displaystyle\uplim_{n\to +\infty}\frac{1}{a_n}.$$
故$\left\{\frac{1}{a_n}\right\}$是收敛数列。当然$\{a_n\}$收敛。
\end{proof}
%%%%%%%%%%%%%%%%%%%%%
\begin{example}
设数列$\{a_n\}, a_n\leq 1,n=1,2,\cdots$,且满足:
$$a_m+a_n-1<a_{m+n}<a_m+a_n+1.$$
证明: (1) $\displaystyle\lim_{n\to +\infty}\frac{a_n}{n}=\omega$,其中$\omega$为有限数;(2) $n\omega -1 \leq a_n \leq n\omega+1$.
\end{example}
\begin{proof}
先证明(1).
$$a_1 -\frac{1}{n}< \frac{a_n}{n} < a_1 +\frac{1}{n}.$$
由此得证(1),且$\omega=a_1$。
由此(2)得证。
\end{proof}
%%%%%%%%%%%%%%%%%%%%%%%%%%%%%%%%
\begin{example}
设$a_n\geq 0,n\in\mathbb{N}$。证明:
$$\displaystyle\lim_{n\to +\infty}\sqrt[n]{a_n}\leq 1 \iff \text{对任何}l > 1,\text{有} \displaystyle\uplim_{n\to +\infty}\frac{a_n}{l^n} = 0.$$
如果删去“任何”两字,结论如何?
\end{example}
\begin{remark}
这个题目的结论是不对的。比如
\begin{equation*}
a_n  = \begin{cases} \frac{1}{2^n}, &\quad n\text{是奇数}\\
1, &\quad n\text{是偶数}
\end{cases}
\end{equation*}
于是
$$\displaystyle\uplim_{n\to +\infty}\frac{a_n}{l^n} \leq \displaystyle\uplim_{n\to +\infty}\frac{1}{l^n}  = 0,$$
但是,$\displaystyle\lim_{n\to +\infty}\sqrt[n]{a_n}$ 不存在。所以下面我们证明:
$$\displaystyle\uplim_{n\to +\infty}\sqrt[n]{a_n}\leq 1 \iff \text{对任何}l > 1,\text{有} \displaystyle\uplim_{n\to +\infty}\frac{a_n}{l^n} = 0.$$
\end{remark}
\begin{proof}
($\Rightarrow$): 对于任何的$l > 1$, 取$\varepsilon=\frac{l-1}{2} > 0$,我们可以找到
$N>0$,使得$$\sqrt[n]{a_n} \leq 1 + \frac{l -1}{2} = \frac{l + 1}{2} < l,\quad\forall n > N.$$ 
于是 $$\frac{a_n}{l^n} < \left(\frac{l + 1}{2l}\right)^n, \quad\forall n > N.$$
由$\frac{l + 1}{2l} < 1$可知,$$\displaystyle\uplim_{n\to +\infty}\frac{a_n}{l^n} = 0.$$

($\Leftarrow$): 由定义可知,对任意的$l > 1$, 存在$N > 0$, 当$n>N$时有,
$$\frac{a_n}{l^n} < 1 \Rightarrow a_n < l^n\Rightarrow \sqrt[n]{a_n} < l.$$
从而$$\displaystyle\uplim_{n\to +\infty}\sqrt[n]{a_n} \leq l\Rightarrow \displaystyle\uplim_{n\to +\infty}\sqrt[n]{a_n} \leq 1.$$

如果删去“任何”两字, 结论不成立。
\end{proof}
%%%%%%%%%%%%%%%%%%%%%%%%%%%%%%
\subsection{思考题}
%%%%%%%%%%%%%%%%%%%
\begin{example}
设数列$\{x_n\}$有界,且$\displaystyle\lim_{n\to +\infty}(x_{n+1}-x_n) = 0$, 令
$$l=\displaystyle\lowlim_{n\to +\infty}x_n, \quad L=\displaystyle\uplim_{n\to +\infty}x_n.$$
证明:$\{a\in\mathbb{R} | \text{有子列}x_{n_k}\rightarrow a(k\rightarrow \infty)\}=[l,L]$.如果删去条件$\displaystyle\lim_{n\to +\infty}(x_{n+1}-x_n) = 0$,结论如何?
\end{example}
\begin{proof}
对于$a\in (l, L)$和任意的$\varepsilon < \frac{1}{2}\min\{a-l, L-a\}$, 我们有存在$N$
使得:
\renewcommand\labelenumi{\normalfont(\theenumi)}
\begin{enumerate}
\item 当$n>N$时,$-\varepsilon \leq a_{n+1} - a_n \leq \varepsilon$.
\item 存在子列$n_{k}$, 使得$a_{n_k} \leq l +\varepsilon, \quad\forall k$
\item 存在子列$n_{l}$, 使得$a_{n_l}\geq L -\varepsilon,\quad\forall l$
\end{enumerate}
选择子列$n_{t}$,并且重新标记下标,使得
\begin{equation*}
a_{n_{t}}
\begin{cases} 
\leq l + \varepsilon, &\quad \text{当t是奇数}\\
\geq L -\varepsilon, &\quad \text{当t是偶数}
\end{cases}
\end{equation*}
对于任意上述子列的任意两个相邻的数$a_{n_{2t}}, a_{n_{2t+1}}$, 在原数列中一定有至少一个$a_{n_{t'}}$落在区间$(a-\varepsilon, a+\varepsilon)$. 从而由极限的定义知这个子列收敛到$a$.命题得证。
\end{proof}
%%%%%%%%%%%%%%%%%%%%%%%%%%
\begin{example}
设$0\leq a_{n+m}\leq a_n\cdot a_m(n,m=1,2,\cdots)$.证明$\displaystyle\uplim_{n\to +\infty}\sqrt[n]{a_n} = \displaystyle\lowlim_{n\to +\infty}\sqrt[n]{a_n}$且$\sqrt[n]{a_n}$收敛。
\end{example}
\begin{proof}
设$\displaystyle\uplim_{n\to +\infty}\sqrt[n]{a_n} 
=a > b=\displaystyle\lowlim_{n\to +\infty}\sqrt[n]{a_n}$。
取$N > 0$使得 $$a_N < b +\frac{a-b}{3}.$$
又取子列$a_{n_k}$使得
$$\displaystyle\uplim_{k\to +\infty}\sqrt[n_{k}]{a_{n_k}} = a.$$
当$k$充分大后,$n_k > N$,且$n_k = m_kN + l$,其中$l = 0, 1, \cdots, N - 1$.
于是$$a_{n_k} \leq a_{m_kN}\cdot a_l = a_N^{m_k}\cdot a_l = a_N^{n_k}\cdot \frac{a_l}{a_N^{l}}.$$
由此可知
$$0 \leq \sqrt[n_k]{a_{n_k}}\leq \sqrt[n_k]{a_N^{n_k}}\sqrt[n_k]{\frac{a_l}{a_N^{l}}}= a_N\cdot\sqrt[n_k]{\frac{a_l}{a_N^{l}}}.$$
于是
$$\displaystyle\uplim_{n\to +\infty}\sqrt[n_k]{a_{n_k}} \leq a_N \leq b+\frac{a-b}{3} < a.$$
这与$\{a_{n_k}\}$的选择矛盾。从而知$a=b$.命题得证。
\end{proof}
%%%%%%%%%%%%%%%%%%%%%%%
\section{Stolz公式}
\subsection{练习题}
\begin{example}
设$C_n^k=\displaystyle\frac{n!}{k!(n-k)!}$为组合数。应用Stolz公式证明:
$$\displaystyle\lim_{n\to +\infty}\frac{\displaystyle\sum_{k=0}^n\ln{C_n^k}}{n^2}=\frac{1}{2}.$$
\end{example}
\begin{proof}我们先计算
$\ln{C_n^k}+\ln{C_n^{n-k}} = 2\ln{n!} - 2\ln{k!} - 2\ln{(n-k)!}, \quad\forall k\leq n$.
所以
$\displaystyle\sum_{k=0}^n\ln{C_n^k} = n\ln{n!} - 2\sum_{k=0}^n\ln{k!}$.
设$x_n=n^2, y_n=\displaystyle\sum_{k=0}^n\ln{C_n^k}$,于是
\begin{equation*}
\begin{split}
\lim_{n\to +\infty}\frac{\displaystyle\sum_{k=0}^n\ln{C_n^k}}{n^2}&\xlongequal{\text{Stolz公式}}\lim_{n\to +\infty}\frac{(n-1)\ln{n}-\ln{n!}}{2n-1}\\&=\lim_{n\to +\infty}\left(\ln{\frac{n}{\sqrt[n]{n!}}}\cdot\frac{n}{2n-1}\right) +\lim_{n\to +\infty}\frac{\ln{n}}{2n-1}=\frac{1}{2}.
\end{split}
\end{equation*}
命题得证
\end{proof}
%%%%%%%%%%%%%%%%%
\begin{example}
应用Stolz公式证明:
\renewcommand\labelenumi{\normalfont(\theenumi)}
\begin{enumerate}
\item $\displaystyle\lim_{n\to +\infty}\frac{\displaystyle\sum_{k=1}^n\sqrt{k}}{n^{\frac{3}{2}}}=\frac{2}{3}$;
\begin{proof}
设$y_n=\sum_{k=1}^n\sqrt{k}, x_n = n^{\frac{3}{2}}$,则
\begin{equation*}
\begin{split}
\lim_{n\to +\infty}\frac{\displaystyle\sum_{k=1}^n\sqrt{k}}{n^{\frac{3}{2}}}&\xlongequal{\text{Stolz公式}}\lim_{n\to +\infty}\frac{\sqrt{n}}{n^{\frac{3}{2}}-(n-1)^{\frac{3}{2}}}\\&=
\lim_{n\to +\infty}\frac{\sqrt{n}(\sqrt{n^3}+\sqrt{(n-1)^3})}{n^3-(n-1)^3}\\
&=\lim_{n\to +\infty}\frac{n^2\left(1+\sqrt{(1-\frac{1}{n})^3}\right)}{3n^2-3n+1}=\frac{2}{3}.
\end{split}
\end{equation*}
\end{proof}
\item $\displaystyle\lim_{n\to +\infty}n\left[\frac{\displaystyle\sum_{k=1}^n\sqrt{k}}{n^{\frac{3}{2}}}-\frac{2}{3}\right]=\frac{1}{2}$.
\begin{proof}我们设$y_n = 3\displaystyle\sum_{k=1}^n\sqrt{k} -2n^{\frac{3}{2}}$, $x_n = 3\sqrt{n}$.于是
\begin{equation*}
\begin{split}
\frac{y_n-y_{n-1}}{x_n-x_{n-1}}&=\frac{3\sqrt{n}-2n^{\frac{3}{2}}+2(n-1)^{\frac{3}{2}}}{3(\sqrt{n}-\sqrt{n-1})}\\
&=\frac{3\sqrt{n}(\sqrt{n^3}+\sqrt{(n-1)^3})-6n^2+6n-2}{3(\sqrt{n}-\sqrt{n-1})(\sqrt{n^3}+\sqrt{(n-1)^3})}\\
&=\frac{3n^2(\sqrt{(1-1/n)^3}-1)+6n-2}{3(\sqrt{n}-\sqrt{n-1})(\sqrt{n^3}+\sqrt{(n-1)^3})}\\
&=\frac{-9n+9-3/n+(6n-2)(\sqrt{(1-1/n)^3}+1)}{3(\sqrt{n}-\sqrt{n-1})(\sqrt{n^3}+\sqrt{(n-1)^3})(\sqrt{(1-1/n)^3}+1)}\\
&=\frac{(-9n+9-3/n+(6n-2)(\sqrt{(1-1/n)^3}+1))(\sqrt{n}+\sqrt{n-1})}{3(\sqrt{n^3}+\sqrt{(n-1)^3})(\sqrt{(1-1/n)^3}+1)}\\
&=\frac{n^{\frac{3}{2}}(-9+9/n-3/n^2+(6-2/n)(\sqrt{(1-1/n)^3}+1))(1+\sqrt{1-1/n})}{3n^{\frac{3}{2}}(1+\sqrt{(1-1/n)^3})(\sqrt{(1-1/n)^3}+1)}
\end{split}
\end{equation*}
于是:
$$\displaystyle\lim_{n\to +\infty}n\left[\frac{\displaystyle\sum_{k=1}^n\sqrt{k}}{n^{\frac{3}{2}}}-\frac{2}{3}\right] \xlongequal{\text{Stolz公式}} \displaystyle\lim_{n\to +\infty}\frac{y_n-y_{n-1}}{x_n-x_{n-1}}=\frac{1}{2}.$$
命题得证
\end{proof}
\end{enumerate}
\end{example}
%%%%%%%%%%%%%%%%%%
\subsection{思考题}
\begin{example}
设$0<x_1<1$,$x_{n+1} = x_n(1-x_n), n =1,2,\cdots.$ 证明:$\displaystyle\lim_{n\to +\infty}nx_n=1$.进而设$0<x_1\leq\frac{1}{q}$,其中$0<q\leq 1$,并且$x_{n+1}=x_n(1-qx_n), n\in\mathbb{N}$.证明:$\displaystyle\lim_{n\to +\infty}nx_n=\frac{1}{q}.$
\end{example}
\begin{proof}
如果我们能证明$\{x_n\}$单调减且$\displaystyle\lim_{n\to +\infty}x_n = 0$,则
\begin{equation*}
\begin{split}
\displaystyle\lim_{n\to +\infty}nx_n&\xlongequal{\text{Stolz公式}}\displaystyle\lim_{n\to +\infty}\frac{n-(n-1)}{\frac{1}{x_n} - \frac{1}{x_{n-1}}}\\&=\displaystyle\lim_{n\to +\infty}\frac{x_{n-1}(1-qx_{n-1})}{qx_{n-1}}\\&=\frac{1}{q}.
\end{split}
\end{equation*}
下面我们证明$\{x_n\}$单调减且$\displaystyle\lim_{n\to +\infty}x_n = 0$.
由于$x_n-x_{n-1} = -qx^2_{n-1}, \forall n\in\mathbb{N}$可知,$\{x_n\}$是单调减的。
很明显$x_n > 0, \forall n\in\mathbb{N}$. 由实数连续性命题(二)可知, $\{x_n\}$是收敛的。
设$\displaystyle\lim_{n\to +\infty}x_n = a$, 则$a=a(1-qa)\Rightarrow qa^2 = 0\Rightarrow a = 0$.命题得证。
\end{proof}
%%%%%%%%%%%%%%%%%%%%%%%%%%%%%
\begin{example}
由Toeplitz定理导出$\frac{\infty}{\infty}$型的Stolz公式。
\end{example}
\begin{proof}
取$$t_{nk} = \frac{x_k - x_{k-1}}{x_n-x_0}, \quad\forall k = 1,2,\cdots, n.$$
\renewcommand\labelenumi{\normalfont(\theenumi)}
\begin{enumerate}
\item 由于$\{x_n\}$是单调增数列,$t_{nk} > 0$. 
\item 因为$\displaystyle\lim_{n\to +\infty}x_n=+\infty$, $\displaystyle\lim_{n\to +\infty}t_{nk} = 0$. 
\item $\displaystyle\sum_{k=1}^{n}t_{nk} = \frac{x_n - x_0}{x_n-x_0}= 1$.
\end{enumerate}
由Toeplitz定理可知
\begin{equation*}
\begin{split}
\lim_{n\to +\infty}\frac{y_n}{x_n} &= \lim_{n\to +\infty}\frac{y_0}{x_n} +\lim_{n\to +\infty}\frac{x_n - x_0}{x_n}\lim_{n\to +\infty}\frac{y_ n - y_0}{x_n-x_0}\\&=\lim_{n\to +\infty}\sum_{k=1}^{n-1}\frac{x_{k+1}-x_{k}}{x_n-x_0}\cdot\frac{y_{k+1}-y_k}{x_{k+1}-x_k}\\&\lim_{n\to +\infty}\sum_{k=1}^{n-1}t_{nk}\frac{y_{k+1}-y_k}{x_{k+1}-x_k}\\
&\xlongequal{\text{Toeplitz公式}}\lim_{n\to +\infty}\frac{y_n-y_{n-1}}{x_{n}-x_{n-1}} = a.
\end{split}
\end{equation*}
命题得证。
\end{proof}
%%%%%%%%%%%%%%%%%%%%%%%%%%%%%
\begin{example}
设数列$\{a_n\}$满足$\displaystyle\lim_{n\to +\infty}a_n \displaystyle\sum_{i=1}^na_i^2 = 1$。证明:$\displaystyle\lim_{n\to +\infty}\sqrt[3]{3n}a_n=1$.
\end{example}
\begin{proof}
记$S_n = \displaystyle\sum_{i=1}^na_i^2.$我们很容易证明以下结论:
\renewcommand\labelenumi{\normalfont(\theenumi)}
\begin{enumerate}
\item $\{S_n\}$是单增的, 且$\displaystyle\lim_{n\to +\infty}S_n=+\infty$
\item $\displaystyle\lim_{n\to +\infty}a_n = 0$
\end{enumerate}
现在计算
\begin{equation*}
\begin{split}
S_n^3 - S_{n-1}^3 &= (S_n-S_{n-1})(S_n^2+S_nS_{n-1}+S_{n-1}^2) \\
&=a_n^2(S_n^2+S_n(S_n - a_n^2)) +(S_n-a_n^2)^2)\\
&=3(a_nS_n)^2-3a_n^3(a_nS_n) +a_n^6
\end{split}
\end{equation*}
从而
$$\displaystyle\lim_{n\to +\infty}\frac{S_n^3}{3n}\xlongequal{\text{Stolz公式}}\displaystyle\lim_{n\to +\infty}\frac{S_n^3-S_{n-1}^3}{3} = 1.$$
于是$$\displaystyle\lim_{n\to +\infty}\frac{1}{3na^3_n} = \displaystyle\lim_{n\to +\infty}\frac{1}{(a_nS_n)^3}\displaystyle\lim_{n\to +\infty}\frac{S^3_n}{3n} = 1.$$
由此可知$\displaystyle\lim_{n\to +\infty}\sqrt[3]{3n}a_n=1$。命题得证。
\end{proof}
%%%%%%%%%%%%%%%%%%%%%%%%%%%%%
\section{复习题1}
%%%%%%%%%%%%%%%%%%%%%%%%%%%%%
未解决的题:11\\
\begin{example}
设$a_0=1$,$a_{n+1}=a_n+\frac{1}{a_n}$,$n=0,1,2,\cdots$.证明:$\displaystyle\lim_{n\to +\infty}\frac{a_n}{\sqrt{2n}}= 1$.
\end{example}
\begin{proof}
由递归定义,
$$a_{n-1}^2+2 < a_{n-1}^2+\frac{1}{a^2_{n-1}} + 2< a_{n}^2,\quad\forall n > 1\Rightarrow a^2_{n}\geq 2*(n-1) + a_1^2=2n-1.$$
于是
$$0\leq \frac{1}{a_n^2} < \frac{1}{2n-1}, \forall n > 1\Rightarrow \displaystyle\lim_{n\to +\infty}\frac{1}{a_n^2}= 0.$$
算术平均
$$\displaystyle\lim_{n\to +\infty}\frac{\displaystyle\frac{1}{a_1^2}+\frac{1}{a_2^2} +\cdots +\frac{1}{a_n^2}}{n} = 0.$$
现在计算
\begin{equation*}
\begin{split}
\frac{a^2_n}{2n} &= \frac{2n-1}{2n} +\frac{\displaystyle\frac{1}{a_1^2}+\frac{1}{a_2^2} +\cdots +\frac{1}{a_{n-1}^2}}{2n}\\
&=\frac{2n-1}{2n} +\frac{\displaystyle\frac{1}{a_1^2}+\frac{1}{a_2^2} +\cdots +\frac{1}{a_{n-1}^2}}{n-1}\cdot\frac{n-1}{2n}
\end{split}
\end{equation*}
由此可知
$$\displaystyle\lim_{n\to +\infty}\frac{a^2_n}{2n}= 1\Rightarrow \displaystyle\lim_{n\to +\infty}\frac{a_n}{\sqrt{2n}}= 1.$$
命题得证。
\end{proof}
%%%%%%%%%%%%%%%%%%
\begin{example}
设$\displaystyle\lim_{n\to +\infty}x_n=\lim_{n\to +\infty}y_n=0$,并且存在常数$K$使得$\forall n\in\mathbb{N}$,有$$|y_1|+|y_2|+\cdots+|y_n|\leq K.$$
令$$z_n=x_1y_n+x_2y_{n-1}+\cdots+x_ny_1,\quad n\in\mathbb{N}.$$证明:$\displaystyle\lim_{n\to +\infty}z_n=0$.
\end{example}
\begin{proof}
由于$\displaystyle\lim_{n\to +\infty}x_n=0$,存在$M > 0$使得$|a_n| < M, \forall n\in\mathbb{N}$.

对$\forall \varepsilon > 0$, 存在$N_1\in\mathbb{N}$,当$n>N_1$时有$$\left|x_n\right| < \frac{\varepsilon}{2K}, \quad\forall n > N_1.$$

设$s_n=\displaystyle\sum_{k=1}^{n}\left|y_k\right|$.很显然$\{s_n\}$是一个收敛数列。于是,存在$N_2\in\mathbb{N}$,当$n>N_2$时有
$$s_{n} - s_{n-N_1} <\frac{\varepsilon}{2MN_1}.$$

综上,
\begin{equation*}
\begin{split}
\left|z_n\right|&=\left|x_1y_n+x_2y_{n-1}+\cdots+x_ny_1\right|\\
&<\left|x_1y_n+x_2y_{n-1}+\cdots+x_{N_1}y_{n-N_1+1}\right| + \left|x_{N_1+1}y_{n-N_1}+x_{N_1+2}y_{n-N_1-1}+\cdots+x_{n}y_1\right|\\
&<MN_1(s_n-s_{n-N_1}) + \frac{\varepsilon}{2K}(|y_1|+|y_2|+\cdots +|y_{n-N_1}|)\\
&<MN_1\frac{\varepsilon}{2MN_1}+K\frac{\varepsilon}{2K} \\&= \varepsilon
\end{split}
\end{equation*}
命题得证。
\end{proof}
%%%%%%%%%%%%%%%
\begin{example}
设数列$\{a_n\}$与$\{b_n\}$满足:
\renewcommand\labelenumi{\normalfont(\theenumi)}
\begin{enumerate}
\item $b_n>0, b_0+b_1+\cdots+b_n\rightarrow +\infty(n\rightarrow+\infty)$;
\item $\displaystyle\lim_{n\to +\infty}\frac{a_n}{b_n}=s$.
\end{enumerate}
应用Toeplitz定理证明:
$$\displaystyle\lim_{n\to +\infty}\frac{a_0+a_1+\cdots+a_n}{b_0+b_1+\cdots+b_n}=s.$$
\end{example}
\begin{proof}
在这题里我们可以取$$t_{nk} = \displaystyle\frac{b_k}{b_0+b_1+b_2+\cdots+b_n}.$$ 
\renewcommand\labelenumi{\normalfont(\theenumi)}
\begin{enumerate}
\item $t_{nk} > 0, \forall n > 0, k=0,2,\cdots,n$;
\item 给定$n$, $\displaystyle\sum_{k=0}^{n}t_{nk} = 1$;
\item 由于$b_0+b_1+\cdots+b_n\rightarrow +\infty(n\rightarrow+\infty)$,给定$k$, $\displaystyle\lim_{n\to +\infty}t_{nk} = 0$;
\end{enumerate}

于是
$$\frac{a_0+a_1+\cdots+a_n}{b_0+b_1+\cdots+b_n}=\displaystyle\sum_{k=0}^{n}t_{nk}\frac{a_k}{b_k}.$$
由Toeplitz定理,命题可以得证。
\end{proof}
%%%%%%%%%%%%%%%%%
\begin{example}
设$p_k>0, k=1,2,\cdots$,且$\displaystyle\lim_{n\to +\infty}\frac{p_n}{p_1+p_2+\cdots+p_n}=0$,$\displaystyle\lim_{n\to +\infty}a_n=a$.证明:
$$\displaystyle\lim_{n\to +\infty}\frac{p_1a_n+p_2a_{n-1}+\cdots+p_na_1}{p_1+p_2+\cdots+p_n}=a.$$
\end{example}
\begin{proof}
对于给定的正整数$k>0$,我们可以证明$\displaystyle\lim_{n\to +\infty}\frac{p_{n-k}}{p_1+p_2+\cdots+p_n}=0$.
这是因为
$$0\leq \frac{p_{n-k}}{p_1+p_2+\cdots+p_n} < \frac{p_{n-k}}{p_1+p_2+\cdots+p_{n-k}}.$$
夹逼定理说明$\displaystyle\lim_{n\to +\infty}\frac{p_{n-k}}{p_1+p_2+\cdots+p_n}=0$。

下面我们就可以用极限的定义来证明命题了。对于$\forall \varepsilon > 0$,存在$N_0$,当$n>N_0$时有$$\left|a_n-a\right| < \frac{\varepsilon}{2}.$$

设$M = \max\left\{\left|a_0 -a\right|, \left|a_1 -a\right|, \cdots, \left|a_{N_0} -a\right|\right\}$.

对于每一个$k, 1\leq k \leq N_0$, 存在$N_k$, 当$n>N_k$时有,
$$\frac{p_{n-k+1}}{p_1+p_2+\cdots+p_n} \leq \frac{\varepsilon}{2MN_0}.$$

取$N=\max\{N_0, N_1, \cdots, N_{N_0}\}$, 当$n>N$时
\begin{equation*}
\begin{split}
\left|\frac{p_1a_n+p_2a_{n-1}+\cdots+p_na_1}{p_1+p_2+\cdots+p_n} - a\right| &= \left|\frac{p_1(a_n-a)+p_2(a_{n-1}-a)+\cdots+p_n(a_1-a)}{p_1+p_2+\cdots+p_n}\right|\\
&<\sum_{k=1}^{N_0} \frac{p_{n-k+1}|a_k - a|}{p_1+p_2+\cdots+p_n} + \frac{p_1 + p_2 + \cdots + p_{n-N_0}}{p_1+p_2+\cdots+p_n}\cdot\frac{\varepsilon}{2} \\
&<\sum_{k=1}^{N_0}\frac{M\varepsilon}{2MN_0} + \frac{\varepsilon}{2} = \varepsilon.
\end{split}
\end{equation*}
命题得证。
\end{proof}
%%%%%%%%%%%%%%%
\begin{example}
设$\{a_n\}$为单调增的数列,令$\sigma_n=\displaystyle\frac{a_1+a_2+\cdots+a_n}{n}$,如果$\displaystyle\lim_{n\to +\infty}\sigma_n = a$,证明:$\displaystyle\lim_{n\to +\infty}a_n = a$.若“单调增”的条件删去,结论是否成立。
\end{example}
\begin{proof}
存在$N > 0$,当$n>N$时,$\sigma_n < a+1$.于是
$$\frac{a_1}{2}+\frac{a_n}{2} < \sigma_{2n} < a + 1.$$从而$$a_n < 2\left(a+1-\frac{a_1}{2}\right).$$
单调增有上界的数列是收敛数列。即$\displaystyle\lim_{n\to +\infty}a_n$存在。
设$\displaystyle\lim_{n\to +\infty}a_n=b$, 由例1.1.15可知,
$$b=\displaystyle\lim_{n\to +\infty}\sigma_n = a.$$

如果$\{a_n\}$不是单调增的,结论不成立。例如$a_n=(-1)^n$,$\sigma_n = 0 \text{或} -\frac{1}{n}$.所以$\displaystyle\lim_{n\to +\infty}\sigma_n = 0$.但是$\{a_n\}$不收敛。
\end{proof}
%%%%%%%%%%%%%%%%%%
\begin{example}
设$\{S_n\}$为数列,$a_n=S_n-S_{n-1}$, $\sigma_n = \displaystyle\frac{S_0+S_1+\cdots+S_n}{n+1}$.如果$\displaystyle\lim_{n\to +\infty}na_n=0$且$\{\sigma_n\}$收敛,证明$\{S_n\}$也收敛,且$\displaystyle\lim_{n\to +\infty}S_n = \displaystyle\lim_{n\to +\infty}\sigma_n$.
\end{example}
\begin{proof}
我们直接计算
\begin{equation*}
\begin{split}
\displaystyle\lim_{n\to +\infty}(S_n - \sigma_n) &= \displaystyle\lim_{n\to +\infty}\frac{nS_n - S_0 -S_1-\cdots-S_{n-1}}{n+1}\\
&\xlongequal{\text{Stolz公式}}\displaystyle\lim_{n\to +\infty}(nS_n-nS_{n-1})\\
&=\displaystyle\lim_{n\to +\infty}a_n = 0.
\end{split}
\end{equation*}
于是$\displaystyle\lim_{n\to +\infty}S_n=\displaystyle\lim_{n\to +\infty}(S_n - \sigma_n) +\displaystyle\lim_{n\to +\infty}\sigma_n = \displaystyle\lim_{n\to +\infty}\sigma_n.$
\end{proof}
%%%%%%%%%%%%%%%%%%%
\begin{example}
设数列$\{x_n\}$满足:$\displaystyle\lim_{n\to +\infty}(x_n-x_{n-2})=0$.证明$\displaystyle\lim_{n\to +\infty}\frac{x_n-x_{n-1}}{n} = 0$.
\end{example}
\begin{proof}如果我们能证明$\displaystyle\lim_{n\to +\infty}\frac{\left(-1\right)^n\left(x_n-x_{n-1}\right)}{n} = 0$,则命题得证。

通过简单的计算,我们有
$$\left(-1\right)^n\left(x_n-x_{n-1}\right) = \sum_{k=1}^{n-2}(-1)^k(x_{k+2}-x_k)+x_2-x_1.$$
由于$\displaystyle\lim_{n\to +\infty}(-1)^n(x_n-x_{n-2})=0$, 可知
$$\lim_{n\to +\infty}\displaystyle\frac{\displaystyle\sum_{k=1}^{n-2}(-1)^k(x_{k+2}-x_k)}{n-2} = 0.$$
于是
$$\displaystyle\lim_{n\to +\infty}\frac{\left(-1\right)^n\left(x_n-x_{n-1}\right)}{n}=\lim_{n\to +\infty}\displaystyle\frac{\displaystyle\sum_{k=1}^{n-2}(-1)^k(x_{k+2}-x_k)}{n-2}\cdot \lim_{n\to +\infty}\frac{n-2}{n}+\lim_{n\to +\infty}\frac{x_2-x_1}{n} = 0.$$
\end{proof}
%%%%%%%%%%%%%%%%
\begin{example}
设$u_0,u_1,\cdots$为满足$u_n=\displaystyle\sum_{k=1}^{\infty}u^2_{n+k}(n=0,1,2,\cdots)$的实数列,且$\displaystyle\sum_{n=1}^{\infty}u_{n}$收敛。证明$\forall k\in\mathbb{N}$,有$u_k=0$.
\end{example}
\begin{proof}
由$u_n$的定义可知,
$$u_n \geq u_{n+1}\geq 0, \quad\forall n.$$
记$S_n = \displaystyle\sum_{k=1}^{n}u_{k}$. 由于$\{S_n\}$收敛,则存在$N > 0$,使得当$n>N$时
$\displaystyle\sum_{k=n}^{\infty}u_{k} < 1$.
\begin{equation*}
\begin{split}
u_{n+1} &\leq u_{n}= \displaystyle\sum_{k=1}^{\infty}u^2_{n+k} \\
&= u_{n+1}\left(u_{n+1} + \frac{u_{n+2}}{u_{n+1}}u_{n+2} +\cdots\right)\\
&\leq u_{n+1}(u_{n+1} + u_{n+2}+\cdots)\\
&\leq u_{n+1}
\end{split}
\end{equation*}
从而,$u_{N+1}=u_{N+2}=u_{N+3}=\cdots = c$. 由于$\displaystyle\sum_{k=N+1}^{\infty}u_{k} < 1$可知,$c=0$.又由$u_{N+1}=u_{N+2}=u_{N+3}=\cdots = 0$,可知$u_N = u_{N-1} = \cdots = u_1 = 0$.
\end{proof}
%%%%%%%%%%%%%%%%%
\begin{example}
设$\displaystyle\lim_{n\to +\infty}a_n =a$证明$\displaystyle\lim_{n\to +\infty}\frac{1}{2^n}\displaystyle\sum_{k=0}^nC_n^ka_k =a$
\end{example}
\begin{proof}
取$t_{nk} = \frac{1}{2^n}C_n^k$, 我们有
\renewcommand\labelenumi{\normalfont(\theenumi)}
\begin{enumerate}
\item $t_{nk} > 0$且 $\displaystyle\sum_{k=0}^nt_{nk} = \frac{1}{2^n}(1+1)^n = 1$;
\item 很容易验证$\displaystyle\lim_{n\to +\infty}t_{nk} = 0$. 因为$2^n=(1+1)^n > C_n^{k+1}$, 从而
$$0\leq t_{nk} \leq \frac{C_n^k}{C_n^{k+1}} = \frac{k+1}{n-k}.$$
由夹逼定理,$\displaystyle\lim_{n\to +\infty}t_{nk} = 0$。
\end{enumerate}
由Toeplitz定理知,$\displaystyle\lim_{n\to +\infty}\frac{1}{2^n}\displaystyle\sum_{k=0}^nC_n^ka_k =a$。
\end{proof}
%%%%%%%%%%%%%%%%%%%%%
\begin{example}
给定实数$a_0,a_1$,并令$$a_n=\frac{a_{n-1}+a_{n-2}}{2}, \quad n = 2,3,\cdots.$$
证明:数列$\{a_n\}$收敛,且$\displaystyle\lim_{n\to +\infty}a_n =\frac{a_0+2a_1}{3}$
\end{example}
\begin{proof}
由递归公式有:
$$a_n-a_{n-1} = \left(-\frac{1}{2}\right)\left(a_{n-1}-a_{n-2}\right)=\cdots= \left(-\frac{1}{2}\right)^{n-1}(a_1-a_0).$$
于是:
\begin{equation*}
\begin{split}
a_n-a_0 &= (a_n-a_{n-1}) + (a_{n-1}-a_{n-2}) +\cdots+(a_1-a_0)\\
&=\left(-\frac{1}{2}\right)^{n-1}(a_1-a_0) + \left(-\frac{1}{2}\right)^{n-2}(a_1-a_0) + \cdots + (a_1-a_0)\\
&=\frac{1-\left(-\frac{1}{2}\right)^{n}}{1+\frac{1}{2}}(a_1-a_0)
\end{split}
\end{equation*}
从而$\displaystyle\lim_{n\to +\infty}a_n =a_0+\displaystyle\lim_{n\to +\infty}\left(\frac{1-\left(-\frac{1}{2}\right)^{n}}{1+\frac{1}{2}}(a_1-a_0)\right)=a_0+\frac{2}{3}(a_1-a_0) = \frac{a_0+2a_1}{3}$。
\end{proof}
%%%%%%%%%%%%%%%%%%
\begin{example}
设$x_1,x_2,\cdots, x_n$为任意给定的实数。令
$$x_i^{(1)}=\frac{x_i+x_{i+1}}{2}, \quad i=1,2,\cdots, n,$$
其中$x_{n+1}$应理解为$x_1$.归纳定义
$$x_i^{(k)}=\frac{x_i^{(k-1)}+x_{i+1}^{(k-1)}}{2}, \quad i=1,2,\cdots, n,$$
$x^{(k-1)}_{n+1}$应理解为$x_1^{(k-1)},k=2,3,\cdots$.证明:
$$\displaystyle\lim_{k\to +\infty}x_i^{(k)}=\frac{x_1+x_2+\cdots+x_n}{n}, \quad \forall i = 1,2,\cdots,n.$$
\end{example}

%%%%%%%%%%%%%%%%%%
\begin{example}
设$\{a_n\}$为一个数列,且$\displaystyle\lim_{n\to +\infty}(a_{n+1}-a_n) = l$。证明:$$\displaystyle\lim_{n\to +\infty}\frac{a_n}{n} = l; \quad \displaystyle\lim_{n\to +\infty}\frac{\displaystyle\sum_{k=1}^na_k}{n^2} =\frac{l}{2}.$$
\end{example}
\begin{proof}
$$a_n = (a_n - a_{n-1}) + (a_{n-1}-a_{n-2})+\cdots + (a_2 - a_1) + a_1.$$
由于$\displaystyle\lim_{n\to +\infty}(a_{n+1}-a_n) = l$,易知
$$\displaystyle\lim_{n\to +\infty}\frac{a_n}{n} = l.$$

现在计算
\begin{equation*}
\begin{split}
\displaystyle\sum_{k=1}^na_k - na_1 &= \displaystyle\sum_{k=2}^{n}(a_k - a_1)\\
&=\displaystyle\sum_{k=2}^{n}\displaystyle\sum_{l=1}^{k-1}(a_{l+1}-a_l)\\
&=\displaystyle\sum_{l=1}^{n-1}(n-l)(a_{l+1}-a_l)
\end{split}
\end{equation*}
如果取$t_{nk}=\displaystyle\frac{2(n-k)}{(n-1)n}$,则
\renewcommand\labelenumi{\normalfont(\theenumi)}
\begin{enumerate}
\item $t_{nk} \geq 0, \forall n\in\mathbb{N}, k=1,2,\cdots, n$且$\displaystyle\sum_{k=1}^{n}t_{nk} = \displaystyle\sum_{k=1}^{n}\frac{2(n-k)}{(n-1)n}=1$;
\item $\displaystyle\lim_{n\to +\infty}t_{nk} = 0, \quad\forall k$.
\end{enumerate}
由Toeplitz定理可知
$$\displaystyle\lim_{n\to +\infty}\frac{\displaystyle\sum_{k=1}^na_k - na_1}{(n-1)n}=\displaystyle\lim_{n\to +\infty}\displaystyle\sum_{k=1}^{n-1}\frac{2(n-k)}{(n-1)n}\frac{a_{k+1}-a_k}{2}=\displaystyle\lim_{n\to +\infty}\displaystyle\sum_{k=1}^{n-1}t_{nk}\frac{a_{k+1}-a_k}{2}=\frac{l}{2}.$$
从而
$$\displaystyle\lim_{n\to +\infty}\frac{\displaystyle\sum_{k=1}^na_k}{n^2}=\displaystyle\lim_{n\to +\infty}\frac{\displaystyle\sum_{k=1}^na_k - na_1}{(n-1)n}\cdot\displaystyle\lim_{n\to +\infty}\frac{(n-1)n}{n^2}+\displaystyle\lim_{n\to +\infty}\frac{a_1}{n}=\frac{l}{2}.$$
命题得证。
\end{proof}
%%%%%%%%%%%%%%%%%%%
\begin{example}
设$x_1\in[0,1], \forall n\geq 2$,令
\begin{equation*}
x_n =
\begin{cases}
\frac{1}{2}x_{n-1}, \quad \text{n为偶数,}\\
\frac{1+x_{n-1}}{2},\quad\text{n为奇数.}
\end{cases}
\end{equation*}
证明:$\displaystyle\lim_{n\to +\infty}x_{2k}=\frac{1}{3};\displaystyle\lim_{n\to +\infty}x_{2k+1}=\frac{2}{3}$.
\end{example}
\begin{proof}
由递推公式可知
\begin{equation*}
\begin{split}
x_{2k} &= \frac{1}{2}\left(\frac{1+x_{2(k-1)}}{2}\right)\\&=\frac{1}{4}+\frac{1}{4}x_{2(k-1)}\\&= \displaystyle\sum_{l=1}^{k-1}\left(\frac{1}{4}\right)^l+\left(\frac{1}{4}\right)^{k-1}x_2 \\&= \frac{1}{3}-\frac{4}{3}\left(\frac{1}{4}\right)^k+\left(\frac{1}{4}\right)^{k-1}x_2.
\end{split}
\end{equation*}
因此$\displaystyle\lim_{n\to +\infty}x_{2k}=\frac{1}{3}$.

$$x_{2k+1}=\frac{1}{2}+\frac{1}{4}x_{2(k-1)+1}
=\frac{1}{2}\displaystyle\sum_{l=0}^{k-1}\left(\frac{1}{4}\right)^{l} + \left(\frac{1}{4}\right)^{k}x_1
=\frac{2}{3}-\frac{2}{3}\left(\frac{1}{4}\right)^{k}+\left(\frac{1}{4}\right)^{k}x_1.$$
因此$\displaystyle\lim_{n\to +\infty}x_{2k+1}=\frac{2}{3}$.
\end{proof}
%%%%%%%%%%%%%%%%%
\begin{example}
定初始值$a_0$,并递推定义$$a_n=2^{n-1}-3a_{n-1},\quad n=1,2,\cdots.$$
求$a_0$的所有可能的值,使得数列$\{a_n\}$是严格增的。
\end{example}
\begin{solution}
考虑数列$\left\{\displaystyle\frac{a_{n}}{3^{n}}\right\}$.显然
\begin{equation*}
\begin{split}
\frac{a_n}{3^n} &=\frac{1}{2}\left(\frac{2}{3}\right)^n - \frac{a_{n-1}}{3^{n-1}} \\&= \frac{1}{2}\left(\frac{2}{3}\right)^n - \frac{1}{2}\left(\frac{2}{3}\right)^{n-1}  + \frac{a_{n-2}}{3^{n-2}}\\
&=\frac{1}{2}\sum_{k=0}^{n-1}(-1)^k\left(\frac{2}{3}\right)^{n-k}+(-1)^na_0\\
&=\frac{1}{5}\left[\left(\frac{2}{3}\right)^n+(-1)^n(5a_0-1)\right]
\end{split}
\end{equation*}
于是
$$a_n=\frac{1}{5}\left[2^n+(-1)^n(5a_0-1)3^n\right].$$
由此
\begin{equation*}
\begin{split}
a_{2n}-a_{2n-1}&=\frac{1}{5}[2^{2n}+(5a_0-1)3^{2n} - 2^{2n-1}+(5a_0-1)3^{2n-1}]\\&=\frac{1}{5}[2^{2n-1}+4(5a_0-1)3^{2n-1}].
\end{split}
\end{equation*}
只有当$a_0\geq\frac{1}{5}$时,$a_{2n}>a_{2n-1},\quad\forall n\in\mathbb{N}.$
\begin{equation*}
\begin{split}
a_{2n-1}-a_{2n-2}&=\frac{1}{5}[2^{2n-1}-(5a_0-1)3^{2n-1} - 2^{2n-2}-(5a_0-1)3^{2n-2}]\\&=\frac{1}{5}[2^{2n-2}-4(5a_0-1)3^{2n-2}].
\end{split}
\end{equation*}
只有当$a_0\leq\frac{1}{5}$时,$a_{2n-1}>a_{2n-2},\quad\forall n\in\mathbb{N}.$

综上,只有$a_0=\frac{1}{5}$时,$\{a_n\}$是严格递增的。
\end{solution}
%%%%%%%%%%%%%%
\begin{example}
设$c>0$,$a_1=\frac{c}{2}$,$a_{n+1} = \frac{c}{2}+\frac{a_n^2}{2}$,$n=1,2,\cdots$.证明:
\begin{equation*}
\displaystyle\lim_{n\to +\infty}a_n=
\begin{cases}
1-\sqrt{1-c}, \quad &0<c\leq 1,\\
+\infty,\quad &c >1.
\end{cases}
\end{equation*}
试问:当时$-3\leq c < 0$,数列$\{a_n\}$的收敛性如何?
\end{example}
\begin{proof}
现证$c>1$的情形:
$$a_{n} \geq \sqrt{ca_{n-1}^2} = \sqrt{c}a_{n-1} \geq \cdots \geq \left(\sqrt{c}\right)^{n-1}a_1.$$
由于$\displaystyle\lim_{n\to +\infty}\left(\sqrt{c}\right)^{n-1} = +\infty$, $\displaystyle\lim_{n\to +\infty}a_{n} = +\infty$.命题得证。

现在$c\leq 1$的情形:
\renewcommand\labelenumi{\normalfont(\theenumi)}
\begin{enumerate}
\item 考虑函数$f(x) = x^2-2x+c$。 当$x\in (1-\sqrt{1-c}, 1+\sqrt{1-c})$, $f(x) < 0$且$\displaystyle\min_{x\in \mathbb{R}}f(x) = -1+c$.
\item 很显然 $a_1=\frac{c}{2} \in (1-\sqrt{1-c}, 1+\sqrt{1-c})$.于是$a_2 - a_1 = \frac{1}{2}f(a_1) < 0 \Rightarrow a_2 < a_1$.
$$a^2_1\in(2-c-2\sqrt{1-c}, 2-c+2\sqrt{1-c})\Rightarrow a_2 = \frac{c}{2} + \frac{a_1^2}{2}\in (1-\sqrt{1-c}, 1+\sqrt{1-c}).$$
\item 假设$n=k$时, $a_k \leq a_{k-1}$ 且$a_k\in (1-\sqrt{1-c}, 1+\sqrt{1-c})$. 显然我们有$a_{k+1} \leq a_k, a_{k+1}\in (1-\sqrt{1-c}, 1+\sqrt{1-c})$.
\end{enumerate}
综上,
$$a_{n+1}\leq a_{n}, \quad a_n\geq 1-\sqrt{1-c}, \quad\forall n.$$
于是$\displaystyle\lim_{n\to +\infty}a_{n}=1-\sqrt{1-c}$.

现考虑$-3\leq c < 0$的情形。
\renewcommand\labelenumi{\normalfont(\theenumi)}
\begin{enumerate}
\item 我们先证明$a_n$是有界的。$\frac{c}{2} < a_1 = \frac{c}{2} \leq 0$.假设$\frac{c}{2} \leq a_k\leq 0$, 则$0 \leq a_k^2 \leq \frac{c^2}{4}$。从而
$$\frac{c}{2} < \frac{c}{2} + \frac{a_k^2}{2} = a_{k+1} \leq \frac{c}{2} + \frac{c^2}{8} = \frac{c}{8}(4 - c) < 0$$
由数学归纳法知,$a_n\in\left[\frac{c}{2}, 0\right], \forall n$.
\item \begin{equation*}
\begin{split}
a_n - a_{n-2} &= \frac{1}{2}\left(a_{n-1}+a_{n-3}\right)\left(a_{n-1}-a_{n-3}\right) \\&= \frac{1}{4}\left(a_{n-1}+a_{n-3}\right)\left(a_{n-2}+a_{n-4}\right)(a_{n-2}-a_{n-4})
\end{split}
\end{equation*}
由于$\frac{1}{4}\left(a_{n-1}+a_{n-3}\right)\left(a_{n-2}+a_{n-4}\right) > 0$, $a_n - a_{n-2}$和$a_{n-2}-a_{n-4}$同号。即$\{a_{2k-1}| k \geq 1\}$
和$\{a_{2k}| k \geq 1\}$是单调数列。
因为$$a_3 - a_1 = \frac{c}{2}+\frac{a_2^2}{2} - \frac{c}{2} = \frac{a_2^2}{2} > 0,$$所以$\{a_{2k-1}| k \leq 1\}$是单调递增的。
又因为
$$a_4-a_2 = \frac{a_3+a_1}{2}\left(a_3 - a_1\right) < 0,$$
所以$\{a_{2k}| k \geq 1\}$是单调递减的。
\end{enumerate}
假设当$k\rightarrow+\infty$时,$a_{2k-1}\rightarrow p, a_{2k}\rightarrow q$。在递推公式两端取极限,我们有
\begin{equation*}
\begin{split}
p&=\frac{c}{2}+\frac{q^2}{2}\\
q&=\frac{c}{2}+\frac{p^2}{2}
\end{split}
\end{equation*}
两式相减可得
$$p-q=\frac{1}{2}(q-p)(p+q)\Rightarrow (p-q)(p+q+2) = 0.$$

如果$p-q = 0$, i.e. $p=q$,则$p=q= 1-\sqrt{1-c}$.

如果$p+q+2 = 0$,则将$p=-2-q$代入第一个方程,我们有$(q+1)^2 =-(c+3)$.所以如果$c>-3$, 则$(q+1)^2 < 0$,从而无解。当$c=-3$时,$q=-1$,从而$p=q=-1$.
\end{proof}
%%%%%%%%%%%%%%%%%%
\begin{example}
数列$\{u_n\}$定义如下: $u_1=b,u_{n+1}=u_n^2+(1-2a)u_n+a^2$,$n\in\mathbb{N}$.问: $a,b$为何值时$\{u_n\}$收敛,并求出其极限值。
\end{example}
\begin{solution}
由递归定义可知:
$$u_{n+1}-a = (u_n-a)^2 + (u_n-a).$$
记$v_n = u_n-a$, 则$v_1 = b -a$.如果$\{u_n\}$收敛,则$\{v_n\}$收敛且$\displaystyle\lim_{n\to +\infty}v_n = 0$.
\renewcommand\labelenumi{\normalfont(\theenumi)}
\begin{enumerate}
\item 由于$v_{n+1}-v_{n} = v_n^2 \geq 0$, $\{v_n\}$是单调增的。
\item 如果$\displaystyle\lim_{n\to +\infty}v_n = 0$,则$v_n \leq 0, \forall n$.
\end{enumerate}
下面我们用归纳法来证明,当$v_1 \leq 0, v_1 + 1 \geq 0$时,$v_n \leq 0$且$v_n + 1 \geq 0,\forall n$.
\renewcommand\labelenumi{\normalfont(\theenumi)}
\begin{enumerate}
\item 当$n=1$时,成立。
\item 假设$n=k$时,我们也有$v_k \leq 0$且$v_k + 1 \geq 0$.
\item 我们证明$n=k+1$时也成立。因为$$v_{k+1} =v_k\cdot(v_k + 1),$$ 所以$v_{k+1} \leq 0$.另一方面$$v_{k+1} +1 = v^2_k + v_k + 1 = \left(v_k + \frac{1}{2}\right)^2 + \frac{3}{4} \geq \frac{3}{4} > 0.$$ 
\end{enumerate}
由此可见,当$b\in[a-1, a]$时,$\{u_n\}$收敛。
\end{solution}
%%%%%%%%%%%%%%%%
\begin{example}
设$A > 0, 0< y_0 < A^{-1}, y_{n+1} = y_n(2-Ay_n), n\in\mathbb{N}$. 证明:$\displaystyle\lim_{n\to +\infty}y_{n}=A^{-1}$.
\end{example}
\begin{proof}如果我们能证明$0< y_n < A^{-1}, \forall n\in\mathbb{N}$,则
$$\frac{y_{n+1}}{y_n} = 2-Ay_n\geq 2-AA^{-1}=1\Rightarrow y_{n+1} \geq y_n.$$
从而$\{y_n\}$是单调递增有上界的数列。故收敛,且$\displaystyle\lim_{n\to +\infty}y_{n}=A^{-1}$.

下面我们就证明$y_n < A^{-1},\forall n\in\mathbb{N}$. 考虑函数$f(x)=x(2-Ax)$.显然该函函数在$x=A^{-1}$是取得最大值。即$f(x) < A^{-1},\forall x\in \mathbb{R}$.由于$0< y_1 < A^{-1}$,由归纳法, $0< y_n < A^{-1}, \forall n\in\mathbb{N}$.
\end{proof}
%%%%%%%%%%%%%%%%%%
\begin{example}
设数列$\{a_n\}$满足$(2-a_n)a_{n+1} = 1$。证明:$\displaystyle\lim_{n\to +\infty}a_{n}=1$。
\end{example}
\begin{proof}
由于$(2-a_n)a_{n+1} = 1$,则$a_n \neq 2$,从而$a_1\neq \frac{3}{2}$.
我们现证无论$a_1$取何值,都存在$N$使得$a_N \leq 1$.
\renewcommand\labelenumi{\normalfont(\theenumi)}
\begin{enumerate}
\item 如果$a_1 \leq 1$, 取$N=1$, $a_N = a_1 \leq 1$
\item 如果$a_1 > \frac{3}{2}$, 则$a_2 > 2$, $a_3 < 0$. 于是取$N=3$, $a_N = a_3 < 0 \leq 1$.
\item 如果$1< a_1 < \frac{3}{2}$.记$a_1 = 1 + h$, 则$$a_k = 1 + \frac{h}{1-(k-1)h}, \quad k = 2, 3\cdots.$$取$k=\left[\frac{1}{h}\right]$, 我们有
$$1-(k-1)h \geq h > 0, \quad \frac{h}{1-(k-1)h}\geq\frac{1}{2}.$$
从而取$N = k+2 = \left[\frac{1}{h}\right]$,我们有$a_{N-2}=a_k\geq \frac{3}{2}, a_{N}< 0\leq 1$.
\end{enumerate}
综上,我们不妨假设$a_1 \leq 1$. 下面我们可以用数学归纳法证明$\{a_n\}$是单调增的,且$a_n \leq 1,\forall n$.
由于$$a_{n+1}- a_n = \frac{(a_n - 1)^2}{2-a_n},$$
我们可知当$a_n < 1$时,$a_{n+1} > a_n$. 又因为$2-a_{n} > 1\rightarrow a_{n+1} \leq 1$. 到此我们证明了$\displaystyle\lim_{n\to +\infty}a_{n}$存在。设$\displaystyle\lim_{n\to +\infty}a_{n}=a$,则$(2-a)a=1$.解方程知$a=1$.
\end{proof}
%%%%%%%%%%%%%%%%%%%
\begin{example}
设数列$\{a_n\}$满足不等式$0\leq a_k\leq 100a_n(n\leq k\leq 2n, n=1,2,\cdots)$,且无穷级数$\displaystyle\sum_{n=1}^{\infty}a_n$收敛。证明$\displaystyle\lim_{n\to +\infty}na_n=0$
\end{example}
\begin{proof}
显然
$a_{2n} \leq 100a_n, \quad a_{2n} \leq 100a_{n+1}, \cdots, a_{2n} \leq 100a_{2n-1}.$
从而
$$0 \leq 2na_{2n} \leq 200\sum_{k=n}^{2n-1}a_k.$$
由于$\displaystyle\sum_{n=1}^{\infty}a_n$收敛,知$\displaystyle\lim_{n\to +\infty}200\sum_{k=n}^{2n-1}a_k=0$. 由夹逼定理知,$\displaystyle\lim_{n\to +\infty}2na_{2n}=0$.

同理,
$a_{2n-1} \leq 100a_n, \quad a_{2n-1}\leq 100a_{n+1}, \cdots, a_{2n-1} \leq 100 a_{2n-2}.$
将所有的不等式加起来,我们有
$$0 \leq (2n-1) a_{2n-1} = a_{2n-1} + 2(n-1)a_{2n-1} \leq 2*100\sum_{k=n}^{2n-2}a_k + a_{2n-1} \leq 200 \sum_{k=n}^{2n-1}a_k.$$
由于$\displaystyle\sum_{n=1}^{\infty}a_n$收敛,知$\displaystyle\lim_{n\to +\infty}200\sum_{k=n}^{2n-1}a_k=0$. 由夹逼定理知,$\displaystyle\lim_{n\to +\infty}(2n-1)a_{2n-1}=0$.

综上所述, 我们有$\displaystyle\lim_{n\to +\infty}na_{n}=0$
\end{proof}
%%%%%%%%%%%%%
\begin{example}
证明:$\displaystyle\lim_{n\to +\infty}\left(1+\frac{1}{n^2}\right)\left(1+\frac{2}{n^2}\right)\cdots\left(1+\frac{n}{n^2}\right)=e^{\frac{1}{2}}$
\end{example}
\begin{proof}
通过简单计算,我们可以得知
$$\left(1+\frac{n}{2n^2}\right)^2 \leq \left(1+\frac{k}{n^2}\right)\left(1+\frac{n-k+1}{n^2}\right), \quad\forall k \leq \frac{n}{2}, k\in\mathbb{N}.$$
由此可见
$$\left(1+\frac{n}{2n^2}\right)^n < \left(1+\frac{1}{n^2}\right)\left(1+\frac{2}{n^2}\right)\cdots\left(1+\frac{n}{n^2}\right).$$

另一方面
$$\left(1+\frac{1}{n^2}\right)\left(1+\frac{2}{n^2}\right)\cdots\left(1+\frac{n}{n^2}\right) \leq \left(\frac{\displaystyle\sum_{k=1}^{n}\left(1+\frac{k}{n^2}\right)}{n}\right)^n=\left(1+\frac{(n+1)}{2n^2}\right)^n.$$

很显然
$$\displaystyle\lim_{n\to +\infty}\left(1+\frac{n}{2n^2}\right)^n=e^{\frac{1}{2}},$$
$$\displaystyle\lim_{n\to +\infty}\left(1+\frac{(n+1)}{2n^2}\right)^n=e^{\frac{1}{2}}.$$
由夹逼定理,命题得证。
\end{proof}
%%%%%%%%%%%%%%%
\begin{example}
设$a_1>b_1>0$,令$$a_n=\displaystyle\frac{a_{n-1}+b_{n-1}}{2},\quad b_n=\frac{2a_{n-1}b_{n-1}}{a_{n-1}+b_{n-1}}, \quad n = 2,3,\cdots.$$ 证明:数列$\{a_n\}$与$\{b_n\}$都收敛,且$\displaystyle\lim_{n\to +\infty}a_n=\displaystyle\lim_{n\to +\infty}b_n=\sqrt{a_1b_1}$.
\end{example}
\begin{proof}很显然
$$b_n \leq \sqrt{a_{n-1}b_{n-1}}\leq a_n,\quad\forall n = 2,3\cdots.$$
于是
$$a_n =\displaystyle\frac{a_{n-1}+b_{n-1}}{2} \leq \displaystyle\frac{a_{n-1}+a_{n-1}}{2} = a_{n-1},$$
$$b_n -b_{n-1} = \frac{(a_{n-1}-b_{n-1})\cdot b_{n-1}}{a_{n-1}+b_{n-1}} \geq 0\Rightarrow b_n \geq b_{n-1}.$$
所以
$$b_1 \leq b_2\leq \cdots \leq b_n \leq \cdots \leq a_n \leq \cdots \leq a_2\leq a_1.$$
由实数连续性命题(二)可知,$\displaystyle\lim_{n\to +\infty}a_n$和$\displaystyle\lim_{n\to +\infty}b_n$都存在。 

设$\displaystyle\lim_{n\to +\infty}a_n=a$,$\displaystyle\lim_{n\to +\infty}b_n=b$,

$$a_nb_n=a_{n-1}b_{n-1}=\cdots = a_1b_1\Rightarrow ab = a_1b_1.$$
$$a_n=\displaystyle\frac{a_{n-1}+b_{n-1}}{2}\Rightarrow a=\frac{a+b}{2}\Rightarrow a=b.$$
于是$a=b=\sqrt{a_1b_1}$.
\end{proof}
%%%%%%%%%%%%%%%%%
\begin{example}
当$n\geq 3$时,证明:
$$\displaystyle\sum_{k=0}^n\frac{1}{k!}-\frac{3}{2n}<\left(1+\frac{1}{n}\right)^n < \displaystyle\sum_{k=0}^n\frac{1}{k!}.$$
\end{example}
\begin{proof}
在证明该题之前,我们先证明如下结论:数列$a_1, a_2, \cdots a_n$满足$n \geq 2$, $a_k \geq -1, k=1,2,\cdots, n$且它们有相同的符号,则
$$\displaystyle\prod_{k=1}^n(1+a_k) \geq 1+\sum_{k=1}^na_k.$$
我们用归纳法来证明此命题:
\renewcommand\labelenumi{\normalfont(\theenumi)}
\begin{enumerate}
\item 当$n=2$时,$(1+a_1)\cdot(1+a_2) = 1 + a_1 + a_2 + a_1a_2 \geq 1 + a_1+a_2$,命题成立。
\item 假设$n=k$时,命题亦成立, i.e. $\displaystyle\prod_{n=1}^k(1+a_n) \geq 1+\sum_{n=1}^ka_n.$
\item 下面证明当$n=k+1$时,命题亦成立。 
\begin{equation*}
\begin{split}
\displaystyle\prod_{n=1}^{k+1}(1+a_n) &= \left(\displaystyle\prod_{n=1}^{k}(1+a_n)\right)\cdot(1+a_{k+1})\\&\geq(1+a_1+\cdots +a_k)\cdot(1+a_{k+1}) ({\text{因为$1+a_{k+1} \geq 0$}})\\
&=1+\sum_{n=1}^{k+1}a_n + \sum_{n=1}^{k}a_na_{k+1}\\
&\geq 1+\sum_{n=1}^{k+1}a_n
\end{split}
\end{equation*}
命题得证。
\end{enumerate}

下面我们用上述命题来证明此题。
\begin{equation*}
\begin{split}
\left(1+\frac{1}{n}\right)^n &=1+\frac{1}{1!} + \sum_{k=2}^n\frac{1}{k!}\left(1-\frac{1}{n}\right)\left(1-\frac{2}{n}\right)\cdots\left(1-\frac{k-1}{n}\right) 
\end{split}
\end{equation*}

首先,由$\left(1-\frac{1}{n}\right)\left(1-\frac{2}{n}\right)\cdots\left(1-\frac{k-1}{n}\right) < 1,\forall k \leq n$,我们可知$$\left(1+\frac{1}{n}\right)^n<\displaystyle\sum_{k=0}^n\frac{1}{k!}.$$
右边的不等式的得证。

另一方面,应用上面证明的结论,可知
$$\left(1-\frac{1}{n}\right)\left(1-\frac{2}{n}\right)\cdots\left(1-\frac{k-1}{n}\right) 
\geq 1 -\sum_{l=1}^{k-1}\frac{l}{n} = 1-\frac{(k-1)k}{2n}.$$
于是
$$\left(1+\frac{1}{n}\right)^n \geq \displaystyle\sum_{k=0}^n\frac{1}{k!} - \frac{1}{2n}\displaystyle\sum_{k=0}^{n-2}\frac{1}{k!} > \displaystyle\sum_{k=0}^n\frac{1}{k!}-\frac{3}{2n}$$
左边的不等式得证。
\end{proof}
%%%%%%%%%%%%%%%%
\begin{example}
设$a_1=1, a_n=n(a_{n-1}+1), n=2,3,\cdots$,且
\[
x_n = \prod_{k=1}^n\left(1+\frac{1}{a_k}\right).
\]求$\displaystyle\lim_{n\to +\infty}x_n$(其中$\displaystyle\prod_{k=1}^n$表示从$k=1$到$k=n$的连乘积).
\end{example}
\begin{solution}
我们先计算
\begin{equation*}
\begin{split}
x_n &= \prod_{k=1}^n\left(1+\frac{1}{a_k}\right) \\
&=\frac{a_1 +1}{a_1}\cdot\frac{a_2+1}{2(a_1+1)}\cdot\frac{a_3+1}{3(a_2+1)}\cdots\frac{a_n+1}{n(a_{n-1}+1)} \\
&=\frac{a_n+1}{n!a_1}
\end{split}
\end{equation*}
另一方面
\begin{equation*}
\begin{split}
a_1 & = 1! \\
a_2 &= 2!\left(1 + \frac{1}{1!}\right)\\
a_3 &= 3!\left(1+\frac{1}{1!} + \frac{1}{2!}\right)\\
&\cdots \\
a_n &= n!\left(1+ \frac{1}{1!} + \frac{1}{2!}+\cdots+\frac{1}{(n-1)!}\right)
\end{split}
\end{equation*}
从而知,$$a_n+1 = n!\left(1+ \frac{1}{1!} + \frac{1}{2!}+\cdots+\frac{1}{n!}\right).$$
于是:$\displaystyle\lim_{n\to +\infty}x_n = \displaystyle\lim_{n\to +\infty}(1+ \frac{1}{1!} + \frac{1}{2!}+\cdots+\frac{1}{n!})=e.$
\end{solution}
%%%%%%%%%%%%%%%%%%
\begin{example}
设$H_n=1+\frac{1}{2}+\cdots+\frac{1}{n}$,$n\in\mathbb{N}$,用$K_n$表示使得$H_k\geq n$的最小下标,求$\displaystyle\lim_{n\to +\infty}\frac{K_{n+1}}{K_n}$
\end{example}
\begin{solution}
我们记$x_n = H_n - \ln{n}$, 则$\displaystyle\lim_{n\to +\infty}x_n=C(\text{Euler常数})$.
现在我们计算
$$x_{K_{n+1}} - x_{K_n} = \left(\frac{1}{K_n+1} + \cdots + \frac{1}{K_{n+1}}\right) -\ln{K_{n+1}} + \ln{K_n}.$$
另一方面
$$1-\frac{1}{K_n}\leq \left(\frac{1}{K_n+1} + \cdots + \frac{1}{K_{n+1}}\right) < 1 + \frac{1}{K_{n+1}}.$$
从而由夹逼定理知
$$0=\displaystyle\lim_{n\to +\infty}(x_{K_{n+1}} - x_{K_n}) = \displaystyle\lim_{n\to +\infty}\left(1 -\ln{K_{n+1}} + \ln{K_n}\right).$$
即$\displaystyle\lim_{n\to +\infty}\frac{K_{n+1}}{K_n}=e$.
\end{solution}
%%%%%%%%%%%%%%%%%%
\begin{example}
设$y_0\geq 2, y_n = y_{n-1}^2 - 2(n\in\mathbb{N})$,
\[
S_n=\frac{1}{y_0}+\frac{1}{y_0y_1} +\cdots + \frac{1}{y_0y_1\cdots y_n}.
\]证明:$\displaystyle\lim_{n\to +\infty}S_n=\frac{y_0-\sqrt{y_0^2-4}}{2}$.
\end{example}
\begin{proof}
如果$y_0=2$, 则$y_n = 2, \forall n > 0$。 从而$S_n = \displaystyle\sum_{k=1}^{n+1}\frac{1}{2^k} = 1 - \frac{1}{2^{n+1}}$.于是
$$\displaystyle\lim_{n\to +\infty}S_n = 1 =\frac{y_0-\sqrt{y_0^2 - 4}}{2}.$$

下面证明当$y_0 > 2$时结论亦成立。取$a =\frac{y_0-\sqrt{y_0^2 - 4}}{2}$. 简单计算可知
$$y_0 = a + \frac{1}{a}.$$
由此
\begin{equation*}
\begin{split}
y_1 &= y_0^2 - 2 =\left(a+\frac{1}{a}\right)^2 - 2 = a^2 +\frac{1}{a^2}\\
y_2 &= y_1^2 - 2 = \left(a^2 +\frac{1}{a^2}\right)^2 -2 = a^4 + \frac{1}{a^4}\\ 
 &\cdots\\
y_n &=y^2_{n-1} -2  = \left(a^{2^{n-1}} +\frac{1}{a^{2^{n-1}}}\right)^2 - 2 = a^{2^n} + \frac{1}{a^{2^n}}
\end{split}
\end{equation*}
于是
\begin{equation*}
\begin{split}
y_0y_1\cdots y_n &= \frac{1}{a-\frac{1}{a}}\left[\left(a-\frac{1}{a}\right)\left(a+\frac{1}{a}\right)\left(a^2 +\frac{1}{a^2}\right)\cdots\left(a^{2^n} + \frac{1}{a^{2^n}}\right)\right] \\
&=\frac{a}{a^2-1}\left[\left(a^{2^n}\right)^2 - \left(\frac{1}{a^{2^n}}\right)^2\right]\\
&=\frac{a}{a^2-1}\frac{a^{2^{n+2}} - 1}{a^{2^{n+1}}}
\end{split}
\end{equation*}
从而
$$\frac{1}{y_0y_1\cdots y_n} = \frac{a^2-1}{a}\frac{a^{2^{n+1}}}{a^{2^{n+2}} -1} =\frac{a^2-1}{a}\left(\frac{1}{a^{2^{n+1}} -1} - \frac{1}{a^{2^{n+2}}-1}\right).$$
由此可知
$$S_n = \frac{a^2-1}{a}\left(\frac{1}{a^2 -1} - \frac{1}{a^{2^{n+2}}-1}\right).$$
进而$\displaystyle\lim_{n\to +\infty}S_n = \frac{a^2-1}{a}\left(\frac{1}{a^2 -1} + 1\right) = a$.命题得证。
\end{proof}
%%%%%%%%%%%%%%%%%%%%%%%
\begin{example}
令数列$\{b_n\}$满足
$$b_n=\displaystyle\sum_{k=0}^n\frac{1}{C^k_n}, \quad n =1,2,\cdots.$$
证明:(1)当$n\geq 2$时,$b_n=\displaystyle\frac{n+1}{2n}b_{n-1} + 1;$\\
(2)$\displaystyle\lim_{n\to +\infty}b_n = 2.$
\end{example}
\begin{proof}
(1). 直接计算
\begin{equation*}
\begin{split}
\frac{n+1}{n}b_{n-1}&=\displaystyle\sum_{k=0}^{n-1}\frac{k!(n-1-k)!}{(n-1)!}\cdot\frac{n+1}{n}\\
&=\displaystyle\sum_{k=0}^{n-1}\frac{k!(n-k)!}{n!}\left(1 + \frac{k+1}{n-k}\right)\\
&=\displaystyle\sum_{k=0}^{n-1}\frac{k!(n-k)!}{n!} + \displaystyle\sum_{k=1}^{n}\frac{k!(n-k)!}{n!}\\
&=\displaystyle\sum_{k=0}^{n-1}\frac{k!(n-k)!}{n!} + \frac{n!(n-n)!}{n!} + \displaystyle\sum_{k=1}^{n}\frac{k!(n-k)!}{n!} + \frac{0!(n-0)!}{n!} -2\\
&=2\cdot\displaystyle\sum_{k=0}^{n}\frac{k!(n-k)!}{n!} - 2\\
&=2b_n - 2
\end{split}
\end{equation*}
所以,$b_n=\displaystyle\frac{n+1}{2n}b_{n-1} + 1$。

(2)当$n > 4$,我们有$b_{n-1} > 2\left(1 + \frac{1}{n-1}\right) = \frac{2n}{n-1}$,从而
$$\frac{b_n}{b_{n-1}} = \frac{n+1}{2n} + \frac{1}{b_{n-1}} < \frac{n+1}{2n} + \frac{n-1}{2n} = 1.$$
即从$n > 4$起,数列$\{b_n\}$单调递减。
另一方面,显然$b_n \geq 2, \forall n \geq 2$.
由此可知$\{b_n\}$是收敛的。对(1)中的等式取极限可知,$\displaystyle\lim_{n\to +\infty}b_n = 2$.
\end{proof}
%%%%%%%%%%%%%%%%%%%%%%%%%%
\begin{example}
设$S_n=1+2^2+3^3+\cdots+n^n$.证明:
$$n^n\left[1+\frac{1}{4(n-1)}\right] < S_n < n^n\left[1+\frac{2}{e(n-1)}\right].$$
\end{example}
\begin{proof}
提取$n^n$可得
$$S_n = n^n\left[1 + \frac{1}{n^n} + \frac{2^2}{n^n}+\cdots + \frac{(n-1)^{n-1}}{n^n}\right].$$
显然
\begin{equation*}
\begin{split}
\frac{1}{n-1}\left(1-\frac{1}{n}\right)^n &= \frac{(n-1)^{n-1}}{n^n} \\ &< \frac{1}{n^n} + \frac{2^2}{n^n}+\cdots + \frac{(n-1)^{n-1}}{n^n} \\
&< \frac{(n-2)^{n-1}}{n^n} + \frac{(n-1)^{n-1}}{n^n} < \frac{2(n-1)^{n-1}}{n^n}\\
&<\frac{2}{n-1}\left(1-\frac{1}{n}\right)^n.
\end{split}
\end{equation*}
我们不难证明$\{\left(1-\frac{1}{n}\right)^n\}$是单调增且$\displaystyle\lim_{n\to +\infty}\left(1-\frac{1}{n}\right)^n = \frac{1}{e}$. 所以
$$\frac{1}{4}=\left(1-\frac{1}{2}\right)^2 < \left(1-\frac{1}{n}\right)^n < \frac{1}{e}.$$
将上述不等式综合起来,我们就证明了此题。
\end{proof}
%%%%%%%%%%%%%%%%%%%
\begin{example}
设$x_n > 0$.证明:
\renewcommand\labelenumi{\normalfont(\theenumi)}
\begin{enumerate}
\item $\displaystyle\uplim_{n\to +\infty}\left(\frac{x_1+x_{n+1}}{x_n}\right)^n\geq e$;
\item 上式中的$e$为最佳常数。
\end{enumerate}
\begin{proof}
\renewcommand\labelenumi{\normalfont(\theenumi)}
\begin{enumerate}
\item $\displaystyle\uplim_{n\to +\infty}\left(\frac{x_1+x_{n+1}}{x_n}\right)^n\geq e$等价于$\displaystyle\uplim_{n\to +\infty}\left(\frac{x_1+x_{n+1}}{x_n}\cdot\frac{n}{1+n}\right)^{n}\geq 1$.我们用反证法来证明该命题。如果
$$\displaystyle\uplim_{n\to +\infty}\left(\frac{x_1+x_{n+1}}{x_n}\cdot\frac{n}{1+n}\right)^{n} < 1,$$则存在$N>0$,当$n > N$时,我们有
$$\frac{x_1+x_{n+1}}{x_n}\cdot\frac{n}{1+n} < 1,\forall n > N.$$
于是
\begin{equation*}
\begin{split}
\frac{x_{N+1}}{N+1}-\frac{x_{N+2}}{N+2} &> \frac{x_1}{N+2}\\
\frac{x_{N+2}}{N+2}-\frac{x_{N+3}}{N+3} &> \frac{x_1}{N+3}, \\
&\cdots\\
\frac{x_{n-1}}{n-1}-\frac{x_{n}}{n} &> \frac{x_1}{n}
\end{split}
\end{equation*}
把以上的不等式加起来,我们有
$$\frac{x_{N+1}}{N+1} > \frac{x_{N+1}}{N+1} - \frac{x_{n}}{n} > \displaystyle\sum_{k=N+1}^n\frac{x_1}{k}\rightarrow+\infty.$$
这显然与$\frac{x_{N+1}}{N+1}$是个有限数矛盾。从而假设不成立,命题得证。
\item 现证明对于任意的$\varepsilon>0$,存在数列$\{x_n\}$使得$$\displaystyle\uplim_{n\to +\infty}\left(\frac{x_1+x_{n+1}}{x_n}\right)^n< e^{1+\varepsilon}.$$我们取$x_1 = \frac{\varepsilon}{2}, x_n = n$,则
$$\displaystyle\uplim_{n\to +\infty}\left(\frac{x_1+x_{n+1}}{x_n}\right)^n = \displaystyle\lim_{n\to +\infty}\left(\frac{\frac{\varepsilon}{2}+n+1}{n}\right)^n=e^{1+\frac{\varepsilon}{2}} < e^{1+\varepsilon}.$$
这说明$e^{1+\varepsilon}$不是下确界。命题得证。
\end{enumerate}
\end{proof}
\end{example}
%%%%%%%%%%%%%%%%%%
\begin{example}
设$a_n>0$.证明:$\displaystyle\uplim_{n\to +\infty}n\left(\frac{1+a_{n+1}}{a_n}-1\right)\geq 1$.
\end{example}
\begin{proof}
如果$\displaystyle\uplim_{n\to +\infty}n\left(\frac{1+a_{n+1}}{a_n}-1\right) < 1$, 
则,存在$N>0$,当$n>N$时有,
$$n\left(\frac{1+a_{n+1}}{a_n}-1\right) < 1\Rightarrow \frac{a_n}{n} - \frac{a_{n+1}}{n+1}>\frac{1}{n+1}.$$
于是:
$$\frac{a_{N+1}}{N+1} > \frac{a_{N+1}}{N+1} - \frac{a_n}{n} > \displaystyle\sum_{k=N+2}^n\frac{1}{k}.$$
当$n\to +\infty$时,$\displaystyle\sum_{k=N+2}^n\frac{1}{k}\to+\infty.$ 这与$\frac{a_{N+1}}{N+1}$是个有限数矛盾。从而假设不成立。
\end{proof}
%%%%%%%%%%%%%%%%%%%%%
\begin{example}
设$2a_{n+1}=1+b_n^2, 2b_{n+1}=2a_n-a_n^2, 0\leq b_n \leq \frac{1}{2}\leq a_n, n = 1,2,\cdots$.证明:数列$\{a_n\}, \{b_n\}$均收敛,并求其极限之值。
\end{example}
\begin{proof}
很容易证得:$a_n\leq \frac{5}{8},\forall n$. 
\begin{equation*}
\begin{split}
a_{n+1} - a_{n} &= \frac{1}{2}(b_n + b_{n-1})(b_n - b_{n-1})\\
b_{n+1} - b_n &= \frac{1}{2}\left(2-a_n-a_{n-1}\right)\left(a_n - a_{n-1}\right)
\end{split}
\end{equation*}
由此可得
\begin{equation*}
\begin{split}
\left|a_{n+1} - a_{n}\right| &\leq \frac{1}{2^2}\left|a_{n-1} - a_{n-2}\right|\\
\left|b_{n+1} - b_n\right| &\leq \frac{1}{2^2}\left|b_{n-1} - b_{n-2}\right|
\end{split}
\end{equation*}
取$A = \max\{\left|a_{2} - a_{1}\right|, \left|a_{3} - a_{2}\right|\}$, 
$B = \max\{\left|b_{2} - b_{1}\right|, \left|b_{3} - b_{2}\right|\}$, 则
\begin{equation*}
\begin{split}
\left|a_{n} - a_{n-1}\right| &\leq \frac{1}{2^{2[\frac{n}{2}]-2}}A\\
\left|b_{n} - b_{n-1}\right| &\leq \frac{1}{2^{2[\frac{n}{2}]-2}}B
\end{split}
\end{equation*}
由此可见,数列$\{a_n\}$和$\{b_n\}$都是Cauchy列。从而都收敛。

假设$\displaystyle\lim_{n\to +\infty}a_n = a, \displaystyle\lim_{n\to +\infty}b_n = b$, 则
\begin{equation*}
\begin{split}
2a&=1+b^2,\\
2b&=2a-a^2.
\end{split}
\end{equation*}
解方程组可知,$(b^2+2b-1)(b^2-2b+3)=0$.解得$b=\sqrt{2} - 1, a = 2-\sqrt{2}$.
\end{proof}
%%%%%%%%%%%%%%%%%%%%%
%%%%%%%%%%%%%%%%%%%%%
%%%%%%%%%%%%%%%%%%%%%
%%%%%%%%%%%%%%%%%%%
\chapter{函数极限与连续}
%%%%%%%%%%%%%%%%%%%%%
%%%%%%%%%%%%%%%%%%%%%
%%%%%%%%%%%%%%%%%%%%%
%%%%%%%%%%%%%%%%%%%
\section{函数极限的概念}
%%%%%%%%%%%%%%%%%%%%%
%%%%%%%%%%%%%%%%%%%%%
%%%%%%%%%%%%%%%%%%%%%
\subsection{练习题}
\begin{example}
在定义2.1.3中就24种情形给出函数极限的定义。并配出相应的图形。
\end{example}
\begin{solution}
不在此赘述。用到的时候在写。
\end{solution}
%%%%%%%%%%%%%%%%%%%
\begin{example}
按函数极限的定义证明:
\renewcommand\labelenumi{\normalfont(\theenumi)}
\begin{enumerate}
\item $\displaystyle\lim_{x\to +\infty}\frac{6x+5}{x} = 6$;
\begin{proof}对$\forall \varepsilon > 0$, 取$\Delta >\frac{1}{\varepsilon}$,当$x > \Delta$时,有
$$\left|\frac{6x+5}{x} - 6 \right| = \left|\frac{5}{6x}\right| < \frac{1}{x} <\frac{1}{\Delta}< \varepsilon.$$
所以,$\displaystyle\lim_{x\to +\infty}\frac{6x+5}{x} = 6$.
\end{proof}
\item $\displaystyle\lim_{x\to 2}(x^2-6x+10) = 2$;
\begin{proof}对$\varepsilon > 0$, 取$\delta = \min\{1, \frac{\varepsilon}{3}\}$, 当$0<|x-2|<\delta$时,有
$$\left|(x^2-6x+10) - 2\right| = \left|x - 2\right|\cdot\left|x - 4\right| < 3\cdot\left|x - 2\right| <\varepsilon.$$
所以,$\displaystyle\lim_{x\to +\infty}(x^2-6x+10) = 2$.
\end{proof}
\item $\displaystyle\lim_{x\to +\infty}\frac{x^2-5}{x^2-1} = 1$;
\begin{proof}对$\forall \varepsilon > 0$, 取$\Delta = \max\left\{2+ \sqrt{5}, \frac{1}{\varepsilon}\right\}$, 当$x >\Delta$时,有
$$\left|\frac{x^2-5}{x^2-1} - 1\right| = \left|\frac{-4}{x^2-1}\right| <\frac{4}{4x} <\frac{1}{x}<\frac{1}{\Delta}<\varepsilon.$$
所以,$\displaystyle\lim_{x\to +\infty}\frac{x^2-5}{x^2-1} = 1$.
\end{proof}
\item $\displaystyle\lim_{x\to 2^{-}}\sqrt{4-x^2} = 0$;
\begin{proof}对$\forall \varepsilon > 0$, 取$\delta = \min\left\{1, \frac{\varepsilon^2}{4}\right\}$, 当$2-\delta < x < 2$时,有
$$\left|\sqrt{4-x^2}\right| = \left|\sqrt{(2-x)(2+x)}\right| < 2\sqrt{2-x}<\varepsilon$$
所以,$\displaystyle\lim_{x\to 2^{-}}\sqrt{4-x^2} = 0$.
\end{proof}
\item $\displaystyle\lim_{x\to 1}\frac{x^4-1}{x-1} = 4$;
\begin{proof}对$\forall \varepsilon > 0$, 取$\delta = \min\left\{1, \frac{\varepsilon}{11}\right\}$, 当$0<\left|x-1\right| < \delta$时,有
$$\left|\frac{x^4-1}{x-1} - 4\right| = \left|x^3 + x^2 + x - 3\right| = \left|x-1\right|\cdot \left|x^2+2x+3\right|< 11\cdot\left|x-1\right| < \varepsilon$$
所以,$\displaystyle\lim_{x\to 1}\frac{x^4-1}{x-1} = 4$.
\end{proof}
\item $\displaystyle\lim_{x\to 3}\frac{x-3}{x^2-9} = \frac{1}{6}$;
\begin{proof}对$\forall \varepsilon > 0$, 取$\delta=\min\left\{1, 30\varepsilon\right\}$, 当$0<\left|x-3\right| < \delta$时,有
$$\left|\frac{x-3}{x^2-9} - \frac{1}{6}\right| =\left|\frac{x-3}{6(x+3)}\right| <\frac{\left|x-3\right|}{30} < \varepsilon$$
所以,$\displaystyle\lim_{x\to 3}\frac{x-3}{x^2-9} = \frac{1}{6}$.
\end{proof}
\item $\displaystyle\lim_{x\to 1^{+}}\frac{x-1}{\sqrt{x^2-1}} = 0$;
\begin{proof}对$\forall \varepsilon > 0$, 取$\delta=2\varepsilon^2$, 当$0< x-1<\delta$时,有
$$\left|\frac{x-1}{\sqrt{x^2-1}}\right| =\left|\frac{\sqrt{x-1}}{\sqrt{x+1}}\right|<\frac{\sqrt{x-1}}{\sqrt{2}}<\varepsilon$$
所以,$\displaystyle\lim_{x\to 1^{+}}\frac{x-1}{\sqrt{x^2-1}} = 0$.
\end{proof}
\item $\displaystyle\lim_{x\to +\infty}(\sqrt{x+1}-\sqrt{x-1}) = 0$;
\begin{proof}对$\forall \varepsilon > 0$, 取$\Delta = \max\left\{1, \frac{4}{\varepsilon^2}\right\}$, 当$x > \Delta$时,有
$$\left|\sqrt{x+1}-\sqrt{x-1}\right| = \frac{2}{\sqrt{x+1}+\sqrt{x-1}}<\frac{2}{\sqrt{x}} < \frac{2}{\Delta} < \varepsilon$$
所以,$\displaystyle\lim_{x\to +\infty}(\sqrt{x+1}-\sqrt{x-1}) = 0$.
\end{proof}
\item $\displaystyle\lim_{x\to \infty}\sqrt{\frac{x^2+2}{x^2-2}}= 1$;
\begin{proof}对$\forall \varepsilon > 0$, 取$\Delta = \max\left\{\frac{1}{\varepsilon}, 2+ \sqrt{6}\right\}$, 当$|x| > \Delta$时,有
$$\left|\sqrt{\frac{x^2+2}{x^2-2}}- 1\right| = \frac{4}{\sqrt{x^4-4} +(x^2 - 2)}<\frac{4}{4|x|}<\frac{1}{\Delta}<\varepsilon$$
所以,$\displaystyle\lim_{x\to \infty}\sqrt{\frac{x^2+2}{x^2-2}}= 1$.
\end{proof}
\end{enumerate}
\end{example}
%%%%%%%%%%%%$$$$$
\begin{example}
设$\displaystyle\lim_{x\to x_0}f(x) =a$. 用$\varepsilon-\delta$法证明:
\renewcommand\labelenumi{\normalfont(\theenumi)}
\begin{enumerate}
\item $\displaystyle\lim_{x\to x_0}f^2(x)=a^2$;
\begin{proof}对$\forall \varepsilon > 0$, 存在$\delta > 0$, 当$0<|x-x_0| < \delta$时, $$|f(x)| < |a|+ 1, \quad|f(x)- a| < \frac{\varepsilon}{2|a|+1}.$$
于是,
$$\left|f^2(x) - a^2\right| = (|f(x)| + |a|)\cdot\left|f(x) - a\right|< (2|a|+1)\left|f(x) - a\right| < \varepsilon$$
所以,$\displaystyle\lim_{x\to x_0}f^2(x)=a^2$.
\end{proof}
\item $\displaystyle\lim_{x\to x_0}\sqrt{f(x)}=\sqrt{a}(a > 0)$;
\begin{proof}
对$\forall \varepsilon > 0$, 存在$\delta > 0$, 当$0<|x-x_0| < \delta$时,
$$f(x) > \frac{a}{4}, \quad |f(x) - a|<\frac{3\sqrt{a}}{2}\varepsilon.$$
于是,
$$\left|\sqrt{f(x)}-\sqrt{a}\right| = \frac{\left|f(x)-a\right|}{\sqrt{f(x)}+\sqrt{a}} < \frac{2}{3\sqrt{a}}\left|f(x)-a\right| < \varepsilon$$
所以,$\displaystyle\lim_{x\to x_0}\sqrt{f(x)}=\sqrt{a}(a > 0)$.
\end{proof}
\item $\displaystyle\lim_{x\to x_0}\sqrt[3]{f(x)}=\sqrt[3]{a}$.
\begin{proof}我们分两种情况证明此命题。
\renewcommand\labelenumi{\normalfont(\theenumi)}
\begin{enumerate}
\item 当$a=0$. 对$\forall \varepsilon > 0$, 取$\delta > 0$, 当$0 < |x-x_0| < \delta$时,有 $|f(x)| < \varepsilon^3$.
于是,$$\left|\sqrt[3]{f(x)}\right| < \varepsilon.$$
所以,$\displaystyle\lim_{x\to x_0}\sqrt[3]{f(x)}=\sqrt[3]{a}$.
\item 当$a\neq 0$. 对$\forall \varepsilon > 0$, 取$\delta > 0$, 当$0 < |x-x_0| < \delta$时,
$$|f(x) -a| < \min\left\{\frac{|a|}{2}, \sqrt[3]{a^2}\varepsilon\right\}.$$
于是,$f(x)a \geq 0, f^2(x)\geq 0$且
$$\left|\sqrt[3]{f(x)}-\sqrt[3]{a}\right|=\frac{|f(x)-a|}{\sqrt[3]{f^2(x)} + \sqrt[3]{f(x)a}+\sqrt[3]{a^2}}<\frac{|f(x)-a|}{\sqrt[3]{a^2}}<\varepsilon.$$
所以,$\displaystyle\lim_{x\to x_0}\sqrt[3]{f(x)}=\sqrt[3]{a}$.
\end{enumerate}
命题得证。
\end{proof}
\end{enumerate}
\end{example}
%%%%%%%%%%%%%%%%%%%%%%%
\begin{example}
设$\displaystyle\lim_{x\to x_0}f(x)=a$。证明:$\displaystyle\lim_{x\to x_0}|f(x)|=|a|$。举例说明反之不成立。问:当且仅当$a$为何值时反之也成立?
\end{example}
\begin{proof}对$\forall \varepsilon > 0$, 取$\delta > 0$, 当$0 < |x-x_0| < \delta$时,有 $|f(x)-a| < \varepsilon$. 于是,
$$\displaystyle\left||f(x)| - |a|\right| \leq |f(x) - a|<\varepsilon.$$
所以,$\displaystyle\lim_{x\to x_0}|f(x)|=|a|$。

设
\begin{equation*}
f(x) = 
\begin{cases}
1, \quad&x\geq 0;\\
-1 \quad&x<0.
\end{cases}
\end{equation*}
则,$\displaystyle\lim_{x\to 0}|f(x)|=1$.但$\displaystyle\lim_{x\to 0^{-}}f(x)=-1$, $\displaystyle\lim_{x\to 0^{+}}f(x)=1$, 所以$\displaystyle\lim_{x\to 0}f(x)$不存在。

当$a=0$时,$\displaystyle\lim_{x\to x_0}|f(x)|=0\Rightarrow\displaystyle\lim_{x\to x_0}f(x)=0$.
\end{proof}
%%%%%%%%%%%%%%%%%%%
\begin{example}
讨论下列函数在点$0$处的极限或左,右极限:
\renewcommand\labelenumi{\normalfont(\theenumi)}
\begin{enumerate}
\item $f(x)=\frac{|x|}{x}$;
\begin{solution}
$\displaystyle\lim_{x\to 0^{-}}f(x)=\displaystyle\lim_{x\to 0^{-}}\frac{x}{-x} = -1$, $\displaystyle\lim_{x\to 0^{+}}f(x)=\displaystyle\lim_{x\to 0^{+}}\frac{x}{x} = 1$.
\end{solution}
\item $f(x)=[x]$;
\begin{solution}
$\displaystyle\lim_{x\to 0^{-}}f(x)=\displaystyle\lim_{x\to 0^{-}}(-1) = -1$, $\displaystyle\lim_{x\to 0^{+}}f(x)=\displaystyle\lim_{x\to 0^{+}}0= 0$.
\end{solution}
\item $f(x)=
\begin{cases}
2^x, \quad&x>0,\\
0, \quad&x=0,\\
1+x^2,\quad&x<0.
\end{cases}
$
\begin{solution}
$\displaystyle\lim_{x\to 0^{-}}f(x)=\displaystyle\lim_{x\to 0^{-}}(1+x^2) = 1$, $\displaystyle\lim_{x\to 0^{+}}f(x)=\displaystyle\lim_{x\to 0^{+}}2^x= 1$. 
从而$\displaystyle\lim_{x\to 0}f(x) = 1$.
\end{solution}
\end{enumerate}
\end{example}
%%%%%%%%%%%%%%%%%%%%%
\begin{example}
设$f(x)=
\begin{cases}
x^2, \quad&x\geq2,\\
-ax, \quad&x<2.
\end{cases}
$
\renewcommand\labelenumi{\normalfont(\theenumi)}
\begin{enumerate}
\item 求$f(2^{+}), f(2^{-})$;
\begin{solution}对$\forall \varepsilon > 0$, 取$\delta=\min\{1, \frac{\varepsilon}{5}\}$, 当$2 < x < 2+\delta$时,有
$$|f(x)-4| = |x^2-4|=(x+2)(x-2)<5(x-2)< \varepsilon.$$
所以,$\displaystyle\lim_{x\to 2^{+}}f(x) = 4$. 

对$\forall \varepsilon > 0$, 取$\delta=\frac{\varepsilon}{|a|+1}$, 当$2-\delta < x < 2$时,有
$$|f(x) + 2a| = |-ax+2a| = |a|(2-x)<\frac{|a|\varepsilon}{|a|+ 1} < \varepsilon.$$
所以,$\displaystyle\lim_{x\to 2^{-}}f(x) = -2a$.
\end{solution}
\item 若$\displaystyle\lim_{x\to 2}f(x)$存在,$a$应为何值。
\begin{solution}
$\displaystyle\lim_{x\to 2}f(x)$存在$\iff \displaystyle\lim_{x\to 2^{+}}f(x) = \displaystyle\lim_{x\to 2^{-}}f(x)$. 所以$a=-2$.
\end{solution}
\end{enumerate}
\end{example}
%%%%%%%%%%%%%%%%%%%%%
\begin{example}
设$f(x_0^-)<f(x_0^+)$.证明: $\exists \delta > 0, s.t.$当$x_0-\delta<x<x_0<y<x_0+\delta$时,有$f(x)<f(y)$.
\end{example}
\begin{proof}取$\varepsilon < \displaystyle\frac{f(x_0^+)-f(x_0^-)}{2}$, 存在$\delta_1 > 0$, 当$x_0-\delta_1 < x < x_0$时,有
$$f(x) - f(x_0^-) < \varepsilon \Rightarrow f(x) < \frac{f(x_0^+)+f(x_0^-)}{2}.$$
对上述的$\varepsilon$, 存在$\delta_2 > 0$, 当$x_0 < y < x_0+\delta_2$时, 有
$$f(y) - f(x_0^+) >-\varepsilon \Rightarrow f(y) > \frac{f(x_0^+)+f(x_0^-)}{2}.$$
取$\delta = \min\{\delta_1,\delta_2\}$, 当$x_0-\delta<x<x_0<y<x_0+\delta$时,有$f(x)<\displaystyle\frac{f(x_0^+)+f(x_0^-)}{2}<f(y)$.
\end{proof}
%%%%%%%%%%%%%%%%%%%%%%
\begin{example}
设$f$在$(-\infty, x_0)$内单调增,且有一数列$\{x_n\}$,适合$x_n<x_0(n\in\mathbb{N})$,$x_n\to x^-_0$及$\displaystyle\lim_{n\to +\infty}f(x_n)=a$。证明:$f(x_0^-)=a=\displaystyle\sup_{x\in U_{-}^o(x_0)}f(x)$。
\end{example}
\begin{proof}对$\forall \varepsilon>0$, 取$N\in\mathbb{N}$, 当$n>N$时,有
$$a -\varepsilon < f(x_n) < a+\varepsilon.$$
取$\delta = \min\{x_0 - x_1, x_0-x_2,\cdots,x_0-x_{N+1}\}$, 我们现在考虑$x$满足$x_0-\delta < x < x_0$.
\renewcommand\labelenumi{\normalfont(\theenumi)}
\begin{enumerate}
\item 由于$\displaystyle\lim_{n\to +\infty}x_n=x^-_0$且$x_n < x_0$, 则存在$N_1 > N$使得$$x_0 - x_{N_1} < \frac{x_0-x}{2}\Rightarrow x_{N_1}>\frac{x_0+x}{2}>x.$$
\item 从而$x_{N+1} < x < x_{N_1}$. 由于$f(x)$单调增,$f(x_{N+1}) \leq f(x) \leq f(x_{N_1})$.
\end{enumerate}
于是,
$$a-\varepsilon < f(x) < a + \varepsilon$$
所以,$f(x_0^-)=a$. 类似地,我们也可以证明$a=\displaystyle\sup_{x\in U_{-}^o(x_0)}f(x)$.
\end{proof}
%%%%%%%%%%%%%%%%%%%%%%
\begin{example}
用肯定的语气表示$\displaystyle\lim_{x\to x_0}f(x)\neq a$.
\end{example}
\begin{solution}
存在$\varepsilon_0 > 0$, 对$\forall \delta > 0$, 存在$x'$满足$0<|x'-x_0|<\delta, |f(x') - a| > \varepsilon_0.$
\end{solution}
%%%%%%%%%%%%%%%%%%%%%%%
\begin{example}
$\forall n\in\mathbb{N}$,$A_n\subset[0,1]$为有限集,且$A_i\bigcap A_j=\emptyset(i\neq j)$,$i$,$j\in \mathbb{N}$。定义函数
\begin{equation*}
f(x)=
\begin{cases}
\frac{1}{n}, \quad &x\in A_n,\\
0, \quad &x\in [0,1]-\displaystyle\bigcup_{n=1}^{\infty}A_n.
\end{cases}
\end{equation*}
$\forall x_0\in[0,1]$,求$\displaystyle\lim_{x\to x_0}f(x)$。
\end{example}
\begin{solution}对$\forall \varepsilon > 0$, 取$N=\left[\displaystyle\frac{1}{\varepsilon}\right]$, 
当$n>N$时,有$$\frac{1}{n} < \varepsilon.$$
对于给定的$x_0\in[0,1]$, 取$\delta = \min\left\{\displaystyle\frac{|x_0 - x'|}{2}\Bigg| x'\in \displaystyle\bigcup_{n=1}^{N}A_n\right\}$,当$0<|x-x_0|<\delta$时,有
$$x\in[0,1]-\displaystyle\bigcup_{n=1}^{N}A_n.$$
于是,$$|f(x)| <\frac{1}{N+1} < \varepsilon.$$
所以,$\displaystyle\lim_{x\to x_0}f(x)=0$.
\end{solution}
%%%%%%%%%%%%%%%%%%%%%%
\begin{example}
叙述函数极限$\displaystyle\lim_{x\to +\infty}f(x)$的归结原理,并应用它证明:$\displaystyle\lim_{x\to +\infty}\sin{x}$与$\displaystyle\lim_{x\to +\infty}\cos{x}$都不存在。
\end{example}
\begin{solution}归结原理:对任意的数列$\{x_n\}$满足$$\displaystyle\lim_{n\to +\infty}x_n=+\infty, \quad \displaystyle\lim_{n\to +\infty}f(x_n)=a,$$
则$\displaystyle\lim_{x\to +\infty}f(x)=a$。

对于$\sin{x}$,我们考察子列$a_n = (2n+\frac{1}{2})\pi, n\in \mathbb{N}$和$b_n=(2n+\frac{3}{2})\pi, n\in \mathbb{N}$.显然$$1=\displaystyle\lim_{n\to +\infty}\sin(a_n)\neq \displaystyle\lim_{n\to +\infty}\sin(b_n)=-1.$$
所以,$\displaystyle\lim_{x\to +\infty}\sin{x}$不存在。

对于$\cos{x}$,我们考察子列$a_n = 2n\pi, n\in \mathbb{N}$和$b_n=(2n+1)\pi, n\in \mathbb{N}$.显然$$1=\displaystyle\lim_{n\to +\infty}\cos(a_n)\neq \displaystyle\lim_{n\to +\infty}\cos(b_n)=-1.$$
所以,$\displaystyle\lim_{x\to +\infty}\cos{x}$不存在。
\end{solution}
%%%%%%%%%%%%%%%%%%%%
\begin{example}
\renewcommand\labelenumi{\normalfont(\theenumi)}
\begin{enumerate}
\item 叙述极限$\displaystyle\lim_{x\to +\infty}f(x)$的Cauchy准则;
\begin{solution}
对$\forall \varepsilon >0$, 存在$\Delta > 0$,当$x_1>\Delta, x_2>\Delta$时,有$$|f(x_1)-f(x_2)| < \varepsilon.$$
\end{solution}
\item 根据Cauchy准则叙述$\displaystyle\lim_{x\to +\infty}f(x)$不存在的充要条件,并应用它证明$\displaystyle\lim_{x\to +\infty}\sin{x}$与$\displaystyle\lim_{x\to +\infty}\cos{x}$不存在。
\begin{solution}
$\displaystyle\lim_{x\to +\infty}f(x)$不存在$\iff$存在$\varepsilon_0 > 0$, 
对$\forall \Delta > 0$,存在$x_1 > \Delta, x_2>\Delta$使得
$$|f(x_1)-f(x_2)| > \varepsilon_0.$$

取$\varepsilon_0 = 1$, 对$\forall \Delta>0$, 取$N = \left[\displaystyle\frac{\Delta}{2\pi}\right]$, $x_1 = (2N+\frac{1}{2})\pi > \Delta, x_2 = (2N+\frac{3}{2})\pi > \Delta$.
于是$$|\sin(x_1) -\sin(x_2)| = 2 >\varepsilon_0.$$
所以,$\displaystyle\lim_{x\to +\infty}\sin{x}$不存在。

同理可以证明$\displaystyle\lim_{x\to +\infty}\cos{x}$不存在。
\end{solution}
\end{enumerate}
\end{example}
%%%%%%%%%%%%%%%%%%%
\begin{example}
设$f$为周期函数,且$\displaystyle\lim_{x\to +\infty}f(x)=0$。证明$f(x)\equiv 0$。
\end{example}
\begin{proof}
假设$f$的周期为$T$,我们只需证明$f(x)\equiv 0,\forall x\in [0,T)$.我们用反证法来证明此命题。

设$\exists x_0\in[0,T)$使得$f(x_0)\neq 0$.考虑数列$\{x_0+nT\}$, 有
$$\displaystyle\lim_{n\to +\infty}(x_0+nT)=+\infty, \quad \displaystyle\lim_{n\to +\infty}f(x_0+nT)=f(x_0)\neq 0.$$
由归结原理,$\displaystyle\lim_{x\to +\infty}f(x)\neq 0$.这与假设矛盾,从而$f(x)\equiv 0,\forall x\in [0,T)$.
\end{proof}
%%%%%%%%%%%%%%%%%%%%
\begin{example}
设$f$在$U^o(x_0)$内有定义。证明: $\forall \{x_n\}\subset U^o(x_0)$且$\displaystyle\lim_{n\to +\infty}x_n=x_0$,极限$\displaystyle\lim_{n\to +\infty}f(x_n)$都存在(实数,或$+\infty$,或$-\infty$),则所有这些极限都相等。
\end{example}
\begin{proof}(反证法):
假设$\{x^1_n\subset U^o(x_0)\}, \displaystyle\lim_{n\to +\infty}x^1_n=x_0$和$\{x^2_n\subset U^o(x_0)\}, \displaystyle\lim_{n\to +\infty}x^2_n=x_0$,但$\displaystyle\lim_{n\to +\infty}f(x^1_n)\neq \displaystyle\lim_{n\to +\infty}f(x^2_n)$. 我们考虑数列$x^3_n=
\begin{cases}
x^1_k,\quad&n=2k-1,\\
x^2_k,\quad&n=2k.
\end{cases}$
显然数列$\{x^3_n\}$满足条件
$$\{x^3_n\subset U^o(x_0)\}, \displaystyle\lim_{n\to +\infty}x^3_n=x_0,$$
但是由归结原理,$\displaystyle\lim_{n\to +\infty}f(x^3_n)$不存在。这与命题假设矛盾。从而
$\displaystyle\lim_{n\to +\infty}f(x^1_n) = \displaystyle\lim_{n\to +\infty}f(x^2_n)$。命题得证。
\end{proof}
%%%%%%%%%%%%%%%%%%%%%
\begin{example}
设$f$为定义$[a,+\infty)$在上的增(减)函数。证明$\displaystyle\lim_{x\to +\infty}f(x)$存在有限的充要条件是$f$在$[a,+\infty)$上有上(下)界。
\end{example}
\begin{proof}我们只证明$f$是单增函数的情形。

($\Rightarrow$)设$\displaystyle\lim_{x\to +\infty}f(x) = b$. 我们证明$f(x) \leq a$。假设存在$x_0$使得$f(x_0) > b$. 考察子列$\{x_0+n\}$.由于$f$是单增的,有
$$f(x_0+n) \geq f(x_0), \forall n\in\mathbb{N}.$$
\renewcommand\labelenumi{\normalfont(\theenumi)}
\begin{enumerate}
\item 如果$\displaystyle\lim_{n\to +\infty}f(x_0+n)$不存在,则由归结原理,$\displaystyle\lim_{x\to +\infty}f(x)$不存在。这与假设矛盾。
\item 如果$\displaystyle\lim_{n\to +\infty}f(x_0+n)$存在,则
$$\displaystyle\lim_{n\to +\infty}f(x_0+n) \geq f(x_0) > b.$$
这与$\displaystyle\lim_{x\to +\infty}f(x)=b$矛盾。
\end{enumerate}
于是,$f(x) \leq b,\forall x\in[a,+\infty)$.

($\Leftarrow$)设$f(x) < M,\forall x\in [0, +\infty)$.取一个子列$\{x_n\}$满足条件
$$\displaystyle\lim_{n\to +\infty}x_n=+\infty.$$我们可以从中取一个单调增的子列$\{x_{n_k}\}$满足条件$$\displaystyle\lim_{k\to +\infty}x_{n_k}=+\infty.$$ 
由于$f$是单调增的,$f(x_{n_k})$也是单调增。从而$\displaystyle\lim_{k\to +\infty}f(x_{n_k})$存在。设$\displaystyle\lim_{k\to +\infty}f(x_{n_k})=b$. 由此我们可以推出
$\displaystyle\lim_{n\to +\infty}f(x_{n})=b$。
同理,另取一个子列$\{x'_n\}$满足条件
$$\displaystyle\lim_{n\to +\infty}x'_n=+\infty.$$从中取一个单调增的子列$\{x'_{n_k}\}$满足条件$$\displaystyle\lim_{k\to +\infty}x'_{n_k}=+\infty$$且 
$$\displaystyle\lim_{k\to +\infty}f(x'_{n_k})=b',\quad \displaystyle\lim_{n\to +\infty}f(x'_{n})=b'$$
将$\{x_{n_k}\}$和$\{x'_{n_k}\}$合起来,组成一个新的递增列,记作$\{y_n\}$。于是,$\{f(y_n)\}$收敛。由归结原理知,$b=b'$. 

自此我们证明了任何数列$\{x_n\}$,如果满足$x_n\rightarrow+\infty$,则$f(x_n)\rightarrow b$.再次利用归结原理,$\displaystyle\lim_{x\to +\infty}f(x)$存在。
\end{proof}
%%%%%%%%%%%%%%%%%%%%%
\begin{example}
设$f$为$U^o(x_0)$上的单调增函数,证明:$f(x_0^-)$与$f(x_0^+)$均存在且有限,且
$$f(x^-_0)=\displaystyle\sup_{x\in U^o_-(x_0)}f(x), \quad f(x^+_0)=\displaystyle\inf_{x\in U^o_+(x_0)}f(x).$$
\end{example}
\begin{proof}
取$x_1\in U^o_-(x_0), x_2\in U^o_+(x_0)$,由$f$是单增函数知,
$$f(x) \leq f(x_2), \forall x\in U^o_-(x_0),$$ $$f(x)\geq f(x_1),\forall x\in U^o_+(x_0).$$
所以,$$f(x_1) \leq \displaystyle\sup_{x\in U^o_-(x_0)}f(x)\leq \displaystyle\inf_{x\in U^o_+(x_0)}f(x) \leq f(x_2).$$

取单调增数列$\{a_n\subset U^o_-(x_0)\}$使得$\displaystyle\lim_{n\to +\infty}a_n = x^-_0$.于是,$\{f(a_n)\}$单调增,且有上界。从而它收敛。记$\displaystyle\lim_{n\to +\infty}f(a_n) = A$.由上确界的定义知,$A \leq \displaystyle\sup_{x\in U^o_-(x_0)}f(x)$. 
设$\{b_n\subset U^o_-(x_0)\}$是一个满足条件$\displaystyle\lim_{n\to +\infty}b_n = x^-_0$的子列。对$\forall \varepsilon > 0$, 存在$N > 0$,当$n>N$时,有
$$A-\varepsilon < f(a_n) < A+\varepsilon.$$
另一方面取$\varepsilon'=x_0 - a_{N+1}$, 
存在$N'$,当$n>N'$时,有$$x_0-b_n < \varepsilon'.$$
又由于$a_n\to x^-_0$,则对任意满足上式的$b_n$, 存在$m > N+1$使得
$$a_{N+1} < b_n < a_m.$$
于是$$A - \varepsilon < f(a_{N+1}) \leq f(b_n) \leq f(a_m) < A + \varepsilon.$$
有极限的定义知,$$\displaystyle\lim_{n\to +\infty}f(b_n)\leq A.$$
由归结原理和上确界的定义知,$$f(x^-_0)=\displaystyle\sup_{x\in U^o_-(x_0)}f(x).$$

类似的办法可证$f(x^+_0)=\displaystyle\inf_{x\in U^o_+(x_0)}f(x)$.
\end{proof}
%%%%%%%%%%%%%%%%%%%%%
%%%%%%%%%%%%%%%%%%%%%
\subsection{思考题}
%%%%%%%%%%%%%%%%%%%%
\begin{example}
\renewcommand\labelenumi{\normalfont(\theenumi)}
\begin{enumerate}
\item 设函数$f$在$(0,+\infty)$上满足方程$f(2x)=f(x)$,且$\displaystyle\lim_{x\to +\infty}f(x)=a$。证明:$f(x)=a$.
\begin{proof}(反证法)设$x_0\in (0,+\infty), f(x_0) \neq a$.由函数方程式知,
$$f(nx_0) = f(x_0).$$
于是,$$\displaystyle\lim_{n\to +\infty}f(nx_0)=f(x_0).$$
由归结原理知,这与假设$\displaystyle\lim_{x\to +\infty}f(x)=a$矛盾。
所以,$f(x)=a,\forall x\in(0,+\infty)$.
\end{proof}
\item 设函数$f$在$(0,+\infty)$上满足方程$f(x^2)=f(x)$,且$$\displaystyle\lim_{x\to 0^+}f(x)=\displaystyle\lim_{x\to +\infty}f(x)=f(1)$$证明:$f(x)\equiv f(1), x\in (0,+\infty)$.
\begin{proof}
如果$x_0<1$, $$\displaystyle\lim_{n\to +\infty}x_0^{2^n}=0.$$
由函数方程式和归结原理知;
$$f(x_0) = \displaystyle\lim_{n\to +\infty}f(x_0) = \displaystyle\lim_{n\to +\infty}f(x_0^{2^n})=f(1).$$

如果$x_0>1$, $$\displaystyle\lim_{n\to +\infty}x_0^{2^n}=+\infty.$$
由函数方程式和归结原理知;
$$f(x_0) = \displaystyle\lim_{n\to +\infty}f(x_0) = \displaystyle\lim_{n\to +\infty}f(x_0^{2^n})=f(1).$$
所以,$f(x)\equiv f(1), x\in(0,+\infty)$.
\end{proof}
\end{enumerate}
\end{example}
%%%%%%%%%%%%%%%%%%%%
\begin{example}
设函数$f:(a, +\infty)\rightarrow \mathbb{R}$在任意有限区间$(a,b)$内有界,且$$\displaystyle\lim_{x\to +\infty}[f(x+1)-f(x)]=A.$$证明:$\displaystyle\lim_{x\to +\infty}\frac{f(x)}{x}=A$.
\end{example}
\begin{proof}设$|f(x)|< M, \forall x\in(a,a+1)$。记$r=x-[x]$.
$$f(x) - f([a] + r) = \displaystyle\sum_{k=[a]}^{[x]-1}(f(k+1+r)-f(k+r)).$$
于是,
\begin{equation*}
\begin{split}
\displaystyle\lim_{x\to +\infty}\frac{f(x)}{x} &= \displaystyle\lim_{x\to +\infty}\frac{f(x)-f([a]+r)}{x}\\&=\displaystyle\lim_{x\to +\infty}\frac{f(x)-f([a]+r)}{[x]-[a]}\cdot\displaystyle\lim_{x\to +\infty}\frac{[x]-[a]}{x} \\&= 
A
\end{split}
\end{equation*}
命题得证。
\end{proof}
%%%%%%%%%%%%%%%%%%%%%%%%%
\begin{example}
设$a>1,b>1$为两个常数,函数$f:\mathbb{R}\rightarrow \mathbb{R}$在$x=0$的近旁有界,且$\forall x\in\mathbb{R}$,有$f(ax)=bf(x)$。证明:$\displaystyle\lim_{x\to 0}f(x)=f(0)$.
\end{example}
\begin{proof}
由函数的定义可知$f(0) = 0$.设$f$在$(-\delta_0, \delta_0)$上有界。记$|f(x)| < M, \forall x\in (-\delta_0, \delta_0)$.对$\forall x \to 0$, 存在$N$
使得$$a^nx\in (-\delta_0, \delta_0), n=1,2,\cdots, N. \quad a^{N+1}x\notin (-\delta_0, \delta_0).$$
显然当$x\to 0$时,$N\to +\infty$.
于是
$$\displaystyle\lim_{x\to 0}f(x) = \displaystyle\lim_{N\to +\infty}\frac{f(a^Nx)}{b^{N}} = 0.$$
命题得证。
\end{proof}
%%%%%%%%%%%%%%%%%%%%%%%%
%%%%%%%%%%%%%%%%%%%%%%%%
%%%%%%%%%%%%%%%%%%%%%%%%
\section{函数极限的性质}
%%%%%%%%%%%%%%%%%%%%%%%%
%%%%%%%%%%%%%%%%%%%%%%%%
\subsection{练习题}
%%%%%%%%%%%%%%%%%%%%%%%
\begin{example}
计算下列极限:
\renewcommand\labelenumi{\normalfont(\theenumi)}
\begin{enumerate}
\item $\displaystyle\lim_{x\to 0}\frac{1+x-2x^3}{1+x^4}$;
\begin{solution}
$\displaystyle\lim_{x\to 0}\frac{1+x-2x^3}{1+x^4} = \frac{\displaystyle\lim_{x\to 0}(1+x-2x^3)}{\displaystyle\lim_{x\to 0}(1+x^4)} =1$.
\end{solution}
\item $\displaystyle\lim_{x\to 1}\frac{x^2-2x+1}{x^3-x}$;
\begin{solution}
$\displaystyle\lim_{x\to 1}\frac{x^2-2x+1}{x^3-x}=\displaystyle\lim_{x\to 1}\frac{(x-1)^2}{x(x-1)(x+1)}=\displaystyle\lim_{x\to 1}\frac{x-1}{x(x+1)}=0$
\end{solution}
\item $\displaystyle\lim_{x\to 0}\frac{\sqrt{1+x}-1}{x-1}$;
\begin{solution}
$\displaystyle\lim_{x\to 0}\frac{\sqrt{1+x}-1}{x-1}=\frac{\displaystyle\lim_{x\to 0}(\sqrt{1+x}-1)}{\displaystyle\lim_{x\to 0}(x-1)} = 0$
\end{solution}
\item $\displaystyle\lim_{x\to 0}\frac{\sqrt{1+x}-\sqrt{1-x}}{x}$;
\begin{solution}
$\displaystyle\lim_{x\to 0}\frac{\sqrt{1+x}-\sqrt{1-x}}{x}=\displaystyle\lim_{x\to 0}\frac{2}{\sqrt{1+x}+\sqrt{1-x}} = 1$
\end{solution}
\item $\displaystyle\lim_{x\to 1}\frac{x^m-1}{x-1}$;
\begin{solution}
\begin{equation*}
\displaystyle\lim_{x\to 1}\frac{x^m-1}{x-1}=
\begin{cases}
\displaystyle\lim_{x\to 1}\frac{x^{-m-1} + \cdots +1}{x^{-m}}=-m, \quad&m<0;\\
0,\quad&m=0;\\
1,\quad&m=1;\\
\displaystyle\lim_{x\to 1}(x^{m-1} + \cdots +1)=m, \quad&m>1.
\end{cases}
\end{equation*}
\end{solution}
\item $\displaystyle\lim_{x\to 1}\frac{x^m-1}{x^n-1}$;
\begin{solution}
\begin{equation*}
\displaystyle\lim_{x\to 1}\frac{x^m-1}{x^n-1}=
\begin{cases}
\text{无定义}, \quad&n=0;\\
0,\quad&m=0;\\
\displaystyle\frac{m}{n},\quad&m\neq 0, n\neq 0.
\end{cases}
\end{equation*}
\end{solution}
\item $\displaystyle\lim_{x\to 0}\frac{(1+x)^{\frac{1}{m}}-1}{x}$;
\begin{solution}
\begin{equation*}
\displaystyle\lim_{x\to 0}\frac{(1+x)^{\frac{1}{m}}-1}{x}=
\begin{cases}
\displaystyle\lim_{x\to 0}\frac{1-(1+x)^{\frac{1}{|m|}}}{x}=-\frac{1}{|m|}=\frac{1}{m}, \quad&m<0;\\
\displaystyle\lim_{x\to 0}\frac{1}{\sqrt[m]{(1+x)^{m-1}}+\cdots + 1}=\frac{1}{m},\quad&m>0
\end{cases}
\end{equation*}
所以,$\displaystyle\lim_{x\to 0}\frac{(1+x)^{\frac{1}{m}}-1}{x}=\frac{1}{m}$.
\end{solution}
\item $\displaystyle\lim_{x\to 1}\frac{x+x^2+\cdots+x^m-m}{x-1}$
\begin{solution}
$\displaystyle\lim_{x\to 1}\frac{x+x^2+\cdots+x^m-m}{x-1}=\displaystyle\lim_{x\to 1}\left[1+(x+1)+\cdots+(x^{m-1}+x^{m-2}+\cdots+1)\right]=\frac{m(m+1)}{2}.$
\end{solution}
\item $\displaystyle\lim_{x\to 0}\frac{(1+mx)^n-(1+nx)^m}{x^2}$
\begin{solution}我们对$n,m$分情况讨论:
\renewcommand\labelenumi{\normalfont(\theenumi)}
\begin{enumerate}
\item 当$m=0$或者$n=0$时,分子为$0$,于是极限为$0$.
\item 当$m=n=1$或者$m=n=-1$是,分子为$0$,于是极限为$0$.
\item 当$m=1,n=-1$时,$$(1+x)^{-1}-(1-x)=\frac{x^2}{1+x}.$$于是,$\displaystyle\lim_{x\to 0}\frac{(1+x)^{-1}-(1-x)}{x^2}=1$.
\item 当$m=-1,n=1$时,$$(1-x)-(1+x)^{-1} = \frac{-x^2}{1+x}.$$于是,$\displaystyle\lim_{x\to 0}\frac{(1-x)-(1+x)^{-1}}{x^2}=-1$.
\item 当$m>0, n<0$时,$$(1+mx)^n-(1+nx)^m =\frac{1-(1+mx)^{|n|}(1+nx)^m}{(1+mx)^{|n|}}.$$于是,
$\displaystyle\lim_{x\to 0}\frac{(1+mx)^n-(1+nx)^m}{x^2}=n^2m^2-C^2_{|n|}m^2-C^2_mn^2$.
\item 当$m<0, n>0$时,$$(1+mx)^n-(1+nx)^m =\frac{(1+mx)^n(1+nx)^{|m|}-1}{(1+nx)^{|m|}}.$$于是,
$\displaystyle\lim_{x\to 0}\frac{(1+mx)^n-(1+nx)^m}{x^2}=C^2_{n}m^2+C^2_{|m|}n^2-n^2m^2$.
\item 当$m>0,n>0$时,$$(1+mx)^n-(1+nx)^m = (C^2_nm^2-C^2_mn^2)x^2+\cdots.$$于是,
$\displaystyle\lim_{x\to 0}\frac{(1+mx)^n-(1+nx)^m}{x^2}=C^2_nm^2-C^2_mn^2$.
\item 当$m<0,n<0$时,$$(1+mx)^n-(1+nx)^m =\frac{(1+nx)^{|m|}-(1+mx)^{|n|}}{(1+mx)^{|n|}(1+nx)^{|m|}}=\frac{(C^2_{|m|}n^2-C^2_{|n|}m^2)x^2+\cdots}{(1+mx)^{|n|}(1+nx)^{|m|}}.$$于是,
$\displaystyle\lim_{x\to 0}\frac{(1+mx)^n-(1+nx)^m}{x^2}=C^2_{|m|}n^2-C^2_{|n|}m^2$.
\end{enumerate}
\end{solution}
\item $\displaystyle\lim_{x\to 0}\frac{(1+nx)^{\frac{1}{m}}-(1+mx)^{\frac{1}{n}}}{x}$
\begin{solution}
$\displaystyle\lim_{x\to 0}\frac{(1+nx)^{\frac{1}{m}}-1}{x}=
\begin{cases}
\displaystyle\lim_{x\to 0}\frac{1-\sqrt[|m|]{1+nx}}{x\cdot\sqrt[|m|]{1+nx}}=-\frac{n}{|m|},\quad&m<0,\\
\displaystyle\lim_{x\to 0}\frac{nx}{\sqrt[m]{(1+nx)^{m-1}}+\sqrt[m]{(1+nx)^{m-2}}+\cdots + 1}=\frac{n}{m},\quad&m>0.
\end{cases}$
即,$\displaystyle\lim_{x\to 0}\frac{(1+nx)^{\frac{1}{m}}-1}{x}=\frac{n}{m}$.

同理,$\displaystyle\lim_{x\to 0}\frac{(1+mx)^{\frac{1}{n}}-1}{x}=\frac{m}{n}$.
所以,
\begin{equation*}
\begin{split}
\displaystyle\lim_{x\to 0}\frac{(1+nx)^{\frac{1}{m}}-(1+mx)^{\frac{1}{n}}}{x}&=\displaystyle\lim_{x\to 0}\frac{(1+nx)^{\frac{1}{m}}-1}{x}-\displaystyle\lim_{x\to 0}\frac{(1+mx)^{\frac{1}{n}}-1}{x}\\&=\frac{n}{m}-\frac{m}{n}\\&=\frac{n^2-m^2}{nm}.
\end{split}
\end{equation*}
\end{solution}
\item $\displaystyle\lim_{x\to 0}\frac{\sqrt{a^2+x}-a}{x}(a>0)$;
\begin{solution}
$\displaystyle\lim_{x\to 0}\frac{\sqrt{a^2+x}-a}{x}=\displaystyle\lim_{x\to 0}\frac{x}{x(\sqrt{a^2+x}+a)}=\frac{1}{2a}.$
\end{solution}
\item $\displaystyle\lim_{x\to +\infty}\frac{(3x+1)^{70}(8x-5)^{20}}{(5x-1)^{90}}$;
\begin{solution}
$\displaystyle\lim_{x\to +\infty}\frac{(3x+1)^{70}(8x-5)^{20}}{(5x-1)^{90}}=\frac{3^{70}8^{20}}{5^{90}}.$
\end{solution}
\item $\displaystyle\lim_{x\to 1}\frac{\sqrt[m]{x}-1}{\sqrt[n]{x}-1}(m,n\in\mathbb{N})$;
\begin{solution}
$\displaystyle\lim_{x\to 1}\frac{\sqrt[m]{x}-1}{\sqrt[n]{x}-1}=\displaystyle\lim_{x\to 1}\frac{(x-1)(\sqrt[n]{x^{n-1}}+\cdots+1)}{(x-1)(\sqrt[m]{x^{m-1}}+\cdots+1)}=\frac{n}{m}.$
\end{solution}
\item $\displaystyle\lim_{x\to 0}\frac{\sqrt[m]{1+\alpha x}\sqrt[n]{1+\beta x} -1}{x}$;
\begin{solution}无论$m>0$或者$m<0$,$$\displaystyle\lim_{x\to 0}\frac{\sqrt[m]{1+\alpha x}-1}{x}=\frac{\alpha}{m}.$$
同理可知,$$\displaystyle\lim_{x\to 0}\frac{\sqrt[n]{1+\beta x}-1}{x}=\frac{\beta}{n}.$$
于是,
\begin{equation*}
\begin{split}
\displaystyle\lim_{x\to 0}\frac{\sqrt[m]{1+\alpha x}\sqrt[n]{1+\beta x} -1}{x}&=
\displaystyle\lim_{x\to 0}\frac{\sqrt[n]{1+\beta x}(\sqrt[m]{1+\alpha x} -1)}{x}
+\displaystyle\lim_{x\to 0}\frac{\sqrt[n]{1+\beta x}-1}{x}\\&=\frac{\alpha}{m}+\frac{\beta}{n}\\&=\frac{n\alpha+m\beta}{mn}.
\end{split}
\end{equation*}
\end{solution}
\item $\displaystyle\lim_{x\to 0}\frac{\sqrt{1+x}-\sqrt{1-x}}{\sqrt[3]{1+x}-\sqrt[3]{1-x}}$;
\begin{solution}
$\displaystyle\lim_{x\to 0}\frac{\sqrt{1+x}-\sqrt{1-x}}{\sqrt[3]{1+x}-\sqrt[3]{1-x}}=\displaystyle\lim_{x\to 0}\frac{\sqrt[3]{(1+x)^2}+\sqrt[3]{(1+x)(1-x)}+\sqrt[3]{(1-x)^2}}{\sqrt{1+x}+\sqrt{1-x}}=\frac{3}{2}$.
\end{solution}
\item $\displaystyle\lim_{x\to \infty}(\sqrt{(a+x)(b+x)}-\sqrt{(a-x)(b-x)})$;
\begin{solution}
$\displaystyle\lim_{x\to \infty}(\sqrt{(a+x)(b+x)}-\sqrt{(a-x)(b-x)})=\displaystyle\lim_{x\to \infty}\frac{2(a+b)x}{\sqrt{(a+x)(b+x)}+\sqrt{(a-x)(b-x)}}=a+b$.
\end{solution}
\item $\displaystyle\lim_{x\to 1}\left(\frac{m}{1-x^m}-\frac{n}{1-x^n}\right)(m,n\in\mathbb{N})$;
\begin{solution}
记$A=\displaystyle\sum_{k=0}^{n-1}x^k, B=\displaystyle\sum_{k=0}^{m-1}x^k$.
显然
$$A-n=\displaystyle\sum_{k=1}^{n-1}(x^k-1)=\displaystyle\sum_{k=1}^{n-1}\displaystyle\sum_{j=0}^{k-1}x^j,$$
$$B-m=\displaystyle\sum_{k=1}^{m-1}(x^k-1)=\displaystyle\sum_{k=1}^{m-1}\displaystyle\sum_{j=0}^{k-1}x^j.$$
于是,
\begin{equation*}
\begin{split}
\frac{m}{1-x^m}-\frac{n}{1-x^n}&=\frac{1}{(1-x)AB}(mA-nB)\\
&=\frac{1}{(1-x)AB}(m(A-n)-n(B-m))\\
&=\frac{1}{AB}\left(m\displaystyle\sum_{k=1}^{n-1}\displaystyle\sum_{j=0}^{k-1}x^j-n\displaystyle\sum_{k=1}^{m-1}\displaystyle\sum_{j=0}^{k-1}x^j\right).
\end{split}
\end{equation*}
所以,
$\displaystyle\lim_{x\to 1}\left(\frac{m}{1-x^m}-\frac{n}{1-x^n}\right)=\frac{1}{mn}\left(m\frac{n(n-1)}{2}-n\frac{m(m-1)}{2}\right)=\frac{n-m}{2}.$
\end{solution}
\item $\displaystyle\lim_{x\to a}\frac{(x^n-a^n)-na^{n-1}(x-a)}{(x-a)^2}(n\in\mathbb{N})$;
\begin{solution}
当$n=1$时,分子为$0$,所以极限为$0$. 当$n>1$时,
\begin{equation*}
\begin{split}
(x^n-a^n)-na^{n-1}(x-a)&=(x-a)\displaystyle\sum_{k=0}^{n-2}a^k(x^{n-1-k}-a^{n-1-k})\\
&=(x-a)^2\displaystyle\sum_{k=0}^{n-2}a^k\displaystyle\sum_{j=0}^{n-2-k}a^jx^{n-2-k-j}.
\end{split}
\end{equation*}
于是,
$\displaystyle\lim_{x\to a}\frac{(x^n-a^n)-na^{n-1}(x-a)}{(x-a)^2}=a^{n-2}\displaystyle\sum_{k=0}^{n-2}(n-1-k)=\frac{n(n-1)}{2}a^{n-2}$.
\end{solution}
\item $\displaystyle\lim_{x\to 1}\frac{x^{n+1}-(n+1)x+n}{(x-1)^2}(n\in\mathbb{N})$;
\begin{solution}我们先计算分子:
\begin{equation*}
\begin{split}
x^{n+1}-(n+1)x+n&=x(x^{n}-1)-n(x-1)\\&=
(x-1)\displaystyle\sum_{k=1}^{n}(x^k-1)\\&=
(x-1)^2\displaystyle\sum_{k=1}^{n}\displaystyle\sum_{j=0}^{k-1}x^j.
\end{split}
\end{equation*}
于是,$\displaystyle\lim_{x\to 1}\frac{x^{n+1}-(n+1)x+n}{(x-1)^2}=\displaystyle\sum_{k=1}^{n}\displaystyle\sum_{j=0}^{k-1}1=\frac{n(n+1)}{2}.$
\end{solution}
\item $\displaystyle\lim_{x\to 1}\frac{(1-\sqrt{x})(1-\sqrt[3]{x})\cdots(1-\sqrt[n]{x})}{(1-x)^{n-1}}(n\in\mathbb{N})$;
\begin{solution}分子有理化,$$1-\sqrt[l]{x}=\frac{1-x}{\displaystyle\sum_{j=0}^{l-1}\sqrt[l]{x^j}}.$$
所以,
$$\frac{(1-\sqrt{x})(1-\sqrt[3]{x})\cdots(1-\sqrt[n]{x})}{(1-x)^{n-1}}=\frac{1}{\displaystyle\prod_{l=2}^{n}\left(\displaystyle\sum_{j=0}^{l-1}\sqrt[l]{x^j}\right)}.$$
于是,$\displaystyle\lim_{x\to 1}\frac{(1-\sqrt{x})(1-\sqrt[3]{x})\cdots(1-\sqrt[n]{x})}{(1-x)^{n-1}}=\frac{1}{n!}.$
\end{solution}
\end{enumerate}
\end{example}
%%%%%%%%%%%%%%%%%%%%%%%
\begin{example}
设$\displaystyle\lim_{x\to x_0}f(x)=a$,$\displaystyle\lim_{x\to x_0}g(x)=b$。
\renewcommand\labelenumi{\normalfont(\theenumi)}
\begin{enumerate}
\item 若$a>b$,则在某$U^o(x_0)$内有$f(x)>g(x)$;
\begin{proof}取$\varepsilon = \displaystyle\frac{a-b}{2}$.由函数极限定义知,存在一个空心邻域$U^o(x_0,\delta_1)$使得
$$g(x)-b < \varepsilon \Rightarrow g(x) < \frac{a+b}{2}, \quad\forall x\in U^o(x_0,\delta_1).$$
取$U^o(x_0,\delta)\subset U^o(x_0,\delta_1)$使得
$$f(x)-a > -\varepsilon\Rightarrow f(x)>\frac{a+b}{2}, \quad\forall x\in U^o(x_0,\delta).$$
所以,$f(x)>g(x),\forall x\in U^o(x_0,\delta)$.
\end{proof}
\item 若在某$U^o(x_0)$内有$f(x)<g(x)$,问:是否必有$a<b$?说明理由。
\begin{solution}不会是严格$a<b$. 例如在$0$的空心邻域$U^0(0,\frac{1}{2})$, $1+\frac{x^2}{3}=f(x) < g(x) = 1+\frac{x^2}{2}$. 但是$\displaystyle\lim_{x\to 0}f(x)=\displaystyle\lim_{x\to 0}g(x)=1$.
\end{solution}
\end{enumerate}
\end{example}
%%%%%%%%%%%%%%%%%%%%%%%
\begin{example}
求下列极限($n\in\mathbb{N}$):
\renewcommand\labelenumi{\normalfont(\theenumi)}
\begin{enumerate}
\item $\displaystyle\lim_{x\to 2^+}\frac{[x]^2-4}{x^2-4}$;
\begin{solution}取$\delta = \frac{1}{2}$,当$2<x < 2+\delta$时,$[x]=2$.所以,$\displaystyle\lim_{x\to 2^+}\frac{[x]^2-4}{x^2-4}=0$.
\end{solution}
\item $\displaystyle\lim_{x\to 2^-}\frac{[x]^2+4}{x^2+4}$;
\begin{solution}$\displaystyle\lim_{x\to 2^-}\frac{[x]^2+4}{x^2+4}=\frac{\displaystyle\lim_{x\to 2^-}([x]^2+4)}{\displaystyle\lim_{x\to 2^-}(x^2+4)}=\frac{5}{8}$.
\end{solution}
\item $\displaystyle\lim_{x\to 1^-}\frac{[4x]}{1+x}$;
\begin{proof}显然$[4x]=3,\forall x\in (1-\frac{1}{4}, 1)$.所以,$\displaystyle\lim_{x\to 1^-}\frac{[4x]}{1+x}=\frac{3}{2}.$
\end{proof}
\item $\displaystyle\lim_{x\to 0^-}\frac{|x|}{x}\cdot\frac{1}{1+x^n}$;
\begin{proof}
$\displaystyle\lim_{x\to 0^-}\frac{|x|}{x}\cdot\frac{1}{1+x^n}=-1$
\end{proof}
\item $\displaystyle\lim_{x\to 0^+}\frac{|x|}{x}\cdot\frac{1}{1+x^n}$;
\begin{proof}
$\displaystyle\lim_{x\to 0^+}\frac{|x|}{x}\cdot\frac{1}{1+x^n}=1$.
\end{proof}
\item $\displaystyle\lim_{x\to 0}\frac{\sqrt[n]{1+x}-1}{x}$;
\begin{proof}
$\displaystyle\lim_{x\to 0}\frac{\sqrt[n]{1+x}-1}{x}=\displaystyle\lim_{x\to 0}\frac{1}{\sqrt[n]{(1+x)^{n-1}}+\sqrt[n]{(1+x)^{n-2}}+\cdots + 1}=\frac{1}{n}$.
\end{proof}
\item $\displaystyle\lim_{x\to \infty}\frac{[x]}{x}$;
\begin{proof}由于$x-1\leq[x]\leq x$,$\displaystyle\lim_{x\to \infty}\frac{[x]}{x}=1$.
\end{proof}
\item $\displaystyle\lim_{x\to +\infty}\frac{\sqrt{x+\sqrt{x+\sqrt{x}}}}{\sqrt{x+1}}$;
\begin{proof}
$\displaystyle\lim_{x\to +\infty}\frac{\sqrt{x+\sqrt{x+\sqrt{x}}}}{\sqrt{x+1}}=1$.
\end{proof}
\end{enumerate}
\end{example}
%%%%%%%%%%%%%%%%%%%%%%%
\begin{example}
设$p(x)=a_1x+a_2x^2+\cdots+a_nx^n$。证明:
$$\displaystyle\lim_{x\to 0}\frac{\sqrt[m]{1+p(x)}-1}{x}=\frac{a_1}{m},$$
其中$n,m\in\mathbb{N}$.
\end{example}
\begin{proof}
分子有理化,
$$\frac{\sqrt[m]{1+p(x)}-1}{x}=\frac{a_1+a_2x+\cdots+a_nx^{n-1}}{\sqrt[m]{(1+p(x))^{m-1}+\sqrt[m]{(1+p(x))^{m-2}}+\cdots + 1}}.$$
于是,$\displaystyle\lim_{x\to 0}\frac{\sqrt[m]{1+p(x)}-1}{x}=\frac{a_1}{m}$.
\end{proof}
%%%%%%%%%%%%%%%%%%%%%%%%
\begin{example}
定出常数$a$与$b$,使得下列等式成立:
\renewcommand\labelenumi{\normalfont(\theenumi)}
\begin{enumerate}
\item $\displaystyle\lim_{x\to \infty}\left(\frac{x^2+1}{x+1}-ax-b\right)=0$;
\begin{solution}
$\displaystyle\frac{x^2+1}{x+1}-ax-b = \frac{(1-a)x-(a+b)+\displaystyle\frac{1-b}{x}}{1+\displaystyle\frac{1}{x}}$.如果极限为$0$,则$1-a = 0, a+b = 0$.于是,$a=1,b=-1$.
\end{solution} 
\item $\displaystyle\lim_{x\to +\infty}\left(\sqrt{x^2-x+1}-ax-b\right)=0$;
\begin{proof}分子有理化
$$\sqrt{x^2-x+1}-ax-b=\frac{(1-a^2)x^2-(1+2ab)x+(1-b^2)}{\sqrt{x^2-x+1}+ax+b}.$$
如果极限为$0$,则$1+a\neq 0, 1-a^2 = 0, 1+2ab=0$.于是,$a=1, b=-\frac{1}{2}$.
\end{proof}
\item $\displaystyle\lim_{x\to -\infty}\left(\sqrt{x^2-x+1}-ax-b\right)=0$;
\begin{proof}分子有理化
$$\sqrt{x^2-x+1}-ax-b=\frac{(1-a^2)x^2-(1+2ab)x+(1-b^2)}{\sqrt{x^2-x+1}+ax+b}.$$
如果极限为$0$,则$1-a\neq 0, 1-a^2 = 0, 1+2ab=0$.于是,$a=-1, b=\frac{1}{2}$.
\end{proof}
\end{enumerate}
\end{example}
%%%%%%%%%%%%%%%%%%%%%%%
\begin{example}
求下列极限:
\renewcommand\labelenumi{\normalfont(\theenumi)}
\begin{enumerate}
\item $\displaystyle\lim_{x\to 0}\frac{\sin{ax}}{\sin{bx}}(b\neq 0)$;
\begin{solution}
$\displaystyle\lim_{x\to 0}\frac{\sin{ax}}{\sin{bx}}=\frac{a}{b}\displaystyle\lim_{x\to 0}\frac{\sin{ax}}{ax}\cdot\displaystyle\lim_{x\to 0}\frac{bx}{\sin{bx}}=\frac{a}{b}$.
\end{solution}
\item $\displaystyle\lim_{x\to 0}\frac{\sin(\sin{x})}{x}$;
\begin{solution}
$\displaystyle\lim_{x\to 0}\frac{\sin(\sin{x})}{x}=\displaystyle\lim_{x\to 0}\frac{\sin(\sin{x})}{\sin{x}}\cdot\displaystyle\lim_{x\to 0}\frac{\sin{x}}{x}=1$
\end{solution}
\item $\displaystyle\lim_{h\to 0}\frac{\sin(x+h)-\sin{x}}{h}$;
\begin{solution}利用三角函数性质,有
$$\sin(x+h)-\sin{x}=\sin{x}(\cos{h}-1)+\cos{x}\sin{h}=-2
\sin^2{\frac{h}{2}}\sin{x}+\cos{x}\sin{h}.$$
所以,$\displaystyle\lim_{h\to 0}\frac{\sin(x+h)-\sin{x}}{h}=\cos{x}$
\end{solution}
\item $\displaystyle\lim_{x\to +\infty}\left(\frac{1+x}{3+x}\right)^x$;
\begin{solution}
$\displaystyle\lim_{x\to +\infty}\left(\frac{1+x}{3+x}\right)^x=\displaystyle\lim_{x\to +\infty}\left(1-\frac{2}{3+x}\right)^{-\frac{3+x}{2}\cdot\frac{-2x}{3+x}}=e^{-2}$.
\end{solution}
\item $\displaystyle\lim_{x\to 0}\frac{\sin{x^3}}{\sin^2{x}}$;
\begin{solution}
$\displaystyle\lim_{x\to 0}\frac{\sin{x^3}}{\sin^2{x}}=\displaystyle\lim_{x\to 0}\frac{\sin{x^3}}{x^3}\cdot\displaystyle\lim_{x\to 0}\frac{x^2}{\sin^2{x}}\cdot\displaystyle\lim_{x\to 0}x=0$.
\end{solution}
\item $\displaystyle\lim_{x\to 0}\frac{\arctan{x}}{x}$;
\begin{solution}
令$y=\arctan{x}$, 则$x\to 0\Rightarrow y\to 0$.
$\displaystyle\lim_{x\to 0}\frac{\arctan{x}}{x}=\displaystyle\lim_{y\to 0}\frac{y\cos{y}}{\sin{y}}=1$.
\end{solution}
\item $\displaystyle\lim_{x\to +\infty}x\sin{\frac{1}{x}}$;
\begin{solution}
$\displaystyle\lim_{x\to +\infty}x\sin{\frac{1}{x}}=\displaystyle\lim_{y\to 0^+}\frac{\sin{y}}{y}=1$.
\end{solution}
\item $\displaystyle\lim_{x\to a}\frac{\sin^2{x}-\sin^2{a}}{x-a}$;
\begin{solution}利用三角函数性质:
\begin{equation*}
\begin{split}
\sin^2{x}-\sin^2{a}&=(\sin{x}+\sin{a})(\sin{x}-\sin{a})\\
&=(\sin{x}+\sin{a})\left[\sin(x-a)\cos{a}-2\sin{a}\sin^2\left(\frac{x-a}{2}\right)\right].
\end{split}
\end{equation*}
所以,$\displaystyle\lim_{x\to a}\frac{\sin^2{x}-\sin^2{a}}{x-a}=2\sin{a}\cos{a}$
\end{solution}
\item $\displaystyle\lim_{x\to 0}\frac{\sin{4x}}{\sqrt{x+1}-1}$;
\begin{solution}
$\displaystyle\lim_{x\to 0}\frac{\sin{4x}}{\sqrt{x+1}-1}=\displaystyle\lim_{x\to 0}\frac{\sin{4x}}{4x}\cdot\displaystyle\lim_{x\to 0}\frac{4x\left(\sqrt{x+1}+1\right)}{x}=2$
\end{solution}
\item $\displaystyle\lim_{x\to 0}\frac{\sqrt{1-\cos{x^2}}}{1-\cos{x}}$.
\begin{proof}
利用三角函数性质:$1-\cos{x}=2\sin^2{\displaystyle\frac{x}{2}}$.于是,
$$\frac{\sqrt{1-\cos{x^2}}}{1-\cos{x}}=\frac{\sqrt{2}\sin{\frac{x^2}{2}}}{2\sin^2{\frac{x}{2}}}=\sqrt{2}\cdot\frac{\sin{\frac{x^2}{2}}}{\frac{x^2}{2}}\cdot\left(\frac{\frac{x}{2}}{\sin{\frac{x}{2}}}\right)^2.$$
所以,$\displaystyle\lim_{x\to 0}\frac{\sqrt{1-\cos{x^2}}}{1-\cos{x}}=\sqrt{2}$.
\end{proof}
\end{enumerate}
\end{example}
%%%%%%%%%%%%%%%%%%%%%%%
\begin{example}
求下列极限:
\renewcommand\labelenumi{\normalfont(\theenumi)}
\begin{enumerate}
\item $\displaystyle\lim_{n\to +\infty}\sin(\pi\sqrt{n^2+1})$;
\begin{solution}利用三角函数性质:
$$\sin(\pi\sqrt{n^2+1}-n\pi)=\sin(\pi\sqrt{n^2+1})\cos{n\pi}\Rightarrow \sin(\pi\sqrt{n^2+1})=\frac{\sin(\pi\sqrt{n^2+1}-n\pi)}{\cos{n\pi}}.$$
所以,
 $\displaystyle\lim_{n\to +\infty}\left|\sin(\pi\sqrt{n^2+1})\right| = \displaystyle\lim_{n\to +\infty}\left|\frac{\sin(\pi\sqrt{n^2+1}-n\pi)}{\frac{\pi}{\sqrt{n^2+1}+n}}\right|\cdot\displaystyle\lim_{n\to +\infty}\frac{\pi}{(\sqrt{n^2+1}+n)}=0$.
于是,$\displaystyle\lim_{n\to +\infty}\sin(\pi\sqrt{n^2+1})=0$。
\end{solution}
\item $\displaystyle\lim_{n\to +\infty}\sin^2(\pi\sqrt{n^2+n})$;
\begin{solution}利用三角函数性质,
$\sin^2(\pi\sqrt{n^2+n})=\sin^2(\pi\sqrt{n^2+n}-n\pi)=\sin^2\left(\pi\displaystyle\frac{n}{\sqrt{n^2+n}+n}\right)$.
所以,$\displaystyle\lim_{n\to +\infty}\sin^2(\pi\sqrt{n^2+n})=1.$
\end{solution}
\end{enumerate}
\end{example}
%%%%%%%%%%%%%%%%%%%%%%%
\begin{example}
\renewcommand\labelenumi{\normalfont(\theenumi)}
\begin{enumerate}
\item 证明:$\forall k\in\mathbb{N}$,有$\displaystyle\lim_{x\to +\infty}\left[\sin{\sqrt{x+k}}-\sin\sqrt{x}\right]$;
\begin{proof}利用三角函数性质,
$$\sin{\sqrt{x+k}}-\sin\sqrt{x}=\sin\left(\frac{k}{\sqrt{x+k}+\sqrt{x}}\right)\cos\sqrt{x}-2\sin\sqrt{x}\sin^2\left(\frac{k}{2(\sqrt{x+1}+\sqrt{x}}\right).$$
于是,$\displaystyle\lim_{x\to +\infty}\left[\sin{\sqrt{x+k}}-\sin\sqrt{x}\right]=0$.
\end{proof}
\item 设常数$a_1,a_2,\cdots,a_n$满足$a_1+a_2+\cdots+a_n=0$,证明:
$$\displaystyle\lim_{x\to +\infty}\displaystyle\sum_{k=1}^na_k\sin{\sqrt{x+k}}=0.$$
\begin{proof}显然
$$\displaystyle\sum_{k=1}^na_k\sin{\sqrt{x+k}}=\displaystyle\sum_{k=2}^na_k\left[\sin{\sqrt{x+k}}-\sin\sqrt{x+1}\right].$$
由上题知,这里和的每一项都收敛于$0$.所以,$\displaystyle\lim_{x\to +\infty}\displaystyle\sum_{k=1}^na_k\sin{\sqrt{x+k}}=0$。
\end{proof}
\end{enumerate}
\end{example}
%%%%%%%%%%%%%%%%%%%%%%%
\begin{example}
证明:$$\displaystyle\lim_{x\to +\infty}\left(\cos{\frac{x}{2}}\cos{\frac{x}{4}}\cdots\cos{\frac{x}{2^n}}\right)=
\begin{cases}
\displaystyle\frac{\sin{x}}{x},\quad&x\neq 0,\\
1,\quad&x=0.
\end{cases}
$$
\end{example}
\begin{proof}
当$x=0$时,$\cos{\frac{x}{2^k}}=\cos{0}=1$.所以,极限为$1$.下面证$x\neq 0$的情形。
$$\cos{\frac{x}{2}}\cos{\frac{x}{4}}\cdots\cos{\frac{x}{2^n}}=\sin{x}\Big{/}\frac{\sin{\frac{x}{2^n}}}{2^n}=\left(\frac{x}{2^n}\Big{/}\sin{\frac{x}{2^n}}\right)\cdot\left(\frac{\sin{x}}{x}\right).$$
所以,$\displaystyle\lim_{x\to +\infty}\left(\cos{\frac{x}{2}}\cos{\frac{x}{4}}\cdots\cos{\frac{x}{2^n}}\right)=\frac{\sin{x}}{x}$.
\end{proof}
%%%%%%%%%%%%%%%%%%%%%%%
\begin{example}
计算极限:
\renewcommand\labelenumi{\normalfont(\theenumi)}
\begin{enumerate}
\item $\displaystyle\lim_{x\to 0}(1-2x)^{\frac{1}{x}}$;
\begin{solution}
$\displaystyle\lim_{x\to 0}(1-2x)^{\frac{1}{x}}=\displaystyle\lim_{x\to 0}(1-2x)^{\frac{-1}{2x}\cdot(-2)}=e^{-2}$.
\end{solution}
\item $\displaystyle\lim_{x\to +\infty}\left(\frac{x+a}{x-a}\right)^x$;
\begin{solution}
$\displaystyle\lim_{x\to +\infty}\left(\frac{x+a}{x-a}\right)^x=\displaystyle\lim_{x\to +\infty}\left(1+\frac{2a}{x-a}\right)^{\frac{x-a}{2a}\cdot\frac{2ax}{x-a}}=e^{2a}$.
\end{solution}
\item $\displaystyle\lim_{x\to 0}\left(\frac{1+\tan{x}}{1+\sin{x}}\right)^{\frac{1}{\sin{x}}}$;
\begin{solution}利用三角函数性质,
\begin{equation*}
\begin{split}
\left(\frac{1+\tan{x}}{1+\sin{x}}\right)^{\frac{1}{\sin{x}}}&=\left(1+\frac{2\sin{x}\sin^2{\frac{x}{2}}}{(1+\sin{x})\cos{x}}\right)^{\frac{1}{\sin{x}}}\\&=
\left(1+\frac{2\sin{x}\sin^2{\frac{x}{2}}}{(1+\sin{x})\cos{x}}\right)^{\displaystyle\frac{(1+\sin{x})\cos{x}}{2\sin{x}\sin^2{\frac{x}{2}}}\cdot\frac{2\sin^2{\frac{x}{2}}}{(1+\sin{x})\cos{x}}}.
\end{split}
\end{equation*}
所以,$\displaystyle\lim_{x\to 0}\left(\frac{1+\tan{x}}{1+\sin{x}}\right)^{\frac{1}{\sin{x}}}=1$。
\end{solution}
\item $\displaystyle\lim_{x\to 0}\left(\frac{\cos{x}}{\cos{2x}}\right)^{\frac{1}{x^2}}$;
\begin{solution}利用三角函数性质,
$$\displaystyle\frac{\cos{x}-\cos{2x}}{\cos{2x}}=\displaystyle\frac{2\cos{x}\sin^2{\frac{x}{2}}+\sin^2{x}}{\cos{2x}}.$$
于是,
$$\displaystyle\lim_{x\to 0}\displaystyle\frac{\cos{x}-\cos{2x}}{\cos{2x}}=\displaystyle\frac{2\cos{x}\sin^2{\frac{x}{2}}+\sin^2{x}}{x^2\cos{2x}}=\frac{3}{2}.$$
所以,$\displaystyle\lim_{x\to 0}\left(\frac{\cos{x}}{\cos{2x}}\right)^{\frac{1}{x^2}}=e^{\frac{3}{2}}$.
\end{solution}
\item  $\displaystyle\lim_{x\to \frac{\pi}{4}}\left(\tan{x}\right)^{\tan{2x}}$;
\begin{solution}
利用三角函数性质:
$$\tan{2x}=\frac{\sin{2x}}{\cos{2x}}=\frac{2\sin{x}\cos{x}}{(\sin{x}+\cos{x})(\cos{x}-\sin{x})}.$$
\begin{equation*}
\begin{split}
(\tan{x})^{\tan{2x}}&=\left(1+\frac{\sin{x}-\cos{x}}{\cos{x}}\right)^{\displaystyle\frac{2\sin{x}\cos{x}}{(\sin{x}+\cos{x})(\cos{x}-\sin{x})}}\\
&=\left[\left(1+\frac{\sin{x}-\cos{x}}{\cos{x}}\right)^{\displaystyle\frac{\cos{x}}{(\sin{x}-\cos{x})}}\right]^{\displaystyle\frac{-2\sin{x}}{(\sin{x}+\cos{x})}}
\end{split}
\end{equation*}
所以,$\displaystyle\lim_{x\to \frac{\pi}{4}}\left(\tan{x}\right)^{\tan{2x}}=e^{-1}$.
\end{solution}
\item $\displaystyle\lim_{x\to \frac{\pi}{2}}\left(\sin{x}\right)^{\tan{x}}$;
\begin{solution}令$x=\frac{\pi}{2}+y$.
\begin{equation*}
\begin{split}
\left(\sin{x}\right)^{\tan{x}}&=\left[\left(1+(\cos{y} - 1)\right)^{\displaystyle\frac{1}{\cos{y} - 1}}\right]^{\displaystyle\frac{(1-\cos{y})\cos{y}}{\sin{y}}}\\
&=\left[\left(1+(\cos{y} - 1)\right)^{\displaystyle\frac{1}{\cos{y} - 1}}\right]^{\displaystyle\frac{\sin{\frac{y}{2}}\cos{y}}{\cos{\frac{y}{2}}}}
\end{split}
\end{equation*}
所以,$\displaystyle\lim_{x\to \frac{\pi}{2}}\left(\sin{x}\right)^{\tan{x}}=e^{0}=1$.
\end{solution}
\item $\displaystyle\lim_{x\to +\infty}\left(\sin{\frac{1}{x}}+\cos{\frac{1}{x}}\right)^{x}$;
\begin{solution}
令$y=\frac{1}{x}$.
$$\displaystyle\lim_{y\to 0^+}\frac{\sin{y}+\cos{y} - 1}{y}=\displaystyle\lim_{y\to 0^+}\frac{\sin{y}}{y}+\displaystyle\lim_{y\to 0^+}\frac{\cos{y} - 1}{y}= 1.$$
于是,
\begin{equation*}
\begin{split}
\displaystyle\lim_{x\to +\infty}\left(\sin{\frac{1}{x}}+\cos{\frac{1}{x}}\right)^{x}&=\displaystyle\lim_{y\to 0^+}\left(\sin{y}+\cos{y}\right)^{\frac{1}{y}}\\
&=\displaystyle\lim_{y\to 0^+}\left[(1+\sin{y}+\cos{y}-1)^{\frac{1}{\sin{y}+\cos{y}-1}}\right]^{\frac{\sin{y}+\cos{y}-1}{y}}\\
&=e
\end{split}
\end{equation*}
\end{solution}
\item $\displaystyle\lim_{x\to 0^+}\left(\cos{\sqrt{x}}\right)^{\frac{1}{x}}$;
\begin{solution}
利用三角函数公式$\cos{\sqrt{x}}-1 = -2\sin^2{\frac{\sqrt{x}}{2}}$.
\begin{equation*}
\begin{split}
\displaystyle\lim_{x\to 0^+}\left(\cos{\sqrt{x}}\right)^{\frac{1}{x}}&=
\displaystyle\lim_{x\to 0^+}\left[(1+\cos{\sqrt{x}}-1)^{\frac{1}{\cos{\sqrt{x}}-1}}\right]^{\frac{\cos{\sqrt{x}}-1}{x}}\\&=
\displaystyle\lim_{x\to 0^+}\left[(1+\cos{\sqrt{x}}-1)^{\frac{1}{\cos{\sqrt{x}}-1}}\right]^{-\frac{1}{2}\cdot\left(\frac{\sin{\frac{\sqrt{x}}{2}}}{\frac{\sqrt{x}}{2}}\right)^2}=e^{-\frac{1}{2}}
\end{split}
\end{equation*}
\end{solution}
\item $\displaystyle\lim_{n\to +\infty}\cos^n{\frac{x}{\sqrt{n}}}$;
\begin{solution}
和上题类似,
\begin{equation*}
\begin{split}
\displaystyle\lim_{n\to +\infty}\cos^n{\frac{x}{\sqrt{n}}}&=
\displaystyle\lim_{n\to +\infty}\left[(1+\cos{\frac{x}{\sqrt{n}}}-1)^{\cos{\frac{x}{\sqrt{n}}}-1}\right]^{n(\cos{\frac{x}{\sqrt{n}}}-1)}\\&=
\displaystyle\lim_{n\to +\infty}\left[(1+\cos{\frac{x}{\sqrt{n}}}-1)^{\cos{\frac{x}{\sqrt{n}}}-1}\right]^{-\frac{x^2}{2}\cdot\left(\frac{\sin{\frac{x}{2\sqrt{n}}}}{\frac{x}{2\sqrt{n}}}\right)^2}=e^{-\frac{x^2}{2}}
\end{split}
\end{equation*}
\end{solution}
\item $\displaystyle\lim_{x\to 0}(2e^{\frac{x}{1+x}}-1)^{\frac{x^2+1}{x}}$;
\begin{solution}由例2.2.13(2)知,
$$\displaystyle\lim_{x\to 0}\frac{e^{\frac{x}{1+x}}-1}{\frac{x}{1+x}} = 1.$$
\begin{equation*}
\begin{split}
\displaystyle\lim_{x\to 0}(2e^{\frac{x}{1+x}}-1)^{\frac{x^2+1}{x}}
&=\displaystyle\lim_{x\to 0}(1+2(e^{\frac{x}{1+x}}-1))^{\frac{1}{2(e^{\frac{x}{1+x}}-1)}2(e^{\frac{x}{1+x}}-1)\frac{x^2+1}{x}}\\
&=\displaystyle\lim_{x\to 0}(1+2(e^{\frac{x}{1+x}}-1))^{\frac{1}{2(e^{\frac{x}{1+x}}-1)}\frac{2(e^{\frac{x}{1+x}}-1)}{\frac{x}{1+x}}\left(\frac{x^2+1}{1+x}\right)}\\
&=e^2
\end{split}
\end{equation*}
\end{solution}
\item $\displaystyle\lim_{x\to a}\left(\frac{\sin{x}}{\sin{a}}\right)^{\frac{1}{x-a}}, a\neq k\pi,k\in\mathbb{Z}$;
\begin{solution}利用三角函数的性质,
\begin{equation*}
\begin{split}
\left(\frac{\sin{x}}{\sin{a}}\right)^{\frac{1}{x-a}}&=\left(1+\frac{\sin{x}-\sin{a}}{\sin{a}}\right)^{\frac{\sin{a}}{\sin{x}-\sin{a}}\cdot \frac{\sin{x}-\sin{a}}{(x-a)\sin{a}}}\\
&=\left(1+\frac{\sin{x}-\sin{a}}{\sin{a}}\right)^{\frac{\sin{a}}{\sin{x}-\sin{a}}\cdot \frac{\sin(x-a)\cos{a}}{(x-a)\sin{a}}\cdot\frac{\sin{a}(\cos(x-a)-1)}{(x-a)\sin{a}}}.
\end{split}
\end{equation*}
于是,$\displaystyle\lim_{x\to a}\left(\frac{\sin{x}}{\sin{a}}\right)^{\frac{1}{x-a}}=e^{\frac{\cos{a}}{\sin{a}}}$.
\end{solution}
\item $\displaystyle\lim_{n\to +\infty}\sqrt{2}\cdot\sqrt[4]{2}\cdot\sqrt[8]{2}\cdots\sqrt[2^n]{2}$;
\begin{solution}记$A=\sqrt{2}\cdot\sqrt[4]{2}\cdot\sqrt[8]{2}\cdots\sqrt[2^n]{2}$,则$$\sqrt[2^n]{2}A=2\Rightarrow A = \frac{2}{\sqrt[2^n]{2}}.$$
于是,$\displaystyle\lim_{n\to +\infty}\sqrt{2}\cdot\sqrt[4]{2}\cdot\sqrt[8]{2}\cdots\sqrt[2^n]{2} = \displaystyle\lim_{n\to +\infty}\frac{2}{\sqrt[2^n]{2}} = 2$.
\end{solution}
\item $\displaystyle\lim_{x\to 0}\frac{(1+x)^{\mu} - 1}{x}$.
\begin{solution}
令$(1+x)^{\mu} = e^y$, 则$x\to 0\iff y\to 0$.
于是,
$$\displaystyle\lim_{x\to 0}\frac{(1+x)^{\mu} - 1}{x}=\displaystyle\lim_{y\to 0}\frac{e^y-1}{(\sqrt[\mu]{e})^y-1}=\mu.$$
\end{solution}
\end{enumerate}
\end{example}
%%%%%%%%%%%%%%%%%%%%%%%
\begin{example}
设$x_n=\underbrace{\sin\cdots\sin}_{\text{n次}}{a}$.证明:$\displaystyle\lim_{n\to +\infty}x_n=0$.
\end{example}
\begin{proof}对$\forall a\in\mathbb{R}$, $$0\leq\underbrace{\sin\cdots\sin}_{\text{(n+1)次}}{a}=x_{n+1} \leq x_{n}=\underbrace{\sin\cdots\sin}_{\text{n次}}{a},\quad \forall n\geq 2.$$
即$\{x_n\}$是单调减有下界的数列,故收敛。设$\displaystyle\lim_{n\to +\infty}x_n=b$,则
$$b=\sin{b},\quad 0\leq b\leq \sin{1}.$$
解方程得$b=0$.
\end{proof}
%%%%%%%%%%%%%%%%%%%%%%%
%%%%%%%%%%%%%%%%%%%%%%%
%%%%%%%%%%%%%%%%%%%%%%%
\subsection{思考题}
%%%%%%%%%%%%%%%%%%%%%%
%%%%%%%%%%%%%%%%%%%%%%
\begin{example}
证明:$\displaystyle\lim_{n\to +\infty}n\sin(2\pi e n!)=2\pi$.(提示:$e=1+\displaystyle\sum_{k=1}^n\frac{1}{k!}+\frac{\theta_n}{n!n}, \quad\frac{n}{n+1}<\theta_n<1$)
\end{example}
\begin{proof}根据提示,计算
$$en!=m+\frac{\theta_n}{n}.$$
所以,$\displaystyle\lim_{n\to +\infty}n\sin(2\pi e n!)=\displaystyle\lim_{n\to +\infty}n\sin{2\pi\frac{\theta_n}{n}}=\displaystyle\lim_{n\to +\infty}2\pi\theta_n\cdot\frac{\sin{2\pi\frac{\theta_n}{n}}}{2\pi\frac{\theta_n}{n}}=2\pi.$
\end{proof}
%%%%%%%%%%%%%%%%%%%%%%%
\begin{example}
证明:$\displaystyle\lim_{n\to +\infty}\left\{\left[(n+1)!\right]^{\frac{1}{n+1}}-(n!)^{\frac{1}{n}}\right\}=\frac{1}{e}$.
\end{example}
\begin{proof}记
$$A=n\cdot\left(\displaystyle\sqrt[n+1]{\frac{n+1}{\sqrt[n]n!}}-1\right), \quad B = \frac{1}{n+1}\left(\ln{\displaystyle\frac{n}{\sqrt[n]{n!}}}+\ln{\frac{n+1}{n}}\right).$$
简单计算可以
$$A=nB\cdot\displaystyle\frac{e^{B}-1}{B}.$$
于是,$\displaystyle\lim_{n\to +\infty}A = \displaystyle\lim_{n\to +\infty}nB\cdot\displaystyle\frac{e^{B}-1}{B}=1$.
所以,
$$\displaystyle\lim_{n\to +\infty}\left\{\left[(n+1)!\right]^{\frac{1}{n+1}}-(n!)^{\frac{1}{n}}\right\}=\displaystyle\lim_{n\to +\infty}\frac{\sqrt[n]{n!}}{n}\cdot A=\frac{1}{e}.$$
命题得证。
\end{proof}
%%%%%%%%%%%%%%%%%%%%%%%
\begin{example}
设$|x| < 1$.证明:
$$\displaystyle\lim_{n\to +\infty}\left(1+\displaystyle\frac{1+x+x^2+\cdots+x^n}{n}\right)^n=e^{\frac{1}{1-x}}.$$
\end{example}
\begin{proof}因为$|x|< 1$, 级数$\displaystyle\sum_{k=0}^{+\infty}x^k = \frac{1}{1-x}$.于是
\begin{equation*}
\begin{split}
\displaystyle\lim_{n\to +\infty}\left(1+\displaystyle\frac{1+x+x^2+\cdots+x^n}{n}\right)^n&=\displaystyle\lim_{n\to +\infty}\left(1+\frac{1-x^{n+1}}{n(1-x)}\right)^{\frac{n(1-x)}{1-x^{n+1}}\frac{1-x^{n+1}}{1-x}}\\&=e^{\frac{1}{1-x}}.
\end{split}
\end{equation*}
命题得证。
\end{proof}
%%%%%%%%%%%%%%%%%%%%%%%
\begin{example}
设$f$与$g$为两个周期函数,且$\displaystyle\lim_{x\to +\infty}[f(x)-g(x)]=0$,证明$f=g$。
\end{example}
\begin{proof}设$f$和$g$的周期分别为$T_f$和$T_g$,则$f(x) = f(x+nT_f)$, $g(x) = g(x+nT_g)$.
考虑子列$\{x+nT_f\}$和$x+nT_g$,有
$$\displaystyle\lim_{n\to +\infty}[f(x+nT_f)-g(x+nT_f)]=0\Rightarrow f(x)=\displaystyle\lim_{n\to +\infty}g(x+nT_f).$$ 
$$\displaystyle\lim_{n\to +\infty}[f(x+nT_g)-g(x+nT_g)]=0\Rightarrow g(x)=\displaystyle\lim_{n\to +\infty}f(x+nT_g).$$
于是,
\begin{equation*}
\begin{split}
f(x)-g(x)&=\displaystyle\lim_{n\to +\infty}(g(x+nT_f)-f(x+nT_g))\\&=\displaystyle\lim_{n\to +\infty}(g(x+nT_f+nT_g)-f(x+nT_g+nT_f))\\&= 0.
\end{split}
\end{equation*}
命题得证。
\end{proof}
%%%%%%%%%%%%%%%%%%%%%%
%%%%%%%%%%%%%%%%%%%%%%%
%%%%%%%%%%%%%%%%%%%%%%%
\section{无穷小(大)量的数量级}
%%%%%%%%%%%%%%%%%%%%%%%%%
%%%%%%%%%%%%%%%%%%%%%%%%
%%%%%%%%%%%%%%%%%%%%%%%
\subsection{练习题}
%%%%%%%%%%%%%%%%%%%%%%
%%%%%%%%%%%%%%%%%%%%%%
\subsection{思考题}
%%%%%%%%%%%%%%%%%%%%%%
%%%%%%%%%%%%%%%%%%%%%%
%\subsubsection{三级节标题}

%\begin{multicols}{2}

%\end{multicols}


%\part{植物多样性分区概述}



\include{angiosperms}

\appendix

%\chapter{}

\renewcommand\indexname{索~~引}
\printindex
\addcontentsline{toc}{chapter}{索~引}

\backmatter

\addcontentsline{toc}{chapter}{参考文献}

\begin{thebibliography}{参考文献}
\bibitem[徐薛]{XX1} 徐森林,薛春华编著 《数学分析》, 清华大学出版社, 2005.
\end{thebibliography}

\chapter{后~~记}

\begin{flushright}

\end{flushright}

\end{document}
